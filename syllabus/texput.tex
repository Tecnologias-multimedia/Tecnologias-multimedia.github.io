% Emacs, this is -*-latex-*-

\newcommand{\TM}{\href{https://tecnologias-multimedia.github.io/}{Tecnologías Multimedia}}

\title{\TM{} - Syllabus}

\maketitle

\section{About}

Tecnologías Multimedia (TM) is an optative subject of the Computer
Science Degree at the UAL (University of Almería).

\section{Course Meeting Times}

See the current \href{https://www.ual.es/estudios/grados/presentacion/plandeestudios/asignatura/4015/40154321}{time-table}.

\section{Scope}

TM is focused on the development of real-time audio processing
techniques, specifically, encoding and transmission.

\section{Main goals}

To develop:
\begin{enumerate}
\item The ability to understand the environment of an organization and
  its needs in the field of information technologies and
  communications, in real time
  (\href{https://www.ual.es/application/files/8516/5061/5446/memoriavig-ing-informatica-4015.pdf}{TI1}).
\item The ability to conceive systems, applications and services based
  on network technologies, including Internet, web, e-commerce,
  multimedia, interactive services and mobile computing
  (\href{https://www.ual.es/application/files/8516/5061/5446/memoriavig-ing-informatica-4015.pdf}{TI6}).
\end{enumerate}

\section{Methodology and Contents}

TM follows the PBL (Project-Based Learning)
\href{http://portafirma.ual.es/pfirma/downloadReport/file?idDocument=4u61Ie5es2&idRequest=ZeBY35LlFa}{methodology}. The
students, helped by the lecturer, develop a project during the
classes. This project is
\href{https://github.com/Tecnologias-multimedia/intercom}{InterCom}(municator),
a Python application that allows networked users to communicate, in
real-time, throught the Internet.

The project is developed as a sequence of milestones. The students, in
groups of up to 4 people, propose solutions for each milestone. Such
solutions are presented to the rest of the class, and each group gives
and receives a feedback from the rest of the class and also from the
lecturer. The best solutions are incorporated to the project, with the
objective of improve it.

The contents of the subject can be found \href{https://tecnologias-multimedia.github.io/contents/}{here}.

\section{Attendance}

Currently,
\href{https://www.ual.es/estudios/grados/presentacion/plandeestudios/asignatura/4015/40154321}{the
  course is organized in 14 lectures of 2 hours/session, and 14
  practicals of 2 hours/session, during 7 weeks (approx.)}, and it
has been developed to be a virtual and blended training. This means
that all the content is available online, but students are expected to
regularly attend the sessions, which are mainly practical.

\section{Evaluation Criteria and Instruments}

TM is evaluated continuously, although there is the possibility of a
single final evaluation, which would be carried out on the day
scheduled for the exam. The continuous evaluation of the students is
carried out from the work developed for each one of the milestones
that are presented in class.

The course evaluation process is the responsibility of the teacher and
follows the following rules:

\begin{enumerate}
\item Sufficiently in advance, the teacher will propose a series of
  milestones associated with the project that is developed in the
  subject. The teacher, both during the master classes and in the
  practical classes, will give all the necessary information so that
  the students can develop the milestones. Generally, all milestones
  can be developed in parallel, although the order in which the
  milestones are completed is not fixed.
\item The teacher will periodically receive information from the
  students about the degree of achievement of the milestones that are
  being developed, with the purpose of directing the development of
  the milestones. When the teacher sees fit, he will propose to the
  students that they present their solutions publicly (in
  class). Normally, the solutions should achieve 100\% of the
  objectives set at each milestone.
\item The grade that students will receive, for groups of up to 4
  people, will depend on the order in which the solutions for the
  milestones have been reached. Specifically, the first group to
  achieve the milestone will receive the highest rating (10) and the
  rest of the groups, in temporal order of achievement of the
  milestone, will be rated with lower and lower grades, which will
  always be higher than 5 if the milestone has been properly
  developed.
\item The groups that do not present any solution will receive a score
  of 0. The groups that have not approved following these rules will
  be qualified as "Not Presented".
\end{enumerate}

This information is extended in the Teaching Guide
(\href{https://portafirma.ual.es/pfirma/downloadReport/file?idDocument=4Jp82utmug&idRequest=QY36GYcOZQ}{Español})/(\href{https://portafirma.ual.es/pfirma/downloadReport/file?idDocument=Zcmom6qigD&idRequest=xXgueuk9oD}{English}).
