\documentclass[legalpaper, 12pt, addpoints]{exam}
%\documentclass[legalpaper, 12pt, addpoints, answers]{exam}

\usepackage[margin=1in]{geometry}
\usepackage[utf8]{inputenc}
\usepackage{graphics}
\usepackage{color}
\usepackage{amssymb}
\usepackage{amsmath}
\usepackage{enumitem}
\usepackage{xcolor}
\usepackage{cancel}
\usepackage{ragged2e}
\usepackage{graphicx}
\usepackage{multicol}
\usepackage{color}
\usepackage{tikz}

\CorrectChoiceEmphasis{\itseries\color{red}}

\begin{document}
%\begin{coverpages}

%---uncomment to add a custom header (replace {header-cufm.png})---- %
%\begin{figure}[t]
%\includegraphics[width=1\textwidth,height=1.2\textheight,keepaspectratio]{%header-cufm.png}
%\end{figure}

\begin{center}
  \textbf{Tecnologías Multimedia} \\
  Examen final\\
  Julio 2021
\end{center}
\extraheadheight{-0.8in}

\vspace{0.25in}
{\fbox{\parbox{6in}{\parbox{6in}{{\textbf{Instrucciones:} Lea cuidadosamente cada pregunta y responda redondeando con un círculo la solición A o B que considere correcta. Tiempo total: 120 minutos.}}}}}
\runningheadrule \extraheadheight{0.14in}

\lhead{\ifcontinuation{Question \ContinuedQuestion\ continues\ldots}{}}
\runningheader{Tecnologñias Multimedia}{Examen final}{Julio 2021}
\runningfooter{}
              {Página \thepage\ de \numpages}
              {}
\vspace{0.15in}

\vspace{0.1in}

%-comment out the next line to display point value for each question -%
%\nopointsinmargin
%\setlength\linefillthickness{0.1pt}
%\setlength\answerlinelength{0.1in}
%\end{coverpages}

\begin{questions}

\question Un intercomunicador como el desarrollado en el proyecto de la asignatura es:

\begin{oneparchoices}
  \choice Una aplicación peer-to-peer que permite a usuarios conectados a la red comunicarse entre sí.\\
  \choice Un servicio Web implementado en Python que permite a los clientes comunicarse entre sí.
\end{oneparchoices}
  
\vspace{0.10in}

\question Una onda sonora es:

\begin{oneparchoices}
  \choice Una señal mecánica de presión que se transmite sólo por el aire.\\
  \choice Cualquier onda mecánica de presión que proporciona información auditiva.
\end{oneparchoices}
  
\question Cuando una onda sonora se representa en el dominio de la frecuencia:

\begin{oneparchoices}
  \choice La amplitud de la señal representa su energía en función del tiempo.\\
  \choice La amplitud de la señal representa de alguna forma su energía en función de la frequencia.
\end{oneparchoices}
  
\vspace{0.10in}

\question La mayoría de los sonidos naturales que percibimos los seres humanos tienden a:

\begin{oneparchoices}
  \choice Acumular la mayor parte de su energía en las bajas frecuencias.\\
  \choice Acumular la mayor parte de su energía en las altas frecuencias.
\end{oneparchoices}
  
\vspace{0.10in}

\question Aproximadamente, un ser humano joven y sano puede percibir sonidos entre:

\begin{oneparchoices}
  \choice 0 Hz y 20000 Hz.\\
  \choice 20 Hz y 20000 Hz.
\end{oneparchoices}
  
\vspace{0.10in}

\question El nivel de intensidad sonora suele medirse en decibelios porque:

\begin{oneparchoices}
  \choice Nuestra precepción de la intensidad sonora no es lineal, sino logarítmica.\\
  \choice Existe un estándar internacional que indica que debe hacerse de esta manera.
\end{oneparchoices}
  
\vspace{0.10in}

\question En general, los humanos percibimos con más facilidad
aquellos sonidos cuyas componentes frecuenciales están entre:

\begin{oneparchoices}
  \choice 3 KHz y 4 KHz, aproximadamente.\\
  \choice 10 KHz y 15 KHz, aproximadamente.
\end{oneparchoices}
  
\vspace{0.10in}

\question La percepción binaural ayuda a los humanos a localizar la fuente de los sonidos:

\begin{oneparchoices}
  \choice Verdadero.\\
  \choice Falso.
\end{oneparchoices}
  
\vspace{0.10in}

\question Una muestra de audio:

\begin{oneparchoices}
  \choice Representa cuantos bits necesita una señal de audio en un instante de tiempo determinado.\\
  \choice Representa el nivel de presión sonora en un instante de tiempo determinado.
\end{oneparchoices}
  
\vspace{0.10in}

\question En la biblioteca \texttt{sounddevice}, un frame es:

\begin{oneparchoices}
  \choice Una pareja de muestras de audio.\\
  \choice La estructura de datos usada para representar un chunk de audio.
\end{oneparchoices}
  
\vspace{0.10in}

\question Una señal de audio estereo se diferencia de una señal de audio mono en que:

\begin{oneparchoices}
  \choice Las calidad de los canales es mejor en la señal estereo.\\
  \choice La señal de audio estéreo transporta información binaural y la mono, no.
\end{oneparchoices}
  
\vspace{0.10in}

\question La percepción binaural ayuda a los humanos a localizar la fuente de los sonidos:

\begin{oneparchoices}
  \choice Verdadero.\\
  \choice Falso.
\end{oneparchoices}
  
\vspace{0.10in}

\question Un frame de audio estereo se diferencia de un frame de audio mono en que:

\begin{oneparchoices}
  \choice Los frames de audio estéreo suelen representarse con 16
  bits/muestra,
  mientras que los frames de audio mono suelen representarse con 8 bits/muestra.\\
  \choice Los frames de audio estéreo representan en un instante de
  tiempo determinado a dos canales mono, uno izquierdo y otro derecho.
\end{oneparchoices}
  
\vspace{0.10in}

\question Con 16 bits/muestra con signo, en general:

\begin{oneparchoices}
  \choice El rango dinámico de la señal de audio comprende entre -32768 y 32767.\\
  \choice El rango dinámico de las muestras estéreo comprende entre -16 y 15, por cada canal.
\end{oneparchoices}
  
\vspace{0.10in}

\question Una única muestra de audio representada en el dominio de la frecuencia:

\begin{oneparchoices}
  \choice Indica el nivel de presión sonora en el dominio de la frecuencia.\\
  \choice No representa nada en el dominio de la frecuencia, puesto
  que no es posible averiguar nada sobre cómo varía el sonido usando
  una única muestra.
\end{oneparchoices}
  
\vspace{0.10in}

\question Cuando usamos 16 bits/muestra, siempre trabajos con audio estéreo:

\begin{oneparchoices}
  \choice Verdadero.\\
  \choice Falso.
\end{oneparchoices}
  
\vspace{0.10in}

\question Una señal de audio en calidad CD implica que:

\begin{oneparchoices}
  \choice Usamos 16 bits/muestra.\\
  \choice Usamos 24 bits/muestra.
\end{oneparchoices}
  
\vspace{0.10in}

\question Una señal de audio en calidad CD implica que:

\begin{oneparchoices}
  \choice Se codifican 3 canales independientes de audio: el canal izquierdo, el derecho y el surround.\\
  \choice La señal de audio es estéreo.
\end{oneparchoices}
  
\vspace{0.10in}

\question Una señal de audio en calidad CD implica que:

\begin{oneparchoices}
  \choice El rango de frecuencias registradas comprende desde 20 Hz a 20 KhZ.\\
  \choice El rango de frecuencias registradas comprende desde 0 Hz a 22050 Hz.
\end{oneparchoices}
  
\vspace{0.10in}

\question Una señal de audio en calidad CD implica que:

\begin{oneparchoices}
  \choice La frecuencia de muestreo es de 44100 Hz, por cada canal.\\
  \choice La frecuencia de muestreo es de 44100 Hz, entre los dos canales.
\end{oneparchoices}
  
\vspace{0.10in}

\question En una señal de audio en calidad CD suele ocurrir que:

\begin{oneparchoices}
  \choice Las bajas frecuencias se representan con mayor calidad que las altas frecuencias.\\
  \choice Todas las frecuencias se representan con la misma calidad.
\end{oneparchoices}
  
\vspace{0.10in}

\question En una señal de audio en calidad CD cada canal genera la misma tasa de bits:

\begin{oneparchoices}
  \choice Verdadero.\\
  \choice Falso.
\end{oneparchoices}
  
\vspace{0.10in}

\question Las señales de audio en calidad CD siempre tienen una tasa de bits igual a:

\begin{oneparchoices}
  \choice 44100 muestras/segundo * 2 canales * 16 bits/muestra.\\
  \choice La calidad no representa necesariamente una única tasa de
  bits porque la señal de audio puede estar comprimida.
\end{oneparchoices}
  
\vspace{0.10in}

\question Según el Teorema del Muestreo, para llegar a representar la
frecuencia 2000 Hz, la tasa de muestreo debe ser:

\begin{oneparchoices}
  \choice 2000 muestras/segundo.\\
  \choice 4000 muestras/segundo.
\end{oneparchoices}
  
\vspace{0.10in}

\question Para verificar el Teorema del Muestreo, las tarjetas de
sonido deben incorporar un filtro paso bajo cuya frecuencia de corte
debe de coincidir con la máxima componente de frecuencia que se desea
registrar):

\begin{oneparchoices}
  \choice Verdadero.\\
  \choice Falso.
\end{oneparchoices}
  
\vspace{0.10in}

\question Un ADC (Analog to Digital Converter) de un PC tiene por misión:

\begin{oneparchoices}
  \choice Descomprimir las señales de audio para que éstas sean audibles por los seres humanos.\\
  \choice Muestrear las señales (presumiblemente de audio) que llegan a su entrada.
\end{oneparchoices}
  
\vspace{0.10in}

\question Un ADC tiene como entrada una señal representada mediante
alguna magnitud física (como voltaje o intensidad luminosa):

\begin{oneparchoices}
  \choice Verdadero.\\
  \choice Falso.
\end{oneparchoices}
  
\vspace{0.10in}

\question Un ADC tiene como salida una señal representada mediante
alguna magnitud física (como voltaje o intensidad luminosa):

\begin{oneparchoices}
  \choice Verdadero.\\
  \choice Falso.
\end{oneparchoices}

\question Un DAC (Digital to Analog Converter) tiene como entrada una
señal representada mediante alguna magnitud física (como voltaje o
intensidad luminosa):

\begin{oneparchoices}
  \choice Verdadero.\\
  \choice Falso.
\end{oneparchoices}
  
\vspace{0.10in}

\question Un DAC tiene como salida una señal representada mediante
alguna magnitud física (como voltaje o intensidad luminosa):

\begin{oneparchoices}
  \choice Verdadero.\\
  \choice Falso.
\end{oneparchoices}
  
\vspace{0.10in}

\question Un DAC tiene como entrada una señal representada mediante
alguna magnitud física (como voltaje o intensidad luminosa):

\begin{oneparchoices}
  \choice Verdadero.\\
  \choice Falso.
\end{oneparchoices}
  
\vspace{0.10in}

\question Un DAC y un ADC son dispositivos de transducción, es decir,
elementos que transforman intensidades físicas en señales digitales, y
viceversa:

\begin{oneparchoices}
  \choice Verdadero.\\
  \choice Falso.
\end{oneparchoices}
  
\vspace{0.10in}

\question La latencia en un sistema de reproducción de audio, en general, mide:

\begin{oneparchoices}
  \choice La diferencia de tiempo que transcurre desde que una señal es capturada hasta que es reproducida.\\
  \choice El tiempo que transcurre desde que la señal se ``envía'' al
  reproductor hasta que dicha señal es finalmente reproducida.
\end{oneparchoices}
  
\vspace{0.10in}

\question Los sistemas de captura de audio devuelven bloques (chunks)
de muestras, y el tamaño de dichos bloques influye en la latencia:

\begin{oneparchoices}
  \choice Verdadero.\\
  \choice Falso.
\end{oneparchoices}
  
\vspace{0.10in}

\question \texttt{sounddevice} genera chunks de audio comprimidos, es
decir, donde la tasa de bits de la representación es inferior a la
tasa de bits de captura/reproducción:

\begin{oneparchoices}
  \choice Verdadero.\\
  \choice Falso.
\end{oneparchoices}
  
\vspace{0.10in}




\end{questions}

\end{document}