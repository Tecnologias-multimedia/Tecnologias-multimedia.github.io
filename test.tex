\documentclass[legalpaper, 12pt, addpoints]{exam}
%\documentclass[legalpaper, 12pt, addpoints, answers]{exam}

\usepackage[spanish]{babel}
\usepackage[margin=1in]{geometry}
\usepackage[utf8]{inputenc}
\usepackage{graphics}
\usepackage{color}
\usepackage{amssymb}
\usepackage{amsmath}
\usepackage{enumitem}
\usepackage{xcolor}
\usepackage{cancel}
\usepackage{ragged2e}
\usepackage{graphicx}
\usepackage{multicol}
\usepackage{color}
\usepackage{tikz}
\usepackage{datetime}

\CorrectChoiceEmphasis{\itseries\color{red}}

\begin{document}
%\begin{coverpages}
%---uncomment to add a custom header (replace {header-cufm.png})---- %
%\begin{figure}[t]
%\includegraphics[width=1\textwidth,height=1.2\textheight,keepaspectratio]{%header-cufm.png}
%\end{figure}

\newcommand{\convocatoria}{Enero, 2022}

\begin{center}
  \textbf{Tecnologías Multimedia} \\
  Test\\
  \convocatoria
\end{center}
\extraheadheight{-0.8in}

\vspace{0.25in} {\fbox{\parbox{6in}{\parbox{6in}{{Lea cuidadosamente
          cada pregunta y responda redondeando con un círculo la
          solución A o B que considere correcta (también existe la
          opción de escribir las respuestas en una hoja aparte,
          indicando el número de la pregunta y la letra de la
          respuesta correcta), y no olvide escribir además su
          nombre, apellidos y DNI. Se recuerda que una respuesta
          incorrecta invalida otra correcta. Todas las preguntas
          tienen el mismo peso en la nota obtenida.  Compilado:
          \today, \currenttime.\\\\\textbf{Tiempo total: 120 minutos}.}}}}}
\runningheadrule \extraheadheight{0.14in}

\lhead{\ifcontinuation{Question \ContinuedQuestion\ continues\ldots}{}}
\runningheader{Tecnologías Multimedia}{Test}{\convocatoria}
\runningfooter{}
              {Página \thepage\ de \numpages}
              {}
\vspace{0.15in}

\vspace{0.1in}

%-comment out the next line to display point value for each question -%
%\nopointsinmargin
%\setlength\linefillthickness{0.1pt}
%\setlength\answerlinelength{0.1in}
%\end{coverpages}

\begin{questions}

%\question Un intercomunicador como el desarrollado en el proyecto de la asignatura es:

%\begin{oneparchoices}
%  \choice Una aplicación peer-to-peer que permite a usuarios conectados a la red comunicarse entre sí.\\
%  \choice Un servicio Web implementado en Python que permite a los clientes comunicarse entre sí.
%\end{oneparchoices}
  
\vspace{0.10in}

\question Una onda sonora es:

\begin{oneparchoices}
  \choice Una señal mecánica de presión que se transmite sólo por el aire.\\
  \choice Cualquier onda mecánica de presión que proporciona información auditiva.
\end{oneparchoices}
  
\question Cuando una onda sonora se representa en el dominio de la frecuencia:

\begin{oneparchoices}
  \choice La amplitud de la señal representa su energía en función del tiempo.\\
  \choice La amplitud de la señal representa de alguna forma su
  energía en función de la frecuencia.
\end{oneparchoices}
  
\vspace{0.10in}

\question La mayoría de los sonidos naturales que percibimos los seres humanos tienden a:

\begin{oneparchoices}
  \choice Acumular la mayor parte de su energía en las bajas frecuencias.\\
  \choice Acumular la mayor parte de su energía en las altas frecuencias.
\end{oneparchoices}
  
\vspace{0.10in}

\question Aproximadamente, un ser humano joven y sano puede percibir sonidos entre:

\begin{oneparchoices}
  \choice 0 Hz y 44100 Hz.\\
  \choice 20 Hz y 20000 Hz.
\end{oneparchoices}
  
\vspace{0.10in}

\question El nivel de intensidad sonora suele medirse en decibelios porque:

\begin{oneparchoices}
  \choice Nuestra percepción de la intensidad sonora no es lineal, sino logarítmica.\\
  \choice Existe un estándar internacional que indica que debe hacerse de esta manera.
\end{oneparchoices}

\vspace{0.10in}

\question En general, los humanos percibimos con más facilidad
aquellos sonidos cuyas componentes frecuenciales están entre:

\begin{oneparchoices}
  \choice 3 KHz y 4 KHz, aproximadamente.\\
  \choice 10 KHz y 15 KHz, aproximadamente.
\end{oneparchoices}
  
\vspace{0.10in}

\question La percepción binaural ayuda a los humanos a localizar espacialmente la fuente de los sonidos:

\begin{oneparchoices}
  \choice Verdadero.\\
  \choice Falso.
\end{oneparchoices}
  
\vspace{0.10in}

\question Una muestra de audio:

\begin{oneparchoices}
  \choice Representa cuantos bits necesita una señal de audio en un instante de tiempo determinado.\\
  \choice Representa el nivel de presión sonora en un instante de tiempo determinado.
\end{oneparchoices}
  
\vspace{0.10in}

\question En la biblioteca \texttt{sounddevice}, un \emph{frame} es:

\begin{oneparchoices}
  \choice Una pareja de muestras de audio.\\
  \choice La estructura de datos usada para representar un \emph{chunk} de audio.
\end{oneparchoices}
  
\vspace{0.10in}

\question Una señal de audio estereo se diferencia de una señal de audio mono en que:

\begin{oneparchoices}
  \choice Las calidad de los canales es mejor en la señal estereo.\\
  \choice La señal de audio estéreo transporta información espacial y la mono, no.
\end{oneparchoices}
  
\vspace{0.10in}

\question Un \emph{frame} de audio estereo se diferencia de un frame
de audio mono en que:

\begin{oneparchoices}
  \choice Los \emph{frames} de audio estéreo suelen representarse con
  16 bits/muestra,
  mientras que los \emph{frames} de audio mono suelen representarse con 8 bits/muestra.\\
  \choice Los \emph{frames} de audio estéreo representan en un
  instante de tiempo determinado a dos canales mono, uno izquierdo y
  otro derecho.
\end{oneparchoices}
  
\vspace{0.10in}

\question Con 16 bits/muestra con signo, en general:

\begin{oneparchoices}
  \choice El rango dinámico de la señal de audio comprende entre -32768 y 32767.\\
  \choice El rango dinámico de las muestras estéreo comprende entre -65536 y 65535, por cada canal.
\end{oneparchoices}
  
\vspace{0.10in}

\question Una única muestra de audio representada en el dominio de la frecuencia:

\begin{oneparchoices}
  \choice Indica el nivel de presión sonora en el dominio de la frecuencia.\\
  \choice No representa nada en el dominio de la frecuencia, puesto
  que no es posible averiguar nada sobre cómo varía el sonido usando
  una única muestra.
\end{oneparchoices}
  
\vspace{0.10in}

\question Cuando usamos 16 bits/muestra, siempre trabajamos con audio estéreo:

\begin{oneparchoices}
  \choice Verdadero.\\
  \choice Falso.
\end{oneparchoices}
  
\vspace{0.10in}

\question Una señal de audio en calidad CD (Compact Disk) implica que:

\begin{oneparchoices}
  \choice Usamos 16 bits/muestra.\\
  \choice Usamos 24 bits/muestra.
\end{oneparchoices}
  
\vspace{0.10in}

\question Una señal de audio en calidad CD implica que:

\begin{oneparchoices}
  \choice Se codifican 3 canales independientes de audio: el canal izquierdo, el derecho y el surround.\\
  \choice La señal de audio es estéreo.
\end{oneparchoices}
  
\vspace{0.10in}

\question Una señal de audio en calidad CD implica que:

\begin{oneparchoices}
  \choice El rango de frecuencias registradas comprende desde 20 Hz a 20 KhZ.\\
  \choice El rango de frecuencias registradas comprende desde 0 Hz a 22050 Hz.
\end{oneparchoices}
  
\vspace{0.10in}

\question Una señal de audio en calidad CD implica que:

\begin{oneparchoices}
  \choice La frecuencia de muestreo es de 44100 Hz, por cada canal.\\
  \choice La frecuencia de muestreo es de 44100 Hz, entre los dos canales.
\end{oneparchoices}
  
\vspace{0.10in}

\question En una señal de audio en calidad CD ocurre que:

\begin{oneparchoices}
  \choice Las bajas frecuencias se representan con mayor calidad que las altas frecuencias.\\
  \choice Todas las frecuencias se representan con la misma calidad.
\end{oneparchoices}
  
\vspace{0.10in}

\question En una señal de audio en calidad CD cada canal genera la misma tasa de bits:

\begin{oneparchoices}
  \choice Verdadero.\\
  \choice Falso.
\end{oneparchoices}
  
\vspace{0.10in}

\question Las señales de audio en formato CD siempre tienen una tasa de bits igual a:

\begin{oneparchoices}
  \choice 44100 muestras/segundo * 2 canales * 16 bits/muestra.\\
  \choice La calidad no representa necesariamente una única tasa de
  bits porque la señal de audio puede estar comprimida.
\end{oneparchoices}
  
\vspace{0.10in}

\question Según el Teorema del Muestreo, para llegar a representar la
frecuencia 2000 Hz, la tasa de muestreo debe ser:

\begin{oneparchoices}
  \choice 2000 muestras/segundo.\\
  \choice 4000 muestras/segundo.
\end{oneparchoices}
  
\vspace{0.10in}

\question Para evitar el \emph{aliasing} durante el proceso de
muestreo digital, las tarjetas de sonido deben incorporar un filtro
paso bajo cuya frecuencia de corte debe de coincidir con la máxima
componente de frecuencia que se desea registrar:

\begin{oneparchoices}
  \choice Verdadero.\\
  \choice Falso.
\end{oneparchoices}
  
\vspace{0.10in}

\question Un ADC (Analog to Digital Converter) tiene por misión:

\begin{oneparchoices}
  \choice Descomprimir las señales de audio para que éstas sean audibles por los seres humanos.\\
  \choice Muestrear las señales (presumiblemente de audio) que llegan a su entrada.
\end{oneparchoices}
  
\vspace{0.10in}

\question Un ADC tiene como entrada una señal representada mediante
alguna magnitud física (como voltaje o intensidad luminosa):

\begin{oneparchoices}
  \choice Verdadero.\\
  \choice Falso.
\end{oneparchoices}
  
\vspace{0.10in}

\question Un ADC tiene como salida una señal representada mediante
alguna magnitud física (como voltaje o intensidad luminosa):

\begin{oneparchoices}
  \choice Verdadero.\\
  \choice Falso.
\end{oneparchoices}

\question Un DAC (Digital to Analog Converter) tiene como entrada una
señal representada mediante alguna magnitud física (como voltaje o
intensidad luminosa):

\begin{oneparchoices}
  \choice Verdadero.\\
  \choice Falso.
\end{oneparchoices}
  
\vspace{0.10in}

\question Un DAC tiene como salida una señal representada mediante
alguna magnitud física (como voltaje o intensidad luminosa):

\begin{oneparchoices}
  \choice Verdadero.\\
  \choice Falso.
\end{oneparchoices}
  
\vspace{0.10in}

\question La latencia en un sistema de reproducción de audio, en general, mide:

\begin{oneparchoices}
  \choice La diferencia de tiempo que transcurre desde que una señal es capturada hasta que es reproducida.\\
  \choice El tiempo que transcurre desde que la señal se ``envía'' al
  reproductor hasta que dicha señal es finalmente reproducida.
\end{oneparchoices}
  
\vspace{0.10in}

\question Los sistemas de captura de audio devuelven bloques (\emph{chunks})
de muestras, y el tamaño de dichos bloques influye en la latencia:

\begin{oneparchoices}
  \choice Verdadero.\\
  \choice Falso.
\end{oneparchoices}
  
\vspace{0.10in}

\question El número de bits/muestra influye en la latencia:

\begin{oneparchoices}
  \choice Verdadero.\\
  \choice Falso.
\end{oneparchoices}
  
\vspace{0.10in}

\question \texttt{sounddevice} genera \emph{chunks} de audio
comprimidos, es decir, donde la tasa de bits de la representación de
la señal de audio es inferior a la tasa de bits de
captura/reproducción:

\begin{oneparchoices}
  \choice Verdadero.\\
  \choice Falso.
\end{oneparchoices}
  
\vspace{0.10in}

\question El tamaño del bloque (\emph{chunk}) usando en \texttt{sounddevice}
influye de forma lineal en la latencia:

\begin{oneparchoices}
  \choice Verdadero.\\
  \choice Falso.
\end{oneparchoices}
  
\vspace{0.10in}

\question La tasa de muestreo influye proporcionalmente en la latencia:

\begin{oneparchoices}
  \choice Sí, de forma inversa.\\
  \choice Sí, de forma exponencial.
\end{oneparchoices}
  
\vspace{0.10in}

\question Cuando la tasa de muestreo y el tamaño del \emph{chunk}
permanecen constantes, el tiempo de \emph{chunk} también lo hace

\begin{oneparchoices}
  \choice Verdadero.\\
  \choice Falso.
\end{oneparchoices}
  
\vspace{0.10in}

\question En Intercom, la tasa de muestreo se varía en tiempo de
ejecución para reducir en la medida de lo posible la tasa de
transmisión

\begin{oneparchoices}
  \choice Verdadero.\\
  \choice Falso.
\end{oneparchoices}
  
\vspace{0.10in}

\question En la implementación de Intercom, donde los \emph{chunks} de
audio no son comprimidos, la tasa de generación de los \emph{chunks}
coincide con la tasa de envío de los \emph{chunks}:

\begin{oneparchoices}
  \choice Verdadero.\\
  \choice Falso.
\end{oneparchoices}
  
\vspace{0.10in}

\question En Intercom, donde los \emph{chunks} de audio sí son
comprimidos (cada uno por separado), la tasa de generación de los
\emph{chunks} coincide con la tasa de envío de los \emph{chunks} (cada
uno por separado), al menos en promedio:

\begin{oneparchoices}
  \choice Verdadero.\\
  \choice Falso.
\end{oneparchoices}
  
\vspace{0.10in}

\question Una forma de conseguir que exista solapamiento en el tiempo
entre la captura, el envío, la recepción y la reproducción de
\emph{chunks} dentro de una sección de código aparentemente secuencial
es mediante el uso de paralelismo a nivel hardware:

\begin{oneparchoices}
  \choice Verdadero.\\
  \choice Falso, si el código es secuencial, dichos pasos no se solapan en el tiempo.
\end{oneparchoices}
  
\vspace{0.10in}

\question Si cuando se captura un \emph{chunk} el hardware de sonido
ya tiene uno disponible, entonces:

\begin{oneparchoices}
  \choice El método retorna inmediatamente con dicho \emph{chunk}.\\
  \choice El método se bloquea hasta que el siguiente \emph{chunk} en ser reproducido ha terminado dicho proceso.
\end{oneparchoices}
  
\vspace{0.10in}

\question Si el tiempo de \emph{chunk} coindice con la cadencia entre
llamadas al método que captura los \emph{chunks}, entonces:

\begin{oneparchoices}
  \choice Esta circunstancia no se puede dar porque no se conoce en general el tiempo de \emph{chunk}.\\
  \choice El método de captura se comporta como si fuera no bloqueante.
\end{oneparchoices}
  
\vspace{0.10in}

\question Para usar Intercom en un entorno real, donde los dos
interlocutores usan hosts diferentes:

%\begin{oneparchoices}
%  \choice Ambos intercomunicadores envían y reciben \emph{chunks}.\\
%  \choice Ambos intercomunicadores envían y reciben \emph{chunks}, y además, estos se transmiten por el enlace de red que conecta dichos hosts.
%\end{oneparchoices}
  
\begin{oneparchoices}
  \choice Es necesario especificar el \emph{end-point} (dirección IP y puerto) destino de los \emph{chunks} que son enviados, y este end-point se recibe con los \emph{chunks} recibidos.\\
  \choice Es necesario especificar el end-point (dirección IP y puerto) destino de los \emph{chunks} que son enviados, y este \emph{end-point} debe comunicarse de forma manual al interlocutor.
\end{oneparchoices}

\vspace{0.10in}

\question En una prueba de Intercom donde dos interlocutores están en
hosts diferentes, la latencia total experimentada por los usuarios es
igual a:

\begin{oneparchoices}
  \choice El tiempo que tarde un \emph{chunk} desde viajar a uno de los hosts hasta que el \emph{chunk} de vuelta es recibido, entendiendo por \emph{chunk} de vuelta el que contiene la respuesta al \emph{chunk} enviado.\\
  \choice La suma de la latencia del enlace más la latencia generada por el intercomunicador.
\end{oneparchoices}
  
\vspace{0.10in}

\question En general, la latencia total que experimentan los usuarios
de Intercom depende de:

\begin{oneparchoices}
  \choice La tasa de transmisión que soporta la red.\\
  \choice La latencia de la red.
\end{oneparchoices}
  
\vspace{0.10in}

\question La variación de la latencia de la red (\emph{jitter})
influye en la tasa de \emph{chunks} enviados:

\begin{oneparchoices}
  \choice Verdadero.\\
  \choice Falso.
\end{oneparchoices}
  
\vspace{0.10in}

\question La variación de la latencia puede influir en la tasa de
pérdida de \emph{chunks}, a pesar de usar un buffer suficientemtente
grande:

\begin{oneparchoices}
  \choice Verdadero.\\
  \choice Falso.
\end{oneparchoices}
  
\vspace{0.10in}

\question La variación de la latencia puede influir en la calidad del
sonido, a pesar de usar un buffer suficientemente grande:

\begin{oneparchoices}
  \choice Verdadero.\\
  \choice Falso.
\end{oneparchoices}
  
\vspace{0.10in}

\question La variación de la latencia de la red influye en la latencia
total experimentada por los usuarios

\begin{oneparchoices}
  \choice Verdadero.\\
  \choice Falso.
\end{oneparchoices}
  
\vspace{0.10in}

\question El buffering de \emph{chunks} aumenta la tasa de transmisión
de \emph{chunks}:

\begin{oneparchoices}
  \choice Verdadero.\\
  \choice Falso.
\end{oneparchoices}
  
\vspace{0.10in}

\question El \emph{buffering} de \emph{chunks} esconde el
\emph{jitter} del \emph{hardware} de sonido:

\begin{oneparchoices}
  \choice Verdadero.\\
  \choice Falso.
\end{oneparchoices}
  
\vspace{0.10in}

\question El \emph{buffering} de \emph{chunks} esconde básicamente el
\emph{jitter} del enlace de transmisión:

\begin{oneparchoices}
  \choice Verdadero.\\
  \choice Falso.
\end{oneparchoices}
  
\vspace{0.10in}

\question El \emph{buffering} de \emph{chunks} esconde el
\emph{jitter} de la CPU, generalmente provocado porque los sistemas
operatios suelen ser multiprogramados:

\begin{oneparchoices}
  \choice Verdadero.\\
  \choice Falso.
\end{oneparchoices}
  
\vspace{0.10in}

\question El tiempo de \emph{buffering} influye en la calidad del
sonido transmitido, siempre y cuando permita acomodar el
\emph{jitter}:

\begin{oneparchoices}
  \choice Verdadero.\\
  \choice Falso.
\end{oneparchoices}
  
\vspace{0.10in}

\question Se puede testear el funcionamiento de Intercom en un único host porque:

\begin{oneparchoices}
  \choice Una instancia de Intercom envía y recibe \emph{chunks}.\\
  \choice No se puede, porque el \emph{hardware} de sonido no puede capturar y reproducir al mismo tiempo.
\end{oneparchoices}
  
\vspace{0.10in}

\question En general, el tiempo de \emph{buffering}:

\begin{oneparchoices}
  \choice Es menor si las instancias de Intercom corren en el mismo host.\\
  \choice Es mayor si sólo se ejecuta una instancia de Intercom porque la CPU debe enviar y recibir.
\end{oneparchoices}
  
\vspace{0.10in}

%\question Uno de los parámetros que tenemos que especificarle a
%Intercom es el end-point (dirección IP y puerto) del intercom
%``interlocutor'':

%\begin{oneparchoices}
%  \choice Verdadero.\\
%  \choice Falso.
%\end{oneparchoices}
  
%\vspace{0.10in}

\question En Intercom, los \emph{chunks} perdidos:

\begin{oneparchoices}
  \choice Los \emph{chunks} no se pierden, se retrasan.\\
  \choice Se rellenan con ceros y son reproducidos.
\end{oneparchoices}
  
\vspace{0.10in}

\question En Intercom, los \emph{chunks} perdidos:

\begin{oneparchoices}
  \choice Generan silencios durante la reproducción.\\
  \choice Simplemente no se reproducen.
\end{oneparchoices}
  
\vspace{0.10in}

\question Cuando se reproduce una secuencia de muestras cuyo valor no varía en el tiempo:

\begin{oneparchoices}
  \choice El sonido que se reproduce se repite en el tiempo.\\
  \choice No se reproduce ningún sonido.
\end{oneparchoices}
  
\vspace{0.10in}

\question A efectos sonoros:

\begin{oneparchoices}
  \choice Es lo mismo reproducir una secuencia de muestras iguales a cero que a cualquier otro valor constante.\\
  \choice Es más agradable reproducir secuencias de ceros.
\end{oneparchoices}
  
\vspace{0.10in}

\question Una tasa de transmisión soportada por el enlace de
comunicación inferior a la tasa de datos generada por Intercom afecta
a la tasa de pérdida de \emph{chunks}:

\begin{oneparchoices}
  \choice Verdadero.\\
  \choice Falso.
\end{oneparchoices}
  
\vspace{0.10in}

\question La tasa de transmisión soportada por el enlace afecta
significativamente a la tasa de pérdida de \emph{frames} individuales,
pero sólo si dicha tasa es inferior a la tasa de bits generada por
Intercom:

\begin{oneparchoices}
  \choice Verdadero.\\
  \choice Falso.
\end{oneparchoices}
  
\vspace{0.10in}

\question Disminuyendo el tamaño de los \emph{chunks} podemos
conseguir usar Intercom con éxito (sin pérdida de datos) sobre un
enlace con baja tasa de transmisión:

\begin{oneparchoices}
  \choice Verdadero.\\
  \choice Falso.
\end{oneparchoices}
  
\vspace{0.10in}

\question Disminuyendo la tasa de \emph{chunks} podemos conseguir usar
Intercom con éxito (sin pérdida de datos) sobre un enlace con baja
tasa de transmisión:

\begin{oneparchoices}
  \choice Verdadero.\\
  \choice Falso.
\end{oneparchoices}
  
\vspace{0.10in}

\question Disminuyendo la tasa de bits de audio podemos conseguir que
se pierdan menos \emph{chunks} de audio sobre un enlace con baja tasa
de transmisión:

\begin{oneparchoices}
  \choice Verdadero.\\
  \choice Falso.
\end{oneparchoices}
  
\vspace{0.10in}

\question Básicamente, el \emph{buffer} usado en Intercom sirve para
retrasar la reproducción respecto de la llegada de \emph{chunks} y así
poder reproducir a tiempo los \emph{chunks} que llegan retrasados:

\begin{oneparchoices}
  \choice Verdadero.\\
  \choice Falso.
\end{oneparchoices}
  
\vspace{0.10in}

\question Cuando se almacena un \emph{chunk} en el \emph{buffer} de Intercom, dicho \emph{chunk}:

\begin{oneparchoices}
  \choice Se reproduce inmediatamente.\\
  \choice Permanece en el \emph{buffer} hasta que le llega el instante de su reproducción.
\end{oneparchoices}
  
\vspace{0.10in}

\question Cuando comprimimos con DEFLATE no existe pérdida de información:

\begin{oneparchoices}
  \choice Verdadero.\\
  \choice Falso.
\end{oneparchoices}
  
\vspace{0.10in}

\question Cuando comprimimos con DEFLATE la tasa de bits de salida es constante:

\begin{oneparchoices}
  \choice Verdadero.\\
  \choice Falso.
\end{oneparchoices}
  
\vspace{0.10in}

\question Cuando comprimimos con DEFLATE la tasa de compresión depende
de los datos de entrada (esto es, del contenido de los \emph{chunks} de
audio):

\begin{oneparchoices}
  \choice Verdadero.\\
  \choice Falso.
\end{oneparchoices}
  
\vspace{0.10in}

\question Cuando comprimimos con DEFLATE, en general, sonidos más
débiles generan tasas de compresión inferiores:

\begin{oneparchoices}
  \choice Verdadero.\\
  \choice Falso, generan tasas de compresión superiores.
\end{oneparchoices}
  
\vspace{0.10in}

\question El tamaño del \emph{chunk} influye en la tasa de compresión
conseguida con DEFLATE:

\begin{oneparchoices}
  \choice Verdadero, porque cada \emph{chunk} se comprime de forma independiente.\\
  \choice Falso.
\end{oneparchoices}
  
\vspace{0.10in}

\question La tasa de bits de audio influye en la tasa de compresión
conseguida con DEFLATE:

\begin{oneparchoices}
  \choice Verdadero.\\
  \choice Falso.
\end{oneparchoices}
  
\vspace{0.10in}

\question El tasa de pérdida de \emph{chunks} influye en la tasa de
compresión conseguida con DEFLATE:

\begin{oneparchoices}
  \choice Verdadero.\\
  \choice Falso.
\end{oneparchoices}
  
\vspace{0.10in}

\question Si cada canal de audio se comprime de forma independiente
con DEFLATE, el número de canales de audio influye en la tasa de
compresión:

\begin{oneparchoices}
  \choice Verdadero.\\
  \choice Falso.
\end{oneparchoices}
  
\vspace{0.10in}

\question \emph{Quantization} es el proceso de discretizar en amplitud una señal:

\begin{oneparchoices}
  \choice Verdadero.\\
  \choice Falso.
\end{oneparchoices}
  
\vspace{0.10in}

\question Un \emph{scalar quantizer} digital es un sistema que acepta
una señal en formato PCM (Pulse Code Modulation) y devuelve otra señal
en formato PCM:

\begin{oneparchoices}
  \choice Verdadero.\\
  \choice Falso.
\end{oneparchoices}
  
\vspace{0.10in}

\question El \emph{quantization step} controla la cantidad de
información que se pierde durante el proceso de \emph{quantization}:

\begin{oneparchoices}
  \choice Verdadero.\\
  \choice Falso, el uso de un \emph{quantizer} siempre es un proceso completamente reversible.
\end{oneparchoices}
  
\vspace{0.10in}

\question Si el \emph{quantization step} es igual a 1, entonces:

\begin{oneparchoices}
  \choice Si las señales ya son digitales (PCM), no existe ningún cambio de representación.\\
  \choice La pérdida de información es mínima.
\end{oneparchoices}
  
\vspace{0.10in}

\question En la implementación actual de Intercom usamos \emph{scalar
quantization} para controlar la pérdida de información:

\begin{oneparchoices}
  \choice Verdadero.\\
  \choice Falso.
\end{oneparchoices}
  
\vspace{0.10in}

\question En la implementación actual de Intercom usamos \emph{vector
quantization} para controlar la pérdida de información:

\begin{oneparchoices}
  \choice Verdadero.\\
  \choice Falso.
\end{oneparchoices}
  
\vspace{0.10in}

\question En la implementación actual de Intercom usamos \emph{scalar
  quantization} para tener cierto control (aproximado) sobre la tasa
de bits transmitidos:

\begin{oneparchoices}
  \choice Verdadero.\\
  \choice Falso.
\end{oneparchoices}
  
\vspace{0.10in}

\question Uno de los efectos de usar un \emph{quantization step}
superior a 1 es que las tasas de compresión, en general, aumentan:

\begin{oneparchoices}
  \choice Verdadero.\\
  \choice Falso.
\end{oneparchoices}
  
\vspace{0.10in}

\question \emph{Quantization} sólo afecta a la forma de onda de la
señal reconstruída, pero no a sus componentes de frecuencia:

\begin{oneparchoices}
  \choice Verdadero.\\
  \choice Falso.
\end{oneparchoices}
  
\vspace{0.10in}

\question La diferencia básica entre un \emph{dead-zone quantizer} y
los otros tipos es que en este primero el \emph{quantization step} es
el doble de grande alrrededor del valor de entrada 0:

\begin{oneparchoices}
  \choice Verdadero.\\
  \choice Falso.
\end{oneparchoices}
  
\vspace{0.10in}

\question Existen \emph{scalar quantizers} donde el \emph{quantization
  step} es variable y depende de la amplitud de la señal de entrada:

\begin{oneparchoices}
  \choice Verdadero.\\
  \choice Falso.
\end{oneparchoices}
  
\vspace{0.10in}

\question Un \emph{quantizer} disminuye el número de muestras:

\begin{oneparchoices}
  \choice Verdadero.\\
  \choice Falso.
\end{oneparchoices}
  
\vspace{0.10in}

\question Un \emph{quantizer} disminuye el número de bits necesarios
para representar las muestras:

\begin{oneparchoices}
  \choice Verdadero.\\
  \choice Falso.
\end{oneparchoices}
  
\vspace{0.10in}

\question El \emph{quantization error} mide la diferencia entre la
señal a discretizar y la señal discretizada:

\begin{oneparchoices}
  \choice Verdadero.\\
  \choice Falso.
\end{oneparchoices}
  
\vspace{0.10in}

\question El \emph{quantization error} depende tanto del
\emph{quantization step} como de la amplitud de la señal discretizada:

\begin{oneparchoices}
  \choice Verdadero.\\
  \choice Falso.
\end{oneparchoices}
  
\vspace{0.10in}

\question Si el \emph{quantization step} es suficientemente pequeño
comparado con la amplitud de la señal de audio discretizada, entonces
el \emph{quantization error} puede considerarse impredecible porque
básicamente se trata de ruido:

\begin{oneparchoices}
  \choice Verdadero.\\
  \choice Falso.
\end{oneparchoices}
  
\vspace{0.10in}

\question El \emph{quantization step} puede usarse para controlar la
tasa de transmisión de \emph{chunks} en Intercom, aun manteniendo el
número de \emph{frames}/\emph{chunk}:

\begin{oneparchoices}
  \choice Verdadero.\\
  \choice Falso.
\end{oneparchoices}
  
\vspace{0.10in}

\question Generalmente, las señales estereo presentan mayor
correlación que la señales mono dentro de un canal:

\begin{oneparchoices}
  \choice Verdadero.\\
  \choice Falso.
\end{oneparchoices}
  
\vspace{0.10in}

\question Generalmente, los canales izquierdo y derecho de las señales
estereo presentan una alta correlación espacial:

\begin{oneparchoices}
  \choice Verdadero.\\
  \choice Falso.
\end{oneparchoices}
  
\vspace{0.10in}

\question Una transformada como la MST (Mid/Side Transform) explota la
redundancia espacial entre las muestras de un canal:

\begin{oneparchoices}
  \choice Verdadero.\\
  \choice Falso.
\end{oneparchoices}
  
\vspace{0.10in}

\question Al menos tal y como se usa en Intercom, la MST reduce la
correlación espacial entre los canales de una señal estereo:

\begin{oneparchoices}
  \choice Verdadero.\\
  \choice Falso.
\end{oneparchoices}
  
\vspace{0.10in}

\question Si existe correlación espacial entre los canales de una
señal estereo, la MST concentra la mayor parte de la energía en una
única subbanda:

\begin{oneparchoices}
  \choice Verdadero.\\
  \choice Falso.
\end{oneparchoices}
  
\vspace{0.10in}

\question La MST es una transformada irreversible:

\begin{oneparchoices}
  \choice Verdadero.\\
  \choice Falso, aunque esto puede depender de si usamos aritmética en punto fijo o flotante.
\end{oneparchoices}
  
\vspace{0.10in}

\question La longitud de los filtros usados en la MST es de 4
coeficientes, es decir, que son necesarias al menos 4 muestras para
poder aplicar la MST:

\begin{oneparchoices}
  \choice Verdadero.\\
  \choice Falso.
\end{oneparchoices}
  
\vspace{0.10in}

\question La MST debe aplicarse necesariamente entre muestras consecutivas en el tiempo:

\begin{oneparchoices}
  \choice Verdadero.\\
  \choice Falso.
\end{oneparchoices}
  
\vspace{0.10in}

\question La MST es una transformada ortogonal y por tanto, cualquiera
de sus filtros de análisis puede derivarse de los demás filtros (de
análisis) usando operaciones sencillas (sumas, restas,
multiplicaciones y divisiones):

\begin{oneparchoices}
  \choice Verdadero.\\
  \choice Falso.
\end{oneparchoices}
  
\vspace{0.10in}

\question La condición de ortogonalidad en una transformada es
interesante porque implica que la subbandas de salida son completamente
independientes (no comparten información que existe en la señal original):

\begin{oneparchoices}
  \choice Verdadero.\\
  \choice Falso.
\end{oneparchoices}
  
\vspace{0.10in}

\question En su totalidad, las subbandas generadas por la MST
contienen tantos coeficientes como muestras son transformadas:

\begin{oneparchoices}
  \choice Verdadero.\\
  \choice Falso.
\end{oneparchoices}
  
\vspace{0.10in}

\question El número de subbandas generadas por la MST en Intercom es:

\begin{oneparchoices}
  \choice Tantas como muestras.\\
  \choice Dos.
\end{oneparchoices}
  
\vspace{0.10in}

\question El ruido de \emph{quantization} generado en una subbanda no
influye en el ruido de \emph{quantization} generado en otra subbanda
diferente si las subbanda son ortogonales:

\begin{oneparchoices}
  \choice Verdadero.\\
  \choice Falso.
\end{oneparchoices}
  
\vspace{0.10in}

\question Si una transformada es ortogonal, entonces el
\emph{quantization error} generado en una de las subbandas no influye
en la información almacenada en las demás subbandas:

\begin{oneparchoices}
  \choice Verdadero.\\
  \choice Falso.
\end{oneparchoices}
  
\vspace{0.10in}

\question El uso de una transformada (como la MST) incrementa el
rendimiento del sistema de codificación desde el punto de vista
\emph{rate/distortion} porque es posible dedicar más bits a aquellas
subbandas que transportan más información de la señal transformada:

\begin{oneparchoices}
  \choice Verdadero.\\
  \choice Falso.
\end{oneparchoices}
  
\vspace{0.10in}

\question La codificación basada en transformadas puede reducir la
redundancia temporal presente en las dos muestras de un \emph{frame}
de audio:

\begin{oneparchoices}
  \choice Verdadero.\\
  \choice Falso.
\end{oneparchoices}
  
\vspace{0.10in}

\question La DWT (Dicrete Wavelet Transform) genera una descomposición
en frecuencia por octavas:

\begin{oneparchoices}
  \choice Verdadero.\\
  \choice Falso.
\end{oneparchoices}
  
\vspace{0.10in}

\question El número de coeficientes presente en una subbanda DWT
depende de los valores de la señal transformada:

\begin{oneparchoices}
  \choice Verdadero.\\
  \choice Falso.
\end{oneparchoices}
  
\vspace{0.10in}

\question En el caso de las señales de audio, la mayor parte de la
energía se concentra en las subbandas de alta frecuencia de la
descomposición DWT:

\begin{oneparchoices}
  \choice Verdadero.\\
  \choice Falso.
\end{oneparchoices}
  
\vspace{0.10in}

\question El número de bits necesarios para representar un coeficiente
DWT en general coincide con el número de bits usado para las muestras
transformadas:

\begin{oneparchoices}
  \choice Verdadero.\\
  \choice Falso.
\end{oneparchoices}
  
\vspace{0.10in}

\question En el caso de Intercom, dentro de una subbanda DWT no existe
correlación temporal:

\begin{oneparchoices}
  \choice Verdadero.\\
  \choice Falso.
\end{oneparchoices}
  
\vspace{0.10in}

\question El proceso de \emph{quantization} aplicado a la subbandas DWT
incrementa significativamente las tasas de compresión conseguidas por
DEFLATE porque:

\begin{oneparchoices}
  \choice Tienden a generarse largas cadenas de ceros especialmente en las subbandas de alta frecuencia.\\
  \choice Las londitudes de las subbandas (en coeficientes) es menor que los \emph{chunks} de audio.
\end{oneparchoices}
  
\vspace{0.10in}

\question El valor de los coeficientes DWT depende de qué filtros son
usados en la transformada:

\begin{oneparchoices}
  \choice Verdadero.\\
  \choice Falso.
\end{oneparchoices}
  
\vspace{0.10in}

\question El proceso de \emph{quantization} aplicado a la subbandas
DWT (sin solapamiento entre \emph{chunks}) puede producir
discontinuidades entre los \emph{chunks} transformados:

\begin{oneparchoices}
  \choice Verdadero.\\
  \choice Falso.
\end{oneparchoices}
  
\vspace{0.10in}

\question La discontinuidad entre \emph{quantized chunks} en el dominio
DWT puede eliminarse si los \emph{chunks} transformados se solapan entre sí:

\begin{oneparchoices}
  \choice Verdadero.\\
  \choice Falso, porque lo que hay que solapar son los canales de los \emph{chunks} en el tiempo.
\end{oneparchoices}
  
\vspace{0.10in}

\question El solapamiento de \emph{chunks} en el dominio del tiempo
también produce un solapamiento en el tiempo entre \emph{chunks}
transfomados:

\begin{oneparchoices}
  \choice Verdadero.\\
  \choice Falso, porque lo que hay que solapar son los canales de los \emph{chunks}.
\end{oneparchoices}
  
\vspace{0.10in}

\question El solapamiento de \emph{chunks} produce un incremento del
número de coeficientes generados por \emph{chunk} solapado (extendido)
respecto de los \emph{chunks} originales:

\begin{oneparchoices}
  \choice Verdadero.\\
  \choice Falso.
\end{oneparchoices}
  
\vspace{0.10in}

\question Los \emph{chunks} solapados y descompuestos en el dominio
DWT no comparten los coeficientes que resultan de las áreas solapadas
entre \emph{chunks}:

\begin{oneparchoices}
  \choice Verdadero.\\
  \choice Falso.
\end{oneparchoices}
  
\vspace{0.10in}

\question El solapamiento de \emph{chunks} durante el uso de la DWT en
Intercom permite disminuir ligeramente la distorsión para una
determinada tasa de bits:

\begin{oneparchoices}
  \choice Verdadero.\\
  \choice Falso.
\end{oneparchoices}
  
\vspace{0.10in}

\question Los \emph{chunks} solapados y descompuestos en el dominio
DWT no comparten los coeficientes que resultan de las áreas solapadas
entre \emph{chunks}:

\begin{oneparchoices}
  \choice Verdadero.\\
  \choice Falso.
\end{oneparchoices}
  
\vspace{0.10in}

\question Los seres humanos pueden percibir sonidos con frecuencia
igual a 0 Hercios:

\begin{oneparchoices}
  \choice Verdadero.\\
  \choice Falso.
\end{oneparchoices}
  
\vspace{0.10in}

\question Los seres humanos perciben los sonidos con una intensidad
que depende de su frecuecia:

\begin{oneparchoices}
  \choice Verdadero.\\
  \choice Falso.
\end{oneparchoices}
  
\vspace{0.10in}

\question El humbral de percepción auditiva varía con la frecuencia:

\begin{oneparchoices}
  \choice Verdadero.\\
  \choice Falso.
\end{oneparchoices}
  
\vspace{0.10in}

\question El humbral de percepción auditiva varía con las personas:

\begin{oneparchoices}
  \choice Verdadero.\\
  \choice Falso.
\end{oneparchoices}
  
\vspace{0.10in}

\question El humbral de percepción auditiva indica que el ruido de
\emph{quantization} puede ser más alto en el rango de frecuencias
centrado en 4 Kz, que en el resto de frecuencias audibles:

\begin{oneparchoices}
  \choice Verdadero.\\
  \choice Falso.
\end{oneparchoices}
  
\vspace{0.10in}

\question En el caso de Intercom, las diferentes subbandas de
frecuencia generadas por la DWT deben ser \emph{quantized} de forma
proporcional al humbral de audición promedio para la correspondiente
subbanda:

\begin{oneparchoices}
  \choice Verdadero.\\
  \choice Falso.
\end{oneparchoices}
  
\vspace{0.10in}

\question La \emph{quantization} perceptual en general mejora la ratio
\emph{rate}/\emph{distortion}:

\begin{oneparchoices}
  \choice Verdadero.\\
  \choice Falso.
\end{oneparchoices}
  
\vspace{0.10in}

\question La \emph{quantization} perceptual en general mejora la ratio
\emph{rate}/\emph{distortion}-perceptual (ratio tasa de bits
\emph{versus} la distorsión percibida):

\begin{oneparchoices}
  \choice Verdadero.\\
  \choice Falso.
\end{oneparchoices}

\vspace{0.10in}

\question En Intercom podemos aplicar la \emph{quantization}
perceptual simplemente \emph{quantizing} los coeficientes DWT antes de
ser entregados al resto del \emph{pipeline} de tareas que
descorrelacionan, comprimen y envían los chunks, y \emph{dequantizing}
los coeficientes DWT después de ser descomprimidos:

\begin{oneparchoices}
  \choice Verdadero.\\
  \choice Falso.
\end{oneparchoices}

\vspace{0.10in}

\vspace{0.25in} {\fbox{\parbox{6in}{\parbox{6in}{{Fin del test.}}}}}

\end{questions}

\end{document}
