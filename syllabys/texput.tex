% Emacs, this is -*-latex-*-

\newcommand{\TM}{\href{https://www.ual.es/estudios/grados/presentacion/plandeestudios/asignatura/4015/40154321?idioma=zh_CN}{Tecnologías Multimedia}}

\title{\TM{} - Syllabus}

\maketitle

\section{About}

Tecnologías Multimedia (TM) is an optative subject of the Computer
Science Degree at the UAL (University of Almería).

\section{Course Meeting Times}

See the current \href{https://www.ual.es/estudios/grados/presentacion/plandeestudios/asignatura/4015/40154321}{time-table}.

\section{Scope}

TM is focused on the development of real-time audio processing
techniques, specifically, encoding and transmission.

\section{Main goals}

To develop:
\begin{enumerate}
\item The ability to understand the environment of an organization and
  its needs in the field of information technologies and
  communications, in real time
  (\href{https://www.ual.es/application/files/8516/5061/5446/memoriavig-ing-informatica-4015.pdf}{TI1}).
\item The ability to conceive systems, applications and services based
  on network technologies, including Internet, web, e-commerce,
  multimedia, interactive services and mobile computing
  (\href{https://www.ual.es/application/files/8516/5061/5446/memoriavig-ing-informatica-4015.pdf}{TI6}).
\end{enumerate}

\section{Methodology and Contents}

TM follows the PBL (Project-Based Learning)
\href{http://portafirma.ual.es/pfirma/downloadReport/file?idDocument=4u61Ie5es2&idRequest=ZeBY35LlFa}{methodology}. The
students, helped by the lecturer, develop a project during the
classes. This project is
\href{https://github.com/Tecnologias-multimedia/intercom}{InterCom}(municator),
a Python application that allows networked users to communicate, in
real-time, throught the Internet.

The project is developed as a sequence of milestones. The students, in
groups of up to 4 people, propose solutions for each milestone. Such
solutions are presented to the rest of the class, and each group gives
and receives a feedback from the rest of the class and also from the
lecturer. The best solutions are incorporated to the project, with the
objective of improve it.

The contents of the subject can be found \href{https://tecnologias-multimedia.github.io/contents/}{here}.

\section{Attendance}

Currently,
\href{https://www.ual.es/estudios/grados/presentacion/plandeestudios/asignatura/4015/40154321}{the
  course is organized in 14 lectures of 2 hours/session, and 14
  practicals of 2 hours/session, during 7 weeks (approx.)}, and it
has been developed to be a virtual and blended training. This means
that all the content is available online, but students are expected to
regularly attend the sessions, which are mainly practical.

\section{Grading Policy}

Grades are determined as it has been depicted in the Teaching Guide
(\href{https://portafirma.ual.es/pfirma/downloadReport/file?idDocument=4Jp82utmug&idRequest=QY36GYcOZQ}{Español})/(\href{https://portafirma.ual.es/pfirma/downloadReport/file?idDocument=Zcmom6qigD&idRequest=xXgueuk9oD}{English}).
