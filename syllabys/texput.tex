% Emacs, this is -*-latex-*-

\newcommand{\TM}{\href{https://tecnologias-multimedia.github.io/}{Tecnologías Multimedia}}

\title{\TM{} - Syllabus}

\maketitle

\section{\href{https://www.ual.es/estudios/grados/presentacion/plandeestudios/asignatura/4015/40154321}{Course Meeting Times}}

\section{Scope}
Currently, Tecnologías Multimedia (TM) is focused on the development
of real-time audio processing techniques, specifically, encoding and
transmission.

\section{Main goals}
\begin{enumerate}
\item Understand and know the basic theory of Audio Processing and Coding, in general.
\item Know how to contribute to a software project, developing new functionality.
\end{enumerate}

\section{\href{http://portafirma.ual.es/pfirma/downloadReport/file?idDocument=4u61Ie5es2&idRequest=ZeBY35LlFa}{Methodology} and \href{https://tecnologias-multimedia.github.io/contents/}{Contents}}
The teaching methodology is based on project-based learning and such
techniques are applied to develop a
\href{https://github.com/Tecnologias-Multimedia/intercom}(project)
strongly related to TM.

\section{Attendance}
Currently,
\href{https://www.ual.es/estudios/grados/presentacion/plandeestudios/asignatura/4015/40154321}{the
  course is organized in 14 lectures of 2 hours/lecture, and 14
  practicals of 2 hours/practice, during 7 weeks (approx.)}, and it
has been developed to be a virtual and blended training. This means
that all the content is available online, but students are expected to
regularly attend the sessions, which are mainly practical.

\section{Grading Policy}
Grades are determined as it has been depicted in the Teaching Guide
(\href{https://portafirma.ual.es/pfirma/downloadReport/file?idDocument=4Jp82utmug&idRequest=QY36GYcOZQ}{Español})/(\href{https://portafirma.ual.es/pfirma/downloadReport/file?idDocument=Zcmom6qigD&idRequest=xXgueuk9oD}{English}).
