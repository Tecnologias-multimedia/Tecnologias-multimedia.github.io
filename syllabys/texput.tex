% Emacs, this is -*-latex-*-

\newcommand{\TM}{\href{https://tecnologias-multimedia.github.io/}{Tecnologías Multimedia}}

\title{\TM{} - Syllabus}

\maketitle

\section{\href{https://www.ual.es/estudios/grados/presentacion/plandeestudios/asignatura/4015/40154321}{Course Meeting Times}}

\section{Scope}
TecnologíasmMultimedia (TM) if focused on the develpment of real-time
audio processing techniques, especifically, encoding and transmission.

\section{Main goals}
\begin{enumerate}
\item Understand and known the basic theory of Audio Processing and Coding, in general.
\item Know how to contribute to a software project, developing new functionality.
\end{enumerate}

\section{\href{http://portafirma.ual.es/pfirma/downloadReport/file?idDocument=4u61Ie5es2&idRequest=ZeBY35LlFa}{Methodology} and \href{https://sistemas-multimedia.github.io/contents/}{Contents}}
The teaching methodology is based in the project based learning and
such techniques are applied to develop a
\href{https://github.com/Sistemas-Multimedia/VCF}(project) strongly
related to SM.

\section{Attendance}
Currently,
\href{https://www.ual.es/estudios/masteres/presentacion/plandeestudios/asignatura/7114/71142105}{the
  course is organized in 8 sessions of 2 hours/session, during 5
  weeks}, and it has been developed to be both, virtual and
blended-training. This means that all the contents are available
online, but students are expected to regularly attend lectures (at
least, remotely), that are mainly practical.

\section{Grading Policy}
Grades are determined as it has been depicted in the
\href{https://www.ual.es/estudios/masteres/presentacion/plandeestudios/asignatura/7114/71142105}{Teaching
  Guide}.

%%%%%%%
  
\section{About Sistemas Multimedia}

This course introduces and develops some of the most used techniques
used in multimedia systems. The teaching methodology is based in the
project based learning and such techniques are applied to develop a
collection of projects, most of them devoted to implement image and
video compressors.



