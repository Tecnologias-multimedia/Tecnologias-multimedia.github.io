\title{\href{https://www.ual.es/estudios/grados/presentacion/plandeestudios/asignatura/4015/40154321?idioma=zh_CN}{Tecnologías Multimedia} - Study Guide - Milestone 1: Git, GitHub and the Workflow}

\maketitle

\begin{figure}
\begin{verbatim}
+-------------------------------------------+   +-----------+
|              Local Host                   |   |Remote Host|
---------------------------------------------   -------------
+-----------+   +-----------+   +-----------+   +-----------+
|  Working  |   |  Staging  |   |   Local   |   |   Origin  |
|   Tree    |   |   Area    |   |   Repo    |   |    Repo   |
+-----------+   +-----------+   +-----------+   +-----------+
      |               |               |               |
      +----- add ---->|               |               |
      |               +----commit --->|               |
      |               |               +---- push ---->|
      |               |               |<--- pull -----+
      |<----------- checkout ---------+               |
      |<------------ merge -----------+               |
\end{verbatim}
\caption{Structure-of and actions-on a Git repo.}
\label{fig:local_repo_structure_and_actions}
\end{figure}

\section{Description}

To work in
\href{https://github.com/Tecnologias-multimedia/intercom}{the InterCom
  project} \cite{intercom} you will need to understand the basics of
Git \cite{Git-book} and
\href{https://github.com/}{GitHub}\footnote{There are other Git-based
hosting services such as \href{https://about.gitlab.com/}{GitLab} and
\href{https://www.atlassian.com/git}{Altassian}/\href{https://bitbucket.org/product}{BitBucket},
but GitHub is the most used one.} \cite{GitHub}, and how to use the
\href{https://guides.github.com/introduction/flow/index.html}{The
  GitHub (Work)Flow} and the
\href{https://github.com/vicente-gonzalez-ruiz/fork_and_branch_git_workflow}{Fork-and-Branch
  Git Workflow} \cite{fork-and-branch-git-workflow}.


You should know that:

\footnote{Technically, a commit copy the modifications
performed in your \emph{working copy} (of your local repo) to the
\emph{staging area} (of your local repo).}

\begin{enumerate}

\item If you want to contribute to a repo (that we will call \emph{the
upstream repo} or simply \emph{upstream}) \cite{Git-workflow} and you
  don't belong to the team of developers (such as it happens in he
  InterCom project), the first action that you must do is to
  \href{https://docs.github.com/en/github/getting-started-with-github/fork-a-repo}{fork}
  the \emph{upstream}. This will create a copy of the repo in your
  GitHub space, and the one thing that distinguish it from a repo
  directly created by youself is that there is a link to the
  \emph{upstream}.

\item You can do a lot of work in a repo hosted by GitHub using the
  Web interface, but it is much more convenient to
  \href{https://docs.github.com/en/github/creating-cloning-and-archiving-repositories/cloning-a-repository}{clone}
  the repo.\footnote{Git is a
  \href{https://nvie.com/posts/a-successful-git-branching-model/}{decentralized
    control system for source code}. Decentralization means that every
  developer has a copy of the \emph{origin repo} (or simply
  \emph{origin}, the first instance of the repo), and that thus, the
  developers can synchromize their local repos with the
  \emph{origin}(al one) or any of the rest of developer's repos, by
  simply adding the right \emph{Git remotes}.}

\item When you copy the \emph{origin} in your local disk, you create
  the data structure shown in the
  Figure~\ref{fig:local_repo_structure_and_actions}. Such structure
  has 3 main parts: (1) The \emph{working tree}, (2) the \emph{staging
  area}, and (3) the \emph{local repo}. The modifications are carried
  on the \emph{working tree}, which can contain files and directories
  that \emph{untracked} (probably temporal files that never should be
  included in the repo). The \emph{staging area} (also called
  \emph{index}) is form by the subset of tracked files of the
  \emph{working tree}. The \emph{local repo} is the data structure
  that holds all the history of the repo. When the \emph{local repo}
  and the \emph{origin} are synchonized, their contents are the same.

\item To update your \emph{local repo} with the modifications carried
  on the \emph{working tree}, you must perform a
  \href{https://docs.github.com/en/github/getting-started-with-github/github-glossary#commit}{commit}.
  
\item At any moment, you can decide to synchronize your local repo
  with the \emph{origin} or any other \emph{remote}. This action is
  called a
  \href{https://docs.github.com/en/github/using-git/pushing-commits-to-a-remote-repository}{push}.
\item Also, at any moment, you can decide to share your modifications
  (or even only your thoughts) with the \emph{upstream}
  (developers). For that, you can do a
  \href{https://docs.github.com/en/github/collaborating-with-issues-and-pull-requests/about-pull-requests}{pull-request}.
\end{enumerate}

You should also understand what is a
Git \href{https://docs.gitlab.com/ee/topics/gitlab_flow.html}{branch}:
\begin{enumerate}
\item \emph{Main branches}, that have infinite lifetime. All repos
  have a \emph{master} branch, which should be always deployable. Some
  have a \emph{testing} branch, in which the code is tested before it
  is included in the master (stable) branch.
\item \emph{Feature/debug branches}, which are aimed to develop a new
  functionality or to debug. Feature branches are also called topic
  branches. These type of branches have usually a short life.
\end{enumerate}

\section{What you have to do?}

\begin{enumerate}
  
\item Have a look to \href{https://git-scm.com/book/en/v2}{the Pro Git
  book} \cite{Git-book}, what as the main source of information for
  Git.

\item If you don't have an GitHub account, please, do the
  \href{https://guides.github.com/activities/hello-world/}{the Hello
    World guide at GitHub} and create one. Be aware that to contribute
  to InterCom an GitHub account is required. Notice that in the Hello
  World guide, the \texttt{master} branch is named the \texttt{main}
  branch. Please, create a \texttt{README.md} file for the Hello World
  repo, as the guide suggests.

\item Now, we are going to to the same that we have done using the
  GitHub web interface (except the Step 1: Create a Repo), but now
  using the terminal which will be the most used interface for dealing
  with Git. First, if Git is not installed in your host (try to run
  \texttt{git} in a terminal), install it with
  \begin{lstlisting}[language=bash]
    sudo apt install git
  \end{lstlisting}

\item
  \href{https://docs.github.com/en/github/getting-started-with-github/github-glossary#clone}{Clone}
  (download) the Hello World repo. You need to click on the ``Code''
  button (select ``the Git protocol'', not ``https'' nor ``download a
  zip file'').

  \begin{lstlisting}[language=bash}
    cd hello_world
  \end{lstlisting}

  Notice that a new directory named as the repo's name at
  GitHub has been created and that inside you can find the
  \texttt{README.md}
  \href{https://daringfireball.net/projects/markdown/}{Markdown} file.

\item \href{https://docs.github.com/en/github/getting-started-with-github/github-glossary#checkout}{Create (and switch to)} a
  \href{https://docs.github.com/en/github/getting-started-with-github/github-glossary#feature-branch}{feature
    branch} called \texttt{improving_readme}. In your terminal write:
  
  \begin{lstlisting}[language=bash}
    git checkout -b improving_readme
  \end{lstlisting}
  
\item Modify the file \texttt{README.md}. Append to it, for example, a
  link to the Hello World guide. Use an ASCII editor (\texttt{nano},
  for example):

  \begin{lstlisting}[language=bash]
    nano README
  \end{lstlisting}

  And write:
  
  \begin{lstlisting}[language=markdown}
    See the [Hello World](https://guides.github.com/activities/hello-world/) guide at GitHub.
  \end{lstlisting}
  
\item
  \href{https://docs.github.com/en/github/getting-started-with-github/github-glossary#commit}{Commit}
  your modification(s):

  \begin{lstlisting}[language=bash]
    git commit -am "Providing the Hello World link"
  \end{lstlisting}

  In your first commit you will be prompted with:

  \begin{lstlisting}[language=bash]
    git config --global user.email "you@example.com"
  \end{lstlisting}

  Plase, input such information.
  
  After the \texttt{commit}, your \emph{local} repo is \emph{ahead} of
  your
  \href{https://docs.github.com/en/github/getting-started-with-github/github-glossary#origin}{\emph{origin}}
  (copy at GitHub of the) remote repo. This means that your
  \emph{local} has modifications that the \emph{origin} doesn't have.

\item Synchronize your \emph{local} and the \emph{origin} using
  \href{https://docs.github.com/en/github/getting-started-with-github/github-glossary#push}{push}.

  Notice that if you have not upoaded a public
  \href{https://www.ssh.com/}{SSH}
  \href{https://www.ssh.com/ssh/identity-key}{key} (or the
  corresponding private key is not properly installed in your
  computer), the GitHub server requests your username and password,
  and this is something that is goint to happen with every push. To
  avoid this repetitive input of your GitHub login information, you
  need
  \href{https://docs.github.com/en/github/authenticating-to-github/connecting-to-github-with-ssh}{to
    identify you at GitHub} using
  \href{https://en.wikipedia.org/wiki/Public-key_cryptography}{public-key
    criptography}. You need to own a pair of keys, one public and
  other private, and upload the public one to GitHub.
  
\item The first step to solve this drawback is to check whether you
  already have a pair of keys (if your are using the just installed
  Xubuntu distribution, obviously you don't need to check anything and
  can go directly to the next step). Simply revise your
  \texttt{\$HOME/.ssh} directory with:

  \begin{lstlisting}[language=bash]
    ls -l ~/.ssh
  \end{lstlisting}

  and if you find a pair of files with almost the same name, and one
  of they finished with \texttt{.pub}, you have a pair of SSH keys.

\item Let's create a pair of keys (if you don't have one or if you
  prefeer to create a new one). Open a terminal and write:

  \begin{lstlisting}[language=bash]
    ssh-keygen -t rsa -b 4096 -C "your_email@example.com"
  \end{lstlisting}

  using the email address you provided when you created the GitHub
  account. Then you are prompted with:

  \begin{lstlisting}[language=bash]
    Enter a file in which to save the key (/home/you/.ssh/id_rsa):
  \end{lstlisting}

  just press the Enter-key, to select such output prefix. Otherwise,
  write a different one, but don't change the path to the
  \texttt{.ssh} directory.

\item Now SSH should request you for a passphrase. If you write one,
  you will be asked for it each time you push your commits to
  GitHub. There are two options to avoid this:

  \begin{enumerate}
  \item Input no passphrase (just by pressing the Enter-key
    again). This has the drawback that if somebody has your keys, he
    could access to GitHub as he were you.
  \item Input a passphrase and configure \texttt{ssh-agent} to send it
    to GitHub by you. This option is the preferable one becase you
    will be asked for the passphrase only when the \texttt{ssh-agent}
    is started.
  \end{enumerate}

\item Now it's time to check whether the \texttt{ssh-agent} is already
  running in your computer. This can be checked with:

  \begin{lstlisting}[language=bash]
    ps aux | grep ssh-agent
  \end{lstlisting}

  and in the case of Xubuntu, you should get something similar to:

  \begin{lstlisting}[language=bash]
    989 ?        Ss     0:00 /usr/bin/ssh-agent /usr/bin/im-launch startxfce4
   1433 pts/0    S+     0:00 grep --color=auto ssh-agent
  \end{lstlisting}

  This means that there are two processes in whose description there
  exists the string \texttt{ssh-agent}. The first entry is the agent
  process. The second one is the \texttt{grep} running at the same
  time that the \texttt{ps}.

\item If the \href{https://www.ssh.com/ssh/agent}{\texttt{ssh-agent}}
  were not running, it can be launched to run in the background with:

  \begin{lstlisting}[language=bash]
    eval "$(ssh-agent -s)"
  \end{lstlisting}

  but you don't need to do that in your Xubuntu installation, because
  the \texttt{ssh-agent} the \href{https://www.xfce.org/}{XFCE}
  lauches it.

\item With your keys, run:

  \begin{lstlisting}[language=bash]  
    ssh-add ~/.ssh/id_rsa
  \end{lstlisting}

  and the passphrase will be prompt.
  
\item Go now to GitHub \texttt{->} Settings \texttt{->} SSH and GPG keys \texttt{->} New SSH key.

  Open a terminal and write:
  
  \begin{lstlisting}[language=bash]  
    cat .ssh/id_rsa.pub
  \end{lstlisting}

  and copy and paste the content of such file (which ends with your
  email address) inside of the space where you can read ``Begins with
  'ssh-rsa', ...''. Don't forget to give a title (something such as
  ``tec-multimedia'') to the key pair.
  
\item When you use the key for the first time (clonning a repo or
  pushing a commit), the SSH client will warn you that the autenticity
  of \texttt{github.com} cannot be established. This is normal and
  should happen only once. Type \texttt{yes}. If this problem
  persists, then you could be suffering a
  \href{https://en.wikipedia.org/wiki/Man-in-the-middle_attack}{man-in-the-middle
    attack}.

\item Revise
  \href{https://github.com/vicente-gonzalez-ruiz/fork_and_branch_git_workflow}{The
    Fork and Branch Git Workflow} \cite{Git-workflow}. Basically, this
  ``protocol'' explains that to contribute to an open-source repo
  hosted by GitHub without belonging to the develop team.

\item Make a fork of the
  \href{https://github.com/Tecnologias-multimedia/intercom}{InterCom}
  project. We will call to this repo the
  \href{https://docs.github.com/en/github/getting-started-with-github/github-glossary#upstream}{\emph{upstream}}, whose URL is
\begin{verbatim}
  git@github.com:Tecnologias-multimedia/intercom.git
\end{verbatim}        
  This info can be found when you clone the InterCom. Notice however,
  that clonning the InterCom is a waste of time because you cannot
  contribute directly to it.

\item Add the remote \emph{upstream} with:

  \begin{lstlisting}[language=bash]  
    git remote add upstream git@github.com:Tecnologias-multimedia/intercom.git
  \end{lstlisting}

  Check that everything has worked with:

  \begin{lstlisting}[language=bash]  
    git remote -v
  \end{lstlisting}

  where you should see two remotes: \emph{origin} and \emph{upstream}.

\end{enumerate}

\section{Timming}

You should reach this milestone at most in one week.

\section{Deliverables}

None.

\section{Resources}

\bibliography{git, intercom}
