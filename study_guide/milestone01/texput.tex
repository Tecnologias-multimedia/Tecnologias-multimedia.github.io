\title{\href{https://www.ual.es/estudios/grados/presentacion/plandeestudios/asignatura/4015/40154321?idioma=zh_CN}{Tecnologías Multimedia} - Study Guide - Milestone 1: Git, GitHub and the Fork-and-Branch Git Workflow}

\maketitle

\section{Description}

To work in the InterCom project \cite{intercom} you will need to
understand the basics of Git and GitHub, and how to make a copy of a
repo(sitory) (by forking it) to know how to ask for changes in the
parent repo (by performing pull-requests).

\section{What you have to do?}

\begin{enumerate}
\item Have a look the Git \cite{Git, Git-book} and the GitHub
  \cite{GitHub} websites. If you don't have a GitHub account, please,
  create one. Otherwise, you will not be able to contribute to the
  Intercom project.
\item Install Git with:
  \begin{lstlisting}[language=bash]
    sudo apt install git
  \end{lstlisting}
\item Create a test project at GitHub using your account. See
  \cite{GitHub-HW}.
\item Revise
  \href{https://github.com/vicente-gonzalez-ruiz/fork_and_branch_git_workflow}{The
    Fork and Branch Git Workflow} \cite{Git-workflow}. Basically, this
  ``protocol'' explains that to contribute to a repo hosted by GitHub
  you need to do:
  \begin{enumerate}
    \item A fork, which is basically a copy of the original
      (``parent'') repo. Such repo is yours and the one thing that
      distinguish it from a repo created by youself is that there is a
      link to the parent repo.
    \item Use branches (that can be understand like parallel states or
      views of your repo) to keep always stable the \emph{master}
      branch of your repo. Such branches can be:
      \begin{enumerate}
        \item \emph{Feature branches}, that use to be quite short in
          time and they are aimed to develop new functionality or to
          debug.
        \item \emph{Developing branches}, which are used exactly for
          the same objetives that the feature branches, but usually
          live forever (like the master branch). If you are
          continuosly working on a repo, probably you will use a
          develping branches.
      \end{enumerate}
  \end{enumerate}
\item Make a fork of the \href{https://github.com/Tecnologias-multimedia/intercom}{InterCom} project.
\item Download (clone) your copy of Intercom (select the Git protocol,
  not https nor download a zip file).
\item Do some modification to your local repo. For example, create and
  add a file named \texttt{deleteme} or something similar (you can
  also propose some more useful modification).
\item Commit the changes. In your first commit you will be prompted with:

  \begin{lstlisting}[language=bash]
    git config --global user.email "you@example.com"
  \end{lstlisting}

\item Push the changes. If you have not upoaded a public
  SSH key (and the private key is not properly installed in your
  computer), the GitHub server should have requested your username and
  password, and this is something that will happen with every push.
\item To avoid this repetitive input of your login information at
  GitHub, you need
  \href{https://docs.github.com/en/github/authenticating-to-github/connecting-to-github-with-ssh}{to
    identify you at GitHub} using
  \href{https://en.wikipedia.org/wiki/Public-key_cryptography}{public-key
    criptography}. You need to own (as said before) a pair of keys,
  one public and other private, and upload the public one to GitHub.
  
\item The first step in this process is to check whether you already
  have a pair of keys (if your are using the just installed Xubuntu
  distribution, obviously you don't and go directly to the next
  step). Simply revise your \texttt{\$HOME/.ssh} directory with:

  \begin{lstlisting}[language=bash]
    ls -l ~/.ssh
  \end{lstlisting}

  and if you find a pair of file with almost the same name, but one of
  them finising in \texttt{.pub}, you have a pair of SSH keys.

\item Let's create a pair of keys. Open a terminal and write:

  \begin{lstlisting}[language=bash]
    ssh-keygen -t rsa -b 4096 -C "your_email@example.com"
  \end{lstlisting}

  using your email address. Then you are prompted with:

  \begin{lstlisting}[language=bash]
    Enter a file in which to save the key (/home/you/.ssh/id_rsa):
  \end{lstlisting}

  just press the Enter-key, to select such output prefix. Otherwise,
  write a different one, but don't change the path to the
  \texttt{.ssh} directory.

\item Now SSH should request you for a passphrase. If you write one,
  you will be asked for it each time you push your commits to
  GitHub. There are two options to avoid this:

  \begin{enumerate}
  \item Input no passphrase (just by pressing the Enter-key
    again). This has the drawback that if somebody has your keys, he
    can access to GitHub as he were you.
  \item Input a passphrase and configure \texttt{ssh-agent} to send it
    to GitHub by you.
  \end{enumerate}

\item It is probable that the \texttt{ssh-agent} is already
  running in your computer. This can be checked with:

  \begin{lstlisting}[language=bash]
    ps aux | grep ssh-agent
  \end{lstlisting}

  and in the case of Xubuntu, you should get something similar to:

  \begin{lstlisting}[language=bash]
    989 ?        Ss     0:00 /usr/bin/ssh-agent /usr/bin/im-launch startxfce4
   1433 pts/0    S+     0:00 grep --color=auto ssh-agent
  \end{lstlisting}

  This means that there are two processes in whose description there
  exists the string \texttt{ssh-agent}. The first entry is the
  agent. The second one is only out \texttt{grep} command.

\item If the \texttt{ssh-agent} were not running, it can be launched to
  run in the background with:

  \begin{lstlisting}[language=bash]
    eval "$(ssh-agent -s)"
  \end{lstlisting}

  but you don't need to do that in your Xubuntu installation.

\item ssh-add ~/.ssh/id_rsa the passphrase will be prompt.
  
\item Settings -> SSH and GPG keys -> New SSH key
  Open a terminal and write:
  cat .ssh/id_rsa.pub
  give a title (something such as ``tec-multimedia'')
  copy and paste the content of such file (which ends with your email address) inside of the space where you can read ``Begins with 'ssh-rsa', ...''.

  \item When you use the key for the first time (clonning a repo or
    pushing a commit), the SSH client will warn you that the
    autenticity of \texttt{github.com} cannot be established. This is
    normal and should happen only once. Type \texttt{yes}.

The ssh-agent will ask for the passphrase the first time you use the keys after a rebooting.
  
\end{enumerate}

\section{Timming}

You should reach this milestone at most in one week.

\section{Deliverables}

None.

\section{Resources}

\bibliography{git, intercom}
