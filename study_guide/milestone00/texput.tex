\title{\href{https://www.ual.es/estudios/grados/presentacion/plandeestudios/asignatura/4015/40154321?idioma=zh_CN}{Tecnologías Multimedia} - Study Guide - Milestone 0: Provisioning}

\maketitle

\section{Description}

InterCom \cite{intercom} is a collection of
\href{https://docs.python.org/3/tutorial/modules.html#}{Python
  modules} \cite{python}. Therefore, you will need Python
\href{https://www.python.org/downloads/}{installed} in your computer
in order to run it.

Most of the current Unix-based operating systems (Linux, FreeBSD and
OSX) use Python for running some of their ``daily tasks'', which means
that a Python interpreter is already available. However, usually it is
better to use our own interpreter because:
\begin{enumerate}
\item We can chose the version of Python and the
  \href{https://docs.python.org/3/tutorial/modules.html#packages}{packages}.
\item We can optimize the compilation of the interpreter depending on
  our needs (for example, including
  \href{https://wiki.python.org/moin/TkInter}{Tk support} or not).
\item By default, all the Python packages will be installed in a
  different repository of the system packages, which eases the
  system/user Python-isolation and the removal of the interpreter.
\end{enumerate}
In Windows you need to install Python, yes or yes.

Alternatively, and this is the option that I (as the teacher)
recommend is to use a dedicated Xubuntu 20.04 (Focal Fossa)
\cite{xubuntu} installation (in an external
\href{https://en.wikipedia.org/wiki/USB_flash_drive}{USB drive} with
at least 8GB of capacity) of for this subject. Notice that you will
need also another USB drive with at least 4GB to hold the disk
installation image.

The advantage of using Xubuntu is that if you need to ask (to
me) some question about Python or the OS, I'll be able to answer
fast. In any case, keep in mind that this is up to you (use the OS
that you prefeer).

\section{What do you have to do?}

Supposing that you have decided to use Xubuntu 20.04, these are the steps
you should perform:

\begin{enumerate}
  \item Download the installation image from
    \href{https://xubuntu.org/download/}{here}.
    
  \item ``Burn'' the (at least) 4GB USB drive with the
    image. Depending on your current OS, use the following
    instructions for
    \href{https://ubuntu.com/tutorials/create-a-usb-stick-on-windows#1-overview}{Windows},
    \href{https://ubuntu.com/tutorials/create-a-usb-stick-on-macos#1-overview}{OSX},
    or
    \href{https://askubuntu.com/questions/372607/how-to-create-a-bootable-ubuntu-usb-flash-drive-from-terminal}{Linux and OSX}.

\item Boot the installer from the USB port. This step depends on your
  computer type. Most of PCs can chose the boat device by pressing the
  F12 key when the PC is booting. On a Mac, you need to keep pressed
  the alt key when it is booting.
  
\item Select the option \texttt{Try Xubuntu without installing}.
\item When the OS is running, configure the network.
\item Insert also the 8GB USB drive where Xubuntu will be installed.
\item Select \texttt{Install Xubuntu 20.04 LTS}.
\item Select English as the language used during the install and the
  installed system.
\item Select your keyboard layout (probably \texttt{Spanish}).
\item Chose \texttt{Download updates while installing Xubuntu} and
  \texttt{Install third-party software for graphics and Wi-Fi hardware
    and additional media formats}.
\item Chose \texttt{Erase disk and install Xubuntu}. Wait for a couple
  of minutes :-/
\item Chose \texttt{Erase disk and install Xubuntu}.
\item Select the drive corresponding to the 8GB USB drive. It's
  difficult to guess the letter associated to this drive, but, it is
  very likely not \texttt{sda}.
\item At this point of the installation you should consider (depending
  on the amount of RAM memory in your computer and the size of the USB
  drive) to create an especific partition for doing swapping. The rule
  of the thumb is to create a partition with the same size that the
  RAM. However, probably you cannot do that in a 8GB USB drive because
  at least 5GB are needed for a Xubuntu installation. Anyway, keep in
  mind that this step is optional because you can always perform
  swapping on a file (a process slightly slower than using the
  dedicated partition).
\item Chose your time zone.
\item Configure you personal account, hostname and logging process.
\item Wait for the end of the installation and boot your new Xubuntu.
\item Go to
  \href{https://github.com/vicente-gonzalez-ruiz/YAPT/blob/master/01-hello_world/02-installation.ipynb}{YAPT/01-hello\_world/02-installation.ipynb}
  \cite{YAPY} and follow the instructions to install CPython 3.8.5.
\item Following also these instructions, create a new virtual
  environment called \texttt{tm}.
\item Remember that you will need to active it when you want to work
  in this project. It is a good idea to append:
  \lstlisting{language=Bash}{
    pyenv activate tm
  }
  to the \texttt{~/.bashrc} file.
\item Install an IDE for programming with Python. I recommend Thonny.
\end{enumerate}

\section{Timming}

You should reach this milestone at most in one week.

\section{Deliverables}

None.

\section{Resources}

\bibliography{intercom, python, linux}
