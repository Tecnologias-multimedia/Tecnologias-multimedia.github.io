\title{\href{https://www.ual.es/estudios/grados/presentacion/plandeestudios/asignatura/4015/40154321?idioma=zh_CN}{Tecnologías Multimedia} - Study Guide - Milestone 0: Provisioning}

\maketitle

\section{Description}

InterCom \cite{intercom} is a collection of
\href{https://docs.python.org/3/tutorial/modules.html#}{Python
  modules} \cite{python}. Therefore, you will need Python
\href{https://www.python.org/downloads/}{installed} in your computer
in order to run it.

Most of the current Unix-based operating systems (Linux, FreeBSD and
OSX) use Python for running some of their ``daily tasks'', which means
that a Python interpreter is already available. However, usually it is
better to use our own interpreter because:
\begin{enumerate}
\item We can chose the version of Python and the
  \href{https://docs.python.org/3/tutorial/modules.html#packages}{packages}.
\item We can optimize the compilation of the interpreter depending on
  our needs (for example, including
  \href{https://wiki.python.org/moin/TkInter}{Tk support} or not).
\item By default, all the Python packages will be installed in a
  different repository of the system packages, which eases the
  system/user Python-isolation and the removal of the interpreter.
\end{enumerate}
In Windows you need to install Python, yes or yes.

Alternatively, and this is the option that I (the teacher) recommend
if you need to ask (to me) some question about the installation of
Python, is to use a dedicated installation (in an external
\href{https://en.wikipedia.org/wiki/USB_flash_drive}{USB drive}) of
Xubuntu (Linux) \cite{xubuntu} for this subject. In any case, keep in
mind that this is up to you.

\section{What do you have to do?}

\begin{enumerate}
\item Have a look the Git \cite{Git, Git-book} and the GitHub
  \cite{GitHub} websites. If you don't have a GitHub account, please,
  create one.
\item Create a test project at GitHub using your account. See
  \cite{GitHub-HW}.
\item See The Fork and Branch Git Workflow \cite{Git-workflow}.
\item Make a fork of the InterCom \cite{intercom} project. 
\end{enumerate}

\section{Timming}

You should reach this milestone at most in one week.

\section{Deliverables}

None.

\section{Resources}

\bibliography{git, intercom}
