\title{Framework}

\maketitle

\section{Description}

The InterCom project \cite{intercom} is a collection of
\href{https://www.python.org/}{Python}
\href{https://docs.python.org/3/tutorial/modules.html#modules}{modules}~\cite{python}.
Although Python has been ported to
\href{https://www.python.org/download/other/}{almost all} the current
OSs, including mobile \href{https://kivy.org/#home}{devices}, we will
develop the project on Linux, and more concretely, on Xubuntu 21.10
(Impish Indri) \cite{xubuntu}, running natively (no
\href{https://en.wikipedia.org/wiki/Virtualization}{virtualization}),
if possible. To receive technical support in a reasonable amount of
time in the case you are in trouble, we will use an especific Linux
distribution: Xubuntu, which is fully functional (at least for
developing our project) and demand a low amount of hardware
resources. Please, follow for this,
\href{https://vicente-gonzalez-ruiz.github.io/Xubuntu_install/}{to
  install Xubuntu}.

You don't need to master Python to follow this course, but it is
convenient for you to rely on some Python programming tutorial, such
as \href{https://docs.python.org/3/tutorial/}{The Python Tutorial}
\cite{python-tutorial} if you realize that the language is a setback
for you. If you need to start with Python from scratch, an
introduction to Python such as this
\href{https://github.com/vicente-gonzalez-ruiz/YAPT/tree/master/workshops/programacion_python_ESO}{workshop
  of YAPT} \cite{YAPT} could also be helpful. See also
\href{http://zetcode.com/lang/python/}{ZetCode's Python
  Tutorial}. Follow this to
\href{https://vicente-gonzalez-ruiz.github.io/Python_install/}{install
  Python}. Alternatively (but reducing the chances of solving any
possible issue), you can use the Python interpreter shipped with your
OS. In this case, it is strongly recommended to use an specific Python
\href{https://docs.python.org/3/library/venv.html}{environment} for
the InterCom project.

Finally, to work in the
\href{https://github.com/Tecnologias-multimedia/intercom}{InterCom}
project \cite{intercom} you must understand the basics of Git
\cite{Git-book} and \href{https://github.com/}{GitHub}\footnote{There
are other Git-based hosting services such as
\href{https://about.gitlab.com/}{GitLab} and
\href{https://www.atlassian.com/git}{Altassian}/\href{https://bitbucket.org/product}{BitBucket},
but GitHub is the most used one.} \cite{GitHub}, and how to use the
\href{https://guides.github.com/introduction/flow/index.html}{The
  GitHub (Work-)Flow} and the
\href{https://github.com/vicente-gonzalez-ruiz/fork_and_branch_git_workflow}{Fork-and-Branch
  Git Workflow} \cite{fork-and-branch-git-workflow}. Be aware that to
contribute to InterCom, an GitHub account is required. Please, follow
this
\href{https://vicente-gonzalez-ruiz.github.io/using_GitHub/}{minimal
  Git guide}. In the guide, you must consider that the
\texttt{<organization\_name>} is \texttt{Tecnologias-multimedia} and
that the \texttt{<repo\_name>} is \texttt{intercom}.

\section{Resources}

\bibliography{python,sound,intercom,git,linux}
