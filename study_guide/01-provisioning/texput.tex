\newcommand{\TM}{\href{https://www.ual.es/estudios/grados/presentacion/plandeestudios/asignatura/4015/40154321?idioma=zh_CN}{Tecnologías Multimedia}}

\title{\TM{} - Study Guide - Milestone 0: OS (Operating System) Provisioning}

\maketitle

\section{Description}

The InterCom project \cite{intercom} is a real-time application with
a high computational (specially in terms of CPU) demand. It is written
in Python \cite{Python}, an interpreted languaje that has been
ported \href{https://www.python.org/download/other/}{almost to all}
the current OSs, including
mobile \href{https://kivy.org/#home}{devices}.

This milestone (the installation of a dedicated Linux distribution for
running InterCom) is optional , but I (as the teacher) highly
recommend to do it because I'll give technical support in a reasonable
time when you are in trouble. For this reason, I recommend you to run
InterCom in a Xubuntu 20.04 (Focal Fossa) \cite{xubuntu} machine,
running natively
(no \href{https://en.wikipedia.org/wiki/Virtualization}{virtualization}).

Therefore, the following ``guide'' helps you to install Xubuntu in an
external \href{https://en.wikipedia.org/wiki/USB_flash_drive}{USB
drive}, which must have at least 8GB of capacity (the minimal
installation of Xubuntu 20.04 needs about 5GB). You will need also a
temporal external USB disk with at leas 4GB to boot from it the
installation Xubuntu image (or to burn
an \href{https://en.wikipedia.org/wiki/Optical_disc}{optical disk}).

\section{What do you have to do?}

Supposing that you have decided to use Xubuntu 20.04 in a USB disk,
these are the steps you should perform (to install Xubuntu in a disk
partition of your computer the instructions are almost the same):

\begin{enumerate}
  \item Download the
    installation \href{https://en.wikipedia.org/wiki/Disk_image}{image}
    from \href{https://xubuntu.org/download/}{here}.
    
  \item ``Burn'' the 4GB USB drive with the image. Depending on your
    current OS, use the following instructions
    for \href{https://ubuntu.com/tutorials/create-a-usb-stick-on-windows#1-overview}{Windows}, \href{https://ubuntu.com/tutorials/create-a-usb-stick-on-macos#1-overview}{OSX}, \href{https://ubuntu.com/tutorials/create-a-usb-stick-on-ubuntu#1-overview}{Ubuntu
    (and derivatives)},
    or \href{https://askubuntu.com/questions/372607/how-to-create-a-bootable-ubuntu-usb-flash-drive-from-terminal}{the
    console}.

\item Boot the image from the USB port. This step depends on your
  computer. Most of PCs can choose the boot device by pressing the
  F12-key when the PC is booting. On a Mac, you need to keep pressed
  the alt-key when it is booting.
  
\item Select the option \texttt{Try Xubuntu without installing}.
  
\item When the OS is running, configure the network.
  
\item Insert also the 8GB USB drive where Xubuntu will be installed.
  
\item Select \texttt{Install Xubuntu 20.04 LTS}.
  
\item Select English as the language used during the installation and the
  installed system. This will help in the case you need to search
  information in the Internet, providing the error descriptions in
  English.
  
\item Select your keyboard layout (probably \texttt{Spanish}).

\item Open a terminal and write:

  \begin{lstlisting}[language=bash]
    df -h
  \end{lstlisting}

  to see all the mounted disk partitions and their capacity. Notice
  that no partition of \texttt{/dev/sda} (the hard disk) should not be
  mounted (although you can do that, you don't need to mount any
  partition of the hard disk), the partition \texttt{/dev/sdb1} (with
  the Xubuntu image) should be mounted, and finally, if the first
  partition of the second external USB drive has been recognized by
  \href{https://gitlab.xfce.org/xfce/thunar}{Thunar} (the default file
  manager in Xubuntu), it should appear as \texttt{/dev/sdc1}. This
  partition should be unmounted to install on it Xubuntu. Anyway, if
  you continue the installation process without unmounting it, the
  installer will ask you to do it. In this description, it has been
  supposed that your computer only has one hard disk.

\item Choose \texttt{Download updates while installing Xubuntu} and
  \texttt{Install third-party software for graphics and Wi-Fi hardware
    and additional media formats}, in order to have access to the
    ultimate software available for Ubuntu (and derivatives).

\item Choose \texttt{Erase disk and install Xubuntu}. Ignore the
  Advanced features. Wait for a couple of minutes :-/

\item Select the drive corresponding to the 8GB USB drive
  (\texttt{/dev/sdc}). Don't choose \texttt{/dev/sda} (the main disk of
  your computer)! Select \texttt{/dev/sdc}!!

\item At this point of the installation you should consider (depending
  on the amount of RAM memory installed in your computer and the size
  of the USB drive) to create an specific partition for doing
  swapping. The rule of the thumb is to create a partition with the
  same size that the RAM. However, probably you cannot do that in a
  8GB USB drive because at least 5GB are needed for a Xubuntu
  installation. Anyway, keep in mind that this step is optional
  because you can always perform swapping on a file (a process
  slightly slower than using the dedicated partition). Consider also
  that InterCom requires only some MB of memory for running and
  therefore, probably you are not going to need to swap any
  \href{https://en.wikipedia.org/wiki/Page_(computer_memory)}{memory
    page} at all. If you decide to create a specific swap partition,
  click on ``advanced partitioning tool'' and do the modifications you
  want, and also check that the boot loader
  (\href{https://www.gnu.org/software/grub/}{GRUB}) will be installed
  in \texttt{/dev/sdc1}. But remember, all the actions described in
  this point are optional (except selecting \texttt{/dev/sdc1} for
  GRUB)..

\item Click on Install Now. Something similar to:

\begin{verbatim}
If you continue, the changes listed below will be written to the disks. Otherwise, you will be able to make further changes manually.

WARNING: This will destroy all data on any partitions you have removed as well as on the partitions that are going to be formatted.

The partition tables of the following devices are changed:
 SCSI8 (0,0,0) (sdc)

The following partitions are going to be formatted:
 partition #1 of SCSI8 (0,0,0) (sdc) as ESP
 partition #2 of SCSI8 (0,0,0) (sdc) as ext4
\end{verbatim}

\item Choose your time zone.
  
\item Configure you personal account, hostname and logging process.
  
\item Wait for the end of the installation and boot your new
  Xubuntu. Don't worry if grub labels Xubuntu as Ubuntu. This is
  normal.
  
\end{enumerate}

\section{Timming}

You should reach this milestone at most in one week.

\section{Deliverables}

None.

\section{Resources}

\bibliography{intercom,python,linux}
