\title{Multimedia Technology - Study Guide - Milestone 1: ``wiring'' the ADC with the DAC}
\author{Vicente González Ruiz - Depto Informática - UAL}

\section{Description}
Let's start the development of InterCom with a simple program that
inputs audio (through
the \href{https://en.wikipedia.org/wiki/Analog-to-digital_converter}{ADC}
of our computer) and inmediatelly outputs it (through
the \href{https://en.wikipedia.org/wiki/Digital-to-analog_converter}{DAC}). You
can download this program (a \href{https://www.python.org}{Python}
module)
from \href{https://github.com/Tecnologias-multimedia/intercom/blob/master/test/sounddevice/wire3.py}{here}. If
you are planning to run this module right now, decrease first the
output volume of your computer to avoid an audio coupling (a feedback)
between the mic(rophone) and the speakers.

As you can see in the code, we use
the \href{https://python-sounddevice.readthedocs.io/en/latest/}{python-sounddevice}
module to access to the sound hardware, using

\begin{lstlisting}[language=Python]
import sounddevice as sd
\end{lstlisting}

Next, we create a I/O raw audio stream
using \href{https://python-sounddevice.readthedocs.io/en/latest/api/raw-streams.html#sounddevice.RawStream}{sounddevice.RawStream()}
\begin{lstlisting}[language=Python]
stream = sd.RawStream(samplerate=44100, channels=2, dtype='int16')
\end{lstlisting}
As you can see in the documentation of `python-sounddevice`, this
constructor method has several parameters. In our example we have used
only the most relevants for our InterCom:
\begin{enumerate}
\item The samling rate.
\item The number of channels.
\item The number of bits of each sample.
\end{enumerate}
Please, review the modules of documentation 

Next, we define a chunk size
\begin{lstlisting}[language=Python]
CHUNK_SIZE = 1024
\end{lstlisting}
for the block of data we are going to capture in the loop that will
be defined just below.

The loop is an infinite loop

\section{What you have to do?}

\section{Resources}

