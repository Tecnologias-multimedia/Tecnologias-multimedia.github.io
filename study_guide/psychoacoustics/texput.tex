\title{Removing psychoacoustical redundancy}

\maketitle

\section{Description}

\subsection{Threshold of Human Hearing (THH)}
%{{{

Psychoacoustics (see
\href{https://vicente-gonzalez-ruiz.github.io/the_sound/}{the sound},
\href{https://vicente-gonzalez-ruiz.github.io/human_auditory_system/}{the
  human auditory system}, and
\href{https://vicente-gonzalez-ruiz.github.io/human_sound_perception/}{the
  human sound perception}) has determined that the HAS (Human Auditory
System) has a sensitivity that depends on the frequency of the sound,
the so called THH
(\href{https://en.wikipedia.org/wiki/Absolute_threshold_of_hearing}{Threshold
  of Human Hearing}). This basically means that some subbands can be
quantized with a larger quantization step than others without a
noticeable increase (from a perfection perspective) of the
quantization noise.

\begin{figure}
  \centering
  \svg{graphics/ToHH}{600}
  \caption{Absolute threshold of human hearing.}
  \label{fig:ToHH}
\end{figure}

A good approximation of the THH (Threshold of Human Hearing) for a 20-years old person can be
obtained with~\cite{bosi2003intro}
\begin{equation}
  T(f)\text{[dB]} = 3.64(f\text{[kHz]})^{-0.8} - 6.5e^{f\text{[kHz]}-3.3)^2} + 10^{-3}(f\text{[kHz]})^4.
  \label{eq:ToHH}
\end{equation}
This equation has been plotted in the Fig.~\ref{fig:ToHH}.

%}}}
\subsection{DWT subbands and quantization steps}
%{{{

The number of DWT subbands
\begin{equation}
  N_{\text{sb}} = N_{\text{levels}} + 1
\end{equation}
where $N_{\text{levels}}$ is the number of levels of the DWT. Except
for the ${\mathbf l}^{N_{\text{levels}}}$ subband (the lowest-pass
frequency of the decomposition), it holds that
\begin{equation}
  W({\mathbf h}^s) = \frac{1}{2}W({\mathbf h}^{s-1}),
\end{equation}
being $W(\cdot)$ the bandwidth of the corresponding
subband $s$. Therefore, considering that the bandwidth of the audio signal
is $22050$ Hz, the bandwidth $W({\mathbf h}^1)$ of the ${\mathbf h}^1$ subband is $11025$ Hz,
$W({\mathbf h} ^2)=22025/4$, and so on. It also holds that
\begin{equation}
  W({\mathbf l}^{N_{\text{levels}}}) = W({\mathbf h}^{N_{\text{levels}}}).
\end{equation}

The idea is to determine, knowning the frequencies represented in each
DWT subband and the THH curve, the quantization step that should be
applied to each subband.

%}}}

\section{What you have to do?}
%{{{

\subsection{Subjective comparison}
%{{{

\begin{enumerate}
\item Using a recording tool such as
  \href{http://audacity.sourceforge.net}{Audacity} or
  \href{http://plugin.org.uk/timemachine/}{JACK Timemachine}, record
  the simulated transmission of a piece of audio and create a
  \texttt{.wav} file, when the audio has been transmitted using
  \texttt{temporal\_overlapped\_DWT\_coding.py} and
  \texttt{threshold.py}, using in both cases the same transmission
  bit-rate. Use the quantization step for controlling the bit-rate.
\item Determine which audio sounds better, from a subjective point of
  view. Repeat this step the number of times you consider necessary.
\end{enumerate}

%}}}

%}}}
    
\section{Deliverables}

The results of your experiments.

\section{Resources}

\bibliography{maths,data-compression,DWT,audio-coding}
