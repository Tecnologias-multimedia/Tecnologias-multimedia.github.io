\newcommand{\TM}{\href{https://www.ual.es/estudios/grados/presentacion/plandeestudios/asignatura/4015/40154321?idioma=zh_CN}{Tecnologías Multimedia}}

\title{\TM{} - Study Guide - Milestone 4: ``wiring'' the ADC with the DAC, and measuring latencies}

\maketitle

\section{Description}

Let's start our understanding of InterCom with a simple program that
inputs digital audio (through the
\href{https://en.wikipedia.org/wiki/Analog-to-digital_converter}{ADC}
of your computer) and, as soon as possible, outputs the audio (through
the
\href{https://en.wikipedia.org/wiki/Digital-to-analog_converter}{DAC}). We
are going to use a minimal ``wiring'' program, that when it is run and
we feed back the audio output with the audio input of our computer,
allows us to measure the
\href{https://en.wikipedia.org/wiki/Latency_(engineering)}{latency} of
this system. To handle the audio hardware we use
\href{https://python-sounddevice.readthedocs.io}{\texttt{sounddevice}}
\cite{sounddevice}, that is wrapper for the
\href{http://www.portaudio.com/}{PortAudio} library.

\section{What you have to do?}

\begin{enumerate}

\item Refresh some (probably high-school) ideas about
  \href{https://vicente-gonzalez-ruiz.github.io/the_sound/}{the
    sound}, \href{https://vicente-gonzalez-ruiz.github.io/human_auditory_system/}{the
    human auditory system},
    and \href{https://vicente-gonzalez-ruiz.github.io/human_sound_perception/}{the
    human sound perception}. %Notice that we are able to perceive
    %sounds whose frequency ranges between 20 Hz and 20 KHz
    %(approximately), and that a speaker is basically a microphone, and
    %viceversa.
  
\item Download the Python
  \href{https://docs.python.org/3/tutorial/modules.html}{module}
  \href{https://raw.githubusercontent.com/Tecnologias-multimedia/intercom/master/test/sounddevice/wire3.py}{wire3.py} with:

  \begin{lstlisting}[language=Bash]
      sudo apt install curl
      curl https://raw.githubusercontent.com/Tecnologias-multimedia/intercom/master/test/sounddevice/wire3.py > wire3.py
  \end{lstlisting}      
      
  This module implements the algorithm:

  \begin{lstlisting}[language=Python]
    while True:
      chunk = sound_card.read(chunk_size = 1024)  # (1)
      sound_card.write(chunk)  # (2)
  \end{lstlisting}
  
  \begin{pseudocode}[display]{}{}
    \BEGIN
      \mathtt{CHUNK\_SIZE} \GETS 1024\\
      \WHILE \TRUE
      \BEGIN
        \mathtt{chunk} \GETS \mathtt{stream}.\mathtt{read}(\mathtt{CHUNK\_SIZE}) \STMTNUM{1in}{st.1}\\
        \mathtt{stream}.\mathtt{write}(\mathtt{chunk}) \STMTNUM{1.03in}{st.2}
      \END
    \END
  \end{pseudocode}

  where {\texttt (1)} captures {\textt 1024} frames from the sound
  card, and (2) plays the chunk of frames
  $\mathtt{chunk}$. In \texttt{sounddevice} a frame is a collection of
  one or more samples (typically, a frame is a single sample if the
  number of channels is 1, or two samples if the number of channels is
  2).

\item If you want to run this module right now, check first that the
  output volumen of your speakers is not too high, otherwise you could
  ``couple'' the speaker and the mic(rophone) of your computer,
  producing a loud and annoying sound. In order to mitigate this
  effect, you can also control the record gain of your mic (if the
  gain is 0, no feedback between the speaker and the mic will be
  possible). In Xubuntu, these controls are available by clicking in
  the speaker icon (situated in the top right corner of your screen)
  of the Xfce window manager.

\item OK, run it (remember to activate first the virtual environment
  \texttt{tm} before using Python), with:

  \begin{lstlisting}[language=Bash]
    pip install sounddevice
    python wire3.py
  \end{lstlisting}

  %If you obtain the message \texttt{ALSA lib
  %  pcm.c:8526:(snd\_pcm\_recover) underrun occurred}, ignore it (this
  %is a consequence of that you are using
  %\href{https://www.freedesktop.org/wiki/Software/PulseAudio/}{PulseAudio}
  %instead of \href{https://alsa-project.org/wiki/Main_Page}{ALSA}
  %directly, and the used \href{https://www.kernel.org/}{kernel} has
  %been compiled without
  %\href{https://en.wikipedia.org/wiki/RTLinux}{real-time process
  %  scheduling}).

\item Now, lets compute, experimentally, the \texttt{wire3.py}
  latency.

  \begin{enumerate} \item First, we need the
  tools \href{http://sox.sourceforge.net/}{SoX}, \href{https://www.audacityteam.org/}{Audacity}
  and \href{https://raw.githubusercontent.com/Tecnologias-multimedia/intercom/master/test/sounddevice/wire3.py}{wire3.py}:
  
    \begin{lstlisting}[language=Bash]
      sudo apt install sox
      sudo apt install audacity
      sudo apt install curl
      curl https://raw.githubusercontent.com/Tecnologias-multimedia/intercom/master/test/sounddevice/plot_input.py > plot_input.py
    \end{lstlisting}

  \item Run:

    \begin{lstlisting}[language=Bash]
      pip install matplotlib
      python plot_input.py
    \end{lstlisting}

    and check that the gain of the mic does not
    produce \href{https://en.wikipedia.org/wiki/Clipping_(audio)}{clipping}
    during the sound recording.

  \item \label{start_point} In a terminal, run:

    \begin{lstlisting}[language=Bash]
      python wire3.py
    \end{lstlisting}

    while you control the output volume of the speakers to produce a
    decaying coupling effect between both devices (the speaker(s) and
    the mic). If your desktop has not
    these \href{https://en.wikipedia.org/wiki/Transducer}{transducers},
    we can use
    a \href{https://www.google.com/search?q=male+to+male+audio+jack+cable&client=firefox-b-d&sxsrf=ALeKk00GZUDGqiOfc0D8xkA_MIYgCuZmSA:1600270049146&source=lnms&tbm=isch&sa=X&ved=2ahUKEwjdvsu-_u3rAhXl0eAKHS90DUoQ_AUoAXoECA0QAw&biw=4288&bih=972}{male-to-male
    jack audio cable} and connect the input and the output of your
    sound card.

  \item In a different terminal (keep \texttt{python wire3.py}
    running), run:

    \begin{lstlisting}[language=Bash]
      sox -t alsa default test.wav
    \end{lstlisting}

    to save the ADC's output to a file.

  \item While \texttt{sox} is recording, produce some short sound (for
    example, hit your laptop or your micro with one or your nails). Do
    this at least a couple of times more, to be sure that you record
    the sound and the feedback of such sound. It's important the sound
    to be short (a few miliseconds) in order to recognize it and it's
    replicas.

  \item Stop \texttt{sox} by pressing the CTRL and C keys (CTRL+C) at
    the same time. This kills \texttt{sox}.

  \item Stop \texttt{python wire3.pt} with CTRL+C.

  \item Load the sound file into Audacity:
    
    \begin{lstlisting}[language=Bash]
      audacity test.wav
    \end{lstlisting}

  \item Localize the first one of your nail-hitting-the-mic sounds in
    the
    \href{https://manual.audacityteam.org/man/audio_tracks.html}{audio
      track}.

  \item Select the region with your sound and a replica using our
    mouse.

  \item Use the
    \href{https://manual.audacityteam.org/man/edit_toolbar.html#zoom_to_selection}{zoom
      to selection buttom} to zoom-in the selected area.

  \item Measure the time between the ocurrence of the sound and
    it's first replica. This time is the real latency of your computer
    runing \texttt{wire3.py}.

  \item Modify the constant $\mathtt{CHUNK\_SIZE}$ and repeat this
    process, starting at the Step \ref{start_point}. Create an ASCII
    file (\texttt{latency\_vs\_chunk\_size.txt}) with the content (use
    tabulators to space the columns):
\begin{verbatim}
# CHUNK_SIZE    real
32              ...
64              ...
128             ...
256             ...
512             ...
1024            ...
2048            ...
4096            ...
8192            ...
\end{verbatim}

  \end{enumerate}

\item At this point, we know the real latency of \texttt{wire3.py} as
  a function of $\mathtt{CHUNK\_SIZE}$. Plot the file
  \texttt{latency\_vs\_chunk\_size.txt} with:

  \begin{lstlisting}[language=Bash]
     sudo apt install gnuplot
     echo "set terminal pdf; set output 'latency_vs_chunk_size.pdf'; set xlabel 'CHUNK\_SIZE (frames)'; set ylabel 'Latency (seconds)'; plot 'latency_vs_chunk_size.txt' title '' with linespoints" | gnuplot
  \end{lstlisting}

\item Let's compute now the buffering latency of a chunk (the chunk-time). If
  $\mathtt{sampling\_rate}$ is the number of frames per second, it holds
  that:
  
  \begin{equation}
    \mathtt{buffering\_latency} = \mathtt{CHUNK\_SIZE} / \mathtt{sampling\_rate}
  \end{equation}

  Add these calculations to \texttt{latency\_vs\_chunk\_size.txt} using
  a third column (remember to use TABs).
\begin{verbatim}
# CHUNK_SIZE    real    minimal
:               :       :
\end{verbatim}

\item Plot both latencies:

  \begin{lstlisting}[language=Bash]
    echo "set terminal pdf; set output 'latency_vs_chunk_size.pdf'; set xlabel 'CHUNK\_SIZE (frames)'; set ylabel 'Latency (seconds)'; set key left; plot 'latency_vs_chunk_size.txt' using 1:2 title 'Real' with linespoints, 'latency_vs_chunk_size.txt' using 1:3 title 'Minimal' with linespoints" | gnuplot
  \end{lstlisting}
  
\item Which seems to be the minimal practical latency (the latency obtained
  when $\mathtt{CHUNK\_SIZE}=1$) in your computer? Justify your
  answers. \label{question}

\end{enumerate}

\section{Timming}

You should reach this milestone at most in one week.

\section{Deliverables}

Answer the question enunciated in the Step ~\ref{question}.

\section{Resources}

\bibliography{python}
