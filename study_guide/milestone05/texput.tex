\title{\href{https://www.ual.es/estudios/grados/presentacion/plandeestudios/asignatura/4015/40154321?idioma=zh_CN}{Tecnologías Multimedia} - Study Guide - Milestone 6: Inter-channel decorrelation in stereo audio}

\maketitle

\section{Description}

\href{https://en.wikipedia.org/wiki/Correlation_and_dependence}{Correlation}
is the term used in statistics for refering to the interdependency
between \href{https://en.wikipedia.org/wiki/Random_variable}{random
  variables}. It can be measured by the
\href{https://www.mathsisfun.com/data/correlation.html}{correlation
  coefficient}~\cite{thinkstats}.

In the case of InterCom, the random variables are the two channels
(left and right) of the
\href{https://en.wikipedia.org/wiki/Stereophonic_sound}{stereo
  signal}. In most cases, both channels are going to be
\href{https://en.wikipedia.org/wiki/Binaural_recording}{highly
  correlated}, which means that we can represent one of them (for
example, the right channel) with respect to the other (the left
channel). From a mathematical point of view, this process can be seen
as a
\href{https://en.wikipedia.org/wiki/Decorrelation}{decorrelation}~\cite{sayood2017introduction}
process.

To perform this inter-channel decorrelation, we are going to use an
orthogonal transform
\begin{equation}
  y = Kx = \frac{1}{\sqrt{2}}\[\begin{array}{cc} 1 & 1 \\ 1 & -1 \end{array}\]x,
\end{equation}
where $x$ stands for a frame (a tuple or L and R samples), $K$ the
$2\times 2$ KLT transform matrix (closely
related with the Haar transform and the S+P transform), and $y$ are
the transform coefficients (in our case, a couple of
coefficients). Notice that this transform is orthonormal because
it holds that,
\begin{equation}
  \sum y_i^2 \neq \sum x_i^2.
\end{equation}

This linear transform can also be viewed 
To perform this inter-channel decorrelation, we are going to suppose
that the left and the right channels are going to transport similar
signals (most of the information that the R channel transports can be
considered redundant). Therefore, the right channel will be
represented as the sample-by-sample difference between the left and
the right channels. Moreover, in order to have also a basic audio
signalIn other words:

\begin{pseudocode}{Inter-channel\_decorrelation}{~}
  \PROCEDURE{Decorrelate}{\mathtt{chunk}}
  \BEGIN
    
    \mathtt{recorded\_chunk} \GETS \mathtt{record()}\\
    \mathtt{outgoing\_packet} \GETS \mathtt{pack(recorded\_chunk)}\\
    \mathtt{send(outgoing\_packet)}\\
    \mathtt{incoming\_packet} \GETS \mathtt{receive()}\\
    \mathtt{chunk\_to\_play} \GETS \mathtt{unpack(incoming\_packet)}\\
    \mathtt{play(chunk\_to\_play)}
  \END
  \ENDPROCEDURE
\end{pseudocode}

Notice that in the case of mono signals, where L and R are identical, 


\section{What you have to do?}

\section{Timming}

You should reach this milestone at most in one week.

\section{Deliverables}

\section{Resources}

\bibliography{maths}
