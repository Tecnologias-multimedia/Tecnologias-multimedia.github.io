% Emacs, this is -*-latex-*-

\title{\href{https://en.wikipedia.org/wiki/Transform_coding}{Transform Coding} for \href{https://en.wikipedia.org/wiki/Data_redundancy}{Redundancy} Removal}

% Use 16 bits/coefficient and CTE (Chunk Truncation Encoding).

\maketitle

\section{Description}
%{{{

\subsection{Spatial decorrelation in stereo audio signals}
%{{{

\subsubsection{An analysis transform}
%{{{ 
InterCom transmits a
\href{https://en.wikipedia.org/wiki/Stereophonic_sound}{stereo} (two
channels)
\href{https://en.wikipedia.org/wiki/Pulse-code_modulation}{PCM
  signal}. In most cases, the channels are
\href{https://en.wikipedia.org/wiki/Binaural_recording}{highly
  correlated} (especially when the microphone is mono because both
channels are identical), which means that we can find a more efficient
representation. To perform this inter-channel
\href{https://en.wikipedia.org/wiki/Decorrelation}{decorrelation}~\cite{thinkstats}
we can use the \href{https://en.wikipedia.org/wiki/Linear_map}{linear
  transform}~\cite{strang4linear}
\begin{equation}
  {\mathbf w} = {\mathbf K}{\mathbf x} =
\begin{bmatrix} \mathbf{K}_0 \\ \mathbf{K}_1 \end{bmatrix} =
\begin{bmatrix} 1 & 1 \\ 1 & -1 \end{bmatrix}
{\mathbf x},
  \label{eq:forward_transform_matrix_form}
\end{equation}
that can be also written as
\begin{equation}
  \begin{bmatrix}
    {\mathbf w}_0 \\
    {\mathbf w}_1
  \end{bmatrix}
  = 
  \begin{bmatrix} 1 & 1 \\ 1 & -1 \end{bmatrix}
  \begin{bmatrix}
    {\mathbf x}_0 \\
    {\mathbf x}_1
  \end{bmatrix},
  \label{eq:forward_transform_matrix_form2}
\end{equation}
where ${\mathbf x}\in\mathbb{Z}^2$ is a stereo frame, ${\mathbf K}$ is
the forward (or analysis) transform matrix, and
${\mathbf w}=\begin{bmatrix} {\mathbf w}_0 & {\mathbf
    w}_1\end{bmatrix}^{\text T}$ is the corresponding
\href{https://en.wikipedia.org/wiki/Discrete_wavelet_transform}{decomposition}. In
this particular transform, the decomposition has two
\href{https://en.wikipedia.org/wiki/Sub-band_coding}{subbands}
${\mathbf w}_0$ and ${\mathbf w}_1$, and each subband has only one
\href{https://web.stanford.edu/class/ee398a/handouts/lectures/07-TransformCoding.pdf}{coefficient}. Notice
that ${\mathbf x}\in\mathbb{Z}^2$ is a vector space\footnote{Adding
  two vectors in the plane produces a third one also in the plane;
  multiplying a vector by a real scalar produces a second vector also
  in the plane. These two ingrained facts make the real plane be a
  vector space.~\cite{vetterli2014foundations}} if we consider also
the required operations.

The proposed matrix ${\mathbf K}$ corresponds to the transform used in
\href{https://en.wikipedia.org/wiki/Joint_encoding#M/S_stereo_coding}{Mid/Side
  (M/S) stereo coding}~\cite{bosi2003intro} that we will call MST
(Mid/Side Transform). This is similar to the $2\times 2$ KLT
\href{https://en.wikipedia.org/wiki/Kosambi%E2%80%93Karhunen%E2%80%93Lo%C3%A8ve_theorem}{(Karhunen-Lo\`eve
  Transform)}, the
\href{http://wavelets.pybytes.com/wavelet/haar/}{Haar
  Transform}~\cite{vetterli1995wavelets} and the $2\times 2$
\href{https://en.wikipedia.org/wiki/Hadamard_transform}{Discrete
  Walsh-Hadamard Transform}~\cite{sayood2017introduction}.

In general (for all the linear transforms),
Eqs.~\eqref{eq:forward_transform_matrix_form} and
\eqref{eq:forward_transform_matrix_form2} can be also expressed as
\begin{equation}
  {\mathbf w}_u = \sum_i {\mathbf K}_{u,i}{\mathbf x}_i,
  \label{eq:forward_transform_linear_combination_form}
\end{equation}
where ${\mathbf K}_{u,i}$ denotes $i$-th element of the $u$-th row of
the matrix ${\mathbf K}$.

A major difference between the transformed data ${\mathbf w}$ and the
original data ${\mathbf x}$ is that the characteristics of the
elements of ${\mathbf w}$ are determined by their position within the
decomposition ${\mathbf w}$~\cite{sayood2017introduction}. Thus, as a
consequence of how the matrix has been defined, the subband ${\mathbf
  w}_0$ represents (very roughly) the low frequencies of ${\mathbf
  x}$, and ${\mathbf w}_1$ the high frequencies. Therefore, the values
of ${\mathbf K}_0$ (the row 0 of ${\mathbf K}$) describe a
\href{https://en.wikipedia.org/wiki/Low-pass_filter}{low-pass filter},
the values of ${\mathbf K}_1$ describe a
\href{https://en.wikipedia.org/wiki/High-pass_filter}{high-pass
  filter}, and ${\mathbf K}$ represents the
\href{https://en.wikipedia.org/wiki/Digital_filter}{filters} of a
\href{https://en.wikipedia.org/wiki/Filter_bank}{filter bank (FB)} with two
filters. This can be also seen in the notebook
\href{https://github.com/Tecnologias-multimedia/Tecnologias-multimedia.github.io/blob/master/study_guide/transform_coding/stereo_transforms_RD.ipynb}{A RD-comparison of ``Stereo'' Transforms}.
%}}}

\subsubsection{The synthesis transform}
%{{{ 
The inverse (or synthesis) transform
\begin{equation}
  {\mathbf x} = {\mathbf K}^{-1}{\mathbf w}
  \label{eq:inverse_transform}
\end{equation}
can be deduced from Eq.~\eqref{eq:forward_transform_matrix_form}, where we
get that
\begin{equation}
  \begin{array}{rcl}
  {\mathbf w}_0 & = & {\mathbf x}_0 + {\mathbf x}_1\\
  {\mathbf w}_1 & = & {\mathbf x}_0 - {\mathbf x}_1.
  \end{array}
\end{equation}
By solving ${\mathbf x}_0$ (adding) and ${\mathbf x}_1$ (substracting) in
these equations, we obtain that
\begin{equation}
  \begin{array}{rcl}
  {\mathbf x}_0 & = & \frac{1}{2}({\mathbf w}_0 + {\mathbf w}_1)\\
  {\mathbf x}_1 & = & \frac{1}{2}({\mathbf w}_0 - {\mathbf w}_1),
  \end{array}
\end{equation}
that in matrix form becomes
\begin{equation}
  \begin{bmatrix}
    {\mathbf x}_0 \\
    {\mathbf x}_1
  \end{bmatrix}
  = \frac{1}{2}
  \begin{bmatrix} 1 & 1 \\ 1 & -1 \end{bmatrix}
  \begin{bmatrix}
    {\mathbf w}_0 \\
    {\mathbf w}_1
  \end{bmatrix}.
\end{equation}
Therefore,
\begin{equation}
  {\mathbf x} = {\mathbf K}^{-1}{\mathbf w} = \frac{1}{2}{\mathbf K}^{\text T}{\mathbf w} = \frac{1}{2}{\mathbf K}{\mathbf w} = \frac{1}{2}\begin{bmatrix} 1 & 1 \\ 1 & -1 \end{bmatrix}{\mathbf w}.
  \label{eq:inverse_transform_matrix_form}
\end{equation}
%}}}

\subsubsection{Orthogonality of the transform}
%{{{ 
As can be seen (if we ignore the $\frac{1}{2}$ scale factor)
the inverse transform is the transpose of the forward transform
(${\mathbf K}^{-1}={\mathbf K}^{\text T}$). This is a characteristic
of all
\href{https://en.wikipedia.org/wiki/Orthogonal_transformation}{orthogonal
  transforms}~\cite{sayood2017introduction}. For the MST,
specifically, it also holds that ${\mathbf K}^{\text T}={\mathbf K}$
because ${\mathbf K}$ is
\href{https://en.wikipedia.org/wiki/Symmetric_matrix}{symmetric}.

Apart from checking that ${\mathbf K}^{-1}={\mathbf K}^{\text T}$,
${\mathbf K}$ is orthogonal if the
\href{https://en.wikipedia.org/wiki/Inner_product_space}{inner
  product}\footnote{The inner product between two vectors is in some
  sense a measure of how ``similar'' they are
  \cite{sayood2017introduction}. In fact, the dot product computes the
  norm (a measure of the distance between
  vectors).~\cite{vetterli2014foundations} Notice also, that the inner
  product is
  \href{https://math.stackexchange.com/questions/476738/difference-between-dot-product-and-inner-product}{also
    called} the \href{https://en.wikipedia.org/wiki/Dot_product}{dot
    product} and the scalar product when we work with
  \href{https://en.wikipedia.org/wiki/Real_number}{real}
  signals.~\cite{vetterli2014foundations}} of the
filters\footnote{When we are working with discrete signals, we usually
  talk about vectors instead of functions. These vectors are sampled
  versions of the corresponding functions, or as happen in our case,
  the
  \href{https://en.wikipedia.org/wiki/Finite_impulse_response}{taps}
  of the filters, each one representing a
  \href{https://en.wikipedia.org/wiki/Basis_(linear_algebra)}{basis
    vectors}.} of ${\mathbf K}$ is $0$ between the different filters
(rows of the matrix)\footnote{If a set of vectors are linearly
  independent, then the set is called a basis for the subspace
  generated by linear combinations of this set. The basis set contains
  the smallest number of linearly independent vectors required to
  represent each element of the vector (sub)space. The number of basis
  vectors required to generate the space is called the dimension of
  the vector space~\cite{sayood2017introduction}. In our case, for the
  MST, we have two basis vectors.}. In our case
${\mathbf K}_0=\begin{bmatrix}1 & 1\end{bmatrix}$~ and
${\mathbf K}_1=\begin{bmatrix} 1 & -1\end{bmatrix}$~, and as we can
see
\begin{equation}
  \langle {\mathbf K}_0,{\mathbf K}_1 \rangle =
  \langle \begin{bmatrix}
    1 & 1
  \end{bmatrix}
  ,
  \begin{bmatrix}
    1 & -1
  \end{bmatrix}
  \rangle =
  \begin{bmatrix}
    1 & 1
  \end{bmatrix}
  \cdot
  \begin{bmatrix}
    1 & -1
  \end{bmatrix}
   = 1\times 1 + 1\times -1 = 0,
\end{equation}
which means that the filters ${\mathbf K}_0$ and ${\mathbf K}_1$ are linearly
independent\footnote{In terms of orthogonality, this means that we
  cannot derive one from the other using the operations that define a
  vector space, and therefore the basis vectors can be a part a basis
  (set)~\cite{strang4linear}.}.

Notice also that
\begin{equation}
  {\mathbf w}_i = \langle {\mathbf x}, {\mathbf K}_i\rangle,
\end{equation}
which basically means\footnote{Remember that for the MST a subband has
  only one coefficient. For other transforms, ${\mathbf w}_i$ can be
  made up of more than one coefficient and therefore, we would be
  speaking of the coefficients of the subband, instead of only one
  coefficient.} that ${\mathbf w}_i$ is proportional to the similarity
between the input signal ${\mathbf x}$ and the
\href{https://en.wikipedia.org/wiki/Finite_impulse_response}{taps} of
the filter ${\mathbf K}_i$. These
\href{https://cseweb.ucsd.edu/classes/fa17/cse166-a/lec13.pdf}{slides}
can help you with this key idea.

Orthogonality is important in compression applications because the
\href{https://en.wikipedia.org/wiki/Correlation_and_dependence}{statistical
  correlation} between subbands is 0, and therefore, the contributions
of the subbands to the reconstruction of the original signal are
independent\footnote{The total
\href{https://en.wikipedia.org/wiki/Distortion}{distortion} is the sum
of the distortion contribution of each
subband~\cite{sayood2017introduction}.}. Another interesting property
satisfied by a lot of famous transforms (such as the
\href{https://en.wikipedia.org/wiki/Fourier_transform}{Fourier
  Transform}) is
\href{https://en.wikipedia.org/wiki/Orthonormality}{orthonormality},
which means that the transform is
\href{https://en.wikipedia.org/wiki/Energy_(signal_processing)}{energy}
preserving~\cite{sayood2017introduction} (or that the
\href{https://en.wikipedia.org/wiki/Parseval%27s_theorem}{Parseval's
  theorem} is satisfied, in both, the analysis and the synthesis
transform).

The MST analysis is not orthonormal, because
\begin{equation}
  \sum_i {{\mathbf w}_i}^2 =
  ({\mathbf x}_0 + {\mathbf x}_1)^2 + ({\mathbf x}_0 - {\mathbf x}_1)^2 =
  ({\mathbf x}_0^2 + 2{\mathbf x}_0{\mathbf x}_1+{\mathbf x}_1^2) + ({\mathbf x}_0^2-{\mathbf 2}x_0{\mathbf x}_1+{\mathbf x}_1^2) =
  2({\mathbf x}_0^2+{\mathbf x}_1^2) =
  2\sum_i {{\mathbf x}_i}^2.
  \label{eq:No_Parseval}
\end{equation}
For this reason, we must divide the synthesized samples by $2$ (see
Eq.~\eqref{eq:inverse_transform_matrix_form}). On the contrary, we would
get $2{\mathbf x}$ as the reconstructed signal instead of ${\mathbf x}$.
%}}}

\subsubsection{Quantization of the subbands}
%{{{ 
Ideally, the quantization step size $\Delta_i$ used for a subband
${\mathbf w}_i$ must operate in the RD curve $f_i$ with the same slope
than the rest of
subbands~\cite{vetterli2014foundations,sayood2017introduction} (this
is the same to say that we must satisfy that $f'_0(x)=f'_1(x)$, where
$f'$ denotes the derivative of $f$). The main drawback of this
approach is that the finding of $f_i$ is computationally intensive (we
must analyze, quantize, compress, decompress, dequantize, synthesize
and compute the distortion of the data for a enoughly high number of
quantization steps), and usually we cannot do that in real-time
applications.\footnote{Notice, however, that this would solve the
problem of controlling the bit-rate because using the RD curves we
know how many bits will require each subband.}

An approximation to this could be to suppose that the RD curves of the
subbands resulting from the analysis of our current
piece\footnote{Remember, two samples of a stereo frame in our case
  because we are removing the spatial redundancy.} of data are similar
to the curves of previous pieces, that has already been compressed and
transmitted, and therefore, we can compute also the distortion. Using
this information, we can estimate a RD curve for the current piece,
and estimate the ``optimal'' quantization step sizes. This procedure
is much faster than the described in the previous paragraph, but it may
still be time-consuming.

For this reason, in the notebook
\href{https://github.com/Tecnologias-multimedia/Tecnologias-multimedia.github.io/blob/master/study_guide/transform_coding/stereo_transforms_RD.ipynb}{A
  RD-comparison of ``Stereo'' Transforms} we explore a different
solution based on the idea that the contribution (in terms of
energy) of the subbands to the reconstruction of the signal
${\mathbf x}$ should be proportional to the
\href{https://en.wikipedia.org/wiki/Filter_(signal_processing)}{gain}
of each analysis filter of ${\mathbf K}$ (remember that we are working
with orthogonal transforms and therefore, the contribution of the
subbands are independent), or what is the same, proportional to the
gain of each column of the synthesis matrix\footnote{Notice that the
  quantization error is generated in the transform domain and
  perceived in the signal domain after appliying the inverse
  transform.}. Thus, if the filters had different gains, the
quantization steps should consider this fact.\footnote{Notice that the
  important here is the relative gain of each subband. For example, if
  the gain of ${\mathbf K}_0$ were $2$ and the gain of ${\mathbf K}_1$
  were $1$, and the dynamic range of ${\mathbf w}_0$ and
  ${\mathbf w}_1$ is the same, we could use $\Delta_1=2\Delta_0$ and
  expect to minimize the distortion.}

By definition, the gain generated in the subband ${\mathbf w}_i$ is
proportional to the
\href{https://en.wikipedia.org/wiki/Lp_space}{L$_2$
  norm}\footnote{L$_2(f)$ (where $f$ is a function) is the set of all
  functions with finite energy and constitues a vector
  space~\cite{sayood2017introduction}. $L_2({\mathbb R})$ of simply
  $L_2$ is the space of all functions $f(t)$ with a well defined
  integral of the square of the modulus of the function. The $L$
  signifies a Lebesque integral, the ``2'' denotes the integral of the
  square of the modulus of the function, and ${\mathbb R}$ states that
  the independent variable of integration is a number over the whole
  real line. For a function $g(t)$ to be a member of that space is
  denoted: $g\in L_2({\mathbb R})$ or simply
  $g\in L_2$~\cite{burrus2013wavelets}. The computation of the L$_2$
  form is equivalent to compute the
  \href{https://en.wikipedia.org/wiki/Euclidean_distance}{Euclidean
    distance} in $N$-dimensional (in our case, $N=2$)
  \href{https://en.wikipedia.org/wiki/Vector_space}{spaces}.} of the
filter ${\mathbf K}_i$ (remember that for the MST, the rows of the
analysis matrix are equal to the columns of the synthesis
matrix). Thus

\begin{equation}
  \begin{array}{l}
    \left\| {\mathbf K}_0 \right\|_2 := \sqrt{\langle \begin{bmatrix}1 & 1\end{bmatrix}, \begin{bmatrix}1 & 1\end{bmatrix} \rangle} = \sqrt{\begin{bmatrix}1 & 1\end{bmatrix} \cdot \begin{bmatrix}1 & 1\end{bmatrix}} = \sqrt{2},\\
    \left\| {\mathbf K}_1 \right\|_2 := \sqrt{\langle \begin{bmatrix}1 & -1\end{bmatrix}, \begin{bmatrix}1 & -1\end{bmatrix} \rangle} = \sqrt{\begin{bmatrix}1 & -1\end{bmatrix}\cdot \begin{bmatrix}1 & -1\end{bmatrix}} = \sqrt{2},
  \end{array}
\end{equation}
resulting that both subbands ${\mathbf w}_1$ and ${\mathbf w}_2$ have
the same gain ($\sqrt{2}$). This result tell us that both subbands
could use the same quantization step size ($\Delta_0=\Delta_1$). In
the notebook
\href{https://github.com/Tecnologias-multimedia/Tecnologias-multimedia.github.io/blob/master/study_guide/transform_coding/stereo_transforms_RD.ipynb}{A
  RD-comparison of "Stereo" Transforms} there are some evidences of
this.

Unfortunately, most of the transform are not implemented using
matrix-vector operations, but using
\href{https://en.wikipedia.org/wiki/Fast_Fourier_transform}{faster
  algorithms} based on a lattice of
\href{https://en.wikipedia.org/wiki/Butterfly_diagram}{computational
  bufferflies} or filter
\href{https://en.wikipedia.org/wiki/Filter_(signal_processing)}{convolutions}
(and therefore, we don't know ${\mathbf K}$). Fortunately, we can
determine ${\mathbf K}_i$ simply by computing the inverse transform of
the decomposition
$\begin{bmatrix} 0 & \cdots & 0 & 1 & 0 & \cdots &
  0 \end{bmatrix}^{\text T}$, where the $1$ value is in the position
$i$ (only the subband ${\mathbf w}_i=1$, the rest are
``zeroed'')).\footnote{Notice that this operation will ``extract'' the
  $i$-th column from ${\mathbf K}^{-1}$ that is equivalent to say that
  will ``extract'' the $i$-th row of ${\mathbf K}$, ${\mathbf K}_i$
  (remember that for orthogonal transforms,
  ${\mathbf K}^{-1}={\mathbf K}^{\text T}$).} In our example, we get
that

\begin{equation}
  \begin{array}{l}
    {\mathbf K}_0 =
    \begin{bmatrix}
      1 & 1 \\
      1 & -1
    \end{bmatrix}
          \begin{bmatrix}
            1 \\
            0
          \end{bmatrix}
    =
    \begin{bmatrix}
      1 & 1
    \end{bmatrix},
    \\
    {\mathbf K}_1 = 
    \begin{bmatrix}
      1 & 1 \\
      1 & -1
    \end{bmatrix}
          \begin{bmatrix}
            0 \\
            1
          \end{bmatrix}
    =
    \begin{bmatrix}
      1 & -1
    \end{bmatrix},
  \end{array}
\end{equation}
that as you can see, correspond to the columns of the inverse
transform matrix ${\mathbf K}^{-1}$. Notice that this is true for all
the orthogonal transforms whose analysis and synthesis matrices are
symmetrical.% because in that case ${\mathbf K}^{-1}={\mathbf K}^{\text T}$.
%}}}

%}}}

\subsection{Temporal coding using the DWT (Discrete Wavelet Transform)}
%{{{

\subsubsection{About temporal redundancy in audio}
%{{{

After exploiting the spatial (stereo) redundancy, the next natural
step in the development of InterCom is to remove the temporal
redundancy that can be found inside of each subband\footnote{Notice
that, beacuse the MST and the transform used in this milestone are
both lineal, the order in which the transforms are applied is
irrelevant. For this reason, we could also have used the temporal
transform inside of each channel of samples, and then, remove the
spatial redundancy.}. As it can be seen in the notebook
\href{https://github.com/Tecnologias-multimedia/intercom/blob/master/tools/audio_viewer.ipynb}{Audio viewer},
most audio signals show ``patterns'' of samples that tends to repeat,
especially locally. Another clear source of temporal redundancy is
that the neighbor audio samples usually show similar amplitude values.

% A decorrelating technique: DPCM
There are several techniques that can be used for removing the
temporal redundancy of a sequence of audio. One of the most
straightforward is
\href{https://en.wikipedia.org/wiki/Differential_pulse-code_modulation}{Differential
  Pulse Code Modulation
  (DPCM)~\cite{sayood2017introduction}}. However, there are more
efficient decorrelation algorithms based on
\href{https://en.wikipedia.org/wiki/Transform_coding}{Transform
  Coding}, such as the described in the previous section and in this
one.

% Another decorrelating technique: transform coding
As it has been explained before, Transform Coding is based on the idea
that we can decompose the input signal into a set of subbands, and if
the used filters are the adecuate ones for removing the (in this case,
temporal) redundancy, we can achieve a high Transform Coding
Gain~\cite{sayood2017introduction}, accumulating the most part of the
signal energy (and presumably most of the information) in a small
number of subbands. When this happens, the quantization of the
subbands will remove basically the least significant information
(usually
\href{https://en.wikipedia.org/wiki/Noise_(electronics)}{electronic
  noise}), allowing better compression ratios than those in which we
apply the same quantization process to the original
samples.\footnote{Notice that is we dead-zone quantize a decomposition
  and most of the coefficients are close to zero, the information
  removed from the signal will be those with a smaller energy.}

%}}}

\subsubsection{Transform and subband coding}
%{{{

% Relation between transform coding and subband coding
The name that has been given to the previous process is
\href{https://en.wikipedia.org/wiki/Sub-band_coding}{Subband
  Coding}\footnote{\href{https://en.wikipedia.org/wiki/Sub-band_coding}{In
    signal processing, sub-band coding (SBC) is any form of transform
    coding that breaks a signal into a number of different frequency
    bands}.}. In this context, our analysis transform matrix
${\mathbf K}$ (see the previous section) represents the taps of a
2-channels analysis
\href{https://en.wikipedia.org/wiki/Filter_bank}{Filter Bank
  (FB)}~\cite{vetterli1995wavelets}, and the forward transform is in
fact ``descomposing'' ${\mathbf x}$ into two subbands ${\mathbf w}_0$
and ${\mathbf w}_1$ (see the Figure~\ref{fig:PRFB}, and the notebook
\href{https://github.com/Tecnologias-multimedia/intercom/blob/master/docs/PRFB.ipynb}{A
  Perfect Reconstruction Filter Bank (PRFB)}). On the other hand, the
synthesis transform matrix ${\mathbf K}^{-1}$ denotes the taps of the
corresponding synthesis FB that allows to recover ${\mathbf x}$
(notice that in the figure, ${\mathbf x}={\mathbf l}^i$,
${\mathbf w}_0={\mathbf l}^{i+1}$, ${\mathbf w}_1={\mathbf h}^{i+1}$,
$\tilde\phi={\mathbf K}_0$, $\tilde\psi={\mathbf K}_1$,
$\phi={\mathbf K}^{-1}_0$, and $\psi={\mathbf K}^{-1}_1$).

\begin{figure}
  \centering
  \myfig{graphics/PRFB}{4cm}{400}
  \caption{A 2-channels PRFB (Perfect Reconstruction Filter Bank).}
  \label{fig:PRFB}
\end{figure}

% An intro to PRFBs
Let's suppose now that the analysis filters (represented by the taps
of) ${\mathbf K}_0$ and ${\mathbf K}_1$ are applied to the input
signal ${\mathbf x}$ (now a sequence of $N$ samples) using a
\href{https://en.wikipedia.org/wiki/Kernel_(image_processing)}{convolution}
(without splitting $\mathbf{x}$ into blocks). Let's also suppose (as
happens in the MST) that ${\mathbf K}_0$ is a low-pass filter and
${\mathbf K}_1$ is a high-pass filter, and that the
\href{https://en.wikipedia.org/wiki/Filter_(signal_processing)#The_transfer_function}{transfer
  function}\footnote{The response of the filter to the
  \href{https://en.wikipedia.org/?title=Unit_impulse&redirect=no}{unit
    impulse}.} of both filters
\href{https://en.wikipedia.org/wiki/Filter_bank#Perfect_reconstruction_filter_banks}{are
  one the inverse of the other}. Under these assumptions, the complete
(analysis/synthesis) transform is called a (2-channels)
\href{https://en.wikipedia.org/wiki/Filter_bank#Perfect_reconstruction_filter_banks}{Perfect
  Reconstruction Filter Bank (PRFB)}, and ${\mathbf x}$ can be
recovered (perfectly) from a subsampled version (in this case
\href{https://en.wikipedia.org/wiki/Downsampling_(signal_processing)}{decimating}
by 2 because we have two channels in the FB) of ${\mathbf w}_0$ and
${\mathbf w}_1$ (see the notebook
\href{https://github.com/Tecnologias-multimedia/intercom/blob/master/docs/PRFB.ipynb}{A
  Perfect Reconstruction Filter Bank (PRFB)}). Notice that this
subsampling is possible because the
\href{https://en.wikipedia.org/wiki/Aliasing}{aliasing}\footnote{Because
  the filters are not ideal, the bandwidth of the filtered signals
  ${\mathbf w}_0$ and ${\mathbf w}_1$ is bigger than half of the
  bandwidth of ${\mathbf x}$. Therefore, subsampling at a ratio of one
  of each two coefficients, we are generating aliasing. See the
  \href{https://en.wikipedia.org/wiki/Nyquist-Shannon_sampling_theorem}{sampling
    theorem}.}  generated in the low-pass subband is attenuated by the
aliasing generated in the high-pass subband. To achieve this, the
\href{https://en.wikipedia.org/wiki/Filter_(signal_processing)#The_transfer_function}{frequency
  response} of ${\mathbf K}_0$ must be equal to the mirrored frequency
response of ${\mathbf K}_1$, and obviously, both filters must have the
same
\href{https://en.wikipedia.org/wiki/Bandwidth_(signal_processing)}{bandwidth}~\cite{sayood2017introduction}. In
this situation, in which ${\mathbf K}_1$ and ${\mathbf K}_2$ are
mirror filters, we say that they form a
\href{https://en.wikipedia.org/wiki/Quadrature_mirror_filter}{Quadrature
  Mirror Filters (QMF) Bank}.

%}}}

\subsubsection{Multichannel filter banks and psychoacoustic frequency resolution}
%{{{

% M-channels PRFB and the frequency resolution of the HAS
Using the suitable filters, it is possible to build $M$-channels
PRFBs.\footnote{Notice that our matrix $K$ would have $M$ rows in this
  case, and also $M$ colums, to satisfy that
  ${\mathbf K}^{-1}={\mathbf K}^{\text T}$ if we are implementing an
  orthogonal transform.}  These filters can analyze (and synthesize)
the signal ${\mathbf x}$, decomposing it in
(\href{https://en.wikipedia.org/wiki/Low-pass_filter#Ideal_and_real_filters}{almost
  for sure}) overlaping frequency subbands with different
bandwidth. The question here is to know how many filters should be
used and what
\href{https://en.wikipedia.org/wiki/Band-pass_filter}{pass-band} width
should they have. At this design point, we must also consider that the
accuracy of the
\href{https://en.wikipedia.org/wiki/Psychoacoustics}{humman perception
  of the sound} depends on the frequency: (as it can be checked in
this
\href{https://github.com/Tecnologias-multimedia/intercom/blob/master/docs/frequency_resolution.ipynb}{notebook})
we are more sensitive to frequency variations when the frequency of
the sound is low. This fact is related with the way in which the
\href{https://en.wikipedia.org/wiki/Critical_band}{critical bands} are
distributed in \href{https://en.wikipedia.org/wiki/Bark_scale}{the
  bark scale}.

%}}}

\subsubsection{The Discrete Wavelet Transform}
%{{{

% The bark scale and the DWT
As it can be seen, the bark scale divides the audible spectrum into 24
subband of (a priori) ``whimsical'' bandwidths. However, it's clear
that a \href{https://en.wikipedia.org/wiki/Octave_band}{dyadic
  partition of the spectrum} fits better than
\href{https://en.wikipedia.org/wiki/Wavelet_transform#Principle}{a
  lineal partition}. Considering this reason, from all the families of
transforms designed to date, the most suitable one, from a frequency
partitioning point of view, is the Discrete Wavelet Transform (DWT).

% Features of the DWT
The DWT has also other interesting features:
\begin{enumerate}
\item It is
  \href{https://en.wikipedia.org/wiki/Discrete_wavelet_transform#Time_complexity}{fast}
  ($O(N)$, where $N$ is the number of ``transformed'' samples).
\item It can represent efficienty
  \href{https://en.wikipedia.org/wiki/Transient_(oscillation)}{transient}
  signals, which can happen frequently in audio.
\item Although we are not going to take advantage of the following
  characteristic (for now), one of the most interesting features of
  the DWT is that it can used to find a
  \href{https://en.wikipedia.org/wiki/Multiresolution_analysis}{multiresolution
    representation} of the signal.
\end{enumerate}

\begin{figure}
  \centering
  \myfig{graphics/cascade}{8cm}{800}
  \caption{A dyadic 2-levels cascade of PRFBs.}
  \label{fig:cascade}
\end{figure}

%}}}

\subsubsection{Implementation of the DWT}
%{{{

% Implementation alternatives for the DWT
The DWT can be implemented in different ways:
\begin{enumerate}
\item Defining the transform matrix ${\mathbf K}$ (see these
  \href{https://cseweb.ucsd.edu/classes/fa17/cse166-a/lec13.pdf}{slides})
  and computing vector-matrix multiplications, which requires a
  calculation time proportional to $O(N^2)$. However, the main problem
  of this type of implementation is generated by the amount of memory
  that ${\mathbf K}$ requires, that is proportional to $N^2$.
\item
  \href{https://en.wikipedia.org/wiki/Discrete_wavelet_transform#Cascading_and_filter_banks}{Cascading
    PRFBs} (see the Figure~\ref{fig:cascade}). Considering that the
  \href{https://en.wikipedia.org/wiki/Convolution}{convolution} is a
  $O(N\log_2N)$ operation (if it is
  \href{https://en.wikipedia.org/wiki/Convolution_theorem}{implemented
    in the frequency domain}), and that the number of levels in the
  cascade is generally small (5 for example), this implementation is
  faster than the based in vector-matrix arithmetic. And most
  importantly, we don't need to store ${\mathbf K}$, but only the taps
  of the filters that in a software implementation of a cascade can be
  as small as the number of different filters.\footnote{Notice that in
    a typical cascade, the filters are always the same.}
\item Using
  \href{https://en.wikipedia.org/wiki/Lifting_scheme}{lifting}~\cite{sweldens1997building},
  which provides an extra speed-up factor of 2 compared to the FB
  implementation. DWTs implemented with lifting do not need to
  downsample and upsample the subbands, an operation that is wasting
  the calculus of half of the coefficients at each level of the
  cascade.
\end{enumerate}

%}}}

\subsubsection{Example of a DWT using the MST filters}
%{{{

% Using the MST filters for building a DWT
In order to clarify the previously introduced concepts, let's build a
DWT using the MST filters and lifting.

\begin{enumerate}
\item Lifting is based on the concept of dyadic
  \href{https://en.wikipedia.org/wiki/Multiresolution_analysis}{multiresolution
    analysis}, and also with the so called
  \href{https://en.wikipedia.org/wiki/Polyphase_matrix}{polyphase
    representation} of signals. In order to do that, we can rewrite
  the MST filter equations (our ${\mathbf K}_0$ and $-{\mathbf K}_1$
  filters in the previous milestone) as
  \begin{equation}
    \begin{array}{rcl}
      {\mathbf l}^1_i & = & {\mathbf x}_{2i} + {\mathbf x}_{2i+1} \\
      {\mathbf h}^1_i & = & {\mathbf x}_{2i+1} - {\mathbf x}_{2i},
    \end{array}
    \label{eq:1dwt}
  \end{equation}
  where the $l$-th subband
  ${\mathbf z}^l=\{{\mathbf z}_i^l~|~0\le i\le 2^{n-l}\}$, being
  $2^n=N$ the number of samples in ${\mathbf x}$, and where, by
  definition, ${\mathbf l}^0={\mathbf x}$, the original resolution
  level of the signal. The subbands ${\mathbf l}^1$ and
  ${\mathbf h}^1$ computed by Eq.~\eqref{eq:1dwt} are the same than the
  decimated subbands computed by a 1-levels PRFB (based on that
  filters), and we say, therefore, that Eq.~\eqref{eq:1dwt} computes the
  1-levels DWT.

  Based on the 1-levels DWT, we define the 2-levels DWT as
  \begin{equation}
    \begin{array}{rcl}
      {\mathbf l}^2_i & = & {\mathbf l}^1_{2i} + {\mathbf l}^1_{2i+1} \\
      {\mathbf h}^2_i & = & {\mathbf l}^1_{2i+1} - {\mathbf l}^1_{2i},
    \end{array}
    \label{eq:2dwt}
  \end{equation}
  that, as we can see, uses as input the output of Eq.~\eqref{eq:1dwt}.

  In general, for a $l$-levels DWT, we get
    \begin{equation}
    \begin{array}{rcl}
      {\mathbf l}^l_i & = & {\mathbf l}^{l-1}_{2i} + {\mathbf l}^{l-1}_{2i+1} \\
      {\mathbf h}^l_i & = & {\mathbf l}^{l-1}_{2i+1} - {\mathbf l}^{l-1}_{2i}.
    \end{array}
    \label{eq:ldwt}
  \end{equation}

  The $l$-levels DWT splits the signal spectrum in $l+1$ subbands. If
  $l=n$ (where $N=2^n$), we have the spectrum partition
  \begin{equation*}
    | {\mathbf l}^l_0 | {\mathbf h}^l_0 | {\mathbf h}^{l-1}_0 {\mathbf h}^{l-1}_1 | {\mathbf h}^{l-2}_0 {\mathbf h}^{l-2}_1 {\mathbf h}^{l-2}_2 {\mathbf h}^{l-2}_3 | \cdots | {\mathbf h}^1_0 {\mathbf h}^1_1 \cdots {\mathbf h}^1_{2^{n-1}-1} |,
  \end{equation*}
  where\footnote{The coefficient ${\mathbf l}^l_0$ is called the DC
    (Direct Current) coefficient, and the rest of ${\mathbf h}$
    coefficients are called AC (Alternating Current) coefficients.} it
  holds that
  \begin{equation}
    1+\sum_{j=1}^l 2^{j-1}=2^n,
  \end{equation}
  i.e., the number of DWT coefficients is also $N$.

\item DWT performs a number of lifting steps, each one with
  2 (sub)steps:
  \begin{enumerate}
  \item A \textbf{predict step}, that computes the ${\mathbf h}$
    subbands as a prediction error (that in general should be
    minimized) between the even samples (usually, the values used to
    predict) and the odd samples (usually, the values predicted). For
    the MST filters, we have that (see Eq.~\eqref{eq:ldwt})
    \begin{equation}
      {\mathbf h}^l_i = {\mathbf l}^{l-1}_{2i+1} - {\mathbf l}^{l-1}_{2i}.
    \end{equation}
    
  \item An \textbf{update step}, which computes the ${\mathbf l}$
    subband considering (only) the even samples and the prediction
    errors. For the MST, we have that (see also Eq.~\eqref{eq:ldwt})
    \begin{equation}
      {\mathbf l}^l_i = 2{\mathbf l}^{l-1}_{2i} + {\mathbf h}^l_i.
    \end{equation}
  \end{enumerate}

  Notice that these steps are invertible:
  \begin{equation}
    \begin{array}{rcl}
      {\mathbf l}^{l-1}_{2i} & = & \frac{1}{2}({\mathbf l}^l_i - {\mathbf h}^l_i)\\
      {\mathbf l}^{l-1}_{2i+1} & = & {\mathbf l}^{l-1}_{2i} + {\mathbf h}^l_i.
    \end{array}
  \end{equation}

\end{enumerate}

%}}}

\subsubsection{Wavelets and filter banks}
%{{{

In the context of the wavelet theory~\cite{burrus2013wavelets}, the
response of the analysis low-pass filter (${\mathbf K}_0$ in the MST)
to the
\href{https://en.wikipedia.org/?title=Unit_impulse&redirect=no}{unit
  impulse}\footnote{The response of a filter to the unit impulse
characterize the filter because the output of the filter is the set of
taps of the filter.} is known as the \emph{scaling function} and is
usually denoted by $\tilde\phi$, the response of the analysis
high-pass filter (${\mathbf K}_1$) is known as the \emph{wavelet
function} and it is usually denoted by $\tilde\psi$, the response of
the synthesis low-pass filter (${\mathbf K}^{-1}_0$) is denoted by
$\phi$ and the synthesis high-pass filter (${\mathbf K}^{-1}_1$) is
represented by $\psi$.

For the MST it holds that $\tilde\phi\bot\tilde\psi$,
$\tilde\phi=\psi$ and $\tilde\psi=\phi$, and this is also true for all
orthogonal DWTs. Another important characteristic of orthogonal DWTs
is that the filters cannot be
\href{https://en.wikipedia.org/wiki/Symmetry}{symmetric}.\footnote{The
symmetry of the filters is important to produce the same type of
artifacts in the boundaries of the signal.}

%The dilated and translated versions of the wavelet function are
%orthogonal~\cite{sayood2017introduction}.

%}}}

\subsubsection{Example of a DWT using high-order filters}
%{{{

The previous MST-based DWT is similar to other transforms such as the
\href{https://en.wikipedia.org/wiki/Haar_wavelet}{Haar transform}, in
which we are using an 1-order predictor for removing the temporal
redundancy. Let's extend the idea of lifting to a prediction of order
two. For that, we define the predict step as
\begin{equation}
  {\mathbf h}^l_i = {\mathbf l}^{l-1}_{2i+1} - \frac{1}{2}({\mathbf l}^{l-1}_{2i} + {\mathbf l}^{l-1}_{2i+2})
\end{equation}
and the update step as
\begin{equation}
  {\mathbf l}^l_i = {\mathbf l}^{l-1}_{2i} + \frac{1}{4}({\mathbf h}^l_{i-1} + {\mathbf h}^l_i),
\end{equation}
where the factor $1/4$ is used to preserve the
energy~\cite{sweldens1997building}. This transform is known as the
\href{https://en.wikipedia.org/wiki/Biorthogonal_wavelet}{biorthogonal}
(2,2) of Cohen-Daubechies-Feauveau.  Biorthogonal\footnote{All
transforms express a change of basis. When the basis are not
orthogonal, the synthesis transform is not the transpose of the
analysis transform. When the synthesis filters are orthogonal to their
corresponding \emph{dual} analysis filters, the transform is said
biorthogonal.~\cite{vetterli2014foundations}} filters can be
\href{http://wavelets.pybytes.com/}{easely recognized} because they
are always symmetric. When the filters of the PRFB are biorthogonal,
they also satisfy that $\psi\bot\tilde\phi$ and $\phi\bot\tilde\psi$.

This linear transform is also invertible by simply reversing the steps:
\begin{equation}
  \begin{array}{rcl}
    {\mathbf l}^{l-1}_{2i} & = & {\mathbf l}^l_i - \frac{1}{4}({\mathbf h}^l_{i-1} + {\mathbf h}^l_i)\\
    {\mathbf l}^{l-1}_{2i+1} & = & {\mathbf h}^l_i + \frac{1}{2}({\mathbf l}^{l-1}_{2i} + {\mathbf l}^{l-1}_{2i+2}).
  \end{array}
\end{equation}

%}}}

%}}}

\subsection{Overlapping the DWT}
%{{{

\subsection{\href{https://en.wikipedia.org/wiki/Lapped_transform}{Overlapped block transforms} for minimizing the distortion}
%{{{

Transform coding implies to split the signal into blocks of data
(chunks), and to compute the transform of each chunk. When the output
coefficients are quantized, it is possible that significative (and
unpleasant) distortions may appear in the border frames of the chunks
(see Fig.~\ref{fig:3_chunks}). This is a consequence that in the
prediction step used by the DWT in the limits of the chunks generate
different predictions at the beginning and the end of the adjacent
chunks.

\begin{figure}
  \centering
  \begin{tabular}{cc}
    \svg{3_chunks}{500} & \svg{without}{500} \\
    \svg{extended}{500} & \svg{reconstructed}{500} \\
  \end{tabular}
  \caption{On the top-left, three consecutive chunks of a real mono
    audio sequence. On the top-right, the reconstruction of the chunks
    without overlapping. On the bottom-left, the extended central chunk. On
    the bottom-right, the reconstruction of the extended chunk. See
    this
    \href{https://github.com/Tecnologias-multimedia/intercom/blob/master/docs/quantization_DWT.ipynb}{notebook}.}
  \label{fig:3_chunks}
\end{figure}

One solution to avoid signal discontinuitites between chunks is to
overlap the chunks. Thus, the current ($i$-th) chunk uses also the
last frames of the previous ($(i-1)$-th) chunk and the first frames of
the next ($(i+1)$-th) chunk to compute the transform of the current
extended ($i$-th) chunk (see the Fig.~\ref{fig:subbands}). This has
been described in the following algorithm:

\subsection*{Encoder:}
\begin{enumerate}
\item ${\mathbf C}_{-1}\leftarrow{\mathbf 0}$, a zero chunk.
\item Input ${\mathbf C}_0$.
\item For $i\in\{0,1,\cdots\}$:   
  \begin{enumerate}               
  \item Input ${\mathbf C}_{i+1}$.
  \item Build extended chunk ${\mathbf E}={\mathbf
    C}_{i-1}[-o:]|{\mathbf C}_i|{\mathbf C}_{i+1}[:o]$, where
    $\cdot|\cdot$ denotes the concatenation of chunks, $o$ is the
    overlapped area size in frames, ${\mathbf C}_{i-1}[-o:]$ the last
    $o$ frames of chunk ${\mathbf C}_{i-1}$, and ${\mathbf
      C}_{i+1}[:o]$ are the first $o$ frames of the chunk ${\mathbf
      C}_{i+1}$.
  \item Compute decomposition ${\mathbf D}_i \leftarrow
    \text{DWT}^l({\mathbf E})$, where $l$ is the number of levels of
    the DWT ($l=2$ in the Fig.~\ref{fig:subbands}).
  \item Output decomposition ${\mathbf D}_i$.
  \item ${\mathbf C}_{i-1}\leftarrow {\mathbf C}_i$ (notice that we can assign the pointers, not the contents).
  \item ${\mathbf C}_i\leftarrow {\mathbf C}_{i+1}$.
  \end{enumerate}
\end{enumerate}

Notice that we are following the
\href{https://numpy.org/doc/stable/reference/}{NumPy}~\cite{numpy,harris2020array}
\href{https://www.pythoninformer.com/python-libraries/numpy/index-and-slice/}{slicing}
notation.

%Notice that we are sending for each chunk $2o+x$ coefficients
%(stereo-coefficients, in the case of using 2 chanels), where $x$ is
%the number of frames/chunk.

\subsection*{Decoder:}
\begin{enumerate}
\item For $i\in\{0,1,\cdots\}$:
  \begin{enumerate}
  \item Input decomposition ${\mathbf D}_i$.
  \item Compute extended chunk ${\mathbf E}\leftarrow\text{DWT}^{-l}({\mathbf D}_i)$.
  \item Output chunk ${\mathbf C}_i={\mathbf E}[o:-o]$.
  \end{enumerate}
\end{enumerate}

%\begin{figure}
%  \centering
%  \svg{graphics/overlapping2}{800}
%  \caption{Chunks overlapping. Notice that $a$ data is repeated in the
%    extended chunks $C_0$ and $C_1$. Something similar happens with
%    $b$ data.}
%  \label{fig:overlapping2}
%\end{figure}

\begin{figure}
  \centering
  \svg{graphics/subbands}{550}
  \caption{Structure in the DWT domain of an extended chunk for
    $l=2$. $o$ is the number of overlapped frames between adajacent
    chunks. ${\mathbf C}_{i-1}[-o:]$ represents the last $o$ frames of
    chunk ${\mathbf C}_{i-1}$, and ${\mathbf C}_{i+1}[:o]$ the first
    $o$ frames of the chunk ${\mathbf C}_{i+1}$.}
  \label{fig:subbands}
\end{figure}

This idea has been implemented in this
\href{https://github.com/Tecnologias-multimedia/intercom/blob/master/docs/overlapped_DWT_I.ipynb}{notebook},
and the result can be seen in the Fig.~\ref{fig:3_chunks}.

%}}}

\subsection{Reducing the data overhead}
%{{{

\label{sec:reducing}

Unfortunately, the previous algorithm sends twice the DWT coefficients
of the overlapped areas (in the Fig.~\ref{fig:subbands}, $\{{\mathbf
  D}_i.{\mathbf l}^2[-o/4:], {\mathbf D}_i.{\mathbf l}^2[:o/4],
{\mathbf D}_i.{\mathbf h}^2[-o/4:], {\mathbf D}_i.{\mathbf
  h}^2[:o/4], {\mathbf D}_i.{\mathbf h}^1[-o/2:], {\mathbf
  D}_i.{\mathbf h}^1[:o/2]\}$). To avoid this waste of bandwidth, we
can reuse the received coefficients of the overlapped areas. This
procedure has been described in the Fig.~\ref{fig:overlapping}, and,
as it can be seen, the encoding algorithm is identical to the previous
one except in that only the central (stereo) coefficients are
sent. The rest of coefficients that are needed to compute the inverse
transform are extracted from the neighbor chunks (represented in the
DWT domain). Notice that now, the number of sent coefficients is
$\text{len}({\mathbf C}_i)$, the number of samples in ${\mathbf C}_i$.

\begin{figure}
  \centering
  \svg{graphics/overlapping}{800}
  \caption{Block overlapping in the DWT domain for $l=2$. Only the
    shadded coefficients are transmitted. Notice that, to be
    reconstructed, each chunk depends on some coefficients of the
    adjacent blocks (only some dependencies have been indicated).}
  \label{fig:overlapping}
\end{figure}

%\subsection*{Encoder:}
%\begin{enumerate}
%\item For $i\in\{0,1,\cdots\}$:
%  \begin{enumerate}
%  \item Input $C_i$ and $C_{i+1}$.
%  \item $D_i \leftarrow \text{DWT}(C_i[o:]|C_{i+1}[:o])$, where
%    $C[o:]$ the last $x-o$ frames of the chunk $C$, and
%    $C_i[o:]|C_{i+1}[:o]$ is the $i$-th right-only extended chunk.
%    \item Send $D_i$.
%  \end{enumerate}
%\end{enumerate}

The codec now can be described by:

\subsection*{Encoder:}
\begin{enumerate}
\item ${\mathbf C}_{-1}\leftarrow{\mathbf 0}$, a zero chunk.
\item Input ${\mathbf C}_0$.
\item For $i\in\{0,1,\cdots\}$:   
  \begin{enumerate}               
  \item Input ${\mathbf C}_{i+1}$.
  \item Build extended chunk ${\mathbf E} = {\mathbf
    C}_{i-1}[-o:]|{\mathbf C}_i|{\mathbf C}_{i+1}[:o]$.
  \item Compute decomposition ${\mathbf D}_i \leftarrow
    \text{DWT}^l({\mathbf E})$.
  \item Output decomposition subset
    $\Big\{{\mathbf D}_i.{\mathbf l}^l[\frac{o}{2^l}:-\frac{o}{2^l}], {\mathbf
      D}_i.{\mathbf h}^l[\frac{o}{2^l}:-\frac{o}{2^l}], {\mathbf D}_i.{\mathbf
      h}^{l-1}[\frac{o}{2^{l-1}}:-\frac{o}{2^{l-1}}], \cdots, {\mathbf D}_i.{\mathbf
      h}^1[\frac{o}{2^1}:-\frac{o}{2^1}]\Big\}$.
  \item ${\mathbf C}_{i-1}\leftarrow {\mathbf C}_i$.
  \item ${\mathbf C}_i\leftarrow {\mathbf C}_{i+1}$.
  \end{enumerate}
\end{enumerate}

%\begin{enumerate}
%\item For $i\in\{0,1,\cdots\}$:   
%  \begin{enumerate}               
%  \item Input chunk $C_i$.
%  \item Compute decomposition $D_i \leftarrow \text{DWT}^l(C_i)$.
%  \item Send $D_i$.
%  \end{enumerate}
%\end{enumerate}

\subsection*{Decoder (ignores overlapping):}
This decoder ignores the adjacent chunks in the DWT domain, but notice that it uses the right coefficients (those computed using overlapping chunks). This should provide reconstructions of the chunks with a higher quality that in the previous milestone.
\begin{enumerate}
\item For $i\in\{0,1,\cdots\}$:
  \begin{enumerate}
  \item Input decomposition subset ${\mathbf D}_i$.
  \item Compute chunk ${\mathbf C}_i\leftarrow\text{DWT}^{-l}({\mathbf D}_i)$.
  \item Output ${\mathbf C}_i$.
  \end{enumerate}
\end{enumerate}

\subsection*{Decoder (uses overlapping):}

\begin{enumerate}
\item ${\mathbf D}_{-1}\leftarrow{\mathbf 0}$.
\item Input decomposition ${\mathbf D}_0$.
\item For $i\in\{0,1,\cdots\}$:
  \begin{enumerate}
  \item Input decomposition ${\mathbf D}_{i+1}$.
  \item Build extended decomposition ${\mathbf E}_i =
    {\mathbf D}_{i-1}.{\mathbf l}^l[-\frac{o}{2^l}:]|{\mathbf D}_i.{\mathbf l}^l|{\mathbf D}_{i+1}.{\mathbf l}^l[:\frac{o}{2^l}]|{\mathbf D}_{i-1}.{\mathbf h}^l[-\frac{o}{2^l}:]|{\mathbf D}_i.{\mathbf h}^l|{\mathbf D}_{i+1}.{\mathbf h}^l[:\frac{o}{2^l}]|\cdots|{\mathbf D}_{i-1}.{\mathbf h}^1[-\frac{o}{2^1}:]|{\mathbf D}_i.{\mathbf h}^1|{\mathbf D}_{i+1}.{\mathbf h}^1[:\frac{o}{2^1}]$.
  \item Compute extended chunk ${\mathbf C}_i\leftarrow\text{DWT}^{-l}({\mathbf E}_i)$.
  \item Output ${\mathbf C}_i[o:-o]$.
  \item ${\mathbf D}_{i-1} \leftarrow {\mathbf D}_i$.
  \item ${\mathbf D}_i \leftarrow {\mathbf D}_{i+1}$.
  \end{enumerate}
\end{enumerate}

%Notice that the only that changes between this codec and the previous
%one is that now, the inverse transform of the extended chunks uses the
%last coefficients of the previous extended chunk. This idea has been
%checked in the
%\href{https://github.com/Tecnologias-multimedia/intercom/blob/master/docs/overlapped_DWT_II.ipynb}{notebook}.

%}}}

%}}}

%}}}

\section{What you have to do?}
%{{{

\subsection{Determine the RD curves for the MST}
%{{{
As we did in the previous milestone, generate the RD curves for a set
of simulated transmission contexts. Use the modules
\texttt{stereo\_MST\_coding\{|\_16|\_32\}.py}. As you can see, they
differs in how the transform has been implemented.
%}}}

\subsection{Determine the RD curves for the DWT}
%{{{

Rebuild the RD curves considering also the temporal decorrelation. Use
\verb|temporal_no_overlapped_DWT_coding.py|. Notice that the number of
levels $l$ of the DWT (computed using
\href{https://pywavelets.readthedocs.io/en/latest/}{PyWavelets}~\cite{lee2019pywavelets})
can have a high impact on the amount of energy concentration achieved
by the DWT, and therefore, on the efficiency of coding system. Show
such impact. Experiment also with the wavelet name.

%}}}

\subsection{Determine the RD curves for the overlapped DWT}
%{{{

\begin{enumerate}
\item In a module named \verb|temporal_overlapped_DWT_coding.py|,
  inherit from \verb|temporal_no_overlapped_DWT_coding.py| the class
  \verb|Temporal_No_Overlapped_DWT|, and create a class named
  \verb|Temporal_Overlapped_DWT| which must implement the codec
  described in the Section~\ref{sec:reducing}.
\item Build the RD curves to see how this improvement impacts on the
  efficiency of intercom.
\item Determine which decoder is better from the RD perspective.
\item Which is the latency of the encoder and the decoders, measured
  in chunk-times?
\end{enumerate}

%}}}

%\subsection{Implement a Chunk Truncation Encoding (CTE)}
%{{{

%}}}

\subsection{Visualize}
%{{{

Visualize the effects (in the DWT domain) of quantization.
\begin{verbatim}
qjackctl & # Select sampling frequency 44100 Hz 
python temporal_overlapped_DWT_coding.py -i 6 -o 6 --show_stats -q 8192
# Connect the output of temporal_overlapped_DWT_coding.py to the input of dwt5.py
python dwt5.py -d 9
\end{verbatim}

%}}}

%}}}

\section{Deliverables}
A description of the experiments and their results.

\section{Resources}

\bibliography{python,maths,data-compression,DWT,audio-coding,signal-processing}
