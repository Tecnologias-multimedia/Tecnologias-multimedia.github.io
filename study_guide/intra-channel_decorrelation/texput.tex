\title{\href{https://www.ual.es/estudios/grados/presentacion/plandeestudios/asignatura/4015/40154321?idioma=zh_CN}{Tecnologías Multimedia} - Study Guide - Milestone 11: Inter-channel decorrelation in stereo audio signals (mid/side stereo coding)}

\maketitle

\section{Description}

\href{https://en.wikipedia.org/wiki/Correlation_and_dependence}{Correlation}
is a term used in statistics which refer to the interdependency
between two \href{https://en.wikipedia.org/wiki/Random_variable}{random
  variables}. It can be measured by the
\href{https://www.mathsisfun.com/data/correlation.html}{correlation
  coefficient}~\cite{thinkstats}.

In the case of InterCom, the random variables are the two channels
(left $L$ and right $R$) of the
\href{https://en.wikipedia.org/wiki/Stereophonic_sound}{stereo
  \href{https://en.wikipedia.org/wiki/Pulse-code_modulation}{PCM}
  signal}~\cite{bosi2003intro}. In most cases, both channels are going
to be \href{https://en.wikipedia.org/wiki/Binaural_recording}{highly
  correlated} (especially if the microphone is mono), which means that
we can represent one of them (for example, the $R$ channel) with
respect to the other (the $L$ channel). From a mathematical point of
view, this process can be seen as a
\href{https://en.wikipedia.org/wiki/Decorrelation}{decorrelation}
process. From a physical perspective, decorrelating implies energy
accumulation in a small number of coefficients~\cite{sayood2017introduction}.

To perform this inter-channel decorrelation we can use the transform
\begin{equation}
  w = Kx = \frac{1}{2}\begin{bmatrix} 1 & 1 \\ -1 & 1 \end{bmatrix}x,
  \label{eq:forward_transform_matrix_form}
\end{equation}
that is equivalent to
\begin{equation*}
  \begin{bmatrix}
    w[0] \\
    w[1]
  \end{bmatrix}
  = \frac{1}{2}
  \begin{bmatrix} 1 & 1 \\ -1 & 1 \end{bmatrix}
  \begin{bmatrix}
    x[0] \\
    x[1]
  \end{bmatrix},
\end{equation*}
where $x[0]$ and $x[1]$ are the samples of the stereo frame $x$.

In Eq.~\ref{eq:forward_transform_matrix_form} $w$ represents the
vector of transform coefficients (in our case, a pair of coefficients
$w[0]$ with the
\href{https://en.wikipedia.org/wiki/Arithmetic_mean}{mean} and $w[1]$
with the difference of the samples divided by $2$).

If the transform represented by the Eq.~\ref{eq:forward_transform_matrix_form} can be
also expressed by
\begin{equation}
  w[u] = \sum_i K_{u,i}x[i],
  \label{eq:forward_transform_linear_combination_form}
\end{equation}
where $K_{u,i}$ denotes $i$-th element of the $u$-th row of the matrix
$K$.

If the forward transform is performing the operations
\begin{equation*}
  \begin{array}{rcl}
  w[0] & = & \frac{1}{2}(x[0] + x[1])\\
  w[1] & = & \frac{1}{2}(x[1] - x[0]),
  \end{array}
\end{equation*}
the inverse transform can be found by solving $x[0]$ and $x[1]$ in
these equations, obtaining that
\begin{equation*}
  \begin{array}{rcl}
  x[0] & = & w[0] - w[1]\\
  x[1] & = & w[0] + w[1],
  \end{array}
\end{equation*}
that in matrix form becomes
\begin{equation*}
  \begin{bmatrix}
    x[0] \\
    x[1]
  \end{bmatrix}
  = 
  \begin{bmatrix} 1 & -1 \\ 1 & 1 \end{bmatrix}
  \begin{bmatrix}
    w[0] \\
    w[1]
  \end{bmatrix}.
\end{equation*}

Therefore, the inverse transform can be written as
\begin{equation}
  x[i] = \sum_u K_{i,u}w[u],
  \label{eq:inverse_transform_linear_combination_form}
\end{equation}
and therefore,
\begin{equation}
  x = K^Tw = K^{-1}w = \begin{bmatrix} 1 & -1 \\ 1 & 1 \end{bmatrix}w.
  \label{eq:inverse_transform_matrix_form}
\end{equation}

Notice that this transform ($K$) is not
\href{https://en.wikipedia.org/wiki/Orthonormal_basis}{orthonormal}
(it is not energy preserving in the transform domain, i.e., the
Parseval's Theorem is not satisfied) because
\begin{equation}
  \sum w[i]^2 =
  \left[\frac{1}{2}(x[0]+x[1])\right]^2 + \left[\frac{1}{2}(x[1]+x[0])\right]^2 =
  \frac{1}{4}(x[0]^2+2x[0]x[1]+x[1]^2) + \frac{1}{4}(x[0]^2-2x[0]x[1]+x[1]^2) =
  \frac{1}{2}(x[0]^2+x[1]^2) =
  \frac{1}{2}\sum x[i]^2,
\end{equation}
but is
\href{https://en.wikipedia.org/wiki/Orthogonal_transformation}{orthogonal}\footnote{Orthogonality
of the transform is a important property because the correlation
between the coefficients (the output of the transform) is 0.}
\cite{sayood2017introduction,burrus2013wavelets}, because the inner
(also named dot) product of the basis vectors $K_1=\begin{bmatrix}1 &
1\end{bmatrix}$ and $K_2=\begin{bmatrix} -1 & 1\end{bmatrix}$ is
\begin{equation*}
  \langle K_1,K_2 \rangle =
  \begin{bmatrix}
    1 & 1
  \end{bmatrix}
  \cdot
  \begin{bmatrix}
    -1 & 1
  \end{bmatrix}
  = 0.
\end{equation*}
This means that $K_1$ and $K_2$ are linearly independent\footnote{We
cannot derive one from the other using the operations that define a
vector space.}, and therefore they form a basis
(set)~\cite{strang4linear}. This can be also verified
with\footnote{The inverse matrix $K^{-1}$, which computes the inverse
transform, is the transpose of the forward transform $K$. This
property is satisfied by all orthogonal
transforms~\cite{sayood2017introduction}.}
\begin{equation}
  KK^T=K^TK=2I,
  \label{eq:ortho_matrix_test}
\end{equation}
where $I$ is the identity matrix.

The factor $2$ in the Eq.~\ref{eq:ortho_matrix_test} is indicating
that the transform has a gain of $2$ (the output of the inverse
transform will be equal to the input to the forward transform
multiplied by $2$, which physically impled that we are amplifying the
audio signal by $2$). To avoid this, we can divide each subband
($w[0]$ and $w[1]$) by $2$ after computing the forward transform.

Eq.~\ref{eq:ortho_matrix_test} is showing also that the gain of each
subband is the same: $2$. This fact can be also verified comparing the
energy of the reconstruction provided by each coefficient:

\begin{equation*}
  \begin{array}{l}
    \begin{bmatrix}
      1 & -1 \\
      1 & 1
    \end{bmatrix}
    \begin{bmatrix}
      1 \\
      0
    \end{bmatrix}
    =
    \begin{bmatrix}
      1 & 1
    \end{bmatrix} = K_1
    \\
    \begin{bmatrix}
      1 & -1 \\
      1 & 1
    \end{bmatrix}
    \begin{bmatrix}
      0 \\
      1
    \end{bmatrix}
    =
    \begin{bmatrix}
      -1 & 1
    \end{bmatrix} = K_2
    \\
    \left\| \begin{bmatrix}1 & 1\end{bmatrix} \right\|_2 := \sqrt{\langle \begin{bmatrix}1 & 1\end{bmatrix}, \begin{bmatrix}1 & 1\end{bmatrix} \rangle} = \sqrt{\begin{bmatrix}1 & 1\end{bmatrix}\cdot \begin{bmatrix}1 & 1\end{bmatrix}} = \sqrt{2}
    \left\| \begin{bmatrix}1 & 1\end{bmatrix} \right\|_2 := \sqrt{\langle \begin{bmatrix}-1 & 1\end{bmatrix}, \begin{bmatrix}-1 & 1\end{bmatrix} \rangle} = \sqrt{\begin{bmatrix}-1 & 1\end{bmatrix}\cdot \begin{bmatrix}-1 & 1\end{bmatrix}} = \sqrt{2},
  \end{array}
\end{equation*}
where $\left\|v \right\|_2$ denotes the L$_2$ norm of the vector $v$.

Notice that this technique allows to find the basis vectors ($K_1$ and
$K_2$ in our case) from the inverse tranform.
  

although both ``subbands''\footnote{In a effort of extrapolate this
discussion to a transform case where each subband has more than only
one coefficient, we have called to $w[0]$ the low-pass subband and
$w[1]$ the high-pass subband. However, notice that each subband has
only one coefficient!} $w[0]$ and $w[1]$ have the same gain
($\frac{1}{2}$, and therefore the same ``importance'' for a
future
\href{https://en.wikipedia.org/wiki/Quantization_(signal_processing)}{quantization}
of $w$, i.e., the same quantization step can be used for the $w[0]$
and $w[1]$ coefficients).

The matrix $K$ proposed is similar to the $2\times 2$ KLT
\href{http://fourier.eng.hmc.edu/e161/lectures/klt/node3.html}{(Karhunen-Lo\`eve
  Transform)}, the
\href{http://wavelets.pybytes.com/wavelet/haar/}{Haar
  Transform}~\cite{vetterli1995wavelets} and the $2\times 2$
\href{https://en.wikipedia.org/wiki/Hadamard_transform}{Discrete
  Walsh-Hadamard Transform}~\cite{sayood2017introduction}).  The
described transform is also similar to the so called
\href{https://en.wikipedia.org/wiki/Joint_encoding#M/S_stereo_coding}{M/S
  stereo coding}, but in our case, the división by 2 is carried on the
forward transform, instead of the inverse (backward) transform. This
is done to ensure that the transform can be computed
\href{https://en.wikipedia.org/wiki/In-place_algorithm}{\emph{in-place}}.



Inspired in the
\href{https://cm-bell-labs.github.io/who/wim/papers/athome/athome.pdf}{the
  Lifting Scheme}~\cite{2006.sweldens}, this transform can be
implemented using the following in-place algorithm:

\begin{pseudocode}{Inter-channel\_decorrelation}{~}
  \PROCEDURE{analyze}{\text{frame}}
  \BEGIN
    \text{frame}[1] -= \text{frame}[0] \\
    \text{frame}[0] += (\text{frame}[1] / 2) \\
    \text{frame}[1] /= 2
  \END
  \ENDPROCEDURE
  \PROCEDURE{synthesize}{\text{frame}}
  \BEGIN
    \text{frame}[1] *= 2 \\
    \text{frame}[0] -= (\text{frame}[1] / 2) \\
    \text{frame}[1] += \text{frame}[0]
  \END
  \ENDPROCEDURE
\end{pseudocode}

where $\text{a}~\mathtt{OPER}= \text{b}$ is a shorter representation of the operation
$\text{a} = \text{a}~\mathtt{OPER}~\text{b}$.

\section{What you have to do?}

\begin{enumerate}
\item In a module named stereo.py, inherit the class
  Quantization and create a class named Stereo\_decorrelation.
\item Override the methods pack() and unpack(). In
  pack() perform the procedure analyze() previously
  described, and in unpack() the
  synthesize(). These procedures should be applied to
  all the frames of a chunk using \href{https://www.oreilly.com/library/view/python-for-data/9781449323592/ch04.html}{vectorized
    operations}.
\item Has been the
  \href{https://en.wikipedia.org/wiki/Data_compression_ratio}{compression
    ratio} improved (on
  \href{https://en.wikipedia.org/wiki/Average}{average})? How much?
\end{enumerate}

\section{Timming}

You should reach this milestone at most one week.

\section{Deliverables}

The module stereo.py. Store it at the
\href{https://github.com/Tecnologias-multimedia/intercom}{root
  directory} of your InterCom's repo.

\section{Resources}

\bibliography{maths,data-compression,DWT,audio-coding}
