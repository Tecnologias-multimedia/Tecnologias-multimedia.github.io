\newcommand{\TM}{\href{https://www.ual.es/estudios/grados/presentacion/plandeestudios/asignatura/4015/40154321?idioma=zh_CN}{Tecnologías Multimedia}}

\title{\TM{} - Study Guide - Milestone 6: Impact of the Jitter}

\maketitle

\section{Description}

\href{https://en.wikipedia.org/wiki/Latency_(engineering)#Communication_latency}{Communication
  latency} (also called
\href{https://en.wikipedia.org/wiki/Network_delay}{network delay} and
\href{https://en.wikipedia.org/wiki/End-to-end_delay}{end-to-end
  delay}) is the time that a piece of data (a
\href{https://en.wikipedia.org/wiki/Network_packet}{packet} in the
case of the Internet) takes to travel from one point of the network to
another. This time is relevant for an intercom because the total
latency $t_u$ that an user experiments is
\begin{equation}
  t_u = t_p + t_i,
  \label{eq:user_latency}
\end{equation}
where $t_p$ is the \emph{propagation time} (or propagation latency) of
the \href{https://en.wikipedia.org/wiki/Telecommunications_link}{link}
and $t_i$ is the \emph{latency generated by the intercom}.

Due to the current design of the Internet~\cite{Tanenbaum,Stallings} (where the
available
\href{https://en.wikipedia.org/wiki/Bandwidth_(computing)}{bandwidth}
is shared on demand by the users of the network) $t_l$ is
time-variying, and cannot be controlled without using
\href{https://en.wikipedia.org/wiki/Quality_of_service}{Quality of
  Service (QoS)}, something that is not accesible to normal network
users~\cite{dordal2020intro}. On the contrary, $t_i$ is constant for a
given intercom's configuration/implementation.

In this milestone we are going to measure the
\href{https://en.wikipedia.org/wiki/Quality_of_experience}{Quality of
  Experience (QoE)} provided by our minimal intercom when the network
latency varies \href{https://en.wikipedia.org/wiki/Randomness}{at
  random}. In this point, we have basically two alternatives:
\begin{enumerate}
\item Run two instances of InterCom in two different
  \href{https://en.wikipedia.org/wiki/Host_(network)}{host}s separated
  by a shared link.
\item Run one instance of InterCom and simulate the network latency.
\end{enumerate}
Each option has pros and cons, from which we can highlight that:
\begin{enumerate}
\item \textbf{Using a real link:} (Pro) The use of real latency is
  going to show the definitive behaviour of InterCom between the used
  hosts, which is quite impredictable (depend basically on the
  \href{https://en.wikipedia.org/wiki/Network_congestion}{network
    congestion}). (Con) To run InterCom in two different hosts we will
  need to establish a direct communication between them and it is very
  likely that we will have to redirect ports in the corresponding
  \href{https://en.wikipedia.org/wiki/Network_address_translation}{NAT}
  devices~\cite{srisuresh1999nat}.
\item \textbf{Using a simulated link:} (Pro) The simulation of the
  link latency is much more straightforward than opening ports in
  our routers and (pro) will allow us to run InterCom in situations
  that are difficult to achieve in a real network. (Con) The running
  environment must provide a way of controlling the latency between
  \href{https://en.wikipedia.org/wiki/Process_(computing)}{processes}. Fortunately,
  in Linux we can control the latency (and the
  \href{https://en.wikipedia.org/wiki/Bit_rate}{bit-rate}) of the
  outgoing (packets) traffic using the command
  \href{https://man7.org/linux/man-pages/man8/tc.8.html}{\texttt{tc}}
  \cite{bert2012lartc}.
\end{enumerate}

\section{What you have to do?}

\subsection{Characterize the latency in different scenarios}

\subsubsection{In your host}

In most cases we will test InterCom in our host. Therefore, it can be
useful to have an idea of how the latencies are distributed, at least
from a statistical point of view.

To mesasure latencies, we will use
\href{https://github.com/torvalds/linux/blob/master/net/ipv4/ping.c}{\texttt{ping}}~\cite{Kurose-Ross,Forouzan},
a tool that
\href{https://en.wikipedia.org/wiki/Ping_(networking_utility)}{sends}
(one or more)
\href{https://en.wikipedia.org/wiki/Internet_Control_Message_Protocol}{ICMP}
Echo Request messages to an IP address and waits for receiving (one or
more) ICMP Echo Reply messages generated by the
(\href{https://en.wikipedia.org/wiki/Operating_system}{OS} of that)
host, measuring the so called
\href{https://en.wikipedia.org/wiki/Round-trip_delay}{RTT} (Round-Trip
Time). For example, in the Figure~\ref{fig:ping_timeline} are
described the different time components in which a RTT can be
decomposed. $t_t$ stands for \emph{transmission time}, and $t_p$
(again) for \emph{propagation time}. A simple link (a cable, for
example) using
\href{https://en.wikipedia.org/wiki/Time-division_multiple_access}{TDM}
(Time-Domain Multiplexing) has been supposed. For that reason, the
propagation and transmission times are identical in both
directions. Notice that if the payload of the \texttt{ping} message
has only 64 bytes (the default value in most \texttt{ping}
implementations) and the bit-rate of the link is high,
then $$t_p>>t_t.$$ For this type of link, it also holds that
\begin{equation}
  \text{RTT} = 2t_p + 2t_t.
  \label{eq:RTT}
\end{equation}
  
\begin{figure}
  \begin{center}
    \myfig{graphics/ping_timeline}{5cm}{500}
  \end{center}
  \caption{Timeline of a ping interaction between two host A and B
    interconnected by simple communication link.}
  \label{fig:ping_timeline}
\end{figure}

Please, do the following steps:

\begin{enumerate}
\item Ping \texttt{localhost}:
   \begin{lstlisting}{language=bash}
ping localhost -c 100 -s <payload_length_in_bytes> > /tmp/ping.dat
  \end{lstlisting}
  Use a payload size compared to the chunk size that you expect to use
  in your InterCom experiments.
\item Compute the expected (considering that the RTT should
  double the) latency to the host of your interlocutor:\\\\
  \texttt{export LC\_NUMERIC=en\_US.UTF-8 \# Use "." instead of "," for the decimal separator}\\
\texttt{grep from < /tmp/ping.dat | cut -f 4 -d "=" | cut -f 1 -d " " | awk
  \textquotesingle\{print \$1/2\}\textquotesingle~> /tmp/localhost\_latencies.dat}\\\\

\item Find the histogram of the expected latencies:
  
  \begin{lstlisting}{language=bash}
cat << EOF | python -
import numpy as np
from scipy import stats
latencies = np.loadtxt("/tmp/localhost_latencies.dat")
average_latency = np.average(latencies)
print("average latency =", average_latency)
max_latency = np.max(latencies)
min_latency = np.min(latencies)
maximum_absolute_deviation = max(max_latency - average_latency, average_latency - min_latency)
print("maximum absolute deviation (jitter) =", maximum_absolute_deviation)
correlation_coefficient = stats.pearsonr(latencies, np.roll(latencies, 1))[0]
print("Pearson correlation coefficient =", correlation_coefficient)
if correlation_coefficient < 0:
  print("Correlation coefficient < 0: use 0 (no correlation between RTT samples) in your experiments")
histogram = np.histogram(latencies)
np.savetxt("/tmp/localhost_histogram.dat", histogram[0])
EOF
  \end{lstlisting}

\item Plot the histogram:
  \begin{lstlisting}{language=bash}
gnuplot
plot "/tmp/localhost_histogram.dat" with histogram
  \end{lstlisting}
  
\item Characterize statistically the latency: which statistical
  distribution is more close to your experimental data?
\end{enumerate}

\subsubsection{In the Internet}

This scenario can be useful to test InterCom in your host but
simulating a real connection between hosts in different home
networks. For doing that:

\begin{enumerate}
  
\item Repeat the previous experiment (the characterization of the
  latencies returned by the \texttt{ping} tool) but using your
  interlocutor's \texttt{<router\_public\_IP\_address>} instead of
  \texttt{localhost}. Call the generated file as
  \texttt{/tmp/<router\_public\_IP\_address>\_latencies.dat}. The IPv4
  address of your router can be determined with:
  
  \begin{lstlisting}{language=bash}
curl ipecho.net/plain
  \end{lstlisting}  
  
\item Request to your interlocutor to ping its router from his/her
  private network, for example, using\footnote{Notice that the private
    IP addres of your router could be different.}:
  
  \begin{lstlisting}{language=bash}
ping -s <chunk_length_in_bytes> 192.168.1.1
  \end{lstlisting}
  
  and to send this data to you. Save this info in
  \texttt{/tmp/<router\_private\_IP\_address>\_latencies.dat} and
  characterize it.

\item Supposing that the latencies are symmetric (the direction of the
  packes does not affect to the latency) and that the overall network
  latency of the link between you a your interlocutor is the sum of
  the latency from your host to the router of your interlocutor added to
  the latency from your interlocutor's host to that router, find a
  characterization for the full link:
  \href{https://en.wikipedia.org/wiki/Average}{average} (arithmetic
  mean) latency, \href{https://en.wikipedia.org/wiki/Jitter}{jitter},
  \href{https://en.wikipedia.org/wiki/Pearson_correlation_coefficient}{Pearson
    correlation coefficient}, and
  \href{https://en.wikipedia.org/wiki/List_of_probability_distributions}{probability
    distribution}.

\end{enumerate}

\subsection{Quantification of the QoE}

Let's measure the QoE using the following classification:
\begin{itemize}
\item \textbf{Perfect}: no loss or delay can be distinguish.
\item \textbf{Good}: if you detect some minimal distortion in the
  rendering of the sound.
\item \textbf{Acceptable}: when the effects of the latency are
  apreciable, but you can communicate with your interlocutor.
\item \textbf{Bad}: you are able to recognize only small parts of the
  received audio.
\item \textbf{No way}: when most of the time only silence is heard.
\end{itemize}

\subsubsection{In your host}

You don't need to control the network traffic in this scenario because
it is already shapped when InterCom uses the loopback network
device. Therefore, simply quantify your QoE when you run InterCom in
your host.

\subsubsection{In the Internet}

\begin{enumerate}

\item Check the current configuration:
  
  \begin{lstlisting}{language=bash}
tc qdisc show dev lo
  \end{lstlisting}
  
  The output should be something like:
  
  \begin{lstlisting}{language=bash}
qdisc noqueue 0: root refcnt 2
  \end{lstlisting}
  
\item Using the characterization of the Internet link previously
  obtained, use the command
  \href{https://man7.org/linux/man-pages/man8/tc.8.html}{\texttt{tc}}
  to simulate this link locally using
  \href{https://man7.org/linux/man-pages/man8/tc-netem.8.html}{netem}:

  \begin{lstlisting}{language=bash}
sudo tc qdisc add dev lo root netem delay <average_delay_in_miliseconds>ms <maximum_average_deviation_in_miliseconds>ms <Pearson_correlation_coefficient_expressed_as_a_percentage>% distribution <uniform|normal|pareto|paretonormal>
  \end{lstlisting}
  where:
  \begin{description}
  \item [\texttt{qdisc}:] Use the default
    \href{https://en.wikipedia.org/wiki/FIFO_(computing_and_electronics)}{FIFO}
    \href{https://wiki.debian.org/TrafficControl}{Queueing DISCipline}
    for the outgoing traffic.
  \item [\texttt{add}:] Add a new traffic control rule.
  \item [\texttt{dev lo}:] The device affected by the
    rule. \texttt{lo} means \texttt{loopback}.
  \item [\texttt{root}:] The rule will be applied to all the outbound
    traffic (it's the root rule of the possible tree of rules).
  \item [\texttt{netem}:] Use the
    \href{https://wiki.linuxfoundation.org/networking/netem}{network
      emulator} to emulate a WAN property.
  \end{description}

  Example:

  \begin{enumerate}
  \item Add the rule:
    
    \begin{lstlisting}{language=bash}
sudo tc qdisc add dev lo root netem delay 100ms 10ms 25% distribution normal
    \end{lstlisting}
    
  \item Check that the rule has been installed with the command:
    
    \begin{lstlisting}{language=bash}
tc qdisc show dev lo
    \end{lstlisting}
    
    that should output:
    
    \begin{lstlisting}{language=bash}
qdisc netem 8009: root refcnt 2 limit 1000 delay 100ms  10ms 25%
    \end{lstlisting}
  \end{enumerate}

\item Measure the QoE.

\item Delete the \texttt{tc} rule with:
  
  \begin{lstlisting}{language=bash}
sudo tc qdisc delete dev lo root netem delay <average_dalay_in_miliseconds>ms <maximum_average_deviation_in_miliseconds>ms <Pearson_correlation_coefficient_expressed_as_a_percentage>% distribution <uniform|normal|pareto|paretonormal>
  \end{lstlisting}

\item (Optional) It's possible to change a working rule with:

  \begin{lstlisting}{language=bash}
sudo tc qdisc change dev lo root netem delay <average_dalay_in_miliseconds>ms <maximum_average_deviation_in_miliseconds>ms <Pearson_correlation_coefficient_expressed_as_a_percentage>% distribution <uniform|normal|pareto|paretonormal>
  \end{lstlisting}
  
\end{enumerate}

\subsection{(Optional) QoE considering the packet loss}

For our application, InterCom, a chunk is lost when it arrives too
late or it never arrives. Therefore, the results of a packet loss or a
packet delay are almost indistinguishable, except by the average
latency experimented by the user (the higher the network latency, the
higher the perceived latency).

For example, a packet loss ratio of $10\%$ can be configured with
\texttt{tc} by running:

  \begin{lstlisting}{language=bash}
sudo tc qdisc add dev lo root netem loss 10%
  \end{lstlisting}

\section{Timming}

Please, finish this milestone in one week.

\section{Deliverables}

A report showing your results.

\section{Resources}

\bibliography{networking,nat}
