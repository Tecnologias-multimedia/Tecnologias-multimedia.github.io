\title{Considering the Threshold of Hearing}

\maketitle

\section{Description}

\subsection{A model of the Threshold of Human Hearing}

Psychoacoustics (see
\href{https://vicente-gonzalez-ruiz.github.io/the_sound/}{the sound},
\href{https://vicente-gonzalez-ruiz.github.io/human_auditory_system/}{the
  human auditory system}, and
\href{https://vicente-gonzalez-ruiz.github.io/human_sound_perception/}{the
  human sound perception}) has determined that the HAS (Human Auditory
System) has a sensitivity that depends on the frequency of the sound,
the so called THH
(\href{https://en.wikipedia.org/wiki/Absolute_threshold_of_hearing}{Threshold
  of Human Hearing}). This basically means that some subbands can be
quantized with a larger quantization step than others without a
noticeable increase (from a perfection perspective) of the
quantization noise.

\begin{figure}
  \centering
  \svg{graphics/ToHH}{600}
  \caption{A model for the threshold of human hearing.}
  \label{fig:ToHH}
\end{figure}

A good approximation of the THH (Threshold of Human Hearing) for a 20-years old person can be
obtained with~\cite{bosi2003intro}
\begin{equation}
  T(f)\text{[dB]} = 3.64(f\text{[kHz]})^{-0.8} - 6.5e^{f\text{[kHz]}-3.3)^2} + 10^{-3}(f\text{[kHz]})^4.
  \label{eq:ToHH}
\end{equation}
This equation has been plotted in the Fig.~\ref{fig:ToHH}.

\subsection{DWT subbands and quantization steps}
The number of DWT subbands
\begin{equation}
  N_{\text{sb}} = N_{\text{levels}} + 1
\end{equation}
where $N_{\text{levels}}$ is the number of levels of the DWT. Except
for the ${\mathbf l}^{N_{\text{levels}}}$ subband (the lowest-pass
frequency of the decomposition), it holds that
\begin{equation}
  W({\mathbf h}^s) = \frac{1}{2}W({\mathbf h}^{s-1}),
\end{equation}
being $W(\cdot)$ the bandwidth of the corresponding
subband $s$. Therefore, considering that the bandwidth of the audio signal
is $22050$ Hz, the bandwidth $W({\mathbf h}^1)$ of the ${\mathbf h}^1$ subband is $11025$ Hz,
$W({\mathbf h} ^2)=22025/4$, and so on. It also holds that
\begin{equation}
  W({\mathbf l}^{N_{\text{levels}}}) = W({\mathbf h}^{N_{\text{levels}}}).
\end{equation}

The idea is to determine, knowing the frequencies represented in each
DWT subband and the THH curve, the quantization step that should be
applied to each subband. This idea has been implemented in the module
threshold.py.

\section{What you have to do?}

\subsection{Determining your threshold of hearing}

The threshold of hearing plotted in the Fig.~\ref{fig:ToHH} can be
different to the curve which corresponds with your current ``hearing
capabilities''.\footnote{For example, your speakers could not have a
  flat frequency response, or your room could attenuate more, some
  freqencies.} For this reason, in the module threshold.py, implement
a procedure for determining your threshold curve. The idea is to
generate a sequence of pure tones localized at different
frequencies. Each tone should start with an amplitude of zero, and
progressively (linearly) increase the amplitude until the user hit a
key when he start listening the tone. After this procedure, you should
be able to plot your threshold curve similar to the depicted in the
Fig.~\ref{fig:ToHH}. Finally, notice that such threshold curve should
be used by your interlocutor, and you should use your intercolutor's
threshold curve. Please, modify threshold.py to implement such
extra\footnote{This fine tunning of the threshold of hearing should be
  optional when you run InterCom.} functionality.

\subsection{Subjective performance}

\begin{enumerate}
\item Using a recording tool such as
  \href{http://audacity.sourceforge.net}{Audacity} or
  \href{http://plugin.org.uk/timemachine/}{JACK Timemachine}, record
  the simulated transmission of a piece of audio and create a
  \texttt{.wav} file, when the audio has been transmitted using
  \texttt{temporal\_overlapped\_DWT\_coding.py} and
  \texttt{threshold.py}, using in both cases the same transmission
  bit-rate. Use the quantization step for controlling the bit-rate.
\item Determine which audio sounds better, from a subjective point of
  view. Repeat this step the number of times you consider necessary.
\end{enumerate}

\section{Deliverables}

The module threshold.py and a report of how your proposal works,
including a subjective performance comparison.

\section{Resources}

\bibliography{maths,data-compression,DWT,audio-coding}

