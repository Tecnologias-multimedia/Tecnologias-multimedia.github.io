\newcommand{\TM}{\href{https://www.ual.es/estudios/grados/presentacion/plandeestudios/asignatura/4015/40154321?idioma=zh_CN}{Tecnologías Multimedia}}

\title{\TM{} - Study Guide - Milestone 2: Installation and basic programming with Python}

\maketitle

\section{Description}

The InterCom project \cite{intercom} are a collection of
\href{https://www.python.org/}{Python}
\href{https://docs.python.org/3/tutorial/modules.html#modules}{modules}
written in Python \cite{python}, that use a number of
\href{https://docs.python.org/3/tutorial/modules.html#packages}{packages}. Therefore,
you will need a
\href{https://docs.python.org/3/tutorial/interpreter.html}{Python
  interpreter} and know how to develop/run Python code.

Most of the current Unix-based operating systems (Linux, FreeBSD and
OSX) use Python for running some of their ``daily tasks'', which means
that a Python interpreter is already available. However, usually it is
better to use our own interpreter because:

\begin{enumerate}

\item We can chose the version of Python and have more flexibility
  with the packages.

\item We can optimize the compilation of the interpreter depending on
  our needs (for example, including
  \href{https://wiki.python.org/moin/TkInter}{Tk support} or not).

\item By default, all the Python packages will be installed in a
  different repository of the system packages, which eases the
  system/user Python-isolation and the removal of the interpreter.

\end{enumerate}

In Windows you need to install Python, yes or yes, from the official
\href{https://www.python.org/downloads/}{website}. However, notice
that this ``guide'' only contemplates the installation of Python in
Unix-based OS machines, and specifically, in a Xubuntu computer.

As said before, it is very likely that your Unix-like OS has Python
installed. However, in order to control the version of the Python
interpreter, we will install a dedicated one.

\section{What you have to do?}

\begin{enumerate}
  
\item Installation of Python.
  
  \begin{enumerate}
    
  \item Go to
    \href{https://github.com/vicente-gonzalez-ruiz/YAPT/blob/master/01-hello_world/02-installation.ipynb}{YAPT/01-hello\_world/02-installation.ipynb}
    \cite{YAPT} and follow the instructions to install the lastest
    stable version
    \href{https://en.wikipedia.org/wiki/CPython}{CPython}, and create
    a new virtual environment called \texttt{tm} with
    \href{https://github.com/pyenv/pyenv}{\texttt{pyenv}}. Basically
    (example for \verb|Python 3.9.7|):

    \begin{enumerate}
    \item Install package dependencies for compiling Python:
      
      \begin{lstlisting}[mathescape=false, language=Bash]
sudo apt-get update; sudo apt-get install make build-essential libssl-dev zlib1g-dev libbz2-dev libreadline-dev libsqlite3-dev wget curl llvm libncursesw5-dev xz-utils tk-dev libxml2-dev libxmlsec1-dev libffi-dev liblzma-dev
\end{lstlisting}

\item Download Pyenv:
  \begin{lstlisting}[mathescape=false, language=Bash]
git clone https://github.com/pyenv/pyenv.git ~/.pyenv
\end{lstlisting}
        
\item (Optional) Compile a dynamic Bash extension to speed up Pyenv.
  \begin{lstlisting}[mathescape=false, language=Bash]
cd ~/.pyenv && src/configure && make -C src; cd ~
\end{lstlisting}

\item Define the \verb|PYENV_ROOT| environment variable to point to
  the path where you cloned the Pyenv repo. For this, put these lines
  into \verb|~/.profile|\footnote{This configuration file is sourced
    by the \texttt{bash} interperter only when a login shell is done
    (when the system requests your identity and you are able to
    privide it.} before the part that sources
  \verb|~/.bashrc|\footnote{This configuration file is sourced by the
    \texttt{bash} interpreter when you request a non-login shell, i.e.,
    when you reinstance the interpreter, for example, running a new
    terminal.}.
        
  \begin{lstlisting}[mathescape=false, language=Bash]
export PYENV_ROOT="$HOME/.pyenv"
export PATH="$PYENV_ROOT/bin:$PATH"
\end{lstlisting}

\item Run:
  \begin{lstlisting}[mathescape=false, language=Bash]
echo 'eval "$(pyenv init -)"' >> ~/.bashrc
echo 'eval "$(pyenv init --path)"' >> ~/.bashrc
\end{lstlisting}
to put these lines at the bottom of \verb|~/.bashrc| to enable
autocompletion and all subcommands.

\item Restart your login session for the changes to take
  effect. E.g. if you're in a GUI session, you need to fully log out
  and log back in.

\item List the available Python interpreters:
  \begin{lstlisting}[mathescape=false, language=Bash]
pyenv install --list
\end{lstlisting}

\item Install the Python interpreter (and some basic tools such as \href{https://pypi.org/project/pip/}{pip}):
  \begin{lstlisting}[mathescape=false, language=Bash]
pyenv install -v 3.9.7
\end{lstlisting}

\item Check what it is currently available:
  \begin{lstlisting}[mathescape=false, language=Bash]
pyenv versions        
\end{lstlisting}
You should get something such as:
\begin{lstlisting}[mathescape=false, language=Bash]
* system (set by /home/<your_home_sir_here>/.pyenv/version)
  3.9.7
\end{lstlisting}

\item Select the Python interperter:
  \begin{lstlisting}[mathescape=false, language=Bash]
pyenv global 3.9.7
\end{lstlisting}

\end{enumerate}
    
  \item Install an
    \href{https://en.wikipedia.org/wiki/Integrated_development_environment}{IDE}
    for programming with Python. I recommend
    \href{https://thonny.org/}{Thonny} if you are not used to any
    other.
    
    \begin{lstlisting}[language=Bash]
pip install thonny
    \end{lstlisting}

  \end{enumerate}

\item Alternatively (but reducing the chances of solving any possible
  issue), you can use the Python interpreter shipped with your OS. In
  this case, it is strongly recommended to use an specific Python
  \href{https://docs.python.org/3/library/venv.html}{environment} for
  the InterCom project.
  
\item Python programming.
  
  \begin{enumerate}
    
  \item You don't need to master Python to follow this course, but it
    is convenient for you to follow some Python programming tutorial,
    such as \href{https://docs.python.org/3/tutorial/}{The Python
    Tutorial} \cite{python-tutorial} if you realize that the language
    is a setback for you. If you need to start with Python from
    scratch, an introduction to Python such as
    this \href{https://github.com/vicente-gonzalez-ruiz/YAPT/tree/master/workshops/programacion_python_ESO}{workshop
    of YAPT} \cite{YAPT} could also be helpful. See
    also \href{http://zetcode.com/lang/python/}{ZetCode's Python
    Tutorial}.
    
  \end{enumerate}

\end{enumerate}

\section{Timming}

There is not time limit for finishing this milestone. Develop it at
your own pace. However, notice that we will be using Python very soon.

\section{Deliverables}

None.

\section{Resources}

\bibliography{python,intercom}
