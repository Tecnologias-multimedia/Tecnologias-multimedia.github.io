\newcommand{\TM}{\href{https://www.ual.es/estudios/grados/presentacion/plandeestudios/asignatura/4015/40154321?idioma=zh_CN}{Tecnologías Multimedia}}

\title{\TM - Study Guide - Milestone 7: Impact of the Bit-rate}

\maketitle

\section{Description}

Along with latency and its variation (jitter), another main aspect to
consider about the transmission link used in an InterCom session is
the transmission bit-rate (bit per second) that it can provide. This
bit-rate depend on the maximum capacity (a characteristic closely
related with the available bandwidth) and its congestion level (that
basically depends on the load) of the link. In general, we can suppose
that the capacity is constant along the time (the bandwidth provided
by the link does not vary with the time). On the contrary, the
congestion level is time-variying and quite unpredictable.

In this milestone we are going to measure the impact of the variation
of the transmission bit-rate on the QoE. Similarly to the procedure
used for measuring the impact of the jitter, we are
going to use
\href{https://man7.org/linux/man-pages/man8/tc.8.html}{\testtt{tc}} to
bound the amount of data that a instance of InterCom will be allowed
to send in a local environment. Notice that this limitation will also
affect to the loss of chunks because if the transmission bit-rate is
smaller than the audio bit-rate, sooner of later the link will discard
those chunk that can not be buffered in the retranmission nodes
(routers and switches).

\section{What you have to do?}

\subsection{Estimate the bit-rate in an Internet link}

Usually, we need to use a tool such as \href{https://iperf.fr/}{iPerf}
to measure the transmission bit-rate between two end-points in the
Internet. This proceduce implies that we must have access to both
hosts to install and run this program, privileges that are not always
allowed.

Alternatively, we can estimate such bandwidth using
\href{https://github.com/torvalds/linux/blob/master/net/ipv4/ping.c}{\texttt{ping}}. This
tool
\href{https://en.wikipedia.org/wiki/Ping_(networking_utility)}{sends}
(one or more)
\href{https://en.wikipedia.org/wiki/Internet_Control_Message_Protocol}{ICMP}
Echo Request messages

\subsection{Quantification of the QoE without packed loss}

Let's measure the QoE using the following classification:
\begin{itemize}
\item Perfect: no loss or delay can be distinguish.
\item Good: if you detect some minimal distortion in the rendering
  of the sound.
\item Acceptable: when the effects of the latency are apreciable, but
  you can communicate with your interlocutor.
\item Bad: you are able to recognize only small parts of the
  received audio.
\item No way: when most of the time only silence is heard.
\end{itemize}

\subsubsection{In your host}

You don't need to control the network traffic in this scenario because
it is already shapped when InterCom uses the loopback network
device. Therefore, simply quantify your QoE when you run InterCom in
your host.

\subsubsection{In the Internet}

\begin{enumerate}
  
\item Using the characterization of the Internet link previously
  obtained, use the command
  \href{https://man7.org/linux/man-pages/man8/tc.8.html}{\texttt{tc}}
  to simulate this link locally using
  \href{https://man7.org/linux/man-pages/man8/tc-netem.8.html}{netem}:

  \begin{lstlisting}{language=bash}
    sudo tc qdisc add dev lo root netem delay <average_dalay_in_miliseconds> <maximum_average_deviation_in_miliseconds> <Pearson_correlation_coefficient_expressed_as_a_percentage> distribution <uniform|normal|pareto|paretonormal>
  \end{lstlisting}
  where:
  \begin{description}
  \item [\texttt{qdisc}:] Use the default
    \href{https://en.wikipedia.org/wiki/FIFO_(computing_and_electronics)}{FIFO}
    \href{https://wiki.debian.org/TrafficControl}{Queueing DISCipline}
    for the outgoing traffic.
  \item [\texttt{add}:] Add a new traffic control rule.
  \item [\texttt{dev lo}:] The device affected by the
    rule. \texttt{lo} means \texttt{loopback}.
  \item [\texttt{root}:] The rule will be applied to all the outbound
    traffic (it's the root rule of the possible tree of rules).
  \item [\texttt{netem}:] Use the
    \href{https://wiki.linuxfoundation.org/networking/netem}{network
      emulator} to emulate a WAN property.
  \end{description}

\item Measure the QoE.

\item Remove the \texttt{tc} rule with.
  
  \begin{lstlisting}{language=bash}
    sudo tc qdisc delete dev lo root netem delay <average_dalay_in_miliseconds> <maximum_average_deviation_in_miliseconds> <Pearson_correlation_coefficient_expressed_as_a_percentage> distribution <uniform|normal|pareto|paretonormal>
  \end{lstlisting}

\item (Optional) You can see the current rules with:

  \begin{lstlisting}{language=bash}
    tc qdisc show
  \end{lstlisting}

\item (Optional) It's possible to change a working rule with:

  \begin{lstlisting}{language=bash}
    sudo tc qdisc change dev lo root netem delay <average_dalay_in_miliseconds> <maximum_average_deviation_in_miliseconds> <Pearson_correlation_coefficient_expressed_as_a_percentage> distribution <uniform|normal|pareto|paretonormal>
  \end{lstlisting}
  
\end{enumerate}

\subsection{(Optional) QoE considering the packet loss}

For our application, InterCom, a chunk is lost when it arrives too
late or it never arrives. Therefore, the results of a packet loss or a
packet delay are almost indistinguishable, except by the average
latency experimented by the user (the higher the network latency, the
higher the perceived latency).

For example, the packet loss ratio of $10\%$ can be controlled with
\texttt{tc} by running:

  \begin{lstlisting}{language=bash}
    sudo tc qdisc add dev lo root netem loss 10%
  \end{lstlisting}

\section{Timming}

Please, finish this milestone in one week.

\section{Deliverables}

A report showing your results.

\section{Resources}

\bibliography{python,intercom}
