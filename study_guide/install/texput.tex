\title{\href{https://www.ual.es/estudios/grados/presentacion/plandeestudios/asignatura/4015/40154321?idioma=zh_CN}{Tecnologías Multimedia} - Study Guide - Milestone 2: Installation and basic programming with Python}

\maketitle

\section{Description}

The InterCom project \cite{intercom} are a collection of
\href{https://docs.python.org/3/tutorial/modules.html#}{Python
  modules} written in Python \cite{Python}. Therefore, you will need
an interpreter and know how to develop/run Python programs
(\href{https://docs.python.org/3/tutorial/modules.html#modules}{modules}
and
\href{https://docs.python.org/3/tutorial/modules.html#packages}{packages}).

Most of the current Unix-based operating systems (Linux, FreeBSD and
OSX) use Python for running some of their ``daily tasks'', which means
that a Python interpreter is already available. However, usually it is
better to use our own interpreter because:

\begin{enumerate}

\item We can chose the version of Python and the packages.

\item We can optimize the compilation of the interpreter depending on
  our needs (for example, including
  \href{https://wiki.python.org/moin/TkInter}{Tk support} or not).

\item By default, all the Python packages will be installed in a
  different repository of the system packages, which eases the
  system/user Python-isolation and the removal of the interpreter.

\end{enumerate}

In Windows you need to install Python, yes or yes, from the official
\href{https://www.python.org/downloads/}{website}. However, notice
that this ``guide'' only contemplates the installation of Python in
Unix-based OS machines.

\section{What you have to do?}

\begin{enumerate}
  
\item Installation of Python.
  
  \begin{enumerate}
    
  \item Go to
    \href{https://github.com/vicente-gonzalez-ruiz/YAPT/blob/master/01-hello_world/02-installation.ipynb}{YAPT/01-hello\_world/02-installation.ipynb}
    \cite{YAPT} and follow the instructions to install CPython 3.8.5,
    and create a new virtual environment called \texttt{tm}. Basically:

\begin{verbatim}
sudo apt-get install -y build-essential libssl-dev zlib1g-dev libbz2-dev libreadline-dev libsqlite3-dev wget curl llvm libncurses5-dev libncursesw5-dev xz-utils tk-dev libffi-dev liblzma-dev python-openssl git
curl https://pyenv.run | bash
cat << EOF >> ~/.bashrh
export PATH="$HOME/.pyenv/bin:$PATH"
eval "$(pyenv init -)"
EOF
source ~/.bashrc
pyenv install -v 3.8.5
pyenv virtualenv 3.8.5 tm
\end{verbatim}

  \item Remember that you will need to active it when you want to
    work in this project:

    \begin{lstlisting}[language=Bash]
      pyenv activate tm
    \end{lstlisting}

    It is a good idea to append this to the \verb|~/.bashrc| file.
    
  \item Install an
    \href{https://en.wikipedia.org/wiki/Integrated_development_environment}{IDE}
    for programming with Python. I recommend
    \href{https://thonny.org/}{Thonny} if you are not used to any
    other.
    
    \begin{lstlisting}[language=Bash]
      pip install thonny
    \end{lstlisting}

  \end{enumerate}
  
\item Python programming.
  
  \begin{enumerate}
    
  \item You don't need to master Python to follow this course, but it
    is convenient for you to follow some Python programming tutorial,
    such as \href{https://docs.python.org/3/tutorial/}{The Python
    Tutorial} \cite{python-tutorial} if you realize that the language
    is a setback for you. If you need to start with Python from
    scratch, an introduction to Python such as
    this \href{https://github.com/vicente-gonzalez-ruiz/YAPT/tree/master/workshops/programacion_python_ESO}{workshop
    of YAPT} \cite{YAPT} could also be helpful. See
    also \href{http://zetcode.com/lang/python/}{ZetCode's Python
    Tutorial}.
    
  \end{enumerate}

\end{enumerate}

\section{Timming}

There is not time limit for finishing this milestone. Develop it at
your own pace.

\section{Deliverables}

None.

\section{Resources}

\bibliography{python,intercom}
