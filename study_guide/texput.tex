\title{Tecnologías Multimedia - Study Guide}

\maketitle

\href{https://www.ual.es/estudios/grados/presentacion/plandeestudios/asignatura/4015/40154321}{TM
  (Tecnologías Multimedia)} is an optative subject of the Computer
Science Degree at the UAL (University of Almería).

\section{Methodology}
TM follows the PBL (Project-Based Learning) methodology. The students,
helped by the teacher, develop a project during the classes. This
project
is \href{https://github.com/Tecnologias-multimedia/intercom}{InterCom}(municator),
an application that allows networked users to communicate throught
the Internet.

The project is implemented as a sequence of milestones. The students,
in groups of up to 4 people, propose solutions for each
milestone. Such solutions are presented to the rest of the class, and
each group gives and receives a score and feedback from the rest of
the class and also from the teacher.

\section{Milestones}

\begin{enumerate}
\item \href{https://tecnologias-multimedia.github.io/study_guide/framework/}{Framework}.
\item \href{https://tecnologias-multimedia.github.io/study_guide/minimal/}{Meeting \texttt{minimal} InterCom}.
\item \href{https://tecnologias-multimedia.github.io/study_guide/latency/}{Hidding the network latency}.
\item \href{https://tecnologias-multimedia.github.io/study_guide/BR_control/}{Bit-rate control}.
\item \href{https://tecnologias-multimedia.github.io/study_guide/transform_coding/}{Transform coding for removing redundancy}.
\item \href{https://tecnologias-multimedia.github.io/study_guide/perceptual_coding/}{Removing psychoacoustical redundancy}.
\item \href{https://tecnologias-multimedia.github.io/study_guide/psychoacoustics/}{Removing psychoacoustical redundancy}.
\end{enumerate}

Week 6: Removing psychoacousting redundancy.
\begin{enumerate}
\item \href{https://tecnologias-multimedia.github.io/study_guide/14-threshold_of_hearing/}{Considering the threshold of hearing}.
\item \href{https://tecnologias-multimedia.github.io/study_guide/15-simultaneous_masking/}{(TODO) Considering the simultaneous masking effect}.
\end{enumerate}

\section{Resources}

\bibliography{data-compression,signal-processing,DWT,linux,python,git,text-compression,maths,image-compression,JPEG2000,intercom,sound,networking,NAT,audio-coding}
