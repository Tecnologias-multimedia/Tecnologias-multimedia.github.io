\title{Tecnologías Multimedia - Study Guide}

\maketitle

\section{Methodology}
TM (Tecnologías Multimedia) follows the PBL (Project-Based Learning) methodology. The students,
helped by the teacher, develop a project during the course. This
project
is \href{https://github.com/Tecnologias-multimedia/intercom}{InterCom}(municator),
an application that allows us to communicate networked users throught
the Internet.

The project is implemented as a sequence of milestones. The students,
in groups of up to 4 people, propose solutions for each
milestone. Such solutions are presented to the rest of the class by
each group, and each group gives and receives a score and feedback
from the rest of the class and also from the teacher.

\section{Milestones}
\begin{enumerate}
\item \href{https://tecnologias-multimedia.github.io/study_guide/milestone00/}{OS (Operating System) Provisioning}.
\item \href{https://tecnologias-multimedia.github.io/study_guide/milestone01/}{Installation and basic programming with Python}.
\item \href{https://tecnologias-multimedia.github.io/study_guide/milestone02/}{Git, GitHub and the Fork-and-Branch Git Workflow}.
\item \href{https://tecnologias-multimedia.github.io/study_guide/milestone03/}{`Wiring'' the ADC with the DAC and measuring latencies}.
%\item \href{https://tecnologias-multimedia.github.io/study_guide/milestone03/}{Transmiting data through the Internet}.
%\item \href{https://tecnologias-multimedia.github.io/study_guide/milestone04/}{Transmiting data through the Internet}.
%\item \href{https://tecnologias-multimedia.github.io/study_guide/milestone05/}{A minimal InterCom}.
\end{enumerate}

