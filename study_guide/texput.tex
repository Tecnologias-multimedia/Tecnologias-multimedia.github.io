\title{Tecnologías Multimedia - Study Guide}

\maketitle

\href{https://www.ual.es/estudios/grados/presentacion/plandeestudios/asignatura/4015/40154321}{TM
  (Tecnologías Multimedia)} is an optative subject of the Computer
Science Degree at the UAL (University of Almería).

\section{Methodology}
TM follows the PBL (Project-Based Learning) methodology. The students,
helped by the teacher, develop a project during the classes. This
project
is \href{https://github.com/Tecnologias-multimedia/intercom}{InterCom}(municator),
an application that allows networked users to communicate throught
the Internet.

The project is implemented as a sequence of milestones. The students,
in groups of up to 4 people, propose solutions for each
milestone. Such solutions are presented to the rest of the class, and
each group gives and receives a score and feedback from the rest of
the class and also from the teacher.

\section{Framework}


\section{Meeting \texttt{minimal} InterCom}

InterCom is an application that captures and plays audio, and therefore, in Linux, runs on the top of one of the following audio services:
\begin{enumerate}
\item \href{https://vicente-gonzalez-ruiz.github.io/ALSA/}{ALSA}.
\item
  \href{https://vicente-gonzalez-ruiz.github.io/PulseAudio/}{PulseAudio}
  (in the case of Xubuntu, this is the audio server that comes with
  Xfce).
\item \href{https://vicente-gonzalez-ruiz.github.io/JACK/}{JACK}.
\end{enumerate}

InterCom uses
\href{https://vicente-gonzalez-ruiz.github.io/intro_to_sounddevice/}{sounddevice}
to capture and play the audio. Using sounddevice we have two alternatives for implementing an intercom.

\subsection{Loop-based algorithm}

Roughtly, InterCom can be divided in 6 steps:

\begin{lstlisting}[language=Python]
  # Loop-based algorithm (to be called in a loop)
  def record_IO_and_play():
    chunk = record()  # (1)
    packed_chunk = pack(chunk)  # (2)
    send(packed_chunk)  # (3)
    packed_chunk = receive()  # (4)
    chunk = unpack(packed_chunk)  # (5)
    play(chunk)  # (6)
\end{lstlisting}

%\begin{pseudocode}{Sequential\_InterCom}{~}
%  \PROCEDURE{Record\_IO\_and\_Play}{~}
%  \BEGIN
%    \mathtt{chunk} \GETS \mathtt{record()}\\
%    \mathtt{packed\_chunk} \GETS \mathtt{pack(chunk)}\\
%    \mathtt{send(packed\_chunk)}\\
%    \mathtt{packed\_chunk} \GETS \mathtt{receive()}\\
%    \mathtt{chunk} \GETS \mathtt{unpack(packed\_chunk)}\\
%    \mathtt{play(chunk)}
%  \END
%  \ENDPROCEDURE
%\end{pseudocode}

Where:

\begin{enumerate}
\item the \verb|record()| method captures a chunk of frames. In
  \verb|sounddevice|, this operation is carried on by the
  \href{https://python-sounddevice.readthedocs.io/en/0.4.0/api/streams.html#sounddevice.Stream.read}{\texttt{read()}
    method}. As it can be seen in
  \href{https://raw.githubusercontent.com/Tecnologias-multimedia/intercom/master/test/sounddevice/wire4.py}{wire4.py}\footnote{
  \texttt{curl
    https://raw.githubusercontent.com/Tecnologias-multimedia/intercom/master/test/sounddevice/wire4.py
    > wire4.py}} and also in the documentation of
  \verb|sounddevice|, if we read only the frames that are available
  in the ADC's buffer, this is a non-blocking operation and the chunk
  size depends on the instant of time in which this method is
  called. Otherwise, if we especify a number of frames different to
  the number of available frames, the operation is blocking and
  \href{https://en.wikipedia.org/wiki/I/O_bound}{I/O-bound} (the
  calling process sleeps until the required chunk size is returned).

\item \verb|pack(chunk)| process the chunk to create a
  \href{https://en.wikipedia.org/wiki/Network_packet}{packet} (or a
  sequence of packets), a structure that can be transmitted through
  the Internet using the
  \href{https://en.wikipedia.org/wiki/Datagram}{Datagram} Model. In
  general, this is a
  \href{https://en.wikipedia.org/wiki/CPU-bound}{CPU-bounded}
  (CPU-intensive) operation, and therefore, it reduces the number of
  executions/second that the \verb|record_IO_and_play()| method can
  reach.

\item \verb|send(packed_chunk)|, sends the packet to the
  InterCom's interlocutor. When datagrams are used, this step is
  not blocking neither CPU-bounding (the CPU usage is very low), as
  long as the number of packets/second is small and the sizes of the
  payloads are also small, as it is expected in InterCom.

\item \verb|receive()|, waits (blocking) for an incoming packet, and
  therefore, this operation is IO-bound. However, most
  \href{https://docs.python.org/3/library/socket.html}{socket
    API}s~\cite{python} offeer a
  \href{https://docs.python.org/3.8/library/socket.html#socket.socket.setblocking}{non-blocking
    option} where when a packet is not available in the kernel's
  buffer associated to the corresponding socket, some kind of
  exception is generated and, in this case, it is resposabability of
  the programmer to generate an ``alternative'' chunk (in our case,
  for example, a chunk filled with zeros that will not produce any
  sound when it is played).

\item \verb|unpack(packed_chunk)| is (like the method
  \texttt{pack(chunk)}) a CPU-intensive step that transforms a
  packed chunk into a chunk of audio.

\item \verb|play(chunk)| renders the chunk. In general, this is an
  I/O-bound
  \href{https://python-sounddevice.readthedocs.io/en/0.4.0/api/streams.html#sounddevice.Stream.write}{blocking}
  action. However, if \verb|play()| is called at the same pace
  than \verb|record()|, and the record and play parameters are
  exactely the same (as happens in our algorithm), the playing of the
  chunk will return inmediately because the time that the
  \verb|play()| method needs to complete would exactly match the
  time that the \verb|record()| method requires (see
  \verb|wire4.py|).
\end{enumerate}

This algorithm works fine if the chunk size is controlled by
\href{http://www.portaudio.com/}{PortAudio}~\cite{portaudio}
(\verb|sounddevice|) and also, the run-time required by
\verb|pack(chunk)| and \verb|unpack(packet)| operations is smaller
than the chunk-time. The first premise (that we can use varying chunk
sizes) complicates slightly the implementation because we would work
with chunks of constantly changing lenghts (that as you can see
running \verb|wire4.py|, in most of the iterations are 0). This also
complicates the control of the latency because the chunk size is
variable. However, the real problem appears when our computer is not
able to satisfy the second requirement, i.e., when the chunk time is
smaller than the time that we need to process the chunks. This only
can be addressed through optimizing (for example, parallelizing) the
code.

\subsection{Timer-based algorithm}

In this algorithm, the task dedicated to record and play the chunks of
audio is called periodically (probably, using some timer provided by
the sound hardware). This procedure guarantees a gliches-free audio-IO
when constant chunk sizes are used because the timer interruption
coincides exactly with the instant of time in which the
\verb|record()| and the \verb|play()| methods we are going to
handle a chunk of audio (of a known size without blocking). The
following algorithm describes the new algorithm that is basically the
previous one, except that the chunk size is fixed.

\begin{lstlisting}[language=Python]
  # Timer-based algorithm (to be called periodically)
  def record_IO_and_play(chunk_size):
    chunk = record(chunk_size)
    packed_chunk = pack(chunk)
    send(packed_chunk)
    packed_chunk = receive()
    chunk = unpack(packed_chunk)
    play(chunk)
\end{lstlisting}

%\begin{pseudocode}{Timer-based\_InterCom}{~}
%  \PROCEDURE{Record\_IO\_and\_Play}{\mathtt{chunk\_size}}
%  \BEGIN
%    \mathtt{chunk} \GETS \mathtt{record(chunk\_size)}\\
%    \mathtt{packed\_chunk} \GETS \mathtt{pack(chunk)}\\
%    \mathtt{send(packed\_chunk)}\\
%    \mathtt{packed\_chunk} \GETS \mathtt{receive()}\\
%    \mathtt{chunk} \GETS \mathtt{unpack(packed\_chunk)}\\
%    \mathtt{play(chunk)}
%  \END
%  \ENDPROCEDURE
%\end{pseudocode}

The current implementation of InterCom uses the Timer-based algorithm.

\section{Hidding the network latency}

\section{Incorporating a bit-rate control}

\section{Removing spatial and temporal redundancy}

\section{Removing psychoacoustical redundancy}

\item Week 2: The simplest InterCom.
\begin{enumerate}
\item \href{https://tecnologias-multimedia.github.io/study_guide/Linux_audio/}{Linux audio basics and \texttt{sounddevice}}.
\item \href{https://tecnologias-multimedia.github.io/study_guide/04-wiring/}{`Wiring'' the ADC with the DAC and measuring latencies}.
\item \href{https://tecnologias-multimedia.github.io/study_guide/05-minimal/}{Minimal InterCom}.
\end{enumerate}

\begin{enumerate}


\item Week 3: Dealing with transmission latencies and bit-rates.
\begin{enumerate}
\item \href{https://tecnologias-multimedia.github.io/study_guide/06-jitter_impact/}{Impact of the jitter}.
\item \href{https://tecnologias-multimedia.github.io/study_guide/07-bit-rate_impact/}{Impact of the bit-rate}.
\item \href{https://tecnologias-multimedia.github.io/study_guide/08-buffer/}{Buffering}.
\end{enumerate}
\item Week 4: Compressing and quantizing.
\begin{enumerate}
\item \href{https://tecnologias-multimedia.github.io/study_guide/09-compress/}{Compressing the audio data with zlib}.
\item \href{https://tecnologias-multimedia.github.io/study_guide/10-br_control/}{Bit-rate control through quantization}.
\end{enumerate}
\item Week 5: Removing temporal and spatial redundancy with transform coding.
\begin{enumerate}
\item \href{https://tecnologias-multimedia.github.io/study_guide/11-stereo_coding/}{Stereo coding of audio signals}.
\item \href{https://tecnologias-multimedia.github.io/study_guide/12-temporal_coding/}{Temporal coding of audio signals}.
\item \href{https://tecnologias-multimedia.github.io/study_guide/13-overlapped_temporal_coding/}{Overlapped DWT}.
\end{enumerate}
\item Week 6: Removing psychoacousting redundancy.
\begin{enumerate}
\item \href{https://tecnologias-multimedia.github.io/study_guide/14-threshold_of_hearing/}{Considering the threshold of hearing}.
\item \href{https://tecnologias-multimedia.github.io/study_guide/15-simultaneous_masking/}{(TODO) Considering the simultaneous masking effect}.
\end{enumerate}
\end{enumerate}

\section{Resources}

\bibliography{intercom,python,linux}
