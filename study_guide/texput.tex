\title{Tecnologías Multimedia - Study Guide}

\maketitle

\href{https://www.ual.es/estudios/grados/presentacion/plandeestudios/asignatura/4015/40154321}{TM
  (Tecnologías Multimedia)} is an optative subject of the Computer
Science Degree at the UAL (University of Almería).

\section{Methodology}
TM follows the PBL (Project-Based Learning) methodology. The students,
helped by the teacher, develop a project during the classes. This
project
is \href{https://github.com/Tecnologias-multimedia/intercom}{InterCom}(municator),
an application that allows networked users to communicate throught
the Internet.

The project is implemented as a sequence of milestones. The students,
in groups of up to 4 people, propose solutions for each
milestone. Such solutions are presented to the rest of the class, and
each group gives and receives a score and feedback from the rest of
the class and also from the teacher.

\section{Preparing the framework}

The InterCom project \cite{intercom} is a collection of
\href{https://www.python.org/}{Python}
\href{https://docs.python.org/3/tutorial/modules.html#modules}{modules}~\cite{python}.
Although Python has been ported to
\href{https://www.python.org/download/other/}{almost all} the current
OSs, including mobile \href{https://kivy.org/#home}{devices}, we will
develop the project on Linux, and more concretely, on Xubuntu 21.10
(Impish Indri) \cite{xubuntu}, running natively (no
\href{https://en.wikipedia.org/wiki/Virtualization}{virtualization}),
if possible. To receive technical support in a reasonable amount of
time in the case you are in trouble, we will use an especific Linux
distribution: Xubuntu, which is fully functional (at least for
developing our project) and demand a low amount of hardware
resources. Please, follow for this,
\href{https://vicente-gonzalez-ruiz.github.io/Xubuntu_install/}{to
  install Xubuntu}.

You don't need to master Python to follow this course, but it is
convenient for you to rely on some Python programming tutorial, such
as \href{https://docs.python.org/3/tutorial/}{The Python Tutorial}
\cite{python-tutorial} if you realize that the language is a setback
for you. If you need to start with Python from scratch, an
introduction to Python such as this
\href{https://github.com/vicente-gonzalez-ruiz/YAPT/tree/master/workshops/programacion_python_ESO}{workshop
  of YAPT} \cite{YAPT} could also be helpful. See also
\href{http://zetcode.com/lang/python/}{ZetCode's Python
  Tutorial}. Follow this to
\href{https://vicente-gonzalez-ruiz.github.io/Python_install/}{install
  Python}. Alternatively (but reducing the chances of solving any
possible issue), you can use the Python interpreter shipped with your
OS. In this case, it is strongly recommended to use an specific Python
\href{https://docs.python.org/3/library/venv.html}{environment} for
the InterCom project.

Finally, to work in the
\href{https://github.com/Tecnologias-multimedia/intercom}{InterCom}
project \cite{intercom} you must understand the basics of Git
\cite{Git-book} and \href{https://github.com/}{GitHub}\footnote{There
are other Git-based hosting services such as
\href{https://about.gitlab.com/}{GitLab} and
\href{https://www.atlassian.com/git}{Altassian}/\href{https://bitbucket.org/product}{BitBucket},
but GitHub is the most used one.} \cite{GitHub}, and how to use the
\href{https://guides.github.com/introduction/flow/index.html}{The
  GitHub (Work-)Flow} and the
\href{https://github.com/vicente-gonzalez-ruiz/fork_and_branch_git_workflow}{Fork-and-Branch
  Git Workflow} \cite{fork-and-branch-git-workflow}. Be aware that to
contribute to InterCom, an GitHub account is required. Please, follow
this \href{}{minimal Git guide}.


\section{A simple InterCom}

\begin{enumerate}
\item \href{https://tecnologias-multimedia.github.io/study_guide/03-git/}{Git, GitHub and the Fork-and-Branch Git Workflow}.
\end{enumerate}

\begin{enumerate}


\item Week 2: The simplest InterCom.
\begin{enumerate}
\item \href{https://tecnologias-multimedia.github.io/study_guide/Linux_audio/}{Linux audio basics and \texttt{sounddevice}}.
\item \href{https://tecnologias-multimedia.github.io/study_guide/04-wiring/}{`Wiring'' the ADC with the DAC and measuring latencies}.
\item \href{https://tecnologias-multimedia.github.io/study_guide/05-minimal/}{Minimal InterCom}.
\end{enumerate}
\item Week 3: Dealing with transmission latencies and bit-rates.
\begin{enumerate}
\item \href{https://tecnologias-multimedia.github.io/study_guide/06-jitter_impact/}{Impact of the jitter}.
\item \href{https://tecnologias-multimedia.github.io/study_guide/07-bit-rate_impact/}{Impact of the bit-rate}.
\item \href{https://tecnologias-multimedia.github.io/study_guide/08-buffer/}{Buffering}.
\end{enumerate}
\item Week 4: Compressing and quantizing.
\begin{enumerate}
\item \href{https://tecnologias-multimedia.github.io/study_guide/09-compress/}{Compressing the audio data with zlib}.
\item \href{https://tecnologias-multimedia.github.io/study_guide/10-br_control/}{Bit-rate control through quantization}.
\end{enumerate}
\item Week 5: Removing temporal and spatial redundancy with transform coding.
\begin{enumerate}
\item \href{https://tecnologias-multimedia.github.io/study_guide/11-stereo_coding/}{Stereo coding of audio signals}.
\item \href{https://tecnologias-multimedia.github.io/study_guide/12-temporal_coding/}{Temporal coding of audio signals}.
\item \href{https://tecnologias-multimedia.github.io/study_guide/13-overlapped_temporal_coding/}{Overlapped DWT}.
\end{enumerate}
\item Week 6: Removing psychoacousting redundancy.
\begin{enumerate}
\item \href{https://tecnologias-multimedia.github.io/study_guide/14-threshold_of_hearing/}{Considering the threshold of hearing}.
\item \href{https://tecnologias-multimedia.github.io/study_guide/15-simultaneous_masking/}{(TODO) Considering the simultaneous masking effect}.
\end{enumerate}
\end{enumerate}

\section{Resources}

\bibliography{intercom,python,linux}
