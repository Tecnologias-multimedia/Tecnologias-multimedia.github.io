\title{Tecnologías Multimedia - Study Guide}

\maketitle

\href{https://www.ual.es/estudios/grados/presentacion/plandeestudios/asignatura/4015/40154321}{TM
  (Tecnologías Multimedia)} is an optative subject of the Computer
Science Degree at the UAL (University of Almería).

\section{Methodology}
TM follows the PBL (Project-Based Learning) methodology. The students,
helped by the teacher, develop a project during the classes. This
project
is \href{https://github.com/Tecnologias-multimedia/intercom}{InterCom}(municator),
an application that allows networked users to communicate throught
the Internet.

The project is implemented as a sequence of milestones. The students,
in groups of up to 4 people, propose solutions for each
milestone. Such solutions are presented to the rest of the class, and
each group gives and receives a score and feedback from the rest of
the class and also from the teacher.

\section{Milestones}

\begin{enumerate}
\item \href{https://tecnologias-multimedia.github.io/study_guide/framework/}{Framework}.
\item \href{https://tecnologias-multimedia.github.io/study_guide/minimal/}{Meeting \texttt{minimal} InterCom}.
\item \href{https://tecnologias-multimedia.github.io/study_guide/latency/}{Hidding the network latency}
\end{enumerate}

\section{Incorporating a bit-rate control}

\section{Removing spatial and temporal redundancy}

\section{Removing psychoacoustical redundancy}

\item Week 2: The simplest InterCom.
\begin{enumerate}
\item \href{https://tecnologias-multimedia.github.io/study_guide/Linux_audio/}{Linux audio basics and \texttt{sounddevice}}.
\item \href{https://tecnologias-multimedia.github.io/study_guide/04-wiring/}{`Wiring'' the ADC with the DAC and measuring latencies}.
\item \href{https://tecnologias-multimedia.github.io/study_guide/05-minimal/}{Minimal InterCom}.
\end{enumerate}

\begin{enumerate}


\item Week 3: Dealing with transmission latencies and bit-rates.
\begin{enumerate}
\item \href{https://tecnologias-multimedia.github.io/study_guide/06-jitter_impact/}{Impact of the jitter}.
\item \href{https://tecnologias-multimedia.github.io/study_guide/07-bit-rate_impact/}{Impact of the bit-rate}.
\item \href{https://tecnologias-multimedia.github.io/study_guide/08-buffer/}{Buffering}.
\end{enumerate}
\item Week 4: Compressing and quantizing.
\begin{enumerate}
\item \href{https://tecnologias-multimedia.github.io/study_guide/09-compress/}{Compressing the audio data with zlib}.
\item \href{https://tecnologias-multimedia.github.io/study_guide/10-br_control/}{Bit-rate control through quantization}.
\end{enumerate}
\item Week 5: Removing temporal and spatial redundancy with transform coding.
\begin{enumerate}
\item \href{https://tecnologias-multimedia.github.io/study_guide/11-stereo_coding/}{Stereo coding of audio signals}.
\item \href{https://tecnologias-multimedia.github.io/study_guide/12-temporal_coding/}{Temporal coding of audio signals}.
\item \href{https://tecnologias-multimedia.github.io/study_guide/13-overlapped_temporal_coding/}{Overlapped DWT}.
\end{enumerate}
\item Week 6: Removing psychoacousting redundancy.
\begin{enumerate}
\item \href{https://tecnologias-multimedia.github.io/study_guide/14-threshold_of_hearing/}{Considering the threshold of hearing}.
\item \href{https://tecnologias-multimedia.github.io/study_guide/15-simultaneous_masking/}{(TODO) Considering the simultaneous masking effect}.
\end{enumerate}
\end{enumerate}

\section{Resources}

\bibliography{intercom,python,linux}
