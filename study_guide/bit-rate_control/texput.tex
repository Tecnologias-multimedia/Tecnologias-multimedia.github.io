\newcommand{\TM}{\href{https://tecnologias-multimedia.github.io/}{Tecnologías Multimedia}}

\title{\TM - Study Guide - Milestone 9: Bit-Rate Control}

\maketitle

\section{Description}

Along with latency and its variation (jitter), the second main aspect
to consider about the transmission link is its end-to-end data
\href{https://en.wikipedia.org/wiki/Bandwidth_(computing)}{transmission
  capacity}. This parameter defines the bit-rate (measured in bits per
second) that the link can sustainably support. This capacity is
directly related with the
\href{https://en.wikipedia.org/wiki/Network_congestion}{network
  congestion}.

Due to the
\href{https://en.wikipedia.org/wiki/Network_congestion}{network
  congestion} which is variable and in most of the cases
unpredictable, the transmission link has a time-variying
capacity. Basically, this means that we can, at most, estimate the
very near future congestion level.



On the other hand, UDP does not implement any network congestion
control policy. When congestion appers, the throughput of the network
decreases dramatically


\section{What you have to do?}

\subsection{Characterize the latency in different scenarios}

\subsubsection{In your host}

In most cases we will test InterCom in your host. Therefore, it can be
useful to have an idea of how the latencies are distributed, at least
from a statistical point of view.

\begin{enumerate}
\item Ping \texttt{localhost}:
   \begin{lstlisting}{language=bash}
    ping localhost -c 100 > /tmp/ping.dat
  \end{lstlisting}
Compute the expected (considering that the
  \href{https://en.wikipedia.org/wiki/Round-trip_delay}{RTT} should
  double the) latency to the host of your interlocutor:\\
\texttt{~~~~grep from < /tmp/ping.dat | cut -f 4 -d "=" | cut -f 1 -d " " | awk
  \textquotesingle\{print \$1/2\}\textquotesingle > /tmp/localhost\_latencies.dat}

\lstset{literate={\$}{{\textcolor{blue}{\$}}}1}
\begin{lstlisting}[language=tex, mathescape]
  grep from < /tmp/ping.dat | cut -f 4 -d "=" | cut -f 1 -d " " | awk '{print $\mbox{\textdollar}$1/2}' > /tmp/localhost\_latencies.dat
\end{lstlisting}

\item Find the histogram of the expected latencies:
  \begin{lstlisting}{language=python}
    cat << EOF | python -
    import numpy as np
    from scipy import stats
    latencies = np.loadtxt("/tmp/localhost_latencies.dat")
    average_latency = np.average(latencies)
    print("average latency =", average_latency)
    max_latency = np.max(latencies)
    min_latency = np.min(latencies)
    maximum_absolute_deviation = max(max_latency - average_latency, average_latency - min_latency)
    print("maximum absolute deviation (jitter)=", maximum_absolute_deviation)
    print("Pearson correlation coefficient =", stats.pearsonr(latencies, np.roll(latencies, 1)[0])
    histogram = np.histogram(latencies)
    np.savetxt("/tmp/localhost_histogram.dat", histogram[0])
    EOF
  \end{lstlisting}

\item Plot the histogram:
  \begin{lstlisting}{language=bash}
    gnuplot plot "/tmp/localhost_histogram.dat" with histogram
  \end{lstlisting}
  
\item Characterize statistically the latency: which statistical
  distribution is more close to your experimental data?
\end{enumerate}

\subsubsection{In the Internet}

This scenario can be useful to test InterCom in your host but
simulating a real connection between hosts in different home
networks. For doing that:

\begin{enumerate}
  
\item Repeat the previous experiment (the characterization of the
  latencies returned by the \texttt{ping} tool) but using your
  interlocutor's \texttt{<router\_public\_IP\_address>} instead of
  \texttt{localhost}. Call the generated file as
  \texttt{/tmp/<router\_public\_IP\_address>.dat}.
  
\item Request to your interlocutor to ping its router from his/her private
  network, and to send this data to you. Save this info in
  \texttt{/tmp/<router\_private\_IP\_address>.dat} and characterize it.

\item Supposing that the latencies are symmetric (the direction of the
  packes does not affect to the latency) and that the overall network
  latency of the link between you a your interlocutor is the sum of
  the latency from your host to the router of your interlocutor and
  the latency from your interlocutor's host to that router, find a
  characterization for the full link:
  \href{https://en.wikipedia.org/wiki/Average}{average} (arithmetic
  mean) latency, \href{https://en.wikipedia.org/wiki/Jitter}{jitter},
  \href{https://en.wikipedia.org/wiki/Pearson_correlation_coefficient}{Pearson
    correlation coefficient}, and
  \href{https://en.wikipedia.org/wiki/List_of_probability_distributions}{probability
    distribution}.

\end{enumerate}

\subsection{Quantification of the QoE without packed loss}

Let's measure the QoE using the following classification:
\begin{itemize}
\item Perfect: no loss or delay can be distinguish.
\item Good: if you detect some minimal distortion in the rendering
  of the sound.
\item Acceptable: when the effects of the latency are apreciable, but
  you can communicate with your interlocutor.
\item Bad: you are able to recognize only small parts of the
  received audio.
\item No way: when most of the time only silence is heard.
\end{itemize}

\subsubsection{In your host}

You don't need to control the network traffic in this scenario because
it is already shapped when InterCom uses the loopback network
device. Therefore, simply quantify your QoE when you run InterCom in
your host.

\subsubsection{In the Internet}

\begin{enumerate}
  
\item Using the characterization of the Internet link previously
  obtained, use the command
  \href{https://man7.org/linux/man-pages/man8/tc.8.html}{\texttt{tc}}
  to simulate this link locally using
  \href{https://man7.org/linux/man-pages/man8/tc-netem.8.html}{netem}:

  \begin{lstlisting}{language=bash}
    sudo tc qdisc add dev lo root netem delay <average_dalay_in_miliseconds> <maximum_average_deviation_in_miliseconds> <Pearson_correlation_coefficient_expressed_as_a_percentage> distribution <uniform|normal|pareto|paretonormal>
  \end{lstlisting}
  where:
  \begin{description}
  \item [\texttt{qdisc}:] Use the default
    \href{https://en.wikipedia.org/wiki/FIFO_(computing_and_electronics)}{FIFO}
    \href{https://wiki.debian.org/TrafficControl}{Queueing DISCipline}
    for the outgoing traffic.
  \item [\texttt{add}:] Add a new traffic control rule.
  \item [\texttt{dev lo}:] The device affected by the
    rule. \texttt{lo} means \texttt{loopback}.
  \item [\texttt{root}:] The rule will be applied to all the outbound
    traffic (it's the root rule of the possible tree of rules).
  \item [\texttt{netem}:] Use the
    \href{https://wiki.linuxfoundation.org/networking/netem}{network
      emulator} to emulate a WAN property.
  \end{description}

\item Measure the QoE.

\item Remove the \texttt{tc} rule with.
  
  \begin{lstlisting}{language=bash}
    sudo tc qdisc delete dev lo root netem delay <average_dalay_in_miliseconds> <maximum_average_deviation_in_miliseconds> <Pearson_correlation_coefficient_expressed_as_a_percentage> distribution <uniform|normal|pareto|paretonormal>
  \end{lstlisting}

\item (Optional) You can see the current rules with:

  \begin{lstlisting}{language=bash}
    tc qdisc show
  \end{lstlisting}

\item (Optional) It's possible to change a working rule with:

  \begin{lstlisting}{language=bash}
    sudo tc qdisc change dev lo root netem delay <average_dalay_in_miliseconds> <maximum_average_deviation_in_miliseconds> <Pearson_correlation_coefficient_expressed_as_a_percentage> distribution <uniform|normal|pareto|paretonormal>
  \end{lstlisting}
  
\end{enumerate}

\subsection{(Optional) QoE considering the packet loss}

For our application, InterCom, a chunk is lost when it arrives too
late or it never arrives. Therefore, the results of a packet loss or a
packet delay are almost indistinguishable, except by the average
latency experimented by the user (the higher the network latency, the
higher the perceived latency).

For example, the packet loss ratio of $10\%$ can be controlled with
\texttt{tc} by running:

  \begin{lstlisting}{language=bash}
    sudo tc qdisc add dev lo root netem loss 10%
  \end{lstlisting}

\section{Timming}

Please, finish this milestone in one week.

\section{Deliverables}

A report showing your results.

\section{Resources}

\bibliography{python,intercom}
