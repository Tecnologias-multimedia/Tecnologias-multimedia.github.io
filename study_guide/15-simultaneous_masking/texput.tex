\newcommand{\TM}{\href{https://tecnologias-multimedia.github.io/}{Tecnologías Multimedia}}

\title{\TM{} - Study Guide - Milestone 15: Considering the simultaneous masking}

\maketitle

\section{Description}

\subsection{Simultaneous masking}
The HAS (Human Auditory System) has a finite frequency resolution,
which basically means that two different tonal sounds with different
amplitudes can be heared only as one (the loud one) when they are
closed enought~\cite{bosi2003intro}. Whe this happens, the DWT
subband~\cite{vetterli1995wavelets} where the quiet sound is placed
can be quantized more severly without perceiving that the quantization
noise in such subband is higher (see the Figure~\ref{fig:SM}). Notice
that, the higher the quantization step, the higher the compression
ratio.

\begin{figure}
  \centering
  \png{simultaneous_masking}{600}
  \caption{Simultaneous masking.}
  \label{fig:SM}
\end{figure}


\section{What you have to do?}

\begin{enumerate}
\item Write a generator of vectors of quantization steps. The
  generator should analyze the energy of each DWT subband and decide,
  considering the simultaneous masking effect, the vector of
  quantization steps (one for subband)..
\end{enumerate}

\section{Timming}

This is a extra milestone. Take your time.

\section{Deliverables}

The module \verb|simultaneous_masking.py| (inheriting from
\verb|threshold.py|). Store it at the
\href{https://github.com/Tecnologias-multimedia/intercom}{root
  directory} of your InterCom's repo.

\section{Resources}

\bibliography{maths,data-compression,DWT,audio-coding}

