\title{\TM - Study Guide - Milestone 5: Impact of the Communication Latency}

\maketitle

\section{Description}

\href{https://en.wikipedia.org/wiki/Latency_(engineering)#Communication_latency}{Communication
  latency} (also called
\href{https://en.wikipedia.org/wiki/Network_delay}{network delay} and
\href{https://en.wikipedia.org/wiki/End-to-end_delay}{end-to-end
  delay}) is the time that a piece of data (a
\href{https://en.wikipedia.org/wiki/Network_packet}{packed} in the
case of the Internet) takes to travel from one point of the network to
another. This time is relevant for an intercom because the total latency $t_u$ that an user is going to experiment is
\begin{equation}
  t_u = t_l + t_i,
  \label{eq:user_latency}
\end{equation}
where $t_l$ is the (tele-communication)
\href{https://en.wikipedia.org/wiki/Telecommunications_link}{link}
latency and $t_i$ is the latency generated by the intercom.

Due to the current design of the Internet (where the available
\href{https://en.wikipedia.org/wiki/Bandwidth_(computing)}{bandwidth}
is shared on demand by the users of the network) $t_l$ is
time-variying, and cannot be controlled without using
\href{https://en.wikipedia.org/wiki/Quality_of_service}{Quality of
  Service (QoS)}, something that is not accesible to normal network
users. On the contrary, $t_i$ is constant for a given intercom's
configuration/implementation.

In this milestone we are going to measure the
\href{https://en.wikipedia.org/wiki/Quality_of_experience}{Quality of
  Experience (QoE)} provided by our minimal intercom when the network
latency varies \href{https://en.wikipedia.org/wiki/Randomness}{at
  random}. At this point, we have basically two alternatives:
\begin{enumerate}
\item Run two instances of InterCom in two different hosts separated
  by a shared link.
\item Run one instance of Intercom and simulate the network latency.
\end{enumerate}
Each option has pros and cons, from which we can highlight that:
\begin{enumerate}
\item (Pro) The use of real latency is going to show the definitive
  behaviour of InterCom between the used hosts, which is quite
  impredictable (depend basically of the
  \href{https://en.wikipedia.org/wiki/Network_congestion}{network
    congestion}). (Con) To run InterCom in two different hosts we will
  need to establish a direct communication between them and it is very
  likely that we will have to redirect ports in the corresponding
  \href{https://en.wikipedia.org/wiki/Network_address_translation}{NAT}
  devices~\cite{}.
\item (Pro) The simulation of the link latency is much more
  straightforward than the opening ports in our routers and (pro) will
  allow us to run InterCom in situations that are difficult to achieve
  in a real network. (Con) The running environment must provide a way
  of controlling the latency between
  \href{https://en.wikipedia.org/wiki/Process_(computing)}{processes}. Fortunately,
  in Linux we can control the latency (and the
  \href{https://en.wikipedia.org/wiki/Bit_rate}{bit-rate}) of the
  outgoing (packets) traffic using
  \href{https://man7.org/linux/man-pages/man8/tc.8.html}{$\mathtt{tc}$}.
\end{enumerate}



The InterCom project \cite{intercom} are a collection of
\href{https://docs.python.org/3/tutorial/modules.html#}{Python
  modules} written in Python \cite{Python}. Therefore, you will need
an interpreter and know how to develop/run Python programs
(\href{https://docs.python.org/3/tutorial/modules.html#modules}{modules}
and
\href{https://docs.python.org/3/tutorial/modules.html#packages}{packages}).

Most of the current Unix-based operating systems (Linux, FreeBSD and
OSX) use Python for running some of their ``daily tasks'', which means
that a Python interpreter is already available. However, usually it is
better to use our own interpreter because:

\begin{enumerate}

\item We can chose the version of Python and the packages.

\item We can optimize the compilation of the interpreter depending on
  our needs (for example, including
  \href{https://wiki.python.org/moin/TkInter}{Tk support} or not).

\item By default, all the Python packages will be installed in a
  different repository of the system packages, which eases the
  system/user Python-isolation and the removal of the interpreter.

\end{enumerate}

In Windows you need to install Python, yes or yes, from the official
\href{https://www.python.org/downloads/}{website}. However, notice
that this ``guide'' only contemplates the installation of Python in
Unix-based OS machines.

\section{What you have to do?}

\begin{enumerate}
  
\item Installation of Python.
  
  \begin{enumerate}
    
  \item Go to
    \href{https://github.com/vicente-gonzalez-ruiz/YAPT/blob/master/01-hello_world/02-installation.ipynb}{YAPT/01-hello\_world/02-installation.ipynb}
    \cite{YAPT} and follow the instructions to install CPython 3.8.5,
    and create a new virtual environment called \texttt{tm}. Basically:

    \begin{lstlisting}[language=Bash]
      sudo apt-get install -y build-essential libssl-dev zlib1g-dev libbz2-dev libreadline-dev libsqlite3-dev wget curl llvm libncurses5-dev libncursesw5-dev xz-utils tk-dev libffi-dev liblzma-dev python-openssl git
      pyenv install -v 3.8.5
      pyenv virtualenv 3.8.5 tm
    \end{lstlisting}

  \item Remember that you will need to active it when you want to
    work in this project:

    \begin{lstlisting}[language=Bash]
      pyenv activate tm
    \end{lstlisting}

    It is a good idea to append this to the \verb|~/.bashrc| file.
    
  \item Install an
    \href{https://en.wikipedia.org/wiki/Integrated_development_environment}{IDE}
    for programming with Python. I recommend
    \href{https://thonny.org/}{Thonny} if you are not used to any
    other.
    
    \begin{lstlisting}[language=Bash]
      pip install thonny
    \end{lstlisting}

  \end{enumerate}
  
\item Python programming.
  
  \begin{enumerate}
    
  \item You don't need to master Python to follow this course, but it
    is convenient for you to follow some Python programming tutorial,
    such as \href{https://docs.python.org/3/tutorial/}{The Python
    Tutorial} \cite{python-tutorial} if you realize that the language
    is a setback for you. If you need to start with Python from
    scratch, an introduction to Python such as
    this \href{https://github.com/vicente-gonzalez-ruiz/YAPT/tree/master/workshops/programacion_python_ESO}{workshop
    of YAPT} \cite{YAPT} could also be helpful. See
    also \href{http://zetcode.com/lang/python/}{ZetCode's Python
    Tutorial}.
    
  \end{enumerate}

\end{enumerate}

\section{Timming}

There is not time limit for finishing this milestone. Develop it at
your own pace.

\section{Deliverables}

None.

\section{Resources}

\bibliography{python,intercom}
