\title{\href{https://www.ual.es/estudios/grados/presentacion/plandeestudios/asignatura/4015/40154321?idioma=zh_CN}{Tecnologías Multimedia} - Study Guide - Milestone 6: Compressing the audio data with \href{https://zlib.net/}{zlib}}

\maketitle

\section{Description}

The process of representing a large—possibly infinite—set of values with a much smaller
set is called quantization.~\cite{sayood}. The set of inputs and outputs of a quantizer can be scalars or vectors. If they are
scalars, we call the quantizers scalar quantizers. If they are vectors, we call the quantizers
vector quantizers.

Scalar Quantization. Digital computation and communication use discrete values rather than continuous
values such as real numbers. ~\cite{foundationsofsignalprocessing} In lossy compression, mean approximation error is called distortion and code
length in bits per scalar input is called rate. Uniform Quantization The simplest and most common form of quantization is
uniform quantization. Typically, it either takes the form of rounding to the nearest
integer multiple of a step size ∆, as in Figure 6.40(a), or using the integer multiples
of the step size as threshold values at which the output jumps, as in Figure 6.40(b).

some applications
(such as certain medical imaging systems) require lossless compression,
while other applications may tolerate some amount of distortion in the
decompressed data in return for a smaller compressed representation.
Quantization is the element of lossy compression systems responsible for
reducing the precision of data in order to make them more compressible.
In most lossy compression systems, it is the only source of distortion.~\cite{taubman}

Quantization of Discrete Time
Signals~\cite{digitalsignalprocessinghandbook}. Quantization is the process of approximating any discrete time, continuous
amplitude signal into one of a finite set of discrete time, continuous amplitude signals based on a
particular distortion or distance measure. A quantizer, Q, is mathematically defined as a mapping [3] Q : Rp → C. This means that the
p-dimensional vectors in the vector space Rp are mapped into a finite collection C of vectors that are
also in Rp. This collection C is called the codebook and the number of vectors in the codebook, N,
is known as the codebook size. The entries of the codebook are known as codewords or codevectors.
If p = 1, we have a scalar quantizer (SQ). If p > 1, we have a vector quantizer (VQ).

--------

It's time to reduce bandwidth comsumption. The $\mathtt{pack()}$ and
the $\mathtt{unpack()}$ methods can compress and decompress,
respectively, the chunks that are handled. To compress and decompress,
we will use a free codec named
\href{https://en.wikipedia.org/wiki/DEFLATE}{DEFLATE}, which is based
on \href{https://en.wikipedia.org/wiki/Lempel%E2%80%93Ziv%E2%80%93Storer%E2%80%93Szymanski}{LZSS}
and \href{https://en.wikipedia.org/wiki/Huffman_coding}{Huffman Coding}~\cite{nelson96datacompression}.

\begin{figure}
  \begin{center}
    \myfig{graphics/reordering}{5cm}{500}
  \end{center}
  \caption{Sample reordering to create two independent channels.}
  \label{fig:reordering}
\end{figure}

\section{What you have to do?}

\begin{enumerate}
\item Create a class named \texttt{Compress}, that inherits from
  \texttt{Buffer} (the class implemented in the previous milestone),
  in which the methods $\mathtt{pack()}$ and $\mathtt{unpack()}$ are
  overriden to compress and decompress the chunks. Use the Python's
  standard library
  \href{https://docs.python.org/3/library/zlib.html}{\texttt{zlib}}. Store
  this class in a module named \texttt{compress.py}.
\item Compress (and decompress) each chunk as a unit (each compressed
  chunk will be transmitted in a different UDP packet). In order to
  increase slightly the
  \href{https://en.wikipedia.org/wiki/Data_compression_ratio}{(data)
    compression ratio}, reorder the samples as it is shown in the
  Figure~\ref{fig:reordering}.
\item Describe the QoE as a function of the transmission bit-rate
  (determine the transmission bit-rates that generate each possible
  QoE clasiffication).
\end{enumerate}

\section{Timming}

Please, finish this milestone at most in one week.

\section{Deliverables}

Create a Python module named \texttt{compress.py} and store it in the
\href{https://github.com/Tecnologias-multimedia/intercom}{root
  directory} of your \texttt{intercom}'s repo.

\section{Resources}

\bibliography{text-compression}
