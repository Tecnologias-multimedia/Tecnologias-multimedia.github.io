\newcommand{\TM}{\href{https://www.ual.es/estudios/grados/presentacion/plandeestudios/asignatura/4015/40154321?idioma=zh_CN}{Tecnologías Multimedia}}

\title{\TM - Study Guide - Milestone 6: Compressing the audio data with \href{https://zlib.net/}{zlib}}

\maketitle

\section{Description}

It's time to reduce bandwidth comsumption. The pack() and the unpack()
methods can compress and decompress, respectively, the chunks that are
handled. To compress and decompress, we will use a free codec named
\href{https://en.wikipedia.org/wiki/DEFLATE}{DEFLATE}, which is based
on
\href{https://en.wikipedia.org/wiki/Lempel%E2%80%93Ziv%E2%80%93Storer%E2%80%93Szymanski}{LZSS}
  and \href{https://en.wikipedia.org/wiki/Huffman_coding}{Huffman
    Coding}~\cite{nelson96datacompression}. See this
  \href{https://github.com/vicente-gonzalez-ruiz/LZ77}{notebook} and
  this
  \href{https://vicente-gonzalez-ruiz.github.io/Huffman_coding/}{notebook}.

\begin{figure}
  \begin{center}
    \myfig{graphics/reordering}{5cm}{500}
  \end{center}
  \caption{Sample reordering to create two independent channels.}
  \label{fig:reordering}
\end{figure}

\section{What you have to do?}

\begin{enumerate}
  
\item Create a class named Compression, that inherits from
  Buffering (the class implemented in the previous milestone),
  in which the methods pack() and unpack() are
  overriden to compress and decompress the chunks. Use the Python's
  standard library
  \href{https://docs.python.org/3/library/zlib.html}{\texttt{zlib}}. Store
  this class in a module named compress.py.
  
\item Compress (and decompress) each chunk as a unit (each compressed
  chunk will be transmitted in a different UDP packet). In order to
  increase slightly the
  \href{https://en.wikipedia.org/wiki/Data_compression_ratio}{(data)
    compression ratio}, reorder the samples as it is shown in the
  Figure~\ref{fig:reordering}.

\item Describe the QoE as a function of the transmission bit-rate
  (determine the transmission bit-rates that generate each possible
  QoE clasiffication). Use
  \href{https://man7.org/linux/man-pages/man8/tc.8.html}{tc}.
\end{enumerate}

\section{Timming}

Please, finish this milestone at most in one week.

\section{Deliverables}

Create a Python module named compress.py and store it in the
\href{https://github.com/Tecnologias-multimedia/intercom}{root
  directory} of your InterCom's repo.

\section{Resources}

\bibliography{text-compression}
