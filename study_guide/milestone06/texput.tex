\title{\href{https://www.ual.es/estudios/grados/presentacion/plandeestudios/asignatura/4015/40154321?idioma=zh_CN}{Tecnologías Multimedia} - Study Guide - Milestone 7: Compressing the audio data with \href{https://zlib.net/}{zlib}}

\maketitle

\section{Description}

It's time to reduce bandwidth comsumption. The $\mathtt{pack()}$ and
the $\mathtt{unpack()}$ methods can compress and decompress,
respectively, the chunks that are handled. To compress and decompress,
we will use a free codec named
\href{https://en.wikipedia.org/wiki/DEFLATE}{DEFLATE}, which is based
on LZSS and Huffman Coding~\cite{nelson96datacompression}.

\section{What you have to do?}

\begin{enumerate}
\item Create a class named \texttt{Intercom\_minimal\_zlib}, that
  inherits from \texttt{Intercom\_minimal}, in which the methods
  $\mathtt{pack()}$ and $\mathtt{unpack()}$ are overriden to compress
  and decompress the chunks. Use the Python's standard library
  \href{https://docs.python.org/3/library/zlib.html}{\texttt{zlib}}. Store
  this class in a module named \texttt{intercom\_minimal\_zlib.py}.
\item Create a class named \texttt{Intercom\_inter\_zlib}, that
  inherits from \texttt{Intercom\_inter} and
  \texttt{Intercom\_minimal\_zlib}, to allow the compression of the
  decorrelated channels.
\item Notice that, in general, the compression ratio provided by
  DEFLATE is higher when the samples (in the case of
  \texttt{Intercom\_minimal}) or the coefficients (in the case of
  \texttt{Intercom\_inter}) are not interleaved (first placing all the
  samples of the first channel/subband, and next, placing all the
  samples of the second channel/subband).
\end{enumerate}

\section{Timming}

You should reach this milestone at most in one week.

\section{Deliverables}

The modules \texttt{intercom\_minimal\_zlib.py} and
\texttt{intercom\_inter\_zlib}. Store them at the
\href{https://github.com/Tecnologias-multimedia/intercom}{root
  directory} of your \texttt{intercom}'s repo.

\section{Resources}

\bibliography{data-compression}
