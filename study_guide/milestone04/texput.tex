\title{\href{https://www.ual.es/estudios/grados/presentacion/plandeestudios/asignatura/4015/40154321?idioma=zh_CN}{Tecnologías Multimedia} - Study Guide - Milestone 5: Minimal InterCom}

\maketitle

\section{Description}

\subsection{Sequential model}

InterCom can be divided into 6 steps:

\begin{pseudocode}{Sequential\_InterCom}{~}
  \BEGIN
    \mathtt{recorded\_chunk} \GETS \mathtt{record()}\\
    \mathtt{outgoing\_packet} \GETS \mathtt{pack(recorded\_chunk)}\\
    \mathtt{send(outgoing\_packet)}\\
    \mathtt{incoming\_packet} \GETS \mathtt{receive()}\\
    \mathtt{chunk\_to\_play} \GETS \mathtt{unpack(incoming\_packet)}\\
    \mathtt{play(chunk\_to\_play)}
  \END
\end{pseudocode}

Where:

\begin{enumerate}
\item $\mathtt{record()}$ captures a chunk of frames. In general, this
  is a \href{https://en.wikipedia.org/wiki/I/O_bound}{I/O-bound}
  blocking operation, depending or not on the availability of frames
  in the ADC.
\item $\mathtt{pack(chunk)}$ process the chunk to create a
  $\mathtt{packet}$, a structure that can be transmitted through the
  Internet using the
  \href{https://en.wikipedia.org/wiki/Datagram}{Datagram} Model. This
  is a non-blocking operation and
  (\href{https://en.wikipedia.org/wiki/CPU-bound}{CPU-bound}.
\item $\mathtt{send(packet)}$, sends the packet to the
  interlocutor's InterCom. When datagrams are used, this step is
  non-blocking, although it cannot considered CPU-bound because the
  CPU usage is very low.
\item $\mathtt{receive()}$, waits (blocking the task) for a incoming
  packet. This operation is IO-bound.
\item $\mathtt{unpack(packet)}$ is a non-blocking
  CPU-intensive (CPU-bound) step that transforms a packet into a chunk
  of audio.
\item $\mathtt{play(chunk)}$ renders the chunk, reproducint it. A
  I/O-bound blocking action.
\end{enumerate}
  
Notice that the execution of these steps, in this order, in two
different InterCom instances has at least two problems:
\begin{enumerate}
\item If a packet is lost, a deadlock is generated, because the
  $\mathtt{receive()}$ never returns.
\item If the time required by $\mathtt{pack(chunk)}$ and
  $\mathtt{unpack(packet)}$ is largest than the chunk-time, a glich
  will be generated.
\end{enumerate}

\subsection{Concurrent model}

The \href{https://en.wikipedia.org/wiki/Parallel_computing}{parallel}
model supposes that there is at least two Processing Units (or PUs,
typically
\href{https://en.wikipedia.org/wiki/Multi-core_processor}{cores}),
that are able to work in parallel. Thus, the sequential model is
divided into two
\href{https://en.wikipedia.org/wiki/Concurrency_(computer_science)}{concurrent}
tasks that should run in parallel.

\begin{pseudocode}{Concurrent\_InterCom}{~}
  \PROCEDURE{Record\_Pack\_and\_Send}{~}
  \BEGIN
    \mathtt{chunk} \GETS \mathtt{record()}\\
    \mathtt{packet} \GETS \mathtt{pack(chunk)}\\
    \mathtt{send(packet)}
  \END
  \ENDPROCEDURE
  \PROCEDURE{Receive\_Unpack\_and\_Play}{~}
  \BEGIN
    \mathtt{packet} \GETS \mathtt{receive()}\\
    \mathtt{chunk} \GETS \mathtt{unpack(packet)}\\
    \mathtt{play(chunk)}
  \END
  \ENDPROCEDURE
\end{pseudocode}

In this case, the loss of a chunk does not generates a deadlock
between the concurrent tasks, because even if the
$\mathrm{Receive\_Unpack\_Play}$ is blocked waiting for a block in the
$\mathtt{receive()}$, the $\mathtt{Record\_Pack\_and\_Send}$ task will
continue sendind chunks.

Unfortunatelly, gliches can still happen in this model if the packets
are lost or delayed, or if $\mathtt{unpack(packet)}$ needs more than a
chunk-time to be completed.

\subsection{Timer-based model}

In this model, a task is called periodically (probably, using some
timer provided by the sound hardware).

\begin{pseudocode}{Timer-based\_InterCom}{~}
  \PROCEDURE{Record\_and\_Play}{\mathtt{recorded\_chunk}, \mathtt{chunk\_to\_play}, \mathtt{chunk\_size}}
  \BEGIN
    \mathtt{recorded\_chunk} \GETS \mathtt{record(chunk\_size)}\\
    
    \mathtt{send(outgoing\_packet)}\\
    \mathtt{play}(\mathtt{chunk\_to\_play})\\
  \END
  \ENDPROCEDURE
\end{pseudocode}

where:
\begin{enumerate}
\item $\mathtt{indata}$ points to the last chunk of audio recorded.
\item $\mathtt{outdata}$ points to the chunk of audio that is going to be played.
\item $\mathtt{chunk\_size}$ is the number of frames processed in each iteration of the callback.
\end{enumerate}

The advantage of this alternative compared to the Concurrent Model is
that we can control the size of the buffers, which in the end controls the latency.

\section{What you have to do?}

\begin{enumerate}

\item
  
\end{enumerate}

\section{Timming}

You should reach this milestone at most in one week.

\section{Deliverables}

\section{Resources}

\bibliography{python}
