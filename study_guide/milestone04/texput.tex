\title{\href{https://www.ual.es/estudios/grados/presentacion/plandeestudios/asignatura/4015/40154321?idioma=zh_CN}{Tecnologías Multimedia} - Study Guide - Milestone 5: Minimal InterCom}

\maketitle

\section{Description}

\subsection{Sequential model}

InterCom can be divided into 6 steps:

\begin{pseudocode}[display]{}{}
  \BEGIN
    \mathtt{chunk} \GETS \mathtt{record()}\\
    \mathtt{packet} \GETS \mathtt{pack(chunk)}\\
    \mathtt{send(packet)}\\
    \mathtt{packet} \GETS \mathtt{receive()}\\
    \mathtt{chunk} \GETS \mathtt{unpack(packet)}\\
    \mathtt{play(chunk)}
  \END
\end{pseudocode}

Where:

\begin{enumerate}
\item $\mathtt{record()}$ captures a chunk of frames. This is a
  \href{https://en.wikipedia.org/wiki/I/O_bound}{I/O-bound} blocking
  operation.
\item $\mathtt{pack(chunk)}$ process the chunk to create a
  $\mathtt{packet}$, a structure that can be transmitted through the
  Internet using the
  \href{https://en.wikipedia.org/wiki/Datagram}{Datagram} Model. This
  is a non-blocking operation and
  (\href{https://en.wikipedia.org/wiki/CPU-bound}{CPU-bound}.
\item $\mathtt{send(packet)}$, sends the packet to the interlocutor's
  InterCom. When datagrams are used, this step is non-blocking,
  although it cannot considered CPU-bound.
\item $\mathtt{receive()}$, waits (blocking the task) for a incoming
  packet. This operation is IO-bound.
\item $\mathtt{unpack(packet)}$ is a non-blocking CPU-intensive
  (CPU-bound) step that transforms a packet into a chunk of audio.
\item $\mathtt{play(chunk)}$ renders the chunk, reproducint it. A
  I/O-bound blocking action.
\end{enumerate}
  
Notice that the execution of these steps, in this order, in two
different InterCom instances has a problem: if a packet is lost, a
deadlock is generated.

\subsection{Parallel model}

The \href{https://en.wikipedia.org/wiki/Parallel_computing}{parallel}
model supposes that there is at least two Processing Units (or PUs,
typically
\href{https://en.wikipedia.org/wiki/Multi-core_processor}{cores}),
that are able to work in parallel. Thus, the sequential model is
divided into two
\href{https://en.wikipedia.org/wiki/Concurrency_(computer_science)}{concurrent}
tasks that should run in parallel.

\begin{pseudocode}[display]{}{}
  \PROCEDURE{Send}{~}\\
  \BEGIN
    \mathtt{chunk} \GETS \mathtt{record()}\\
    \mathtt{packet} \GETS \mathtt{pack(chunk)}\\
    \mathtt{send(packet)}
  \END
  \ENDPROCEDURE
%\end{pseudocode}
%\begin{pseudocode}[display]{}{}
  \BEGIN
    \mathtt{packet} \GETS \mathtt{receive()}\\
    \mathtt{chunk} \GETS \mathtt{unpack(packet)}\\
    \mathtt{play(chunk)}
  \END
\end{pseudocode}

In this case, the loss of a chunk does not generates a deadlock
between the concurrent tasks.

\section{What you have to do?}

\begin{enumerate}

\item
  
\end{enumerate}

\section{Timming}

You should reach this milestone at most in one week.

\section{Deliverables}

\section{Resources}

\bibliography{python}
