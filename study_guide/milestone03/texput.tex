\title{\href{https://www.ual.es/estudios/grados/presentacion/plandeestudios/asignatura/4015/40154321?idioma=zh_CN}{Tecnologías Multimedia} - Study Guide - Milestone 3: ``wiring'' the ADC with the DAC}

\maketitle

\section{Description}

Let's start our development of InterCom with a simple program that
inputs digital audio (through the
\href{https://en.wikipedia.org/wiki/Analog-to-digital_converter}{ADC}
of your computer) and, as soon as possible, outputs it (through the
\href{https://en.wikipedia.org/wiki/Digital-to-analog_converter}{DAC}). We
are going to download a minimal audio-wire program and to measure its
latency. To handle the audio hardware we use
\href{https://python-sounddevice.readthedocs.io}{\texttt{sounddevice}}
\cite{sounddevice}, that is wrapper for the
\href{http://www.portaudio.com/}{PortAudio} library.

\section{What you have to do?}

\begin{enumerate}

\item Download the Python \href{https://docs.python.org/3/tutorial/modules.html}{module} \href{https://github.com/Tecnologias-multimedia/intercom/blob/master/test/sounddevice/wire3.py}{wire3.py}.

\item If you want to run this module right now, check first that the
  output volumen of your speakers is not too high (an audio coupling
  (a feedback) between the mic(rophone) and the speakers could
  happen).

As you can see in the code of \texttt{wire3.py}:

\begin{lstlisting}[language=Python]
import sounddevice as sd
\end{lstlisting}

we are using the
\href{https://python-sounddevice.readthedocs.io/en/latest/}{python-sounddevice}
\cite{sounddevice} module to access to the sound hardware.

Next, we create a I/O raw audio stream
using \href{https://python-sounddevice.readthedocs.io/en/latest/api/raw-streams.html#sounddevice.RawStream}{sounddevice.RawStream()},
using
\begin{lstlisting}[language=Python]
stream = sd.RawStream(samplerate=44100, channels=2, dtype='int16')
\end{lstlisting}

As you can see in \cite{sounddevice}, this constructor method accepts
several parameters. In our example we have used only the most
relevants for us right now:
\begin{enumerate}
\item The samling rate.
\item The number of channels.
\item The number of bits of each sample.
\end{enumerate}
Please, refeer \cite{harmonic-analysis} for knowing more detailed
information about these parameters.

Please, review the modules of documentation 

Next, we define a chunk size
\begin{lstlisting}[language=Python]
CHUNK_SIZE = 1024
\end{lstlisting}
for the block of data we are going to capture in the loop that will
be defined just below.

The loop is an infinite loop

\item Measure the theoretical latency.

\item Write a pulse generator that periodically plays a very short
  sound, and where we can control the period.

\item Capture this sound with \texttt{wire3.py} and modify the period
  until you hear that the input and the output are synchronized.

\item Compute the period in secods, this is also the latency of
  \texttt{wire3.py}.

\item Compute the difference between this latency and the theoretical
  one.

\end{enumerate}

\section{Resources}

\bibliography{python}
