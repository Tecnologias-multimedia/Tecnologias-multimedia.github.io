\newcommand{\TM}{\href{https://www.ual.es/estudios/grados/presentacion/plandeestudios/asignatura/4015/40154321?idioma=zh_CN}{Tecnologías Multimedia}}

\title{\TM{} - Study Guide - Milestone 13: Overlapped DWT (I)}

\maketitle

\section{Description}

\subsection{\href{https://en.wikipedia.org/wiki/Lapped_transform}{Overlapped transforms} for minimizing the distortion}
Transform coding implies to split the signal into chunks, and to
compute the transform of each chunk. When the output coefficients are
quantized, it is possible that significative (and unpleasant)
distortions may appear between the border samples of the chunks (see
this the Fig~\ref{fig:3_chunks}).

\begin{figure}
  \centering
  \begin{tabular}{cc}
    \svg{3_chunks}{500} & \svg{without}{500} \\
    \svg{extended}{500} & \svg{reconstructed}{500} \\
  \end{tabular}
  \caption{On the top-left, three consecutive chunks of a real audio
    sequence. On the top-right, the reconstruction of the chunks
    without overlapping. On the bottom-left, the extended chunk. On
    the bottom-right, the reconstruction of the extended chunk. See
    this
    \href{https://github.com/Tecnologias-multimedia/intercom/blob/master/docs/quantization_DWT.ipynb}{notebook}.}
  \label{fig:3_chunks}
\end{figure}

One solution for the current ($i$-th) chunk is to use the last samples
of the previous ($(i-1)$-th) chunk and the first samples of the next
($(i+1)$-th) chunk for computing the transform of the current ($i$-th)
chunk. Therefore, to avoid signal discontinuities between chunks, we
can overlap them and use the following algorithm:

\subsection*{Encoder:}
\begin{enumerate}
\item For $i\in\{0,1,\cdots\}$:   
  \begin{enumerate}               
  \item Input chunks $i-1$, $i$ and $i+1$, $C_{i-1}$, $C_i$ and $C_{i+1}$.
  \item Let's the overlapped area size $o$, $C[:o]$ the first $o$
    samples of the chunk $C$, and $C[-o:]$ the last $o$ samples.
  \item $D_i \leftarrow \text{DWT}(C_{i-1}[-o:]|C_i|C_{i+1}[:o])$, where $D_i$ is the
    decomposition of the $i$-th extended chunk, $\cdot|\cdot$ denotes
    the concatenation of samples operator, defined by
    \begin{equation}
      a|b = \{a,b\},
    \end{equation}
    and $C_{i-1}[-o:]|C_i|C_{i+1}[:o]$ is the $i$-th extended chunk.
  \item Send $D_i$.
  \end{enumerate}
\end{enumerate}

\subsection*{Decoder:}
\begin{enumerate}
\item For $i\in\{0,1,\cdots\}$:
  \begin{enumerate}
  \item Receive $D_i$.
  \item Perform $C_{i-1}[-o:]|C_i|C_{i+1}[:o]\leftarrow\text{DWT}^{-1}(D_i)$.
  \item Output $C_i$.
  \end{enumerate}
\end{enumerate}

This idea has been implemented in this
\href{https://github.com/Tecnologias-multimedia/intercom/blob/master/docs/overlapped_DWT_I.ipynb}{notebook},
and the result can be seen in the Fig.~\ref{fig:3_chunks}.

\section{What you have to do?}

\begin{enumerate}
\item In a module named overlapped\_temporal\_decorrelate.py, inherit
  the class Temporal\_decorrelation and create a class named
  Overlapped\_temporal\_decorrelation.
\item Compare the RD (Rate/Distortion) curves with the previous one.
\end{enumerate}

\section{Timming}

Please, present your results in one week.

\section{Deliverables}

The module overlapped\_temporal\_decorrelate.py. Store it at the
\href{https://github.com/Tecnologias-multimedia/intercom}{root
  directory} of your InterCom's repo.

\section{Resources}

\bibliography{maths,data-compression,DWT,audio-coding}
