\title{\href{https://www.ual.es/estudios/grados/presentacion/plandeestudios/asignatura/4015/40154321?idioma=zh_CN}{Tecnologías Multimedia} - Study Guide - Milestone 3: Git, GitHub and the Fork-and-Branch Git Workflow}

\maketitle

\lstset{
  showstringspaces=false,
  upquote=true
}

\section{Description}

To work in
\href{https://github.com/Tecnologias-multimedia/intercom}{the InterCom
  project} \cite{intercom} you must understand the basics of
Git \cite{Git-book} and
\href{https://github.com/}{GitHub}\footnote{There are other Git-based
hosting services such as \href{https://about.gitlab.com/}{GitLab} and
\href{https://www.atlassian.com/git}{Altassian}/\href{https://bitbucket.org/product}{BitBucket},
but GitHub is the most used one.} \cite{GitHub}, and how to use the
\href{https://guides.github.com/introduction/flow/index.html}{The
  GitHub (Work)Flow} and the
\href{https://github.com/vicente-gonzalez-ruiz/fork_and_branch_git_workflow}{Fork-and-Branch
  Git Workflow} \cite{fork-and-branch-git-workflow}.

\section{What you have to do?}

\begin{enumerate}
  
\item Have a look to the \href{https://git-scm.com/book/en/v2}{Pro Git} 
  book \cite{Git-book}, what is the main source of information about
  Git.

\item If you don't have an GitHub account, please, do the
  \href{https://guides.github.com/activities/hello-world/}{the Hello
    World guide at GitHub} and create one. Be aware that to contribute
  to InterCom an GitHub account is required. Notice that in the Hello
  World guide, the \texttt{master} branch is named the \texttt{main}
  branch. Please, create a \texttt{README.md} file for the Hello World
  repo, as the guide suggests.

\item Now, we are going to to the same that we would have done using
  the GitHub web interface (except the Step 1: Create a Repo), but now
  using the terminal which will be the most used interface for dealing
  with Git. First, if Git is not installed in your host (try to run
  \texttt{git} in a terminal), install it with:

  \begin{lstlisting}[language=bash]
    sudo apt install git
  \end{lstlisting}

\item
  \href{https://docs.github.com/en/github/getting-started-with-github/github-glossary#clone}{Clone}
  (download) the Hello World repo. You need to click on the ``Code''
  button (select ``https'', not ``download a
  zip file''). Then run:

  \begin{lstlisting}[language=bash]
    cd hello_world
  \end{lstlisting}

  Notice that a new directory named as the repo's name at GitHub has
  been created, and that inside you can find the \texttt{README.md}
  file written in
  \href{https://daringfireball.net/projects/markdown/}{Markdown}.

\item \href{https://docs.github.com/en/github/getting-started-with-github/github-glossary#checkout}{Create (and switch to)} a
  \href{https://docs.github.com/en/github/getting-started-with-github/github-glossary#feature-branch}{feature
    branch} called \texttt{improving\_readme}. In your terminal write:
  
  \begin{lstlisting}[language=bash]
    git checkout -b improving_readme
  \end{lstlisting}
  
\item Modify the file \texttt{README.md}. Append to it, for example, a
  link to the Hello World guide. Use an ASCII editor (\texttt{nano},
  for example):

  \begin{lstlisting}[language=bash]
    nano README
  \end{lstlisting}

  And write:
  
  \begin{lstlisting}
    See the [Hello World](https://guides.github.com/activities/hello-world/) guide at GitHub.
  \end{lstlisting}
  
\item
  \href{https://docs.github.com/en/github/getting-started-with-github/github-glossary#commit}{Commit}
  your modification(s):

  \begin{lstlisting}[language=bash, showstringspaces=false, upquote=true]
    git commit -am "Providing the Hello World link"
  \end{lstlisting}

  In your first commit you will be prompted with:

  \begin{lstlisting}[language=bash]
    git config --global user.email "your_email@example.com"
  \end{lstlisting}

  Please, input such information.
  
  After the \texttt{commit}, your \href{https://docs.github.com/en/github/getting-started-with-github/github-glossary#fetch}{\emph{local}} repo is \emph{ahead} of
  your
  \href{https://docs.github.com/en/github/getting-started-with-github/github-glossary#origin}{\emph{origin}}
  (copy at GitHub of the) remote repo. This means that your
  \emph{local} has modifications that the \emph{origin} doesn't have.

\item Synchronize your \emph{local} and the \emph{origin} using
  \href{https://docs.github.com/en/github/getting-started-with-github/github-glossary#push}{push}:

  \begin{lstlisting}[language=bash]
    git push
  \end{lstlisting}

  Notice that if you have not upoaded a public
  \href{https://www.ssh.com/}{SSH}
  \href{https://www.ssh.com/ssh/identity-key}{key} (or the
  corresponding private key is not properly installed in your
  computer), the GitHub server requests your username and password,
  and this is something that is going to happen with every
  \texttt{push}. To avoid this repetitive input of your GitHub login
  information, you need
  \href{https://docs.github.com/en/github/authenticating-to-github/connecting-to-github-with-ssh}{to
    login at GitHub} using
  \href{https://en.wikipedia.org/wiki/Public-key_cryptography}{public-key
    criptography}. For that, you must own a pair of keys, one public
  and other private, and upload the public one to GitHub.
  
\item The first step is to check whether you already have a pair of
  keys (if your are using the just installed Xubuntu distribution,
  obviously you don't need to check anything and can go directly to
  the next step). Simply revise your \texttt{\$HOME/.ssh} directory
  with:

  \begin{lstlisting}[language=bash]
    ls -l ~/.ssh
  \end{lstlisting}

  and if you find a pair of files with almost the same name, and one
  of they finising in \texttt{.pub}, you probably own a pair of SSH
  keys.

\item Let's create a pair of keys (if you don't have one or if you
  prefeer to create a new one). Open a terminal and write:

  \begin{lstlisting}[language=bash]
    ssh-keygen -t rsa -b 4096 -C "your_email@example.com"
  \end{lstlisting}

  using the email address you provided when you created your GitHub
  account. Then, when you are prompted with:

  \begin{lstlisting}[language=bash]
    Enter a file in which to save the key (/home/you/.ssh/id_rsa):
  \end{lstlisting}

  just press the Enter-key, to select such output prefix. Otherwise,
  write a different one, but don't change the path to the
  \texttt{.ssh} directory.

\item Now SSH should request you for a passphrase. If you write one,
  you will be asked for it each time you push your commits to
  GitHub. There are two options to avoid this:

  \begin{enumerate}
  \item Input no passphrase (just by pressing the Enter-key again in
    the previous step). This has the drawback that if somebody steals
    your keys, he could access to GitHub as he were you.
  \item Input a passphrase and configure
    \href{https://www.ssh.com/ssh/agent}{\texttt{ssh-agent}} to send
    it to GitHub by you. This option is the preferable one becase you
    will be asked for the passphrase only when the \texttt{ssh-agent}
    is started (\href{https://www.xfce.org/}{Xfce} does that by you).
  \end{enumerate}

\item Now it's time to check whether the \texttt{ssh-agent} is already
  running in your computer. This can be done with:

  \begin{lstlisting}[language=bash]
    ps aux | grep ssh-agent
  \end{lstlisting}

  and in the case of Xubuntu, you should get something similar to:

\begin{verbatim}
 989 ?        Ss     0:00 /usr/bin/ssh-agent /usr/bin/im-launch startxfce4
1433 pts/0    S+     0:00 grep --color=auto ssh-agent
\end{verbatim}

  This means that there are two processes in whose description there
  exists the string \texttt{ssh-agent}. The first entry is the agent
  process. The second one is the \texttt{grep} running at the same
  time that the
  \href{https://man7.org/linux/man-pages/man1/ps.1.html}{\texttt{ps}}.

\item If the \texttt{ssh-agent} were not running, it can be launched
  to run in
  the \href{https://en.wikipedia.org/wiki/Background_process}{background}
  with:

  \begin{lstlisting}[language=bash]
    eval "$(ssh-agent -s)"
  \end{lstlisting}

  but you don't need to do that in your Xubuntu installation, because
  (remember) the \texttt{ssh-agent} the
  Xfce desktop environment lauches it.

\item With your keys, run:

  \begin{lstlisting}[language=bash]  
    ssh-add ~/.ssh/id_rsa
  \end{lstlisting}

  and the passphrase will be prompt.
  
\item Go now to GitHub \texttt{->} Settings \texttt{->} SSH and GPG
  keys \texttt{->} New SSH key. Open a terminal and write:
  
  \begin{lstlisting}[language=bash]  
    cat .ssh/id_rsa.pub
  \end{lstlisting}

  and copy and paste the content of such file (which ends with your
  email address) inside of the space where you can read
  ``\texttt{Begins with 'ssh-rsa', ...}''. Don't forget to give a
  title (something such as ``\texttt{tm}'' (tecnonogías multimedia))
  to the key pair.
  
\item When you use the key for the first time (clonning a repo or
  pushing a commit), the SSH client will warn you that the autenticity
  of \texttt{github.com} cannot be established. This is normal and
  should happen only once. Type \texttt{yes}. If this problem
  persists, then you could be suffering a
  \href{https://en.wikipedia.org/wiki/Man-in-the-middle_attack}{man-in-the-middle
    attack}.

\item Revise
  \href{https://github.com/vicente-gonzalez-ruiz/fork_and_branch_git_workflow}{The
    Fork and Branch Git Workflow}. Basically, this ``protocol''
  explains that to contribute to an open-source repo hosted by GitHub
  without belonging to the develop team, you must do some
  \texttt{git}-steps (and some of them are below).

\item Make a fork of the
  \href{https://github.com/Tecnologias-multimedia/intercom}{InterCom}
  project. We will call to this repo the
  \href{https://docs.github.com/en/github/getting-started-with-github/github-glossary#upstream}{\emph{upstream}}, whose URL is
\begin{verbatim}
  git@github.com:Tecnologias-multimedia/intercom.git
\end{verbatim}        
  This info can be found when you
  \href{https://docs.github.com/en/github/creating-cloning-and-archiving-repositories/cloning-a-repository}{clone}
  the InterCom. Notice however, that action of clonning the InterCom
  is a waste of time because you cannot contribute directly to it
  (remember, you must \texttt{clone} your own repo).

\item Add the
  \href{https://docs.github.com/en/github/getting-started-with-github/github-glossary#remote}{remote}\footnote{Git
  is a
  \href{https://nvie.com/posts/a-successful-git-branching-model/}{decentralized
    control system for source code}. Decentralization means that every
  developer has a copy of the \emph{origin}, and that thus, the
  developers can synchronize their \emph{locals} with any of the
  \emph{remotes}.}  \emph{upstream} with:

  \begin{lstlisting}[language=bash]
    git remote add upstream git@github.com:Tecnologias-multimedia/intercom.git
  \end{lstlisting}

  Check that everything has worked with:

  \begin{lstlisting}[language=bash]  
    git remote -v
  \end{lstlisting}

  where you should see two remotes: \emph{origin} and \emph{upstream}. Something similar to:

\begin{verbatim}
origin   git@github.com:Tecnologias-multimedia/intercom.git (fetch)
origin   git@github.com:Tecnologias-multimedia/intercom.git (push)
upstream git@github.com:you_at_GitHub/intercom.git (fetch)
upstream git@github.com:you_at_GitHub/intercom.git (push)
\end{verbatim}

\end{enumerate}

\section{Timming}

You should reach this milestone at most in one week.

\section{Deliverables}

None.

\section{Resources}

\bibliography{git,intercom}
