% Emacs, this is -*-latex-*-

\title{Bit-Rate Control in InterCom}

% USE 16 bits/coefficient (2 byte-planes).

\maketitle

\tableofcontents

\section{Description}
%{{{ 
\subsection{Impact of the link throughput}
%{{{ 
Along with the latency and its variation (jitter), another main aspect
to consider about the
\href{https://en.wikipedia.org/wiki/Telecommunications_link}{transmission
  link} used in an InterCom session is the
\href{https://en.wikipedia.org/wiki/Channel_capacityhttps://en.wikipedia.org/wiki/Network_throughput}{link
  throughput}\footnote{Measured in bits per second or a $10$-multiple
  of this transmission
  \href{https://en.wikipedia.org/wiki/Bandwidth_(computing)}{capacity}.}
that it can provide~\cite{Forouzan,Tanenbaum}. This bit-rate depends on
the maximum
\href{https://en.wikipedia.org/wiki/Channel_capacity}{capacity} (a
characteristic closely related with the available
\href{https://en.wikipedia.org/wiki/Bandwidth_(signal_processing)}{bandwidth})
and the
\href{https://en.wikipedia.org/wiki/Network_congestion}{congestion
  level} (that basically depends on the load) of the link. In general,
we can suppose that the capacity is constant along the time (the
bandwidth provided by the link does not vary with the time). On the
contrary, the throughput is time-variying and quite unpredictable,
because it depends on the congestion level that which in turn depdends
on the behaviour of the network users.

In this milestone we are going to measure the impact of the link
throughput on the
\href{https://en.wikipedia.org/wiki/Quality_of_experience}{QoE}\footnote{Follow
  the same rubric than in the case of the measurement of the impact of
  the latency.} provided by the current implementation of InterCom
(\href{https://github.com/Tecnologias-multimedia/InterCom/blob/master/src/buffer.py}{\texttt{buffer.py}}). Similarly
to the procedure used for measuring the impact of the latency and the
jitter, we will use
\href{https://man7.org/linux/man-pages/man8/tc.8.html}{\texttt{tc}}~\cite{bert2012lartc}
to control the amount of data that an InterCom instance will be
allowed to send in a local environment, with the aim of simulating a
real environment. Notice that this upper bound in the bit-rate will
also affect to the loss of chunks because if the link capacity is
smaller than the audio bit-rate (the throughput), sooner of later the
link will discard those chunks that can not be buffered in the
retransmission nodes (routers and switches)\footnote{Notice that, in
  this case, we would be at least contributing, if not causing, the
  link congestion.}.
%}}}

\subsection{Compressing the audio data with \href{https://zlib.net/}{zlib}}
%{{{ ...
To reduce the bit-rate, we need some way of compressing the data,
action that also will reduce the throughput in InterCom. The
\verb|pack()| and the \verb|unpack()| methods can compress and
decompress, respectively, the chunks that are processed. To compress
and decompress, we will use a free
\href{https://en.wikipedia.org/wiki/Codec}{codec} named
\href{https://en.wikipedia.org/wiki/DEFLATE}{DEFLATE}, which is based
on
\href{https://en.wikipedia.org/wiki/Lempel%E2%80%93Ziv%E2%80%93Storer%E2%80%93Szymanski}{LZSS}
  and \href{https://en.wikipedia.org/wiki/Huffman_coding}{Huffman
    Coding}~\cite{nelson96datacompression} (see this
  \href{https://github.com/vicente-gonzalez-ruiz/LZ77}{notebook} and
  this
  \href{https://vicente-gonzalez-ruiz.github.io/Huffman_coding/}{notebook}). The
  DEFLATE algorithm is implemented in the Python's standard library
  \href{https://docs.python.org/3/library/zlib.html}{\texttt{zlib}}. %We
%have used this facility for compressing and decompressing the chunks
%that we are sending and receiving in the methods \verb|pack()| and
%\verb|unpack()|, respectively.

In order to compare the performance of different alternatives, the
avobe methods are implemented in the following modules, with a different
functionality:

\begin{enumerate}
\item
  \href{https://github.com/Tecnologias-multimedia/InterCom/blob/master/src/DEFLATE\_raw.py}{\texttt{DEFLATE\_raw.py}}:
  Compress the raw chunks with DEFLATE.
\item
  \href{https://github.com/Tecnologias-multimedia/InterCom/blob/master/src/DEFLATE\_serial.py}{\texttt{DEFLATE\_serial.py}}:
  Compress the chunk after concatenating the channels (see
  Fig.~\ref{fig:reordering}). Notice that with this data-shuffling,
  the samples are not interleaved and the
  \href{https://en.wikipedia.org/wiki/Correlation}{correlation}
  between consecutive bytes is slighly increased. This should also
  increase the
  \href{https://en.wikipedia.org/wiki/Data_compression_ratio}{(data)
    compression ratio}.
\begin{figure}
  \begin{center}
    \myfig{graphics/reordering}{5cm}{500}
  \end{center}
  \caption{Sample reordering to create two independent channels.}
  \label{fig:reordering}
\end{figure}
\item
  \href{https://github.com/Tecnologias-multimedia/InterCom/blob/master/src/DEFLATE\_serial2.py}{\texttt{DEFLATE\_serial2.py}}:
  Similar to \verb|compress_serial.py|, but reseting DEFLATE at each
  new chunk-channel, i.e., compressing each chunk-channel
  independtly. The idea here is to see if DEFLATE is exploiting the
  redundancy between the consecutive channels.
\item
  \href{https://github.com/Tecnologias-multimedia/InterCom/blob/master/src/DEFLATE\_byteplanes2.py}{\texttt{DEFLATE\_byteplanes2.py}}:
  Similar to \verb|compress_serial.py| (samples are de-interleaved),
  but 2 code-streams are generated, one for the LSB (Low Significant
  Byte) plane and another for the MSB (Most Significant Byte) plane,
  working with 16 bits/sample. The idea here is to see if the MSB can
  be compressed more efficiently because it can contain runs of zeros,
  especially when the audio sequence is \emph{quiet}.
\item
  \href{https://github.com/Tecnologias-multimedia/InterCom/blob/master/src/DEFLATE\_byteplanes3.py}{\texttt{DEFLATE\_byteplanes3.py}}:
  Similar to \verb|DEFLATE_byteplanes2.py| but considering three
  byte-planes. This would enable the compression of
  \emph{coefficients}\footnote{Used in a future improvements of
  intercom.} that require more than two bytes to be represented.
\item
  \href{https://github.com/Tecnologias-multimedia/InterCom/blob/master/src/DEFLATE\_byteplanes4.py}{\texttt{DEFLATE\_byteplanes4.py}}:
  Considers four byte-planes.
\item
  \href{https://github.com/Tecnologias-multimedia/InterCom/blob/master/src/DEFLATE_byteplanes2\_interlaced.py}{\texttt{DEFLATE\_byteplanes2\_interlaced.py}}:
  Similar to \verb|DEFLATE_byteplanes2.py| but using the raw chunks
  (without concatenating the channels).
\end{enumerate}

Finally, notice that the number of sent UDP packets (that now will be
length-variying) remains constant respect to \verb|buffer.py|.
%}}}

\subsection{Quantization}
%{{{
At the hardware level, the samples of audio are usually represented
using \href{https://en.wikipedia.org/wiki/Pulse-code_modulation}{Pulse
  Code Modulation (PCM)}. In a PCM sample, the number of levels that
the signal can take depends on the
\href{https://en.wikipedia.org/wiki/Audio_bit_depth}{number of
  bits/sample} (16 bits in our case).

Another key aspect to consider is that the processing that the
\href{https://en.wikipedia.org/wiki/Auditory_system}{Human
  Auditory System (HAS)} performs to understand audio signals has several
\href{https://en.wikipedia.org/wiki/Psychoacoustics}{\emph{sources} of
  perceptual redundancy}. One of these sources is the
\href{https://en.wikipedia.org/wiki/Equal-loudness_contour}{finite
  number of different volumen levels that a human being can
  recognize}~\cite{bosi2003intro}. In this milestone we will profit of
that fact to decrease the transmission bit-rate by sacrificing
quality. In most
\href{https://en.wikipedia.org/wiki/Lossy_compression}{lossy
  compression} schemes, quantization is the only source of
\href{https://en.wikipedia.org/wiki/Distortion}{distortion}~\cite{taubman2002jpeg2000}.

\href{https://en.wikipedia.org/wiki/Quantization_(signal_processing)}{Scalar
  Quantization} (SQ) is the process of decreasing the number of
discrete levels that a signal can
take~\cite{sayood2017introduction}. \href{https://en.wikipedia.org/wiki/Vector_quantization}{Vector
  Quantization} (VQ) is similar, but is applied to tuples of samples
at the same time~\cite{vetterli2014foundations}. SQ is used when the
samples are decorrelated or althought are correlated, the
decorrelation will be exploited in a
\href{https://en.wikipedia.org/wiki/Entropy_encoding}{entropy coding}
stage (which in our case is DEFLATE), because the
\href{https://en.wikipedia.org/wiki/Quantization_(signal_processing)#Rate%E2%80%93distortion_optimization}{coding
  efficiency} provided by VQ is marginal in this
context~\cite{vetterli2014foundations}, and requires in general higher
computational resources.

Quantizers can also be classified into
\href{https://en.wikipedia.org/wiki/Quantization_(signal_processing)#Mid-riser_and_mid-tread_uniform_quantizers}{uniform}
and
\href{https://nptel.ac.in/content/storage2/courses/117104069/chapter_5/5_5.html}{non-uniform}~\cite{sayood2017introduction,vetterli2014foundations}. An
uniform quantizer distributes the available representation levels
uniformely over the range of input values. Non-uniform quantizers use
higher density of representation levels (more output levels per input
different values) to those intervals of input values that occur more
often.\footnote{The decision intervals and the representation levels
in each interval can be also optimized using other criteria, such as,
minimizing the rate/distortion at a given point of the RD curve.}
Non-uniform quantizers can also be classified into static and
\href{https://en.wikipedia.org/wiki/Adaptive_differential_pulse-code_modulation}{adaptive
  quantizers}. In the first case, the
\href{https://en.wikipedia.org/wiki/Probability_distribution}{distribution}
of the representation levels remains constant during the quantization
stage, and in the second case, the quantizer parameters are adapted
dinamically to the characteristics of the input signal. In this
milestone we will use an
\href{https://en.wikipedia.org/wiki/Quantization_(signal_processing)#Dead-zone_quantizers}{uniform
  dead-zone scalar static quantizer}, which can be implemented
efficiently (in software) for digital signals. Moreover, dead-zone
quantizers tend to produce more quantization indices equal to 0 (which
increases compression rates) at the cost of generating more
quantization noise for values of the input signal close to 0, or what
is the same, decreasing the
\href{https://en.wikipedia.org/wiki/Signal-to-noise_ratio}{SNR} for
small signal values. A priori, this could be seen as a problem, but in
reality it is not because precisely when the amplitude of the signal
is small and the noise is independent of its amplitude (which usually
happens with
\href{https://en.wikipedia.org/wiki/Noise_(electronics)}{electronic
  noise}), the SNR of the input signal has its lowest value precisely
for those values close to 0. Therefore, the quantizer basically is
going to change the electronic noise by quantization
noise\footnote{The error generated by the quantization stage.} (see
this
\href{https://github.com/vicente-gonzalez-ruiz/signal_quantization}{introduction
  to signal quantization} and this
\href{https://github.com/vicente-gonzalez-ruiz/scalar_quantization}{comparative
  between digital scalar quantizers}). Finally, although this is a
feature that we are not going to exploit for now, dead-zone quantizers
extracts bit-planes when the quantization steps sizes are powers of
two, allowing the design of progressive entropy encoding schemes, if
required.

%}}}

\subsection{(Bit-) Rate control and distortion}
%{{{ 
The number of representation levels used by a quantizer depends
basically on the quantization step (size), typically denoted
by $\Delta$. The higher the $\Delta$, the smaller the number of
representation levels, and therefore, the higher the distortion
generated by the quantization error (see the previous notebooks), and
the smaller the output bit-rate. This generates a
\href{https://en.wikipedia.org/wiki/Rate%E2%80%93distortion_theory}{rate/distortion}
  trade-off that is descriptive of all lossy compressors.

The control of the bit-rate through the $\Delta$ values is a technique
that can be used in real-time transmission systems to minimize the
jitter and the loss of packets when
\href{https://en.wikipedia.org/wiki/Network_congestion}{congestion}
occurs. However, notice that depending of the
\href{https://en.wikipedia.org/wiki/Entropy_(information_theory)}{entropy}
coding stage and the characteristics of the signal
(\href{https://en.wikipedia.org/wiki/Variance}{variance}, entropy) may
not exist a clear relationship between the $\Delta$ and the output
bit-rate. This happens using DEFLATE.

Notice also that any rate-control algorithm based on quantization has a
characteristic RD (Rate/Distortion) curve, in which the X axis
represents the (in the case of InterCom, received) bit-rate and the Y
axis the distortion in the reconstruction (in the case of InterCom,
the played audio sequence) obtained after the quantization. Some
examples can be found
in \href{https://github.com/Tecnologias-multimedia/Tecnologias-multimedia.github.io/blob/master/study_guide/BR_control/audio_quantization.ipynb}{this
notebook}.
%}}}

\subsection{The current implementation(s) for the control of the bit-rate}
%{{{ 
Bit-Rate (BR) control through quantization has been implemented in the
class \verb|BR_Control*| of the modules \texttt{BR\_control*.py}. This
class overrides the inherited methods \verb|pack()| and
\verb|unpack()|, performing now:

\begin{lstlisting}[language=Python]
  def pack(chunk_number, chunk):
    quantized_chunk = quantize(chunk)  # (1)
    packed_chunk = Buffering.pack(chunk_number, quantized_chunk)  # (2)
    return packed_chunk  # (3)
\end{lstlisting}

\begin{lstlisting}[language=Python]
  def unpack(packed_chunk):
    (chunk_number, quantized_chunk) = Buffering.unpack(packed_chunk)  # (1)
    chunk = dequantize(quantized_chunk)  # (2)
    return (chunk_number, chunk)  # (3)
\end{lstlisting}

Notice that, regarding the bit-rate control, you will find four
implementations related to this milestone:
\begin{enumerate}
\item \verb|BR_control_no.py|: Uses a constant
  $\Delta>0$.\footnote{$\Delta$ must be always bigger than $0$, by
definition, and this does not depend on the bit-rate control.} There
  is not BR control.
\item \verb|BR_control_add_lost.py|: Every second runs:
  \begin{equation}
    \Delta = \Delta + L - 1
  \end{equation}
  where $L$ is the number of lost (received) chunks in the previous
  second. Notice that this heuristic (and the following ones) supposes
  that the interlocutor is lossing (on average) the same number of
  chunks.
\item \verb|BR_control_lost.py|: Every second runs:
  \begin{equation}
    \Delta = L - 1.
  \end{equation}
\item \verb|BR_control_conservative.py|: Every second runs:
  \begin{equation}
    \left\{
    \begin{array}{ll}
      \Delta = 2\Delta & \quad\text{if}~L>2 \\
      \Delta = \frac{10}{11}\Delta & \quad\text{always}.
    \end{array}
    \right.
  \end{equation}
\end{enumerate}
%}}}

%}}}

\section{What do you have to do?}
%{{{ 
\subsection{Estimate the bit-rate in an Internet link}
%{{{ 
Usually, we need to use a tool such as \href{https://iperf.fr/}{iPerf}
to
\href{https://en.wikipedia.org/wiki/Measuring_network_throughput}{measure
  the link capacity} between
two \href{https://datatracker.ietf.org/doc/html/rfc4113}{end-points}
(of
different \href{https://en.wikipedia.org/wiki/Host_(network)}{hosts})
in the Internet. This proceduce implies that we must have access to
both hosts to install and run this program, privileges that are not
always available.

Alternatively, we can estimate the
\href{https://en.wikipedia.org/wiki/Throughput}{link throughput} using
\href{https://github.com/torvalds/linux/blob/master/net/ipv4/ping.c}{\texttt{ping}}
(see the
\href{https://tecnologias-multimedia.github.io/study_guide/latency/}{previous
  milestone}). At this point is important to realize that
\texttt{ping} has been designed to measure latencies, not bit-rates,
and that for this reason, we will be able only to estimate
throughputs, rougtly.

The transmission bitrate available between two directly reachable IP
devices can be estimated as
\begin{equation}
  b=\frac{B}{t_t},
  \label{eq:b}
\end{equation}
where $B$ is the number of bits sent in a \verb|ping|
\href{https://en.wikipedia.org/wiki/Payload_(computing)}{payload}, and
\begin{equation}
  t_t = \frac{\text{RTT}_{\text{max}}-2t_p}{2},
  \label{eq:tt}
\end{equation}
an equation that can be determined by revisiting the \verb|ping|
timeline (Fig.~1) and the Eq.~2 of the
\href{https://tecnologias-multimedia.github.io/study_guide/latency/}{previous
  milestone}.

Notice that if the payload of the ping is small enough ($0$ bytes
ideally, and therefore $t_t=0$), we can consider that
\begin{equation}
  t_p \approx \frac{\text{RTT}_{\text{min}}}{2},
  \label{eq:tp}
\end{equation}
and that for measuring the $\text{RTT}$ in Eq.~\eqref{eq:tt}, we
should use a payload as large as possible (ideally, a infinite number
of bytes), in order to make $t_t$ a significative amount of time, ease
to measure. Notice also that we should use the smallest RTT value of a
\texttt{ping} session of measurements in the Eq.~\eqref{eq:tp} because
the minimum link latency is a constant and the congestion of the link
always increases it.

Therefore, supposing that $t_p$ is constant and using the
average\footnote{Remember that with the \texttt{ping} command we
iterate several times and the tool returns average values.}
$\text{RTT}$ for transmitting $B$ bits ($B$ as large as possible), the
bit-rate value $b$ determined by Eq.~\eqref{eq:b} can be considered as
a good estimation for the link of the near future.

So, for determining $b$ we must run the following steps:
\begin{enumerate}
\item Run \texttt{ping} using the smallest possible payload that
  provides RTTs values.\footnote{The minimal payload for a
    \texttt{ping} message can be 0, but depending on the
    implementation of \texttt{ping} you may be forzed to use a larger
    value.} Use it for computing $t_p$.
  
\item Run \texttt{ping} using the largest possible payload to achieve
  that the most part of the average RTT provided by \texttt{ping}
  corresponds to $t_t$. It is necessary to use the highest
  payload-size because the accuracy of the cronometer used by
  \texttt{ping}, in general, is not high enough.
  
\item Use Eq.~\eqref{eq:tt} to find $t_t$ as a function of the average
  RTT and the minimal $t_p$.
  
\item Finally, use Eq.~\eqref{eq:b} to determine $b$.
\end{enumerate}

If the devices cannot be \texttt{ping}-ed, and supposing that there is
only one point in the communication link that is filtering the ICMP
Echo Request traffic in each direction, then we can apply the same
technique that was used in the
\href{https://tecnologias-multimedia.github.io/study_guide/latency/}{previous
  milestone}, which basically consists in computing the total RTT as
the sum of the RTTs from the devices that we want to connect (where
the InterCom is it supposed to be run) to the filter (possiblely, one
of your home routers). A similar throughput estimation should be
obtained considering only the slower part of the total link.

(Optional) In order to gain experience with all this stuff, measure
your bandwith (using your local host) with different \texttt{ping}
servers, try to compute the link throughput:
\begin{enumerate}
\item In \href{https://en.wikipedia.org/wiki/Localhost}{\texttt{localhost}}.
\item In \texttt{localhost}, limiting the bit-rate with \texttt{tc}.
\item Between two hosts in your
  \href{https://en.wikipedia.org/wiki/Local_area_network}{local
    network} (your host and your router, for example). Notice that in
  this case, you should not obtain a bit-rate higher than the capacity
  of your
  \href{https://en.wikipedia.org/wiki/Network_interface_controller}{network
    adapter}.
\item Between your host and a public host of the Internet. Try
  \href{https://www.meter.net/}{\texttt{www.meter.net}} for example.
\end{enumerate}
%}}}

\subsection{Simulate the link}
%{{{ 
With the bit-rate value $b$ computed from the previous section, simulate
such link (locally) with:

\begin{lstlisting}{language=bash}
  sudo tc qdisc add dev lo root handle 1: tbf rate <bit-rate_in_kbps>kbit burst 32kbit limit 32kbit
  sudo tc qdisc add dev lo root parent 1:1 handle 10: netem delay <average_delay_in_miliseconds>ms <maximum_average_deviation_in_miliseconds>ms <Pearson_correlation_coefficient_expressed_as_a_percentage>% distribution <uniform|normal|pareto|paretonormal>
\end{lstlisting}

Examples:

\begin{enumerate}
\item Check the current configuration:
  
  \begin{lstlisting}{language=bash}
    tc qdisc show dev lo
  \end{lstlisting}
  
  The output should be something like:
  
  \begin{lstlisting}{language=bash}
    qdisc noqueue 0: root refcnt 2
  \end{lstlisting}
  
\item Define the rule for the bit-rate control (example for $200$ kbps):
  
  \begin{lstlisting}{language=bash}
    sudo tc qdisc add dev lo root handle 1: tbf rate 200kbit burst 32kbit limit 32kbit
  \end{lstlisting}
  
\item Define the rule for the latency control
  (example for $100$ ms, $10$ ms of jitter, and
  \href{https://wiki.linuxfoundation.org/networking/netem}{with the
    next random transmission depending $25$\% on the last one}):
  
  \begin{lstlisting}{language=bash}
    sudo tc qdisc add dev lo parent 1:1 handle 10: netem delay 100ms 10ms 25% distribution normal
  \end{lstlisting}
  
\item After adding these rules, this should be the configuration:
  
  \begin{lstlisting}{language=bash}
    tc qdisc show dev lo
  \end{lstlisting}
  
  The output should be something like:
  
  \begin{lstlisting}{language=bash}
    qdisc tbf 1: root refcnt 2 rate 200Kbit burst 4Kb lat 0us 
    qdisc netem 10: parent 1:1 limit 1000 delay 100ms  10ms 25%
  \end{lstlisting}

\end{enumerate}

Remember to delete the rules after their are unnecessary (example to delete the previous rules):

\begin{lstlisting}{language=bash}
  sudo tc qdisc delete dev lo parent 1:1 handle 10: netem delay 100ms 10ms 25% distribution normal
  sudo tc qdisc delete dev lo root handle 1: tbf rate 200kbit burst 32kbit limit 32kbit
\end{lstlisting}
%}}}

\subsection{Which data-ordering performs better?}
%{{{ 
Determine empirically which ordering of the chunk data is the most
efficient from a lossless data compression point of view (the smaller
the bit-rates, the higher the compression). Use the audio sequence you
want. Some samples are stored in the \verb|data| directory of
intercom.
%}}}

\subsection{Compute RD (Rate/Distortion) curves}
%{{{ 
Considering the
\href{https://en.wikipedia.org/wiki/Root-mean-square_deviation}{RMSE
  (Root Mean Square Error)} as distortion measure between the sent and
the received audio signal (using only one instance of InterCom),
generate the RD curve considering a set of different simulated
transmission environments (use
\href{https://man7.org/linux/man-pages/man8/tc.8.html}{tc}) of one
audio sequence (remember that you can use one of the samples found in
the \verb|data| directory of the InterCom's repo), for the four
implementations. Notice that you can use the
\verb|--minimal_quantization_step| parameter to generate the different
points of the RD curves.
%}}}

%}}}

\section{Deliverables}

A report documenting the experiments and the results.

\section{Resources}

\bibliography{networking,text-compression,JPEG2000,audio-coding,data-compression,signal-processing}
