\newcommand{\TM}{\href{https://www.ual.es/estudios/grados/presentacion/plandeestudios/asignatura/4015/40154321?idioma=zh_CN}{Tecnologías Multimedia}}

\title{\TM{} - Study Guide - Milestone 13: Considering the Threshold of Hearing}

\maketitle

\section{Description}

\subsection{Absolute Threshold of Humman Hearing}
Psychoacoustics has determined that the HAS (Human Auditory System) has a
sensitivity
(\href{https://en.wikipedia.org/wiki/Absolute_threshold_of_hearing}{threshold
  of human hearing}) that depends on the frequency of the sound. This
basically means that some subbands can be quantized with a larger
quantization step than others without a noticeable increase of the
quantization noise.

\begin{figure}
  \centering
  \svg{graphics/ToHH}{600}
  \caption{Absolute threshold of humman hearing.}
  \label{fig:ToHH}
\end{figure}

A good approximation of the ToHH (Threshold of Humman Hearing) for a 20-years old person can be
obtained with~\cite{bosi2003intro}
\begin{equation}
  T(f)\text{[dB]} = 3.64(f\text{[kHz]})^{-0.8} - 6.5e^{f\text{[kHz]}-3.3)^2} + 10^{-3}(f\text{[kHz]})^4.
  \label{eq:ToHH}
\end{equation}
This equation has been plotted in the Fig.~\ref{fig:ToHH}.

\subsection{DWT subbands and quantization steps}
The number of DWT subbands
\begin{equation}
  N_{\text{sb}} = N_{\text{levels}} + 1
\end{equation}
where $N_{\text{levels}}$ is the number of levels of the DWT. Except for
the $L^{N_{\text{levels}}}$ subband, it holds that
\begin{equation}
  W(H^s) = \frac{1}{2}W(H^{s-1}),
\end{equation}
being $W(\cdot)$ the bandwidth of the corresponding
subband $s$. Therefore, considering that the bandwidth of the audio signal
is 22050 Hz, the bandwidth $B(H^1)$ of the $H^1$ subband is 11025 Hz,
$B(H^2)=22025/4$, and so on. It holds that
\begin{equation}
  B(L^{N_{\text{levels}}}) = B(H^{N_{\text{levels}}}).
\end{equation}

\section{What you have to do?}

\begin{enumerate}
\item Given a $N_{\text{levels}}$, determine the bandwidth of each
  subband. Using these bandwidths, find the interval of frequencies of
  each subband. Using these intervals, the Eq.~\ref{eq:ToHH}, and
  considering that $\Delta$ is the ``master'' quantization step that
  will control the overall loss of quality, determine a suitable
  quantization step $\Delta_s$ for each subband $s$.
\end{enumerate}

\section{Timming}

This is the final milestone. Present your results in the exam time.

\section{Deliverables}

The module threshold.py (inheriting from temporal.py). Store it at the
\href{https://github.com/Tecnologias-multimedia/intercom}{root
  directory} of your InterCom's repo.

\section{Resources}

\bibliography{maths,data-compression,DWT,audio-coding}

