\newcommand{\TM}{\href{https://www.ual.es/estudios/grados/presentacion/plandeestudios/asignatura/4015/40154321?idioma=zh_CN}{Tecnologías Multimedia}}

\title{\TM{} - Study Guide - Milestone 14: Considering the Threshold of Hearing}

\maketitle

\section{Description}

\subsection{Absolute Threshold of Human Hearing}

\sub
\section{What you have to do?}

%and
%  considering that $\Delta$ is the ``master'' quantization step that
%  will control the overall loss of quality,
  
\begin{enumerate}
\item Given a $N_{\text{levels}}$, determine the bandwidth of each
  subband. Using these bandwidths, find the interval of frequencies of
  each subband. Using these intervals and the Eq.~\ref{eq:ToHH},
  determine a suitable quantization step $\Delta_s$ for each subband
  $s$, supposing that the subband with the highest sentibility (those
  that represents audio signals close to $3$ KHz) should not be
  quantized (or quantized with a $\Delta=1$). Use the average of the
  function defined by Eq.~\ref{eq:ToHH} in the corresponding subband
  $s$ as $\Delta_s$.
\item Quantize and dequantize each subband with the corresponding
  $\Delta_s$, using a dead-zone quantizer.
\end{enumerate}

\section{Timming}

This is the final milestone. Present your results in the exam time.

\section{Deliverables}

The module threshold.py (inheriting from temporal\_overlapped\_DWT\_coding.py). Store it at the
\href{https://github.com/Tecnologias-multimedia/intercom/src}{src}
  directory of your InterCom's repo.

\section{Resources}

\bibliography{maths,data-compression,DWT,audio-coding}

