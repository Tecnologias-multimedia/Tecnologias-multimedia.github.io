% Emacs, this is -*-latex-*-

\title{Perceptual Coding Considering the Threshold of Hearing}

\maketitle
\tableofcontents

\section{A model of the Threshold of (Human) Hearing}

Psychoacoustics (see
\href{https://vicente-gonzalez-ruiz.github.io/the_sound/}{the sound},
\href{https://vicente-gonzalez-ruiz.github.io/human_auditory_system/}{the
  human auditory system}, and
\href{https://vicente-gonzalez-ruiz.github.io/human_sound_perception/}{the
  human sound perception}) has determined that the HAS (Human Auditory
System) has a sensitivity that depends on the frequency of the sound,
the so called ToH
(\href{https://en.wikipedia.org/wiki/Absolute_threshold_of_hearing}{Threshold
  of (Human) Hearing}). This basically means that some subbands
(intervals of frequencies) can be quantized with a larger quantization
step than others without a noticeable increase (from a perceptual
perspective) of the quantization noise~\cite{sayood2017introduction}.

\begin{figure}
  \centering
  %\svg{graphics/ToHH}{1800}
  \myfig{graphics/ToHH}{47cm}{1400}
  \caption{A model for the threshold of human hearing.}
  \label{fig:ToHH}
\end{figure}

A good approximation of ToH for a 20-year-old person can be
obtained with~\cite{bosi2003intro}
\begin{equation}
  T(f)\text{[dB]} = 3.64(f\text{[kHz]})^{-0.8} - 6.5e^{f\text{[kHz]}-3.3)^2} + 10^{-3}(f\text{[kHz]})^4.
  \label{eq:ToHH}
\end{equation}
This equation has been plotted in Fig.~\ref{fig:ToHH}.

\section{Dyadic DWT subbands and quantization steps}
The number of dyadic DWT subbands
\begin{equation}
  N_{\text{sb}} = N_{\text{levels}} + 1
\end{equation}
where $N_{\text{levels}}$ is the number of levels of the
dyadic DWT~\cite{vetterli1995wavelets}. Except for the
${\mathbf l}^{N_{\text{levels}}}$ subband (the lowest-pass frequency
of the decomposition), it holds that
\begin{equation}
  W({\mathbf w}_s) = \frac{1}{2}W({\mathbf w}_{s-1}),
\end{equation}
being $W(\cdot)$ the bandwidth of the corresponding
subband. Therefore, considering that (by default, in InterCom) the
bandwidth of the audio signal is $22050$ Hz, the bandwidth
$W({\mathbf w}_1)=11025$ Hz, $W({\mathbf w}_2)=22025/4$, etc. It also
holds that
\begin{equation}
  W({\mathbf l}^{N_{\text{levels}}}) = W({\mathbf w}^{N_{\text{levels}}}).
\end{equation}

The idea is to decide, knowing the frequencies represented in each DWT
subband and the ToH curve (see
\href{https://github.com/Tecnologias-multimedia/InterCom/blob/master/docs/2-hours_seminar.ipynb}{
  InterCom: a Real-Time Digital Audio Full-Duplex
  Transmitter/Receiver}), the QSS (Quantization Step Size) that should
be applied to each subband.

This idea is already implemented in a module named \verb|dyadic_ToH.py|.

\section{Deliverables}

Subjectively compare the audio quality obtained by
\verb|dyadic_ToH.py| and its predecessor,
\verb|temporal_overlapped_DWT_coding.py|. Subjectively means that, in
groups, you must determine, for the same bit-rate and content
configuration, which implementation sounds better.

Mark: \textbf{0 points}.

\section{Resources}

\bibliography{maths,data_compression,DWT,audio_coding}

