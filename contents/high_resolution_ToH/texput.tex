% Emacs, this is -*-latex-*-

\title{A Higher Resolution Subband Decomposition of the Threshold of Hearing}

\maketitle

\section{Description}

The frequency resolution of the dyadic subband
partitioning~\cite{vetterli1995wavelets} generated by the DWT could not
be high enough to map the ToH curve accurately. To overcome this, we
will decompose each DWT subband into a configurable number of new
subbands where all of them will have the same bandwidth. Use the
\href{https://pywavelets.readthedocs.io/en/latest/ref/2d-decompositions-overview.html#wavelet-packet-transform}{Wavelet
  Packet Transform} to perform the subband decomposition of each
wavelet subband.

\section{Deliverables}

\subsection{Implement \texttt{advanced\_ToH.py}}

Using the new subband decomposition proposed previously, find the
corresponding quantization step sizes as it is done in the module
\verb|basic_ToH.py|, and use them. Store your implementation in a new
module named \verb|advanced_ToH.py|. Remember that the quantization
noise~\cite{sayood2017introduction} can be higher in those subbands
where the threshold of hearing is higher.

\begin{comment}
\subsection{Subjective performance}

\begin{enumerate}
\item Using a recording tool such as
  \href{http://audacity.sourceforge.net}{Audacity} or
  \href{http://plugin.org.uk/timemachine/}{JACK Timemachine}, record
  the simulated transmission of a piece of audio and create a
  \texttt{.wav} file, when the audio has been transmitted using
  \texttt{temporal\_overlapped\_DWT\_coding.py} and
  \texttt{threshold.py}, using in both cases the same transmission
  bit-rate. Vary the quantization step size for controlling the
  bit-rate.
\item Determine which audio sounds better from a subjective point of
  view. Repeat this step the number of times you consider necessary.
\end{enumerate}
\end{comment}

\section{Resources}

\bibliography{maths,data_compression,DWT,audio_coding}

