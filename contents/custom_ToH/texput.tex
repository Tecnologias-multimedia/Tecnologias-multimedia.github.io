% Emacs, this is -*-latex-*-

\title{Custom ToH (Threshold of Hearing) Curves}

\maketitle
\tableofcontents

\section{Not everyone hears the same}

The ToH curve~\cite{bosi2003intro} varies between individuals:
\begin{enumerate}
\item In general, women hear better than men.
\item With age, we lose sensitivity to high frequencies.
\item Exposure to loud noises over time can elevate the ToH,
  especially in individuals exposed to loud sound.
\item Auditory training can help detect sounds at lower intensities or
  distinguish subtle nuances in tone.
  % \item Short-duration sounds may require higher intensities to be
  %   perceived compared to longer-duration ones. -> a un nuevo
  %   hito. https://en.wikipedia.org/wiki/Critical_band
\end{enumerate}

And this can be said without considering your local audio
infraestructure.\footnote{For example,
  your speakers could not have a flat frequency response, or your room
  could attenuate some frequencies.}

Therefore, it is quite improbable that, at least in practice, you
have a ToH identical to the theoretical curve shown in the previous
milestone.

\section{Why to use custom ToH curves?}

If you are talking to a semi-deaf interlocutor, why would you send
them auditory information that they will not be able to perceive? For
this reason, this step proposes the calculation of your own ToH curve,
information that must be transmitted to your interlocutor so that they
can use the quantization steps adapted to your perception.

\section{A procedure to determine your YoH curve}

An optimal-user-specific ToH
should take into consideration your noticeable
\href{https://en.wikipedia.org/wiki/Quantization_(signal_processing)}{quantization
  noise} in each subband, defining a set of QSSs (one per subband). To
find such set, the following algorithm can be used:

\begin{enumerate}
\item %Let
  %$\{{\mathbf l}^{N_{\text{levels}}}, {\mathbf h}^{N_{\text{levels}}},
  %{\mathbf h}^{N_{\text{levels}}-1},\cdots, {\mathbf h}^1\}$ the
  %wavelet representation of a
  %chunk. %, being ${\mathbf l}^{N_{\text{levels}}}$ the lowest frequency subband.
  Starting with the lowest frequency subband (at the first
  iteration the rest of subbands are zero). While the noise
  (suppose that the quantization noise follows an
  \href{https://en.wikipedia.org/wiki/Continuous_uniform_distribution}{uniform
    distribution}) is imperceptible:
  \begin{enumerate}
  \item Increase the amplitude of the noise in the subband.
  \item Compute the inverse transform, generating a chunk of audio.
  \item Reproduce the generated chunk, alternating every second with
    a silence (the play of an empty chunk).
  \end{enumerate}
\item Continue with the next subband, but keeping the
  highest unperceptible noise in the previously processed
  ones.
\end{enumerate}

It's a good idea to have the option of determining the ToH of a given
subband, without reconsidering the rest of them.

\section{Deliverables}

Implement the previous procedure in a Python module named
\verb|create_ToH.py|. The output of such module should be a text file,
named \verb|custom_ToH.txt|, with three columns (example):
\begin{verbatim}
# Initial_frequency_in_Herts Band-width_in_Herts ToH
20                           50                  100
70                           60                  80
130                          80                  60
210                          100                 50
310                          150                 40
460                          180                 30
:                            :                   :
\end{verbatim}

Modify the module \verb|dyadic_ToH.py| to use the file
\verb|custom_ToH.txt|, when available.

\section{Resources}

\bibliography{maths,data_compression,DWT,audio_coding}

