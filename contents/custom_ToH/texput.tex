% Emacs, this is -*-latex-*-

\title{Custom Thresholds of Hearing}

\maketitle

\section{Description}
In general, the ToH (Threshold of Hearing) is different for each
person. Added to this, not all the auditory information that is
received from our iterlocutor is transduced with the same
fidelity\footnote{For example, your speakers could not have a flat
  frequency response, or your room could attenuate more, some
  frequencies.} Knowing this, we can take advantage of this fact by
avoiding having to transmit information that we will not be able to
perceive. In other words, we can compute a set of quantization step
sizes that our interlocutor can use to quantize the audio that he/she
is going to send us.

To find such QSSs, we can simulate
\href{https://en.wikipedia.org/wiki/Quantization_(signal_processing)}{quantization
  noise}~\cite{sayood2017introduction} in each
subband~\cite{vetterli1995wavelets} and use the amplitude of the noise
when it starts to be noticeable. Supposing that the quantization noise
follows an
\href{https://en.wikipedia.org/wiki/Continuous_uniform_distribution}{uniform
  distribution}, the next algorithm can be implemented:

\begin{enumerate}
\item Let
  $\{{\mathbf l}^{N_{\text{levels}}}, {\mathbf h}^{N_{\text{levels}}},
  {\mathbf h}^{N_{\text{levels}}-1},\cdots, {\mathbf h}^1\}$ the
  wavelet representation of a
  chunk. %, being ${\mathbf l}^{N_{\text{levels}}}$ the lowest frequency subband.
  Starting with ${\mathbf l}^{N_{\text{levels}}}$ (at the first
  iteration the rest of subbands are zero), split the wavelet subband
  into a (configurable) number of Fourier subbands. While the noise stays
  imperceptible:
  \begin{enumerate}
  \item Increase the amplitude of the noise in the Fourier subband.
  \item Compute the inverse FFT/wavelet-transform.
  \item Reproduce the generated chunk, alternating every second with
    the play of an empty chunk.
  \end{enumerate}
\item Continue with the next subband, but keeping the
  highest unperceptible uniform noise in the previously processed
  ones. In this way, for the subband ${\mathbf h}^1$ (the last
  one to be processed), the rest of subbands should be already
  noisy.
\end{enumerate}

Again, notice that the QSSs are determined for the sound that you are
going to play (not for the audio that you are generating). Therefore,
you should use your interlocutor's QSSs and vice versa. It is also
necessary the transmission of this information.

Finally, use the new QSSs instead of the default ones (those generated
using the standard ToH).

\section{Deliverables}

\subsection{Implement \texttt{find\_custom\_ToH.py}}

Provide an implementation of the previous algorithm for determining
your ToH, and sending it to your interlocutor.

\subsection{Implement \texttt{use\_custom\_ToH.py}}

Provide an implementation that receives the ``custom'' QSSs, and use to
encode the audio that you are sending.

\section{Resources}

\bibliography{maths,data_compression,DWT,audio_coding}

