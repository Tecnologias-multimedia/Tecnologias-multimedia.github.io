\newcommand{\TM}{\href{https://www.ual.es/estudios/grados/presentacion/plandeestudios/asignatura/4015/40154321?idioma=zh_CN}{Tecnologías Multimedia}}

\title{Linux audio basics and \texttt{sounddevice}}

\maketitle

\section{The Linux Audio (probably partial) Stack}

\begin{center}
\begin{tabular}{|c|c|}
\hline
\multicolumn{2}{c}{\href{https://en.wikipedia.org/wiki/List\_of\_Linux\_audio\_software}{Linux Audio Applications}} \\
\hline
\href{https://jackaudio.org/}{JACK} & \href{https://www.freedesktop.org/wiki/Software/PulseAudio/}{Pulse Audio} \\
\hline
\multicolumn{2}{c}{\href{https://www.alsa-project.org}{ALSA}} \\
\hline
\multicolumn{2}{c}{\href{https://www.alsa-project.org/wiki/Matrix:Main}{Sound hardware}} \\
\hline
\end{tabular}
\end{center}

\section{Advanced Linux Sound Architecture (ALSA)}
ALSA is a software
\href{https://docs.kernel.org/sound/kernel-api/index.html}{framework}
and part of the Linux kernel (as a set of
\href{https://wiki.archlinux.org/title/Kernel_module}{modules}) that
provides an application programming interface (API), in
\href{https://en.wikipedia.org/wiki/C_(programming_language)}{C}, for
sound device drivers~\cite{phillips2005user}. For example, to show the
modules related with ALSA currently loaded, run:
\begin{verbatim}
lsmod | grep '^snd' | column -t
\end{verbatim}

\subsection{\texttt{amixer} and \texttt{alsamixer}}
\texttt{\href{https://linux.die.net/man/1/amixer}{amixer}} (through the command line) and \texttt{\href{https://en.wikipedia.org/wiki/Alsamixer}{alsamixer}} (using also the command line but being ncurses-based) can be used to control the gain of the audio inputs and outputs. For example, to run \texttt{alsamixer}, in your terminal write:
\begin{verbatim}
alsamixer
\end{verbatim}

\subsection{\texttt{arecord}}
\texttt{\href{https://linux.die.net/man/1/arecord}{arecord}} runs on
ALSA to capture
\href{https://en.wikipedia.org/wiki/Pulse-code_modulation}{PCM (Pulse
  Code Modulation)} (by default, using the
\href{https://en.wikipedia.org/wiki/WAV}{WAV} audio format) audio
data, and provides information about the hardware, and also, about the
audio servers. Some examples of use are:

\begin{itemize}

\item List the physical audio \href{https://en.wikipedia.org/wiki/Analog-to-digital_converter}{ADC}(s):
\begin{verbatim}
arecord -l      
\end{verbatim}

\item Show information about the ADC(s) capabilities (if \href{https://www.freedesktop.org/wiki/Software/PulseAudio/}{PulseAudio} is not running, remove \texttt{pasuspender --}):
\begin{verbatim}
pasuspender -- arecord -D hw:0,0 --dump-hw-params
\end{verbatim}

\item Idem, but using the default audio server (usually, PulseAudio):
\begin{verbatim}
arecord --dump-hw-params > /dev/null
\end{verbatim}
Usually, the software capabilities are higher than the hardware ones.

\item List the virtual ADC(s) (\href{https://alsa.opensrc.org/Pcm-device}{PCM} outputs):
\begin{verbatim}
arecord -L
\end{verbatim}

\item Record raw (without header) audio in \href{https://en.wikipedia.org/wiki/Compact_Disc_Digital_Audio#Audio_format}{CD audio format}:
\begin{verbatim}
arecord -t raw -f cd > signed_int16_bits__little_endian__44100_frames_per_second__2_samples_per_frame.raw
\end{verbatim}

\end{itemize}

\subsection{\texttt{aplay}}
\texttt{\href{https://linux.die.net/man/1/aplay}{aplay}} is the
companion PCM player of \texttt{arecord}. The playing capabilities
used to be exactly the same than the provides by \texttt{arecord}.

\begin{itemize}

\item List the physical audio \href{https://en.wikipedia.org/wiki/Digital-to-analog_converter}{DAC}(s):
\begin{verbatim}
aplay -l      
\end{verbatim}

\item Idem, but listing the virtual DACs (PCM outputs):
\begin{verbatim}
aplay -L
\end{verbatim}

\item (Recording and) Playing directly through ALSA, using the physical DAC and DAC:
\begin{verbatim}
arecord -t raw -f cd | aplay -t raw -f cd
pasuspender -- arecord -D hw:0,0 -t raw -f cd | pasuspender -- aplay -D hw:0,0 -t raw -f cd
\end{verbatim}

\item (Recording and) Playing throught the default sound server (usually PulseAudio), using the virtual DAC and DAC:
\begin{verbatim}
arecord -t raw -f cd | aplay -t raw -f cd
\end{verbatim}
\end{itemize}

\subsection{\texttt{speaker-test}}
\texttt{\href{https://linux.die.net/man/1/speaker-test}{speaker-test}} allows to generate (by default) \href{https://en.wikipedia.org/wiki/Pink_noise}{pink noise} or a pure tone through the audio outputs. Examples:
\begin{verbatim}
speaker-test                       # Analog (copper or Bluetooth, if available) mono
speaker-test -c 2                  # Analog (copper of Bluethooh, if available) stereo
speaker-test -Dplug:spdif -c2      # Digital (SPDIF) stereo, if available
speaker-test -c 8                  # Surround (usually through HDMI) 7.1, if available
\end{verbatim}

\section{PulseAudio}

\href{https://www.freedesktop.org/wiki/Software/PulseAudio/}{PulseAudio
  is a sound server system for POSIX OSes, meaning that it is a proxy
  for your sound applications. It is an integral part of all relevant
  modern Linux distributions and is used in various mobile devices, by
  multiple vendors. It performs advanced operations on sound data as
  it passes between your application and hardware. Things like
  transferring audio to a different machine, changing the sample
  format or channel count, or mixing several sounds into one
  input/output, are easily achieved using
  PulseAudio.}~\cite{newmarch2017pulseaudio}

Like ALSA, PulseAudio is dinamically configured using
PulseAudio\href{https://www.freedesktop.org/wiki/Software/PulseAudio/Documentation/User/Modules/}{modules}. \texttt{\href{https://linux.die.net/man/1/pactl}{pactl}}
is the command line tool for loading and downloading modules. For
example, to use the equalizer
\texttt{\href{https://www.freedesktop.org/wiki/Software/PulseAudio/Documentation/User/Equalizer/}{qpaeq}}
that should be already installed in your computer if you are using
PulseAudio, run:

\begin{verbatim}
pactl load-module module-equalizer-sink
pulseaudio --kill && pulseaudio --start
qpaeq &
\end{verbatim}
% Ver también: pulseaudio-equalizer-ladspa

Another important PulseAudio application is the
\texttt{\href{https://freedesktop.org/software/pulseaudio/pavucontrol/}{pavcontrol}}
that is a mixer/VU-meter that handles sound applications, input and
output audio devices.

\section{JACK}


\section{sounddevice}
\href{https://python-sounddevice.readthedocs.io}{sounddevice} is a
Python \href{https://docs.python.org/3/tutorial/modules.html}{module}
that provides bindings for the
\href{http://www.portaudio.com/}{PortAudio
  library}~\cite{sounddevice}.

Let's see some examples of what sounddevice can do:

\subsection*{Wiring the ADC and the DAC using a loop}
\lstinputlisting[language=Python]{/home/vruiz/intercom/test/sounddevice/wire5.py}

\subsection*{Wiring the ADC and the DAC using an interruption handler}
\lstinputlisting[language=Python]{/home/vruiz/intercom/test/sounddevice/wire.py}

\section{Measuring latencies}
Both examples are ``wiring'' programs which input digital audio
(through the
\href{https://en.wikipedia.org/wiki/Analog-to-digital_converter}{ADC}
of your computer) and, as soon as possible, outputs the audio (through
the
\href{https://en.wikipedia.org/wiki/Digital-to-analog_converter}{DAC}), which means that we will listen the signal captured by the mic(rophone) after a delay, that in this context is usually called \href{https://en.wikipedia.org/wiki/Latency_(engineering)}{latency}. In our implementations, the latency can be controlled by variying the buffer size. In the terminology of \texttt{sounddevice}, the number of \href{https://python-sounddevice.readthedocs.io/en/0.3.12/api.html}{frames} that are in a chunk. Notice that each frame represents a stereo sample using the audio CD format (2 channels, 16 bits/sample, signed integers where the ausence of energy is represented by the zero value).

\begin{enumerate}

\item Refresh some (probably high-school) ideas about
  \href{https://vicente-gonzalez-ruiz.github.io/the_sound/}{the
    sound}, \href{https://vicente-gonzalez-ruiz.github.io/human_auditory_system/}{the
    human auditory system},
    and \href{https://vicente-gonzalez-ruiz.github.io/human_sound_perception/}{the
    human sound perception}. %Notice that we are able to perceive
    %sounds whose frequency ranges between 20 Hz and 20 KHz
    %(approximately), and that a speaker is basically a microphone, and
    %viceversa.
  
\item Download the Python
  \href{https://docs.python.org/3/tutorial/modules.html}{module}
  \href{https://raw.githubusercontent.com/Tecnologias-multimedia/intercom/master/test/sounddevice/wire3.py}{wire3.py} with:

  \begin{lstlisting}[language=Bash]
sudo apt install curl
curl https://raw.githubusercontent.com/Tecnologias-multimedia/intercom/master/test/sounddevice/wire5.py > wire5.py
  \end{lstlisting}      

\item If you want to run this module right now and you are not using a
  \href{https://en.wikipedia.org/wiki/Headset_(audio)}{headset}, check
  first that the output volumen of your
  \href{https://en.wikipedia.org/wiki/Headset_(audio)}{speakers} is
  not too high, otherwise you could involuntary
  \href{https://www.youtube.com/watch?v=rI90lhYAffo}{``couple''} the
  speaker and the
  \href{https://en.wikipedia.org/wiki/Microphone}{mic(rophone)} of
  your computer, producing a loud and annoying
  \href{https://www.cirrusresearch.co.uk/blog/2012/03/tonal-noise-analysis-with-the-optimus-green-sound-level-meters/}{tonal
    sound}. In order to mitigate this
  effect, you can also control the gain of your mic (if the
  gain is 0, no feedback between the speaker and the mic will be
  possible). In Xubuntu, these controls are available through clicking in
  the speaker icon (situated in the top-right corner of your screen)
  of the \href{https://www.xfce.org/}{Xfce window manager}.

\item Run the module with:

  \begin{lstlisting}[language=Bash]
pip install sounddevice  # Only once
python wire5.py
  \end{lstlisting}

  Stop (killing) the module by clicking the CTRL- and c-keys (CTRL+c),
  simultaneously.

  %If you obtain the message \texttt{ALSA lib
  %  pcm.c:8526:(snd\_pcm\_recover) underrun occurred}, ignore it (this
  %is a consequence of that you are using
  %\href{https://www.freedesktop.org/wiki/Software/PulseAudio/}{PulseAudio}
  %instead of \href{https://alsa-project.org/wiki/Main_Page}{ALSA}
  %directly, and the used \href{https://www.kernel.org/}{kernel} has
  %been compiled without
  %\href{https://en.wikipedia.org/wiki/RTLinux}{real-time process
  %  scheduling}).


\item Now, lets compute, experimentally, the latency experimented by
  \texttt{wire5.py}.

  \begin{enumerate}
  \item First, we need the
    tools: \href{http://sox.sourceforge.net/}{SoX}, \href{https://www.audacityteam.org/}{Audacity},
    and \href{https://raw.githubusercontent.com/Tecnologias-multimedia/intercom/master/test/sounddevice/plot_input.py}{plot\_input.py}:
    
    \begin{lstlisting}[language=Bash]
      sudo apt install sox
      sudo apt install audacity
      curl https://raw.githubusercontent.com/Tecnologias-multimedia/intercom/master/test/sounddevice/plot_input.py > plot_input.py
    \end{lstlisting}
    
  \item Run:
    
    \begin{lstlisting}[language=Bash]
pip install matplotlib
python plot_input.py
    \end{lstlisting}

    and check that the gain of the mic does not
    produce \href{https://en.wikipedia.org/wiki/Clipping_(audio)}{clipping}
    during the sound recording.

  \item \label{start_point} In a terminal, run:

    \begin{lstlisting}[language=Bash]
python wire3.py
    \end{lstlisting}

    while you control the output volume of the speakers to produce a
    decaying coupling noisy effect between both devices (the
    speaker(s) and the mic). If your desktop has not these
    \href{https://en.wikipedia.org/wiki/Transducer}{transducers}, we
    can use a
    \href{https://www.google.com/search?q=male+to+male+audio+jack+cable&client=firefox-b-d&sxsrf=ALeKk00GZUDGqiOfc0D8xkA_MIYgCuZmSA:1600270049146&source=lnms&tbm=isch&sa=X&ved=2ahUKEwjdvsu-_u3rAhXl0eAKHS90DUoQ_AUoAXoECA0QAw&biw=4288&bih=972}{male-to-male
      jack audio cable} and connect the line-output of your soundcard
    to the input of your sound card.

  \item In a different terminal (keep \texttt{python wire3.py}
    running), run:

    \begin{lstlisting}[language=Bash]
sox -t alsa default test.wav
    \end{lstlisting}

    to save the ADC's output to the file \texttt{test.wav}.

  \item While \texttt{sox} is recording, produce some short sound (for
    example, hit your laptop or your micro with one or your nails). Do
    this at least a couple of times more, to be sure that you record
    the sound and also the feedback of such sound. It's important the
    sound to be short (a few miliseconds) in order to visually
    recognize it and it's replicas.

  \item Stop \texttt{sox} by pressing the CTRL+c (at
    the same time). This kills \texttt{sox}.

  \item Kill \texttt{python wire3.pt} with CTRL+c.

  \item Load the sound file into Audacity:
    
    \begin{lstlisting}[language=Bash]
      audacity test.wav
    \end{lstlisting}

  \item Localize the first one of your hitting-nail sounds in the
    \href{https://manual.audacityteam.org/man/audio_tracks.html}{audio
      track} of Audacity.

  \item Select (using the mouse) the region that contains your sound and a replica.

  \item Use the
    \href{https://manual.audacityteam.org/man/edit_toolbar.html#zoom_to_selection}{zoom
      to selection buttom} to zoom-in the selected area.

  \item Measure the time between the ocurrence of the hit (of the
    nail) and the recording of its first replica produced by the
    speaker-to-the-mic feedback. This time is the real latency of your
    computer runing \texttt{wire3.py}.

  \item Modify the constant $\mathtt{CHUNK\_SIZE}$ in the module and
    repeat this process, starting at the Step
    \ref{start_point}. Create an ASCII file (named
    \texttt{latency\_vs\_chunk\_size.txt}) with the content (use
    TAB-ulators to space the columns):
\begin{verbatim}
# CHUNK_SIZE    real
32              ...
64              ...
128             ...
256             ...
512             ...
1024            ...
2048            ...
4096            ...
8192            ...
\end{verbatim}
    with the real (practical) latency.
  \end{enumerate}

\item At this point, we know the real latency of \texttt{wire3.py} as
  a function of $\mathtt{CHUNK\_SIZE}$. Plot the file
  \texttt{latency\_vs\_chunk\_size.txt} with:

  \begin{lstlisting}[language=Bash]
 sudo apt install gnuplot
 echo "set terminal pdf; set output 'latency_vs_chunk_size.pdf'; set xlabel 'CHUNK\_SIZE (frames)'; set ylabel 'Latency (seconds)'; plot 'latency_vs_chunk_size.txt' title '' with linespoints" | gnuplot
  \end{lstlisting}

\item Let's compute now the buffering latency of a chunk (the
  chunk-time). If $\mathtt{sampling\_rate}$ is the number of frames
  per second during the recording process, it holds that:
  
%\begingroup
%\Large
  \begin{equation}
    %\displaystyle
    \mathtt{minimal\_buffering\_latency} = \mathtt{CHUNK\_SIZE} / \mathtt{sampling\_rate}
  \end{equation}
  %\endgroup

  Add these calculations to \texttt{latency\_vs\_chunk\_size.txt} using
  a third column (remember to use TABs).
\begin{verbatim}
# CHUNK_SIZE    real    minimal
:               :       :
\end{verbatim}

\item Plot both latencies:

  \begin{lstlisting}[language=Bash]
echo "set terminal pdf; set output 'latency_vs_chunk_size.pdf'; set xlabel 'CHUNK\_SIZE (frames)'; set ylabel 'Latency (seconds)'; set key left; plot 'latency_vs_chunk_size.txt' using 1:2 title 'Real' with linespoints, 'latency_vs_chunk_size.txt' using 1:3 title 'Minimal' with linespoints" | gnuplot
  \end{lstlisting}
  
\item Which seems to be the minimal practical (real) latency (the
  latency obtained ideallly when $\mathtt{CHUNK\_SIZE}=1$ ... however,
  notice that depending on your computer, this chunk size can be too
  small, overwhelming the CPU) in your computer?  Justify your
  answers. \label{question}

\end{enumerate}

    
      
\section{Resources}

\bibliography{python,sound}
