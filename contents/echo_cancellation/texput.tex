% Emacs, this is -*-latex-*-

\title{Echo Cancellation}

\maketitle

\section{Description}

\subsection{The simplified problem}

One of the first problems we encounter with the use the
\texttt{buffer.py} module\footnote{And obviously, any other parent
  version of \texttt{buffer.py}.} is that, if we don't use headphones,
the sound that comes out of our PC's speaker reaches our microphone
and some time later, that sound reaches our interlocutor in the form
of an echo\footnote{Notice that the output of our speaker is the voice
  of our interlocutor.}, which is reproduced by his/her speaker, which
can be captured again by his microphone, and so on, until generating a
rather unpleasant feedback. In other words, if ${\mathbf s}$ is the
signal played by the speaker, ${\mathbf n}$ is the signal generated by
the ``near-end'' person\footnote{Our interlocutor would be the
  ``far-end'' person.}, and ${\mathbf m}$ is the signal recorded by
our microphone, we have that
\begin{equation}
   {\mathbf m}_i = {\mathbf n}_i + a{\mathbf s}_{i-d},
  \label{eq:echo_problem}
\end{equation}
where ${\mathbf m}_i$ is the $i$-th sample of the signal
${\mathbf m} = \{{\mathbf m}_i\}$, $a$ is a real number that expresses
an attenuation\footnote{Usually, $a<1$ because in general the amount
  of signal that our micro captures from our speaker is smaller (in
  \href{https://en.wikipedia.org/wiki/Energy_(signal_processing)}{energy}~\cite{vetterli2014foundations})
  than the signal ${\mathbf s}$.}, and $d$ is an integer number that
indicates the delay (in sample-times\footnote{A sample-time is the
  interval of time that separates two consecutive samples (or frames)
  and depends on the sampling frequency.}) that exists between when
the received signal ${\mathbf s}$ is reproduced by our speaker until
such signal is captured by our ADC (Analog Digital Converter).

\subsection{The trivial partial solution}
One way to minimize this problem\footnote{Apart from using headphones,
  which is by far the better solution.} is to reduce as much as
possible (as long as it is audible, of course) the gain of the
amplifier that feeds our speaker(s). In terms of the
Eq.~\eqref{eq:echo_problem}, this means to make $a$ close to
zero. Unfortunately, this is not always possible, among other reasons,
because if the volume is too low, we will not hear our interlocutor
(the ``far-end'').

\subsection{Delay-and-substract solution}
Another simple\footnote{The AEC (Audio Echo Cancellation) problem has
  been extensively estudied and there are several solutions. This one
  is probably the simplest one.} solution is to determine $a$ and $d$,
and compute
\begin{equation}
  {\mathbf n}_i = {\mathbf m}_i - a{\mathbf s}_{i-d}.
  \label{eq:echo_cancellation}
\end{equation}

$d$ can be found by measuring the time that a played signal spends to
be recorded by our mic, and $a$ computing the ratio between the
energies of ${\mathbf s}$ (the played signal) and ${\mathbf m}$ (the
recorded signal). For example, Skype finds $d$ and $a$ at the
beginning of the session using a ``call signal'' (a sequence of
more-or-less tonal sounds).

\subsection{Considering the frequency respose of the near audioset}
In general, Eq.~\eqref{eq:echo_problem} is oversimplified because our
local audioset (speaker, mic, walls, our body, etc.) does not have a
flat response in the frequency domain. Therefore, a more realistic
model of the echo effect is
\begin{equation}
   {\mathbf m}_i = {\mathbf n}_i + \{f({\mathbf s})\}_{i-d},
  \label{eq:more_realistic_echo_problem}
\end{equation}
where $f({\mathcal s})$ is a filtered version of the signal
${\mathbf s}$, in which some frequencies are partially attenuated
(notice that now there is a possiblely different $a$-value for each
frequency of ${\mathbf s}$).

Now the problem (apart from finding $d$, obviously) is how to
determine $f(\cdot)$. Again, a simple way of doing this is to use a
call signal to measure the frequency response of the near-end
audioset. For this, a good call signal can be a sequence of
impulses\footnote{A impulse generates a flat
  \href{https://en.wikipedia.org/wiki/Spectral_density}{spectrum}~\cite{kovacevic2013fourier,Oppenheim2}
  for a short period of time.} of a sequence of uniform random noise
signals\footnote{Uniform random noise (or
  \href{https://en.wikipedia.org/wiki/White_noise}{white noise}) has a
  flat spectrum.}. Notice that the frequency response (the filter
coefficients of $f(\cdot)$) can be found comparing the flat spectrum
signal ${\mathbf s}$ generated by the speaker with the supposedly
non-flat spectrum of the signal ${\mathbf m}$ captured by the
microphone. To apply $f(\cdot)$ we can multiply
${\mathcal F}({\mathbf s})$ by the filter coefficients of $f(\cdot)$
in the Fourier domain, being ${\mathcal F}(\cdot)$ the Fourier
Transform~\cite{kovacevic2013fourier,Oppenheim2}.

\subsection{Continuous monitoring of $d$ and $f(\cdot)$}
Both parameters can change if we change, for example, the inclination
of the screen of our laptop. Therefore, $d$ and $f(\cdot)$ (or only
$a$ in the case of using the simplified model described in the
Eq.~\eqref{eq:echo_problem}) should be monitorized constantly, maybe
with a cadence of one second.

\section{Deliverables}
\begin{enumerate}
\item A Python module called \texttt{echo\_cancellation.py} that
  inherits from \texttt{buffer.py} and that implements at least one of
  the previously described solutions.
\item A working experimental setup using a ``localhost connection'' to
  check the performance of your implementation. You will requested to
  run your experiment during the presentation.
\end{enumerate}
\section{Resources}

\bibliography{signal_processing}

\begin{comment}
http://www.seas.ucla.edu/dsplab/index.html
https://es.mathworks.com/help/signal/ug/echo-cancelation.html
https://dsp.stackexchange.com/questions/26617/echo-cancelling-using-autocorrelation-function
https://pypi.org/project/adaptfilt/
http://www.diva-portal.org/smash/get/diva2:280596/fulltext01
https://github.com/ThomasHaubner/e2e_dnn_ad_control_for_lin_aec
https://scicoding.com/4-ways-of-calculating-autocorrelation-in-python/
\end{comment}