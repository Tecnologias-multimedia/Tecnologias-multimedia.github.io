% Emacs, this is -*-latex-*-

\title{NAT Traversal}

\maketitle

\section{NAT-ing for what?}

One of the main problems that applications that transmit data over the
Internet must face is the use of NATs (Network Address Translators). A
NAT is a network device that converts public end-points\footnote{An
  end-point is a tuple made up of an IP address and a port.} to
private end-points and vice versa. Due to this, the consumption rate
of IPv4 addresses has dropped drastically and has meant that IPv6 is
still not the dominant network communication protocol, as expected.

The real reason for this is that only public servers really need a
public IP address, and the vast majority of Internet users use private
IP addresses on their local devices. This basically means that in each
of the companies that do not offer IP services from their corporate
network, and that in each of our homes, where we do not install public
servers either, we are reusing the same range of private IP addresses
over and over again.

\section{Negative consequences}

The main drawback of using NAT devices is that for two network
applications, A and B, that use IP and run on two hosts belonging to
two different private networks, with NAT-A and NAT-B respectively, to
be able to communicate, it is necessary for A to know the end-point
used in NAT-B for incoming traffic from NAT-A to B (through NAT-A),
and obviously, B must know the end-point used in NAT-A for incoming
traffic to A (from NAT-B).

\section{NAT configuration}

But how do we tell NAT-A to let traffic to A from NAT-B through, and
vice versa? Well, there are basically 2 alternatives:

\begin{enumerate}
\item Open ports ``by hand'', in both NATs, to indicate the public
  port open for incoming traffic to the process. For example, if A
  runs on end-point 192.168.1.2:4000 and the open public port is 5000,
  all UDP traffic arriving at port 5000 from the Internet should be
  sent to end-point 192.168.1.2: 4000. Normally, to do this, it is
  necessary to have access to the NAT as an administrator.

\item Create the appropriate entries in the NAT tables so that
  communication between A and B is possible. This should happen if
  both A and B know the end-point of their interlocutor and send UDP
  traffic from A to B and from B to A at the same time.
\end{enumerate}
  
Obviously, the most transparent option for the user is the second.

\section{Deliverables}

A Python module called \texttt{NAT\_traversal.py} that inherits from
\texttt{minimal.py} and that implements the previously presented idea.

\section{Resources}

\bibliography{networking,nat}
