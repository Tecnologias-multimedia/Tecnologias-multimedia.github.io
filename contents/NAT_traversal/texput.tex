% Emacs, this is -*-latex-*-

\title{NAT Traversal}

\maketitle

\section{What is a ``NAT''?}

\begin{figure}
\begin{verbatim}
+----------+    +--------+    +----------+
| Internal +----+ NAT    +----+ External |
| Network  |    | Device |    | Network  |
+----------+    +--------+    +----------+
                               (Internet)
\end{verbatim}
\caption{NAT framework.}
\label{fig:NAT_framework}
\end{figure}

NAT stands for Network Address Translation, and it is a method of
remapping an IP address space into another by modifying network
address information in the IP header of packets while they are in
transit across a traffic routing device (see the
Fig~\ref{fig:NAT_framework}). When a network device on a internal
(usually private) network sends a packet P to the Internet, the NAT
device translates the internal IP address in the packet header to a
external (usually public) IP address. When a packet from the Internet
arrives at the NAT device, it translates the external IP address in
the packet header to the internal IP address of the device that sent
the packet P.

\section{Pros and cons}

NAT was designed because it provides the following advantages:
\begin{enumerate}
\item It allows multiple devices on a private network to share a
  single public IP address and access the internet. Therefore, a NAT
  device conserves public IP addresses, which are in short supply.
\item It can improve security by hiding the internal IP addresses of
  devices on a private network.
\end{enumerate}

However, NAT also has (mainly) one disadvantage:
\begin{enumerate}
\item It can be difficult (and sometimes, impossible) to host servers
  (for example, a Web server) on a private network, because the
  incoming traffic will be filtered by the NAT device at least the
  server initiates the conections (something that by definition is not
  done by servers).
\end{enumerate}

\section{IP addresses, ports, end-points, connections and peers}

An IP address (Internet Protocol address) is a unique address that
identifies a device on a network. It is an integer number (32 bits in
the case of IPv4 and 128 bit in IPv6) assigned to devices that use the
Internet to communicate. For example, computers that communicate over
the Internet or via local networks share information to a specific
location using IP addresses.

A port in the context of networking is a integer number (usually 16
bits long) associated with a network protocol that receives or
transmits communication for a specific service (for example, port 80
is used for the Hypertext Transfer Protocol (HTTP), which is used to
transfer web pages). Ports are used to identify a specific process or
service on a computer. When a computer receives a packet of data, the
operating system looks at the port number in the packet header to
determine which process or service should handle the data.

An end-point is the combination of a IP address and a port. End-points
identify networked actors (usually computer processes) that
interchange IP packets. For example, the end-point \texttt{8.8.8.8:53}
identify a name-server provided by Google Inc.

A conection is the action of transmitting at least one packet from one
end-point to another. Notice that connections are determined by a pair
of end-points.

In general, we will call peers to the networked entities.

\section{The Translation Table (TT)}

A NAT device uses a TT to track the connections that are established
between ``internal'' and ``external'' peers (see the
Fig.~\ref{fig:NAT_framework}). When a outgoing packet (generated in
the private network) arrives, the TT indicates the private end-point
of the peer that has sent this packet. When a incomming packet
(generated in the Internet) arrives, the TT indicates the private
end-point of a peer.

Notice that the entries in the TT are only generated for out-going
traffic.

\section{Types of NAT}

\subsection{Static and dynamic}

Depending on the number of available IP public addresses:

\begin{enumrate}
\item A static NAT maps a single private IP address to a single public
  IP address. This is useful for devices that need to be accessible
  from the internet, such as web servers.
\item A dynamic NAT maps multiple private IP addresses to a pool of
  public IP addresses. This is the most common type of NAT and is used
  in most home and office networks.
\end{enumerate}

\subsection{Types of ports mapping}

Depending on how the public ports are assigned:

\begin{enumerate}
\item EIM (Endpoint Independent Mapping) NAT devices reuse the same
  (external) port $P(X:y)$ for all traffic sent from the same internal
  end-point $X:y$ to any external endpoint. Therefore, it holds that
  \begin{equation}
    P(X:y)=y.
  \end{equation}
  This algorithm is used in Cone NATs and Port-Preservation Symmetric
  NAT devices.
\item In CDM (Connection Dependent Mapping) NAT devices the assigned
  (external) port $P(X:x,Y:y)$ is different for each connection
  $(X:x,Y:y)$, where $X:x$ is an internal end-point and $Y:y$ is an
  external end-point. In this case, we can distinguish between:
  \begin{enumerate}
    \item Fixed-$\Delta$ CDM, if the NAT device selects $P(X:x,Y:y)$
      using a constant port step $\Delta$.
    \item Random-$\Delta$ CDM, if the NAT device selects $P(X:x,Y:y)$
      using a random port step $\Delta$.
  \end{enumerate}
  These algorithms are used in symmetric NAT devices. Notice that,
  even if our device is fixed-$\Delta$, it could behave as a
  Random-$\Delta$ one due to the outgoing traffic generated by
  neighbor peers.
\end{enumerate}

\subsection{Types of filtering}

Depending on how the incomming traffic is accepted (or not), NAT are
classified as:

\begin{enumerate}
\item EIF (Endpoint Independent Filtering) when the incoming packets
  are forwarded without any extra restriction. This is used in Full
  Cone NAT devices.
\item ARF (Address Restricted Filtering) when only those incoming
  packets having the same external IP address as the one in the TE is
  forwarded to the internal endpoint. This policy is used by
  Restricted Cone NAT devices.
\item ERF (Endpoint Restricted Filtering) when only external incoming
  packets hav- ing the same external source endpoint like the one in
  the TE is forwarded to the internal endpoint. This policy is used in
  Port Restricted Cone (PRCN) and Symmetric NAT devices.
\end{enumerate}
  
\section{NAT-ing for what?}

One of the main problems that applications that transmit data over the
Internet must face is the use of NATs (Network Address Translators). A
NAT is a network device that converts public end-points\footnote{An
end-point is a tuple made up of an IP address and a port.} to private
end-points and vice versa. Due to this, the consumption rate of IPv4
addresses has dropped drastically and has meant that IPv6 is still not
the dominant network communication protocol, as expected.

The real reason for this is that only public servers really need a
public IP address, and the vast majority of Internet users use private
IP addresses on their local devices. This basically means that in each
of the companies that do not offer IP services from their corporate
network, and that in each of our homes, where we do not install public
servers either, we are reusing the same range of private IP addresses
over and over again.

\section{Negative consequences}

The main drawback of using NAT devices is that for two network
applications, A and B, that use IP and run on two hosts belonging to
two different private networks, with NAT-A and NAT-B respectively, to
be able to communicate, it is necessary for A to know the end-point
used in NAT-B for incoming traffic from NAT-A to B (through NAT-A),
and obviously, B must know the end-point used in NAT-A for incoming
traffic to A (from NAT-B).

\section{NAT configuration}

But how do we tell NAT-A to let traffic to A from NAT-B through, and
vice versa? Well, there are basically 2 alternatives:

\begin{enumerate}
\item Open ports ``by hand'', in both NATs, to indicate the public
  port open for incoming traffic to the process. For example, if A
  runs on end-point 192.168.1.2:4000 and the open public port is 5000,
  all UDP traffic arriving at port 5000 from the Internet should be
  sent to end-point 192.168.1.2: 4000. Normally, to do this, it is
  necessary to have access to the NAT as an administrator.

\item Create the appropriate entries in the NAT tables so that
  communication between A and B is possible. This should happen if
  both A and B know the end-point of their interlocutor and send UDP
  traffic from A to B and from B to A at the same time.
\end{enumerate}
  
Obviously, the most transparent option for the user is the second.

\section{Deliverables}

A Python module called \texttt{NAT\_traversal.py} that inherits from
\texttt{minimal.py} and that implements the previously presented idea.

\section{Resources}

\bibliography{networking,nat}
