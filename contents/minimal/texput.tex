% Emacs, this is -*-latex-*-

\title{Meeting \href{https://github.com/Tecnologias-multimedia/InterCom/blob/master/src/minimal.py}{\texttt{minimal}} InterCom}

\maketitle
\tableofcontents

\section{Description}

InterCom is an application that captures and plays audio, and
therefore, in Linux, runs on the top of one of the following audio
services:
\begin{enumerate}
\item \href{https://vicente-gonzalez-ruiz.github.io/ALSA/}{ALSA}.
\item
  \href{https://vicente-gonzalez-ruiz.github.io/PulseAudio/}{PulseAudio}
  (in the case of Xubuntu, this is the audio server that comes with
  \href{https://www.xfce.org/}{Xfce}).
\item \href{https://vicente-gonzalez-ruiz.github.io/JACK/}{JACK}.
\item \href{https://pipewire.org/}{Pipewire}.
\end{enumerate}

To abstract InterCom from the available audio resources, InterCom uses
\href{http://www.portaudio.com/}{PortAudio}~\cite{portaudio} (through
\href{https://vicente-gonzalez-ruiz.github.io/intro_to_sounddevice/}{sounddevice})
to capture and play the audio. Using sounddevice, we have two
alternatives for implementing InterCom (and in general, for any
real-time audio-processing application):

\subsection{Loop-based algorithm}

Roughtly, InterCom can be divided into 6 steps:

\begin{lstlisting}[language=Python,numbers=left]
  # Loop-based algorithm (to be called in a loop)
  def record_IO_and_play(chunk_size):
     chunk = record(chunk_size) # (1)
     packed_chunk = pack(chunk) # (2)
     send(packed_chunk) # (3)
     packed_chunk = receive() # (4)
     chunk = unpack(packed_chunk) # (5)
     play(chunk) # (6)
\end{lstlisting}

%\begin{pseudocode}{Sequential\_InterCom}{~}
%  \PROCEDURE{Record\_IO\_and\_Play}{~}
%  \BEGIN
%    \mathtt{chunk} \GETS \mathtt{record()}\\
%    \mathtt{packed\_chunk} \GETS \mathtt{pack(chunk)}\\
%    \mathtt{send(packed\_chunk)}\\
%    \mathtt{packed\_chunk} \GETS \mathtt{receive()}\\
%    \mathtt{chunk} \GETS \mathtt{unpack(packed\_chunk)}\\
%    \mathtt{play(chunk)}
%  \END
%  \ENDPROCEDURE
%\end{pseudocode}

where:

\begin{enumerate}
\item The \verb|record(chunk_size)| method captures a chunk (a
  fragment) of frames\footnote{A stereo sample, usually 16 + 16
    bits.}. In \verb|sounddevice|, this operation is carried on by the
  \href{https://python-sounddevice.readthedocs.io/en/0.4.0/api/streams.html#sounddevice.Stream.read}{\texttt{read()}
    method}. As can be seen in
  \href{https://raw.githubusercontent.com/Tecnologias-multimedia/intercom/master/test/sounddevice/wire4.py}{wire4.py}\footnote{
    \texttt{curl
      https://raw.githubusercontent.com/Tecnologias-multimedia/intercom/master/test/sounddevice/wire4.py
      > wire4.py}} and also in the documentation of
  \verb|sounddevice| if we read only the frames that are available in
  the soundcard's buffer, this generates a non-blocking operation and
  the chunk size depends on the instant of time in which this method
  is called. Otherwise, if we specify a number of frames different to
  the number of frames available, the operation can\footnote{The
    reading will be blocking if the number of requested frames is
    larger than the number of available frames.} be
  \href{https://python-sounddevice.readthedocs.io/en/0.4.0/api/streams.html#sounddevice.Stream.write}{blocking}
  and \href{https://en.wikipedia.org/wiki/I/O_bound}{I/O-bound} (the
  calling process sleeps until the required chunk size is returned).

\item \verb|pack(chunk)| process the chunk to create a
  \href{https://en.wikipedia.org/wiki/Network_packet}{packet} (or a
  sequence of packets), a structure that can be transmitted through
  the \href{https://en.wikipedia.org/wiki/Internet}{Internet} using the
  \href{https://en.wikipedia.org/wiki/Datagram}{Datagram} Model. In
  general, this is a
  \href{https://en.wikipedia.org/wiki/CPU-bound}{CPU-bounded}
  (CPU-intensive) operation because the payload of the packet can be
  compressed in order to minimize the transmission
  \href{https://en.wikipedia.org/wiki/Bit_rate}{bit-rate}.

\item \verb|send(packed_chunk)| sends the packet to our
  interlocutor. When datagrams are used, this step is not blocking nor
  CPU-bounding (the CPU usage is very low), as long as the number of
  packets/second is small and the sizes of the payloads are also
  small, as it is expected in InterCom.

\item \verb|receive()|, waits (blocking the calling process) for an
  incoming packet, and therefore, this operation is IO-bound. However,
  most \href{https://docs.python.org/3/library/socket.html}{socket
    API}s~\cite{python} offeer a
  \href{https://docs.python.org/3.8/library/socket.html#socket.socket.setblocking}{non-blocking
    option} where when a packet is not available in the kernel's
  buffer associated with the corresponding socket after a
  predetermined amount of time, some kind of exception is generated,
  and, in this case, it is resposabability of the programmer to
  generate an ``alternative'' chunk (in our case, for example, a chunk
  filled with
  \href{https://en.wikipedia.org/wiki/Digital_audio#Overview}{zeros}
  that will not produce any sound when it is played).

\item \verb|unpack(packed_chunk)| is (like the method
  \texttt{pack(chunk)}) a CPU-intensive step that transforms a
  packed chunk into a chunk of audio.

\item \verb|play(chunk)| renders the chunk. In general, this is an
  I/O-bound blocking action. However, if \verb|play()| is called at
  the same pace as \verb|record()|, and the record and play parameters
  are exactly the same (as usually happens in InterCom), the playing of
  the chunk should return immediately because the time that the
  \verb|play()| method needs to complete would match exactly the time
  that the \verb|record()| method requires (see
  \href{https://github.com/Tecnologias-multimedia/InterCom/blob/master/test/sounddevice/wire4b.py}{wire4b.py}).
\end{enumerate}

\begin{comment}
This algorithm works fine if the chunk size is controlled by
(\verb|sounddevice|) and also, the run-time required by the operations
\verb|pack(chunk)| and \verb|unpack(packet)| is shorter than the chunk
time. Unfortunately, the first premise (that we can use varying chunk
sizes) complicates slightly the implementation because we would work
with chunks of different lengths (that as you can see running
\verb|wire4.py|, in most of the iterations the size of the read chunk
is 0). This also complicates the control of the latency because the
chunk size is variable. However, the real problem appears when our
computer is not able to satisfy the second requirement, i.e., when the
chunk time is smaller than the time that we need to process the
chunks. This can only be addressed through code optimization, for
example, parallelization.
\end{comment}

\subsection{Timer-based algorithm}

The to-be-called-in-a-loop implementation depicted in the previous
section works fine, but finally InterCom uses a
to-be-called-by-an-interruption implementation because it is allows to
have running another task for free.

In this algorithm, the task dedicated to record and play the chunks of
audio is called periodically (probably, using some
\href{https://en.wikipedia.org/wiki/Timer}{timer} provided by the
sound hardware). This procedure guarantees a
\href{https://en.wikipedia.org/wiki/Glitch}{gliches}-free audio-IO
when constant chunk sizes are used, because the timer interruption
coincides exactly with the instant of time in which the
\verb|record()| and the \verb|play()| methods can be used without
blocking.
\begin{comment}
The following pseudo-code describes the new algorithm, which
is basically the previous one, except that the chunk size is fixed.

\begin{lstlisting}[language=Python,numbers=left]
  # Timer-based algorithm (to be called periodically)
  def record_IO_and_play(chunk_size):
     chunk = record(chunk_size)
     packed_chunk = pack(chunk)
     send(packed_chunk)
     packed_chunk = receive()
     chunk = unpack(packed_chunk)
     play(chunk)
\end{lstlisting}
\end{comment}

%\begin{pseudocode}{Timer-based\_InterCom}{~}
%  \PROCEDURE{Record\_IO\_and\_Play}{\mathtt{chunk\_size}}
%  \BEGIN
%    \mathtt{chunk} \GETS \mathtt{record(chunk\_size)}\\
%    \mathtt{packed\_chunk} \GETS \mathtt{pack(chunk)}\\
%    \mathtt{send(packed\_chunk)}\\
%    \mathtt{packed\_chunk} \GETS \mathtt{receive()}\\
%    \mathtt{chunk} \GETS \mathtt{unpack(packed\_chunk)}\\
%    \mathtt{play(chunk)}
%  \END
%  \ENDPROCEDURE
%\end{pseudocode}

The
\href{https://github.com/Tecnologias-multimedia/InterCom/blob/master/src/minimal.py#L111}{current
  implementation} of InterCom uses the Timer-based
(interruption-based) algorithm.

\section{Deliverables}

\textbf{None}. However, you should understand the meaning, purpose and
use of each of the parameters of \texttt{minimal.py}. For this, try
the following:

\begin{enumerate}
\item Run \texttt{minimial.py} using \texttt{localhost} as the
  destination for your chunks of audio (notice that this is the
  default configuration for the parameters
  \verb|--destination_address| and \verb|--destination_port|). Check
  that you can listen to yourself after some delay, and that the
  quality of the sound is good (it is recommended to used a
  headset). If you think that the quality is not high enough, comment
  this with your teacher. Remember that the parameters
  \verb|--show_stats|, \verb|--show_samples|, and
  \verb|--show_spectrum|, can provide some information about what is
  happening. Can you determine the delay?
\item Always using the command line, modify the parameter
  \verb|--frames_per_chunk|. Do you notice any changes in latency
  or audio quality?
\item Modify the parameter \verb|--frames_per_second|. Again, do
  you notice any changes in latency or audio quality? You should.
\item With one of your group mates, try to communicate using the same
  \href{https://en.wikipedia.org/wiki/Local_area_network}{LAN} (you
  will need to modify the parameter
  \verb|--destination_address|). Notice that some LANs, such as the
  provided by the UAL could filter your InterCom traffic. Good
  alternative options are you LAN at home or an ad-hoc LAN created
  with a mobile device.
\item Repeat the last experiment when two (or more) group mates send
  you their chunks. What do you think that is happening?
\item Finally, are you able to communicate with an interlocutor that
  is in a different LAN than you (for example, when two InterCom
  instances are running in different home networks)?
\end{enumerate}

Mark: \textbf{0 points}.

\begin{comment}
\section{What do you have to do?}

\begin{enumerate}

\item Read the source code of
  \href{https://github.com/Tecnologias-multimedia/intercom/blob/master/src/minimal.py}{\texttt{minimal.py}}
  and identify in the code the interrupt handler that implements
  the timer-based algorithm. Here you can find information about
  \href{https://github.com/vicente-gonzalez-ruiz/YAPT/blob/master/03-IO/networking/sockets.ipynb}{sockets}~\cite{YAPT} in Python.
  
\item Evaluate the performance (quality of sound) in different configurations:
  \begin{enumerate}
  \item When \verb|minimal.py| runs on the same computer. Can you
    describe how the parameter \verb|FRAMES_PER_SECOND| is modified?
  \item When \verb|minimal.py| runs in different hosts. In this case,
    try that the computers do not belong to the same local
    network. This
    \href{https://www.noip.com/support/knowledgebase/general-port-forwarding-guide/}{guide}
    can help. Consider in your analysis that
    \href{https://en.wikipedia.org/wiki/User_Datagram_Protocol}{UDP}~\cite{UDP}
    is used to transport the audio data. If you cannot use different
    local networks, create a WiFi zone using your mobile phone and try
    to produce a packet loss by separating the devices a
    sufficient distance.
  \end{enumerate}
\end{enumerate}
\end{comment}

\section{Resources}

\bibliography{python,sound,networking}
