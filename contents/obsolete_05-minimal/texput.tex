\newcommand{\TM}{\href{https://www.ual.es/estudios/grados/presentacion/plandeestudios/asignatura/4015/40154321?idioma=zh_CN}{Tecnologías Multimedia}}

\title{\TM{} - Study Guide - Milestone 5: Minimal InterCom (obsolete)}

\maketitle

\section{Description}


\section{What you have to do?}

\begin{enumerate}

\item Read the source code of
  \href{https://github.com/Tecnologias-multimedia/intercom/blob/master/src/minimal.py}{\texttt{minimal.py}}
  and identify in the code the interruption handler that implements
  the timer-based algorithm. Here you can find information about
  \href{https://github.com/vicente-gonzalez-ruiz/YAPT/blob/master/03-IO/networking/sockets.ipynb}{sockets}~\cite{YAPT} in Python.
  
\item Evaluate the performance (quality of the sound) in different configurations:
  \begin{enumerate}
  \item When \verb|minimal.py| runs in the same computer.
  \item When \verb|minimal.py| runs in different hosts. In this case,
    try that the computers do not belong to the same local
    network. This
    \href{https://www.noip.com/support/knowledgebase/general-port-forwarding-guide/}{guide}
    can help. Consider in your analysis that
    \href{https://en.wikipedia.org/wiki/User_Datagram_Protocol}{UDP}~\cite{UDP}
    is used to transport the audio data. If you cannot use different
    local networks, create a WiFi zone using your mobile phone and try
    to produce a loss of packets by separating the devices a
    sufficient distance.
  \end{enumerate}
   
%\item Implement a timer-based InterCom. Use as reference
%  \href{https://github.com/Tecnologias-multimedia/intercom/blob/master/test/sounddevice/wire.py}{wire.py}\footnote{
%  \texttt{curl
%    https://raw.githubusercontent.com/Tecnologias-multimedia/intercom/master/test/sounddevice/wire.py
%    > wire.py} }. Your implementation should be able to record,
%  \href{https://github.com/vicente-gonzalez-ruiz/YAPT/blob/master/03-IO/networking/sockets.ipynb}{transmit}~\cite{YAPT}
%  (using
%  \href{https://en.wikipedia.org/wiki/User_Datagram_Protocol}{UDP}~\cite{UDP})
%  and play raw (this means that the
%  $\mathtt{pack()}$/$\mathtt{unpack()}$ operations are not required for
%    this milestone) audio data between two computers.
  
\end{enumerate}

\section{Timming}

You should reach this milestone at most in one week.

\section{Deliverables}

The results of the experiments proposed in the previous section, that
will be presented to the rest of the class.

%A Python module with the implementation of the minimal InterCom. Name
%it \texttt{intercom\_minimal.py}. Store it at the
%\href{https://github.com/Tecnologias-multimedia/intercom}{root
%  directory} of your \texttt{intercom}'s repo.

\section{Resources}

\bibliography{python, networking, sound}
