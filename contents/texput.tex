% Emacs, this is -*-latex-*- 

\title{\href{https://tecnologias-multimedia.github.io/}{Tecnologías Multimedia} \newline Contents}

\maketitle

\begin{enumerate}
\item \href{https://tecnologias-multimedia.github.io/contents/framework/}{Framework}. % Día 0, hora 0.
\item \href{https://tecnologias-multimedia.github.io/contents/minimal/}{Meeting \texttt{minimal} InterCom}. % Hora 1. Sin presentación de alumnos.
\item \href{https://tecnologias-multimedia.github.io/contents/NAT_traversal/}{NAT Traversal}. % TODO
\item \href{https://tecnologias-multimedia.github.io/contents/latency/}{Hidding the Network Latency}. % 1 hora, sin presentación.
\item \href{https://tecnologias-multimedia.github.io/contents/echo_cancellation/}{Echo Cancellation}. % TODO
\item \href{https://tecnologias-multimedia.github.io/contents/threshold_of_hearing/}{Perceptual coding I: Threshold of Hearing}.
\item \href{https://tecnologias-multimedia.github.io/contents/simultaneous_masking/}{Perceptual coding II: Simultaneous Masking}.
%\item \href{https://tecnologias-multimedia.github.io/contents/simultaneous_masking/}{(TODO) Considering the simultaneous masking effect}.
\item \href{https://tecnologias-multimedia.github.io/contents/BR_control/}{Bit-rate control}. % Estudiar otras propuestas. TODO
\item \href{https://tecnologias-multimedia.github.io/contents/transform_coding/}{Transform coding for removing redundancy}. % Separar
%\item \href{https://tecnologias-multimedia.github.io/contents/perceptual_coding/}{Removing psychoacoustical redundancy}.
%\item \href{https://tecnologias-multimedia.github.io/contents/psychoacoustics/}{Removing psychoacoustical redundancy}.
\end{enumerate}

%\section{Resources}

\bibliography{data_compression,signal_processing,DWT,linux,python,git,text_compression,maths,image_compression,JPEG2000,intercom,sound,networking,NAT,audio_coding}
