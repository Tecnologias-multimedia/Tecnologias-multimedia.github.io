% Emacs, this is -*-latex-*-

\title{Increasing the Number of Subbands}

\maketitle
\tableofcontents

\section{The dyadic frequency partition is not enough}

In general, the number of subbands provided by the dyadic wavelet
domain~\cite{sayood2017introduction,vetterli1995wavelets} (remember,
if $l$ is the number of levels of the DWT, we obtain $l+1$ subbands)
is insufficient to accurately represent the diverse auditory
thresholds present in a typical hearing threshold curve
\cite{bosi2003intro}.

To address this issue two different algorithms can be used: (1) change
the dyadic decomposition by a linear decomposition, and (2) decompose
the dyadic subbands that we already have in smaller subbands.

Obviously, once we have created the frequency subbands, it would be a
matter of determining the corresponding quantization step sizes (QSS)
based on the ToH curve. We need one QSS per subband.

\section{Linear decomposition using Wavelet Packet Transform}
\label{sec:linear}
As an alternative to the Discrete Wavelet Transform (DWT), the Wavelet
Packet Transform (WPT) allows for a linear decomposition of the
signal's frequency range. This is essentially achieved by recursively
applying wavelet filters to both the low-frequency and high-frequency
subbands (see Milestone
\href{https://tecnologias-multimedia.github.io/contents/transform_coding/}{\emph{Transform
    Coding for Redundancy Removal}}). Consequently, if $l$ represents
the number of levels, then a total of $2^l$ subbands are
generated. For instance, with $l=6$, the DWT yields only $7$ subbands,
whereas the WPT produces $64$ subbands.

In more detail, this is what would need to be implemented:
\begin{enumerate}
\item Extend the chunk (this can be inherited from the class
  \verb|Temporal_Overlapped_DWT|).
\item Compute the WPT of the extended chunk (an example of this (but
  notice that for a non-extended chunk) can be found in the class
  \verb|Linear_ToH_NO|).
\item As we did with the DWT in \verb|Temporal_Overlapped_DWT|,
  extract the central parts of each subband to obtain a total number
  of frame-coefficients that matches to the number of frames in the
  chunk.
\end{enumerate}

\section{Linear decomposition of the dyadic subbands}
\label{sec:dyadic-linear}
Another solution (more close to the
\href{https://en.wikipedia.org/wiki/Bark_scale}{Bark scale}) is to
divide each one of the dyadic subbands into a number of
subbands. Thus, if we have $l+1$ dyadic subbands and now we decompose
each subband into $n$ (sub)subbands, we get a total number of $n(l+1)$
subbands.

For this, we can use (again) the WPT applied to each dyadic subband
generated by the DWT of the extended chunk. The idea here is to:
\begin{enumerate}
\item Extend the chunk (this can be inherited from the class
  \verb|Temporal_Overlapped_DWT|).
\item Compute the dyadic DWT transform of the extended chunk (this can
  be also reused from the class \verb|Temporal_Overlapped_DWT|). Let
  $l_{\text{DWT}}$ be the number of levels of this transform.
\item Compute the WPT of each dyadic (extended) subband. Let
  $l_{\text{WPT}}=\log_2(n)$ be the number of levels of this
  transform.
\item Extract the central parts of each packet-subband to obtain a
  total number of frame-coefficients that matches to the number of frames in
  the chunk.
\end{enumerate}
  
Notice that WPT performs a lineal decomposition. Therefore, for
example, if the sampling frequency is $48000$ Hz, $l_{\text{DWT}}=3$
and $l_{\text{WPT}}=1$ the lowest frequency dyadic subband goes from
$0$ Hz to $3000=\frac{24000}{2^3}$ Hz, and it will de divided into two
subbands with a size (bandwidth) of $1500$ Hz.

\section{Deliverables}

\begin{enumerate}
\item Implement a Python module called \verb|linear_ToH.py| where the
  functionality described in Section~\ref{sec:linear} has been
  implemented.
\item Implement a Python module called \verb|dyadic_linear_ToH.py|
  where the functionality described in Section~\ref{sec:dyadic-linear}
  has been implemented.
\end{enumerate}

\section{Resources}

\bibliography{maths,data_compression,DWT,audio_coding}

