% Emacs, this is -*-latex-*-

\newcommand{\TM}{\href{https://www.ual.es/estudios/grados/presentacion/plandeestudios/asignatura/4015/40154321?idioma=zh_CN}{Tecnologías Multimedia}}

\title{Considering the simultaneous masking (in progress)}

% Ante la presencia de suficiente energía en una subbanda, los quantization step sizes (QSS) del resto de subbandas podrían aumentarse porque la subbanda más energética enmascara a las demás. En el dominio wavelet, donde más o menos tenemos una distribución diádica de frecuencias entre subbandas, podemos suponer que si la subbanda h^i es la más energética y E(h^i) es su energía, podemos suponer que las subbandas h^{i+1} y h^{i-1} podrían unar un QSS*2, y que las subbandas h^{i+2} y h^{i-2} un QSS*4, etc. Si l^i es la más energética, entonces la QSS para h^i sería QSS*2, y así sucesivamente.

\maketitle

\section{Description}

\subsection{Simultaneous masking}
The HAS (Human Auditory System) has a finite frequency resolution,
which basically means that two different tonal sounds with different
amplitudes can be heard only as one (the louder) when they are
closed enough~\cite{bosi2003intro} (in frequency). When this happens,
the DWT subband~\cite{vetterli1995wavelets} where the quiet sound is
placed can be quantized more severely without perceiving that the
quantization noise in such subband is higher (see
Figure~\ref{fig:SM}). Note that the higher the quantization step,
the higher the compression ratio.

\begin{figure}
  \centering
  \png{simultaneous_masking}{600}
  \caption{An example of simultaneous masking. The threshold of
    hearing has been modified (increased) in the in the vicinity of
    the pure tone of 1 KHz.}
  \label{fig:SM}
\end{figure}

Masking depends on:
\begin{enumerate}
\item The proximity in frequency of the maskee and masker signals.
\item Their overlap over time.
\end{enumerate}

How much quantization noise can be generated without it being audible?
Psychoacoustics has the answer: A weaker signal (maskee signal) becomes
inaudible in the presence of (is masked by) a louder signal (masker signal). 

\subsection{An algorithm}

In the presence of sufficient energy in one subband, the quantization
step sizes (QSS) of the rest of the subbands could be increased
because the most energetic subband masks the others. In the wavelet
domain, where we have a dyadic distribution of frequencies between
subbands, we can assume that if the subband $h^i$ is the most
energetic and $E(h^i)$ is its energy, we can assume that the subbands
$h^{i+1}$ and $h^{i-1}$ could use a QSS*2, and the subbands $h^{i+2}$
and $h^{i-2}$ a QSS*4, etc. If $l^i$ is the most energetic, then the
QSS for $h^i$ would be QSS*2, and so on.

Notice that these modifications of the QSS should take into
consideration the ToH curve computed in the previous milestone.
  
\section{What you have to do?}

\begin{enumerate}
\item Write a generator of vectors of quantization steps. The
  generator should analyze the energy of each DWT subband and decide,
  considering the simultaneous masking effect, the vector of
  quantization steps (one for the subband).
\item Extend this procedure to the case of using Fourier/Wavelet
  subbands (see the previous milestone).
\end{enumerate}

\section{Deliverables}

The module \verb|simultaneous_masking.py| (inherited from
\verb|threshold.py|). Store it at the
\href{https://github.com/Tecnologias-multimedia/intercom}{root
  directory} of your InterCom repo.

\section{Resources}

\bibliography{maths,data_compression,DWT,audio_coding}

