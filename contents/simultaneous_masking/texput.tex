% Emacs, this is -*-latex-*-

\newcommand{\TM}{\href{https://www.ual.es/estudios/grados/presentacion/plandeestudios/asignatura/4015/40154321?idioma=zh_CN}{Tecnologías Multimedia}}

\title{\TM{} - Study Guide - Milestone 15: Considering the simultaneous masking}

\maketitle

\section{Description}

\subsection{Simultaneous masking}
The HAS (Human Auditory System) has a finite frequency resolution,
which basically means that two different tonal sounds with different
amplitudes can be heard only as one (the louder) when they are
closed enough~\cite{bosi2003intro} (in frequency). When this happens,
the DWT subband~\cite{vetterli1995wavelets} where the quiet sound is
placed can be quantized more severely without perceiving that the
quantization noise in such subband is higher (see
Figure~\ref{fig:SM}). Note that the higher the quantization step,
the higher the compression ratio.

\begin{figure}
  \centering
  \png{simultaneous_masking}{600}
  \caption{Simultaneous masking. The threshold of hearing has been modified (increased) in the in the vicinity of the pure tone of 1 KHz.}
  \label{fig:SM}
\end{figure}

Masking depends on:
\begin{enumerate}
\item The frequency of the maskee and masker signals.
\item Their overlap over time.
\end{enumerate}

How much quantization noise can be generated without it being audible?
Psychoacoustics has the answer: A weaker signal (maskee signal) becomes
inaudible in the presence of (is masked by) a louder signal (masker signal). 

\section{What you have to do?}

\begin{enumerate}
\item Write a generator of vectors of quantization steps. The
  generator should analyze the energy of each DWT subband and decide,
  considering the simultaneous masking effect, the vector of
  quantization steps (one for the subband).
\item Extend this procedure to the case of using Fourier/Wavelet
  subbands (see the previous milestone).
\end{enumerate}

\section{Deliverables}

The module \verb|simultaneous_masking.py| (inherited from
\verb|threshold.py|). Store it at the
\href{https://github.com/Tecnologias-multimedia/intercom}{root
  directory} of your InterCom repo.

\section{Resources}

\bibliography{maths,data_compression,DWT,audio_coding}

