% Emacs, this is -*-latex-*-

\title{Bit-Rate Control}

% USE 16 bits/coefficient (2 byte-planes).

\maketitle

\tableofcontents

\section{Description}
%{{{ 
\subsection{Impact of the network throughput}
%{{{ 
Along with the latency and its variation (jitter), another main aspect
to consider about the
\href{https://en.wikipedia.org/wiki/Telecommunications_link}{transmission
  link} (used in an InterCom session) is the
\href{https://en.wikipedia.org/wiki/Channel_capacityhttps://en.wikipedia.org/wiki/Network_throughput}{link
  throughput}\footnote{Measured in bits per second or a $10$-multiple
  of this transmission
  \href{https://en.wikipedia.org/wiki/Bandwidth_(computing)}{capacity}.} (or bit-rate)
that it can provide~\cite{Forouzan,Tanenbaum}. This bit-rate depends on
the maximum
\href{https://en.wikipedia.org/wiki/Channel_capacity}{capacity} (a
characteristic closely related with the available
\href{https://en.wikipedia.org/wiki/Bandwidth_(signal_processing)}{bandwidth})
and the
\href{https://en.wikipedia.org/wiki/Network_congestion}{congestion
  level} (that basically depends on the load) of the link. In general,
we can suppose that the capacity is constant over time (the
bandwidth provided by the link does not vary). On the
contrary, the throughput is time-varying and quite unpredictable,
because it depends on the congestion level that, in turn, depends
on the behavior of the network users.

In this milestone, we will measure the impact of the link throughput
on the \href{https://en.wikipedia.org/wiki/Quality_of_experience}{QoE}
provided by the current implementation of InterCom
(see \href{https://github.com/Tecnologias-multimedia/InterCom/blob/master/src/echo_cancellation.py}{\texttt{feedback\_supression.py}}). Similarly
to the procedure used to measure the impact of latency and jitter, we
will use
\href{https://man7.org/linux/man-pages/man8/tc.8.html}{\texttt{tc}}~\cite{bert2012lartc}
to control the amount of data\footnote{Notice that this upper bound in
  the bit-rate will also affect to the loss of chunks because if the
  link capacity is smaller than the audio bit-rate (throughput),
  sooner or later the transmission link will discard those chunks that
  cannot be buffered in the retransmission nodes (routers and
  switches). In this case, we would be at least contributing, if not
  causing, the link congestion.} that an InterCom instance will be
allowed to send in a local environment, with the aim of simulating a
real environment.\footnote{If \texttt{tc} (or a similar tool) is not
  avaiable in your OS, you can use a real transmission environment,
  but you must take into consideration that you will need to control
  de bit-rate in order obtain the points of the RD curve.}
%}}}

\subsection{Compressing the audio data with \href{https://zlib.net/}{zlib}}
%{{{ ...
To reduce the bit-rate, we need some way of compressing the data. The
\verb|pack()| and \verb|unpack()| methods can compress and
decompress, respectively, the chunks that are processed. To compress
and decompress, we will use a free data
\href{https://en.wikipedia.org/wiki/Codec}{codec} named
\href{https://en.wikipedia.org/wiki/DEFLATE}{DEFLATE}, which is based
on
\href{https://en.wikipedia.org/wiki/Lempel%E2%80%93Ziv%E2%80%93Storer%E2%80%93Szymanski}{LZSS}
  and \href{https://en.wikipedia.org/wiki/Huffman_coding}{Huffman
    Coding}~\cite{nelson96datacompression} (see this
  \href{https://github.com/vicente-gonzalez-ruiz/LZ77/blob/master/index.ipynb}{notebook} and
  this
  \href{https://github.com/vicente-gonzalez-ruiz/Huffman_coding/blob/master/index.ipynb}{notebook}). The
  DEFLATE algorithm is implemented in the Python's standard library
  \href{https://docs.python.org/3/library/zlib.html}{\texttt{zlib}}. %We
%have used this facility for compressing and decompressing the chunks
%that we are sending and receiving in the methods \verb|pack()| and
%\verb|unpack()|, respectively.

In order to compare the performance of different alternatives, the
above methods are implemented in the following modules, with different
functionality:

\begin{enumerate}
\item
  \href{https://github.com/Tecnologias-multimedia/InterCom/blob/master/src/DEFLATE\_raw.py}{\texttt{DEFLATE\_raw.py}}:
  Compress the raw chunks with DEFLATE.
\item
  \href{https://github.com/Tecnologias-multimedia/InterCom/blob/master/src/DEFLATE\_serial.py}{\texttt{DEFLATE\_serial.py}}:
  Compress the chunk after concatenating the channels (see
  Fig.~\ref{fig:reordering}). Note that with this data-shuffling,
  the samples are not interleaved and the
  \href{https://en.wikipedia.org/wiki/Correlation}{correlation}
  between consecutive samples is slightly increased. This should also
  increase the
  \href{https://en.wikipedia.org/wiki/Data_compression_ratio}{(data)
    compression ratio}.
\begin{figure}
  \begin{center}
    \myfig{graphics/reordering}{5cm}{500}
  \end{center}
  \caption{Sample reordering to create two independent channels.}
  \label{fig:reordering}
\end{figure}
\item
  \href{https://github.com/Tecnologias-multimedia/InterCom/blob/master/src/DEFLATE\_serial\_reset.py}{\texttt{DEFLATE\_serial\_reset.py}}:
  Similar to \verb|compress_serial.py|, but reseting DEFLATE at each
  new chunk-channel, i.e., compressing each chunk-channel
  independtly. The idea here is to see if DEFLATE is exploiting the
  redundancy between the consecutive channels.
\item
  \href{https://github.com/Tecnologias-multimedia/InterCom/blob/master/src/DEFLATE\_byteplanes2.py}{\texttt{DEFLATE\_byteplanes2.py}}:
  Similar to \verb|compress_serial.py| (samples are de-interleaved),
  but 2 code-streams are generated, one for the LSB (Low Significant
  Bytes) plane and another for the MSB (Most Significant Bytes) plane,
  working with 16 bits/sample (two planes). The idea here is to see if the MSB can
  be compressed more efficiently because it can contain runs of zeros,
  especially when we are sending \emph{quiet} audio sequences.
\item
  \href{https://github.com/Tecnologias-multimedia/InterCom/blob/master/src/DEFLATE\_byteplanes3.py}{\texttt{DEFLATE\_byteplanes3.py}}:
  Similar to \verb|DEFLATE_byteplanes2.py| but considering three
  byte-planes. This would allow compression of
  \emph{coefficients}\footnote{Used in a future improvements of
  intercom.} that requires more than two bytes to be represented.
\item
  \href{https://github.com/Tecnologias-multimedia/InterCom/blob/master/src/DEFLATE\_byteplanes4.py}{\texttt{DEFLATE\_byteplanes4.py}}:
  Consider four-byte planes.\footnote{Coefficient can require 4 bytes.}
\item
  \href{https://github.com/Tecnologias-multimedia/InterCom/blob/master/src/DEFLATE_byteplanes2\_interlaced.py}{\texttt{DEFLATE\_byteplanes2\_interlaced.py}}:
  Similar to \verb|DEFLATE_byteplanes2.py| but using the raw chunks
  (without concatenating the channels).
\end{enumerate}

Finally, notice that the number of UDP packets sent (which now will be
variable in length each of them) remains constant.
%}}}

\subsection{Quantization}
%{{{
At the hardware level, audio samples are usually represented
using \href{https://en.wikipedia.org/wiki/Pulse-code_modulation}{Pulse
  Code Modulation (PCM)}. In a PCM sample, the number of levels
the signal can take depends on the
\href{https://en.wikipedia.org/wiki/Audio_bit_depth}{number of
  bits/sample} (16 bits in our case).

Another key aspect to consider is that the processing that the
\href{https://en.wikipedia.org/wiki/Auditory_system}{Human
  Auditory System (HAS)} performs to understand audio signals has several
\href{https://en.wikipedia.org/wiki/Psychoacoustics}{\emph{sources} of
  perceptual redundancy}. One of these sources is the
\href{https://en.wikipedia.org/wiki/Equal-loudness_contour}{finite
  number of different volumen levels that a human being can
  recognize}~\cite{bosi2003intro}. In this milestone, we will profit from
such fact to decrease the transmission bit-rate by reducing
quality. In most
\href{https://en.wikipedia.org/wiki/Lossy_compression}{lossy
  compression} schemes, quantization is the main source of
\href{https://en.wikipedia.org/wiki/Distortion}{distortion}~\cite{taubman2002jpeg2000}.

\href{https://en.wikipedia.org/wiki/Quantization_(signal_processing)}{Scalar
  Quantization} (SQ) is the process of decreasing the number of
discrete levels that a signal can
take~\cite{sayood2017introduction}. \href{https://en.wikipedia.org/wiki/Vector_quantization}{Vector
  Quantization} (VQ) is similar, but it is applied to tuples of samples
at the same time~\cite{vetterli2014foundations}. SQ is used when
the samples are decorrelated or, although correlated,
decorrelation will be exploited in a posterior
\href{https://en.wikipedia.org/wiki/Entropy_encoding}{entropy coding}
stage (which in our case is DEFLATE), because the
\href{https://en.wikipedia.org/wiki/Quantization_(signal_processing)#Rate%E2%80%93distortion_optimization}{coding
  efficiency} provided by VQ is marginal in this
context~\cite{vetterli2014foundations}, and generally requires higher
computational resources.

Quantizers can also be classified into
\href{https://en.wikipedia.org/wiki/Quantization_(signal_processing)#Mid-riser_and_mid-tread_uniform_quantizers}{uniform}
and
\href{https://nptel.ac.in/content/storage2/courses/117104069/chapter_5/5_5.html}{non-uniform}~\cite{sayood2017introduction,vetterli2014foundations}. An
uniform quantizer distributes the available representation levels
uniformely over the range of input values. Non-uniform quantizers use
higher density of representation levels (more output levels per input
different values) to those intervals of input values that occur more
often.\footnote{The decision intervals and the representation levels
  in each interval can be also optimized using other criteria, such
  as, minimizing the rate/distortion at a given point of the RD
  curve.} Quantizers can also be classified into static
and
\href{https://en.wikipedia.org/wiki/Adaptive_differential_pulse-code_modulation}{adaptive
  quantizers}. In the first case, the
\href{https://en.wikipedia.org/wiki/Probability_distribution}{distribution}
of the representation levels remains constant during the quantization
stage, and in the second case, the quantizer parameters are adapted
dynamically to the characteristics of the input signal.

In this milestone we use an
\href{https://en.wikipedia.org/wiki/Quantization_(signal_processing)#Dead-zone_quantizers}{uniform
  dead-zone scalar static quantizer}, which can be implemented
efficiently (in software) for digital signals. Moreover, dead-zone
quantizers tend to produce more quantization indices equal to 0 (which
increases compression ratios) at the cost of generating more
quantization noise for values of the input signal close to 0, or what
is the same, decreasing the
\href{https://en.wikipedia.org/wiki/Signal-to-noise_ratio}{SNR} for
small signal values. A priori, this could be seen as a problem, but in
reality it is not because precisely when the amplitude of the signal
is small and the noise is independent of its amplitude (which usually
happens with
\href{https://en.wikipedia.org/wiki/Noise_(electronics)}{electronic
  noise}), the SNR of the input signal has its lowest value precisely
for those values close to 0. Therefore, the quantizer will basically
change electronic noise by quantization noise\footnote{The error
  generated by the quantization stage.} (see this
\href{https://github.com/vicente-gonzalez-ruiz/signal_quantization}{introduction
  to signal quantization} document and this
\href{https://github.com/vicente-gonzalez-ruiz/scalar_quantization}{comparative
  between digital scalar quantizers} document). Finally, although this
is a feature that we are not going to exploit for now, dead-zone
quantizers are equivalent to encode the signal by bit-planes when the
quantization steps sizes are powers of two, allowing the design of
progressive entropy encoding schemes, if required.

%}}}

\subsection{(Bit-)Rate control and distortion}
%{{{ 
The number of representation levels used by a quantizer depends on the
quantization step (size), typically denoted by $\Delta$. The higher
the $\Delta$, the smaller the number of representation levels, and
therefore the higher the distortion generated by the quantization
error, and the smaller the output bit-rate! This generates a
\href{https://en.wikipedia.org/wiki/Rate%E2%80%93distortion_theory}{rate/distortion}
  trade-off that is descriptive of all lossy compressors (more bits,
  less distortion, and viceversa).

  In order to minimize the lost of data, the rate can be controlled in
  real-time transmission systems by modifiying $\Delta$ when
  \href{https://en.wikipedia.org/wiki/Network_congestion}{congestion}
  occurs. However, notice that depending of the
  \href{https://en.wikipedia.org/wiki/Entropy_(information_theory)}{entropy}
  coding stage and the characteristics of the signal
  (\href{https://en.wikipedia.org/wiki/Variance}{variance}, entropy,
  etc.) may not exist a clear relationship between $\Delta$ and the
  output bit-rate. This happens using DEFLATE.

Notice also that any rate control algorithm based on quantization has
a characteristic RD (Rate/Distortion) curve, in which the X axis
represents the (in the case of InterCom, received) (bit-)rate, and the Y
axis the distortion in the reconstruction (in the case of InterCom,
the played audio sequence) obtained after the
``de-quantization''\footnote{From a signal processing point of view,
  the term ``de-quantization'' refers to restore the original dynamic
  range of the signal, but notice that this does not imply that the
  original signal will be restored. This only happens when
  $\Delta=1$.}. Some examples can be found in
\href{https://github.com/Tecnologias-multimedia/Tecnologias-multimedia.github.io/blob/master/contents/BR_control/audio_quantization.ipynb}{this
  notebook}.
%}}}

\subsection{The current implementation(s) for the control of the bit-rate}
%{{{ 
Bit-Rate (BR) control through quantization has been implemented in the
class \verb|BR_Control*| of the modules \texttt{BR\_control*.py}. This
class overrides the inherited methods \verb|pack()| and
\verb|unpack()|, performing now (remember that the chunks are already
``DEFLATE-encoded and -decoded''):

\begin{lstlisting}[language=Python]
  def pack(chunk_number, chunk):
    quantized_chunk = quantize(chunk)  # (1)
    packed_chunk = Buffering.pack(chunk_number, quantized_chunk)  # (2)
    return packed_chunk  # (3)
\end{lstlisting}

\begin{lstlisting}[language=Python]
  def unpack(packed_chunk):
    (chunk_number, quantized_chunk) = Buffering.unpack(packed_chunk)  # (1)
    chunk = dequantize(quantized_chunk)  # (2)
    return (chunk_number, chunk)  # (3)
\end{lstlisting}

Notice that, regarding the bit-rate control, you will find four
implementations related to this milestone:
\begin{enumerate}
\item \verb|BR_control_no.py|: Uses a constant
  $\Delta>0$.\footnote{$\Delta$ must be always bigger than $0$, by
definition, and this does not depend on the bit-rate control.} There
  is not BR control.
\item \verb|BR_control_add_lost.py|: Every second runs:
  \begin{equation}
    \left\{
    \begin{array}{ll}
      \Delta = \Delta + L - 1 & \quad\text{always}, \\
      \Delta = \Delta_{\text{min}} & \quad\text{if}~\Delta < \Delta_{\text{min}}.                                 
    \end{array}
    \right.
  \end{equation}
  %\begin{equation}
  %  \Delta = \Delta + L - 1
  %\end{equation}
  where $L$ is the number of lost (received) chunks in the previous
  second. Notice that this heuristic (and the following ones) supposes
  that our interlocutor is losing (on average) the same number of
  chunks that we are losing.
\item \verb|BR_control_lost.py|: Every second runs:
  \begin{equation}
    \left\{
    \begin{array}{ll}
      \Delta = L - 1 & \quad\text{always}, \\
      \Delta = \Delta_{\text{min}} & \quad\text{if}~\Delta < \Delta_{\text{min}}.
    \end{array}
    \right.
  \end{equation}
  %\begin{equation}
  %  \Delta = L - 1\quad\text{if}~L>2.
  %\end{equation}
\item \verb|BR_control_conservative.py|: Every second runs:
  \begin{equation}
    \left\{
    \begin{array}{ll}
      \Delta = 2\Delta & \quad\text{if}~L>2, \\
      \Delta = \frac{10}{11}\Delta & \quad\text{always}, \\
      \Delta = \Delta_{\text{min}} & \quad\text{if}~\Delta < \Delta_{\text{min}}.
    \end{array}
    \right.
  \end{equation}
\end{enumerate}
%}}}

%}}}

\section{Deliverables}

Remember to report both, the experiments and the results. Notebooks are appreciated.
\begin{enumerate}
\item \textbf{Which data-ordering performs better?}:
  % {{{
  
Determine empirically which ordering of the chunk data is the most
efficient from a lossless data compression point of view (the smaller
the bit-rates, the higher the compression). Use the audio sequence you
want.\footnote{Some samples are stored in the \texttt{data} directory of
  InterCom.} Notice that using different audio files you could obtain
different results.

%Mark: \textbf{1 point}.

% }}}

\item \textbf{Which BR control algorithm shows the best RD (Rate/Distortion) curve?}:
% {{{
  
Considering the
\href{https://en.wikipedia.org/wiki/Root-mean-square_deviation}{RMSE
  (Root Mean Square Error)} as a distortion measure between the sent
and the received audio signal, for a concrete audio sequence,
generate the RD curve considering a set of different simulated
transmission environments (use
\href{https://vicente-gonzalez-ruiz.github.io/about_tc/}{tc} or a
any other similar tool).

Remember that the X-axis must express bit-rate and the Y-axis,
distortion.

%Mark: \textbf{2 points}.

\end{enumerate}


%}}}

\section{Resources}

\bibliography{networking,text_compression,JPEG2000,audio_coding,data_compression,signal_processing}
