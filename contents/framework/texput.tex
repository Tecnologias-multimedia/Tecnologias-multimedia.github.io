% Emacs, this is -*-latex-*-

\title{Framework}

\maketitle

\section{Description}

The InterCom project \cite{intercom} is a collection of
\href{https://www.python.org/}{Python}
\href{https://docs.python.org/3/tutorial/modules.html#modules}{modules}~\cite{python}.
Python has been ported to
\href{https://www.python.org/download/other/}{almost all} the current
OSs, including mobile \href{https://kivy.org/#home}{devices}. Among
all available development environments, we will develop the project on
Linux, and more concretely, on
\href{https://xubuntu.org/download/}{Xubuntu 22.04 (Jammy Jellyfish)}
\cite{xubuntu}, running natively (no
\href{https://en.wikipedia.org/wiki/Virtualization}{virtualization}),
if possible. Please, follow for this,
\href{https://vicente-gonzalez-ruiz.github.io/Xubuntu_install/}{to
  install Xubuntu}. Alternatively, you can use any other Linux
distribution (or even any other OS), but you should be able to
provide technical support yourself, in case you need it.

You don't need to master Python to follow this course, but it is
convenient for you to rely on some Python programming tutorial, such
as \href{https://docs.python.org/3/tutorial/}{The Python Tutorial}
\cite{python-tutorial} if you realize that the language is a setback
for you. If you need to start with Python from the beginning, an
introduction to Python such as this
\href{https://github.com/vicente-gonzalez-ruiz/YAPT/tree/master/workshops/programacion_python_ESO}{workshop
  of YAPT} \cite{YAPT} could also be helpful. See also
\href{http://zetcode.com/lang/python/}{ZetCode's Python
  Tutorial}.

To run InterCom, the following alternatives should work:
\begin{enumerate}
\item Use the Python interpreter shipped with your OS. If you are
  using the recommended Xubuntu version in this guide, the Python
  interpreter is fine. In this case, use a virtual
  \href{https://docs.python.org/3/library/venv.html}{environment} for
  InterCom. To do this, run:
  \begin{lstlisting}[language=Bash]
    python3 -m venv ~/enviroments/intercom     # Create (only the first time)
    source ~/enviroments/intercom/bin/activate # Activate (allways you run InterCom)
  \end{lstlisting}
  And to deactivate the environment, run:
  \begin{lstlisting}[language=Bash]
    deactivate
  \end{lstlisting}
  
\item Use a specific version of Python installed in a virtual environment. In this case, follow this 
\href{https://vicente-gonzalez-ruiz.github.io/Python_install/}{guide}.

\end{enumerate}

Finally, if you want to contribute to the
\href{https://github.com/Tecnologias-multimedia/intercom}{InterCom}
project \cite{intercom} you must understand the basics of Git
\cite{Git-book} and \href{https://github.com/}{GitHub}\footnote{There
are other Git-based hosting services such as
\href{https://about.gitlab.com/}{GitLab} and
\href{https://www.atlassian.com/git}{Altassian}/\href{https://bitbucket.org/product}{BitBucket},
but GitHub is the most used one.} \cite{GitHub}, and how to use the
\href{https://guides.github.com/introduction/flow/index.html}{The
  GitHub (Work-)Flow} and the
\href{https://github.com/vicente-gonzalez-ruiz/fork_and_branch_git_workflow}{Fork-and-Branch
  Git Workflow} \cite{fork-and-branch-git-workflow}. Be aware that to
contribute to InterCom, an GitHub account is required. Please, follow
this
\href{https://vicente-gonzalez-ruiz.github.io/using_GitHub/}{minimal
  Git guide}. In the guide, you must consider that
\texttt{<organization\_name>} is \texttt{Tecnologias-multimedia} and
that \texttt{<repo\_name>} is \texttt{intercom}.

\section{Resources}

\bibliography{python,sound,intercom,git,linux}
