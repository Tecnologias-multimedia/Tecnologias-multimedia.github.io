% Emacs, this is -*-latex-*-

\title{\href{https://tecnologias-multimedia.github.io/}{Tecnologías Multimedia} \newline Framework}

\maketitle
\tableofcontents

\section{Description}

The InterCom project \cite{intercom} is a collection of
\href{https://www.python.org/}{Python}
\href{https://docs.python.org/3/tutorial/modules.html#modules}{modules}~\cite{python}.
Python has been ported to
\href{https://www.python.org/download/other/}{almost all} the current
OSs, including mobile \href{https://kivy.org/#home}{devices}. Among
all available development environments, the prefered one is Linux, and
more concretely, on \href{https://xubuntu.org/download/}{Xubuntu 24.04
  (Noble Numbat)} \cite{xubuntu}, running natively (no
\href{https://en.wikipedia.org/wiki/Virtualization}{virtualization}),
if possible (see
\href{https://vicente-gonzalez-ruiz.github.io/Xubuntu_install/}{How to
  install xubuntu}). Alternatively, you can use any other Linux
distribution, such as Debian or Arch, or even any other OS (OSX or
Windows).

InterCom has been written in Python. You don't need to master Python
to follow this course, but it is convenient for you to rely on some
Python programming tutorial, such as
\href{https://docs.python.org/3/tutorial/}{The Python Tutorial}
\cite{python-tutorial} if you realize that the language is a setback
for you. If you need to start with Python from the beginning, an
introduction to Python such as this
\href{https://github.com/vicente-gonzalez-ruiz/YAPT/tree/master/workshops/programacion_python_ESO}{workshop
  of YAPT} \cite{YAPT} could also be helpful. See also
\href{http://zetcode.com/lang/python/}{ZetCode's Python Tutorial}, for
example (Internet if full of Python manuals).

To run InterCom, the following alternatives (both) should work:
\begin{enumerate}
\item Use the Python interpreter shipped with your OS. If you are
  using the recommended xubuntu version, the Python interpreter
  provided by the OS is fine. In this case, use a virtual
  \href{https://docs.python.org/3/library/venv.html}{environment} when
  working with InterCom. To do this, run:
  \begin{lstlisting}[language=Bash]
    # Supposing that the ~/envs/ folder contains the virtual environments 
    python3 -m venv ~/envs/TM # Create the environment TM (only the first time)
    source ~/envs/TM/bin/activate # Activate it (allways you run InterCom)
  \end{lstlisting}
  And to deactivate the environment, run:
  \begin{lstlisting}[language=Bash]
    deactivate
  \end{lstlisting}
  This is the recommended alternative.
  
\item Use a specific version of Python installed in a virtual environment. In this case, follow this 
\href{https://vicente-gonzalez-ruiz.github.io/Python_install/}{guide}.

\end{enumerate}

Finally, if you want to contribute to the
\href{https://github.com/Tecnologias-multimedia/intercom}{InterCom}
project \cite{intercom} you must understand the basics of Git
\cite{Git-book} and \href{https://github.com/}{GitHub}\footnote{There
are other Git-based hosting services such as
\href{https://about.gitlab.com/}{GitLab} and
\href{https://www.atlassian.com/git}{Altassian}/\href{https://bitbucket.org/product}{BitBucket},
but GitHub is the most used one.} \cite{GitHub}, and how to use the
\href{https://guides.github.com/introduction/flow/index.html}{The
  GitHub (Work-)Flow} and the
\href{https://github.com/vicente-gonzalez-ruiz/fork_and_branch_git_workflow}{Fork-and-Branch
  Git Workflow} \cite{fork-and-branch-git-workflow}. Be aware that to
contribute to InterCom, an GitHub account is required. Please, follow
this
\href{https://vicente-gonzalez-ruiz.github.io/using_GitHub/}{minimal
  Git guide}. In the guide, you must consider that
\texttt{<organization\_name>} is \texttt{Tecnologias-multimedia} and
that \texttt{<repo\_name>} is \texttt{intercom}.

\section{Resources}

\bibliography{python,sound,intercom,git,linux}
