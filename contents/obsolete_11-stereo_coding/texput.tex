\newcommand{\TM}{\href{https://www.ual.es/estudios/grados/presentacion/plandeestudios/asignatura/4015/40154321?idioma=zh_CN}{Tecnologías Multimedia}}

\title{\TM{} - Study Guide - Milestone 11: Spatial decorrelation in stereo audio signals}

\maketitle

\section{Description}

\begin{comment}
Unfortunately, this
estatement is only true if the
\href{https://en.wikipedia.org/wiki/Entropy_encoding}{entropy coding}
stage generates the same number of bits for both subbands, something
that rarely happens because we are compressing the coefficients of the
subbands considering the complete chunk (remember that we are
exploiting the
\href{https://en.wikipedia.org/wiki/Correlation_and_dependence}{statistical
  correlation} between the sequence of coefficients generated by all
the frames of a chunk). In general, the amount of information provided
by each subband $w_i$ is different, and therefore, the discrete
\href{https://en.wikipedia.org/wiki/Rate%E2%80%93distortion_theory}{Rate
  Distortion} (RD) curve\footnote{A discrete RD curve is defined by
the
\href{https://en.wikipedia.org/wiki/Multi-objective_optimization}{Pareto
  front} form by the RD points.} generated by each subband is distint.

The standard solution for this problem is to select a $\Delta_i$ value
for each $w_i$ that select RD points with the same RD
\href{https://en.wikipedia.org/wiki/Slope}{slope}~\cite{vetterli2014foundations,sayood2017introduction}.
A RD point is defined as a pair or $(r,d)$ values where $r$ represents
a bit-rate (typically expressed in bits/sample) and $d$ represents a
distortion (that uses to be the
\href{https://en.wikipedia.org/wiki/Root-mean-square_deviation}{RMSE}
when we use the L$_2$ norm to measure distances). Therefore, to find
the two RD curves for the current chunk, we should apply the stereo
transform, use a set of quantization steps to each subband, and
compress the resulting quantization indexes for each quantization
step. This would find the $r$ values of our RD curve. Then,
decompress, dequantize and find the distortion for the chunk. This
would find the $d$ values. Finally, with this RD curve, we should
select the $\Delta_i$ values that provides the same slope for both
subbands.

Obviously, we can not use the previous algoritm for computing the RD
curves in a real-time application such as InterCom.\footnote{The
amount of computational resources would increase significatively.} We
need to make some assumptions in order to reduce the computational
cost of finding the RD curves. The first of our assumptions is that
between (temporally) adjacent chunks the RD curves are going to be
similar. Therefore, we can build the RD curve for the current chunk by
using the RD points generated\footnote{Each chunk is quantized and
compressed, so, we only need to compute the distortion to have the RD
point used for the chunk.} by the compression of previously processed
chunks. The second assumption is that we can estimate the average
slope of the complete RD curve by using only 2 RD points. Using this
information, we will try to use, for the current chunk, a pair of
$\Delta_i$ quantization steps that produce two RD curves (one curve
per subband) with the same average slope.
\end{comment}

\section{What you have to do?}


\begin{comment}
\begin{enumerate}
\item In a module named stereo\_coding.py, inherit the class
  Quantization and create a class named Stereo\_Coding.
\item Override the methods pack() and unpack(). In pack() perform the
  analysis transform previously described, and in unpack() the
  synthesis transform. These procedures should be applied to all the
  frames of a chunk using
  \href{https://www.oreilly.com/library/view/python-for-data/9781449323592/ch04.html}{vectorized
    operations}.
\item Has the
  \href{https://en.wikipedia.org/wiki/Data_compression_ratio}{compression
    ratio} been improved (on
  \href{https://en.wikipedia.org/wiki/Average}{average})? How much?
\end{enumerate}
\end{comment}

\section{Timming}

You should reach this milestone at most one week.

\section{Deliverables}

The RD curves and the required information to redo them.

\section{Resources}

\bibliography{maths,data-compression,DWT,audio-coding,signal-processing}
